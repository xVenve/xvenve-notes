\documentclass[12pt, twoside, openright]{report} %fuente a 12pt, formato doble pagina y chapter a la derecha
\raggedbottom % No ajustar el contenido con un salto de pagina

% MÁRGENES: 2,5 cm sup. e inf.; 3 cm izdo. y dcho.
\usepackage[
a4paper,
vmargin=2.5cm,
hmargin=3cm
]{geometry}

% INTERLINEADO: Estrecho (6 ptos./interlineado 1,15) o Moderado (6 ptos./interlineado 1,5)
\renewcommand{\baselinestretch}{1.15}
\parskip=6pt

% DEFINICIÓN DE COLORES para portada y listados de código
\usepackage[table]{xcolor}
\definecolor{azulUC3M}{RGB}{0,0,102}
\definecolor{gray97}{gray}{.97}
\definecolor{gray75}{gray}{.75}
\definecolor{gray45}{gray}{.45}

% Soporte para GENERAR PDF/A
\usepackage[a-1b]{pdfx}

% ENLACES
\usepackage{hyperref}
\hypersetup{colorlinks=true,
	linkcolor=black, % enlaces a partes del documento (p.e. índice) en color negro
	urlcolor=blue} % enlaces a recursos fuera del documento en azul

% Añadir pdfs como partes del documento
\usepackage{pdfpages}

% Quitar la indentación de principio de los parrafos
\setlength{\parindent}{0em}

% EXPRESIONES MATEMATICAS
\usepackage{amsmath,amssymb,amsfonts,amsthm}

\usepackage{txfonts} 
\usepackage[T1]{fontenc}
\usepackage[utf8]{inputenc}

% Insertar graficas y fotos
\usepackage{tikz}
\usepackage{pgfplots}

\usepackage[spanish, es-tabla]{babel} 
\usepackage[babel, spanish=spanish]{csquotes}
\AtBeginEnvironment{quote}{\small}

% diseño de PIE DE PÁGINA
\usepackage{fancyhdr}
\pagestyle{fancy}
\fancyhf{}
\renewcommand{\headrulewidth}{0pt}
\fancyfoot[LE,RO]{\thepage}
\fancypagestyle{plain}{\pagestyle{fancy}}

% DISEÑO DE LOS TÍTULOS de las partes del trabajo (capítulos y epígrafes o subcapítulos)
\usepackage{titlesec}
\usepackage{titletoc}
\titleformat{\chapter}[block]
{\large\bfseries\filcenter}
{\thechapter.}
{5pt}
{\MakeUppercase}
{}
\titlespacing{\chapter}{0pt}{0pt}{*3}
\titlecontents{chapter}
[0pt]                                               
{}
{\contentsmargin{0pt}\thecontentslabel.\enspace\uppercase}
{\contentsmargin{0pt}\uppercase}                        
{\titlerule*[.7pc]{.}\contentspage}                 

\titleformat{\section}
{\bfseries}
{\thesection.}
{5pt}
{}
\titlecontents{section}
[5pt]                                               
{}
{\contentsmargin{0pt}\thecontentslabel.\enspace}
{\contentsmargin{0pt}}
{\titlerule*[.7pc]{.}\contentspage}

\titleformat{\subsection}
{\normalsize\bfseries}
{\thesubsection.}
{5pt}
{}
\titlecontents{subsection}
[10pt]                                               
{}
{\contentsmargin{0pt}                          
	\thecontentslabel.\enspace}
{\contentsmargin{0pt}}                        
{\titlerule*[.7pc]{.}\contentspage}  


% DISEÑO DE TABLAS.
\usepackage{multirow} % permite combinar celdas 
\usepackage{caption} % para personalizar el título de tablas y figuras
\usepackage{floatrow} % utilizamos este paquete y sus macros \ttabbox y \ffigbox para alinear los nombres de tablas y figuras de acuerdo con el estilo definido. Para su uso ver archivo de ejemplo 
\usepackage{array} % con este paquete podemos definir en la siguiente línea un nuevo tipo de columna para tablas: ancho personalizado y contenido centrado
\newcolumntype{P}[1]{>{\centering\arraybackslash}p{#1}}
\DeclareCaptionFormat{upper}{#1#2\uppercase{#3}\par}

% Diseño de tabla para ingeniería
\captionsetup[table]{
	format=hang,
	name=Tabla,
	justification=centering,
	labelsep=colon,
	width=.75\linewidth,
	labelfont=small,
	font=small,
}

% DISEÑO DE FIGURAS.
\usepackage{graphicx}
\graphicspath{{img/}} %ruta a la carpeta de imágenes

% Diseño de figuras para ingeniería
\captionsetup[figure]{
	format=hang,
	name=Fig.,
	singlelinecheck=off,
	labelsep=colon,
	labelfont=small,
	font=small		
}

% NOTAS A PIE DE PÁGINA
\usepackage{chngcntr} %para numeración continua de las notas al pie
\counterwithout{footnote}{chapter}

% LISTADOS DE CÓDIGO
% soporte y estilo para listados de código. Más información en https://es.wikibooks.org/wiki/Manual_de_LaTeX/Listados_de_código/Listados_con_listings
\usepackage{listings}

% definimos un estilo de listings
\lstdefinestyle{estilo}{ frame=Ltb,
	framerule=0pt,
	aboveskip=0.5cm,
	framextopmargin=3pt,
	framexbottommargin=3pt,
	framexleftmargin=0.4cm,
	framesep=0pt,
	rulesep=.4pt,
	backgroundcolor=\color{gray97},
	rulesepcolor=\color{black},
	%
	basicstyle=\ttfamily\footnotesize,
	keywordstyle=\bfseries,
	stringstyle=\ttfamily,
	showstringspaces = false,
	commentstyle=\color{gray45},     
	%
	numbers=left,
	numbersep=15pt,
	numberstyle=\tiny,
	numberfirstline = false,
	breaklines=true,
	xleftmargin=\parindent
}

\captionsetup[lstlisting]{font=small, labelsep=period}
% fijamos el estilo a utilizar 
\lstset{style=estilo}
\renewcommand{\lstlistingname}{\uppercase{Código}}

\pgfplotsset{compat=1.17} 
%-------------
%	DOCUMENTO
%-------------

\begin{document}
\pagenumbering{roman} % Se utilizan cifras romanas en la numeración de las páginas previas al cuerpo del trabajo
	
%----------
%	PORTADA
%----------	
\begin{titlepage}
	\begin{sffamily}
	\color{azulUC3M}
	\begin{center}
		\begin{figure}[H] %incluimos el logotipo de la Universidad
			\makebox[\textwidth][c]{\includegraphics[width=16cm]{Portada_Logo.png}}
		\end{figure}
		\vspace{2.5cm}
		\begin{Large}
			Grado en Ingeniería Informática\\			
			20XX-20XX\\
			\vspace{2cm}		
			\textsl{Apuntes}\\
			\bigskip
		\end{Large}
		 	{\Huge SUBJECT}\\
		 	\vspace*{0.5cm}
	 		\rule{10.5cm}{0.1mm}\\
			\vspace*{0.9cm}
			{\LARGE Jorge Rodríguez Fraile\footnote{\href{mailto:100405951@alumnos.uc3m.es}{Universidad: 100405951@alumnos.uc3m.es}  |  \href{mailto:jrf1616@gmail.com}{Personal: jrf1616@gmail.com}}}\\ 
			\vspace*{1cm}
	\end{center}
	\vfill
	\color{black}
		\includegraphics[width=4.2cm]{img/creativecommons.png}\\
		Esta obra se encuentra sujeta a la licencia Creative Commons\\ \textbf{Reconocimiento - No Comercial - Sin Obra Derivada}
	\end{sffamily}
\end{titlepage}

%----------
%	ÍNDICES
%----------	

%--
% Índice general
%-
\tableofcontents
\thispagestyle{fancy}

%--
% Índice de figuras. Si no se incluyen, comenta las líneas siguientes
%-
\listoffigures
\thispagestyle{fancy}

%--
% Índice de tablas. Si no se incluyen, comenta las líneas siguientes
%-
\listoftables
\thispagestyle{fancy}

%----------
%	TRABAJO
%----------	

\pagenumbering{arabic} % numeración con múmeros arábigos para el resto de la publicación	


%----------
%	COMENZAR A ESCRIBIR AQUI
%----------	


\part{Separador}
\chapter{Titulo 1}
\section{Titulo 2}
\subsection{Titulo 3}
\subsubsection{Titulo 4}
\paragraph{paragraph}
\subparagraph{subparagraph} 

\newpage

\section{Ejemplos}

\subsection{Imagen}
\begin{figure}[H]
	\ffigbox[\FBwidth]
	{\caption{Figura 1}}
	{\includegraphics[scale=.8]{Portada_Logo.png}}
\end{figure}


\subsection{Tabla}
\begin{table} [h] %  https://www.tablesgenerator.com/latex_tables
	\begin{tabular}{|p{1.75cm}|p{2cm}|p{2.5cm}|p{1.5cm}|p{2.5cm}|p{1.5cm}|}
		\hline
		\rowcolor{gray75}	
		Nombre & Apellido & Teléfono & Años & Salario & Ítems \\
		\hline 
		Dominik & Perkins & 324-7244-52 & 24 & 189491 & 2 \\
		\hline
		Alissa & Wright & 754-9052-35 & 9 & 74458 & 2 \\
		\hline
		Derek & Hunt & 400-4833-00 & 27 & 100900 & 3 \\
		\hline
		Dominik & Payne & 659-3850-02 & 4 & 89965 & 3 \\
		\hline
		Rafael & Henderson & 856-0458-34 & 2 & 148677 & 2 \\
		\hline
		Violet & Wells & 212-0225-15 & 22 & 168367 & 0\\
		\hline
		Adele & Nelson & 130-5688-00 & 15 & 176676 & 1 \\
		\hline
		Cherry & Lloyd & 887-2631-74 & 27 & 192515 & 5 \\
		\hline
		Aston & Andrews & 108-8120-94 & 14 & 179556 & 3 \\
		\hline
		\multicolumn{6}{|l|}{Fuente INE} \\
		\hline
	\end{tabular}
	\caption{Tabla Ejemplo}
\end{table}


\subsection{Grafica}
\begin{figure}[H]
	\ffigbox[\FBwidth]
	{\caption{Figura 1}}
	{\begin{tikzpicture}[scale=.8]
		\begin{axis}[
			title={Titulo},
			xlabel={$x$},
			ylabel={$y$},
			legend pos=north west,
			ymajorgrids=true,
			xmajorgrids=true,
			grid style=dashed,
			axis lines=middle,
			xmin=-6, xmax=6, ymin=-0, ymax=40,
			axis x line=center,
			axis y line=center,
		]
		
		\addplot[color=blue, mark=*]
			% coordinates {(1,1)(2,2)(3,3)(4,4)};
			{x^2 - x + 4};
			\legend{Funcion}
			
		\end{axis}
	\end{tikzpicture}
	\begin{tikzpicture}
		\begin{axis}[
			ticks=none,
			xtick distance=1,
			ytick distance=1,
			axis equal image=true,
			grid,
			grid style={gray!50},
			grid=both,
			xlabel={$x$},
			ylabel={$y$},
			axis lines=middle,
			xmin=-4, xmax=9, ymin=-5, ymax=4,
			axis x line=center,
			axis y line=center,
		]
			\addplot[thick, smooth, color=blue] plot coordinates
			{
				(-3, -1)
				(-.5, -3)
				(.5, -1.9)
				(1.5, -2.8)
				(3.5, 1)
				(5.5, 3)
				(7.5, -1.95)
				(8, -1.5)
			};
		\end{axis}
	\end{tikzpicture}}
\end{figure}


\subsection{Lista}
\begin{enumerate}
	\item The labels consists of sequential numbers.
	      \begin{itemize}
		    \item The individual entries are indicated with a black dot, a so-called bullet.
		    \item The text in the entries may be of any length.
	    \end{itemize}
	\item The numbers starts at 1 with every call to the enumerate environment.
\end{enumerate}
 
\subsection{Code listing}
\begin{lstlisting}[language=Python]
parents, babies = (1, 1)
while babies < 100:
	print ('This generation has {0} babies'.format(babies))
	parents, babies = (babies, parents + babies)
\end{lstlisting}

\part{PDF (siguiente hoja)}
\includepdf[pages=-]{docs/pdfImport.pdf}


\end{document}