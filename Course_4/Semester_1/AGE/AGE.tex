\documentclass[12pt, twoside, openright]{report} %fuente a 12pt, formato doble pagina y chapter a la derecha
\raggedbottom % No ajustar el contenido con un salto de pagina

% MÁRGENES: 2,5 cm sup. e inf.; 3 cm izdo. y dcho.
\usepackage[
a4paper,
vmargin=2.5cm,
hmargin=3cm
]{geometry}

% INTERLINEADO: Estrecho (6 ptos./interlineado 1,15) o Moderado (6 ptos./interlineado 1,5)
\renewcommand{\baselinestretch}{1.15}
\parskip=6pt

% DEFINICIÓN DE COLORES para portada y listados de código
\usepackage[table]{xcolor}
\definecolor{azulUC3M}{RGB}{0,0,102}
\definecolor{gray97}{gray}{.97}
\definecolor{gray75}{gray}{.75}
\definecolor{gray45}{gray}{.45}

% Soporte para GENERAR PDF/A
\usepackage{etoolbox}
\makeatletter
\@ifl@t@r\fmtversion{2021-06-01}%
 {\AddToHook{package/after/xmpincl}
   {\patchcmd\mcs@xmpincl@patchFile{\if\par}{\ifx\par}{}{\fail}}}{}
\makeatother
\usepackage[a-1b]{pdfx}

% ENLACES
\usepackage{hyperref}
\hypersetup{colorlinks=true,
  linkcolor=black, % enlaces a partes del documento (p.e. índice) en color negro
  urlcolor=blue} % enlaces a recursos fuera del documento en azul

% Añadir pdfs como partes del documento
\usepackage{pdfpages}

% Quitar la indentación de principio de los parrafos
\setlength{\parindent}{0em}

% EXPRESIONES MATEMATICAS
\usepackage{amsmath,amssymb,amsfonts,amsthm}

\usepackage{txfonts} 
\usepackage[T1]{fontenc}
\usepackage[utf8]{inputenc}

% Insertar graficas y fotos
\usepackage{tikz}
\usepackage{pgfplots}

\usepackage[spanish, es-tabla]{babel} 
\usepackage[babel, spanish=spanish]{csquotes}
\AtBeginEnvironment{quote}{\small}

% diseño de PIE DE PÁGINA
\usepackage{fancyhdr}
\pagestyle{fancy}
\fancyhf{}
\renewcommand{\headrulewidth}{0pt}
\fancyfoot[LE,RO]{\thepage}
\fancypagestyle{plain}{\pagestyle{fancy}}

% DISEÑO DE LOS TÍTULOS de las partes del trabajo (capítulos y epígrafes o subcapítulos)
\usepackage{titlesec}
\usepackage{titletoc}
\titleformat{\chapter}[block]
{\large\bfseries\filcenter}
{\thechapter.}
{5pt}
{\MakeUppercase}
{}
\titlespacing{\chapter}{0pt}{0pt}{*3}
\titlecontents{chapter}
[0pt]                                               
{}
{\contentsmargin{0pt}\thecontentslabel.\enspace\uppercase}
{\contentsmargin{0pt}\uppercase}                        
{\titlerule*[.7pc]{.}\contentspage}                 

\titleformat{\section}
{\bfseries}
{\thesection.}
{5pt}
{}
\titlecontents{section}
[5pt]                                               
{}
{\contentsmargin{0pt}\thecontentslabel.\enspace}
{\contentsmargin{0pt}}
{\titlerule*[.7pc]{.}\contentspage}

\titleformat{\subsection}
{\normalsize\bfseries}
{\thesubsection.}
{5pt}
{}
\titlecontents{subsection}
[10pt]                                               
{}
{\contentsmargin{0pt}                          
  \thecontentslabel.\enspace}
{\contentsmargin{0pt}}                        
{\titlerule*[.7pc]{.}\contentspage}  


% DISEÑO DE TABLAS.
\usepackage{multirow} % permite combinar celdas 
\usepackage{caption} % para personalizar el título de tablas y figuras
\usepackage{floatrow} % utilizamos este paquete y sus macros \ttabbox y \ffigbox para alinear los nombres de tablas y figuras de acuerdo con el estilo definido. Para su uso ver archivo de ejemplo 
\usepackage{array} % con este paquete podemos definir en la siguiente línea un nuevo tipo de columna para tablas: ancho personalizado y contenido centrado
\newcolumntype{P}[1]{>{\centering\arraybackslash}p{#1}}
\DeclareCaptionFormat{upper}{#1#2\uppercase{#3}\par}

% Diseño de tabla para ingeniería
\captionsetup[table]{
  format=hang,
  name=Tabla,
  justification=centering,
  labelsep=colon,
  width=.75\linewidth,
  labelfont=small,
  font=small,
}

% DISEÑO DE FIGURAS.
\usepackage{graphicx}
\graphicspath{{img/}} %ruta a la carpeta de imágenes

% Diseño de figuras para ingeniería
\captionsetup[figure]{
  format=hang,
  name=Fig.,
  singlelinecheck=off,
  labelsep=colon,
  labelfont=small,
  font=small    
}

% NOTAS A PIE DE PÁGINA
\usepackage{chngcntr} %para numeración continua de las notas al pie
\counterwithout{footnote}{chapter}

% LISTADOS DE CÓDIGO
% soporte y estilo para listados de código. Más información en https://es.wikibooks.org/wiki/Manual_de_LaTeX/Listados_de_código/Listados_con_listings
\usepackage{listings}

% definimos un estilo de listings
\lstdefinestyle{estilo}{ frame=Ltb,
  framerule=0pt,
  aboveskip=0.5cm,
  framextopmargin=3pt,
  framexbottommargin=3pt,
  framexleftmargin=0.4cm,
  framesep=0pt,
  rulesep=.4pt,
  backgroundcolor=\color{gray97},
  rulesepcolor=\color{black},
  %
  basicstyle=\ttfamily\footnotesize,
  keywordstyle=\bfseries,
  stringstyle=\ttfamily,
  showstringspaces = false,
  commentstyle=\color{gray45},     
  %
  numbers=left,
  numbersep=15pt,
  numberstyle=\tiny,
  numberfirstline = false,
  breaklines=true,
  xleftmargin=\parindent
}

\captionsetup[lstlisting]{font=small, labelsep=period}
% fijamos el estilo a utilizar 
\lstset{style=estilo}
\renewcommand{\lstlistingname}{\uppercase{Código}}

\pgfplotsset{compat=1.17} 
%-------------
% DOCUMENTO
%-------------

\begin{document}
\pagenumbering{roman} % Se utilizan cifras romanas en la numeración de las páginas previas al cuerpo del trabajo

%----------
% PORTADA
%---------- 
\begin{titlepage}
	\begin{sffamily}
		\color{azulUC3M}
		\begin{center}
			\begin{figure}[H] %incluimos el logotipo de la Universidad
				\makebox[\textwidth][c]{\includegraphics[width=16cm]{Portada_Logo.png}}
			\end{figure}
			\vspace{2.5cm}
			\begin{Large}
				Grado en Ingeniería Informática\\
				2021-2022\\
				\vspace{2cm}
				\textsl{Apuntes}\\
				\bigskip
			\end{Large}
			{\Huge Algoritmos Genéticos y Evolutivos}\\
			\vspace*{0.5cm}
			\rule{10.5cm}{0.1mm}\\
			\vspace*{0.9cm}
			{\LARGE Jorge Rodríguez Fraile\footnote{\href{mailto:100405951@alumnos.uc3m.es}{Universidad: 100405951@alumnos.uc3m.es}  |  \href{mailto:jrf1616@gmail.com}{Personal: jrf1616@gmail.com}}}\\
			\vspace*{1cm}
		\end{center}
		\vfill
		\color{black}
		\includegraphics[width=4.2cm]{img/creativecommons.png}\\
		Esta obra se encuentra sujeta a la licencia Creative Commons\\ \textbf{Reconocimiento - No Comercial - Sin Obra Derivada}
	\end{sffamily}
\end{titlepage}

%----------
% ÍNDICES
%---------- 

%--
% Índice general
%-
\tableofcontents
\thispagestyle{fancy}

%--
% Índice de figuras. Si no se incluyen, comenta las líneas siguientes
%-
\listoffigures
\thispagestyle{fancy}

%--
% Índice de tablas. Si no se incluyen, comenta las líneas siguientes
%-
\listoftables
\thispagestyle{fancy}

%----------
% TRABAJO
%---------- 

\pagenumbering{arabic} % numeración con múmeros arábigos para el resto de la publicación  


%----------
% COMENZAR A ESCRIBIR AQUI
%---------- 

\chapter{Información}

\section{Profesores}
\begin{quote}
	Magistral: Pedro Isasi, pedro.isasi@uc3m.es, 2.1.B13.

	Práctica: Yago Sáez, yago.saez@uc3m.es, 2.1.C13

	Miércoles clases Práctica / Jueves clase Magistral.
\end{quote}

Las tutorías solicitarlas por correo electrónico, si no responden en 24-48 horas volver a enviar el mensaje, mejor insistir que perder el tiempo ante posible perdida.

Se utilizará portátil en las clases prácticas, traerlo para ir resolviendo los ejercicios de programación.

\section{Objetivos}
\begin{itemize}
	\item Trataremos de aprender métodos para replicar fundamentos biológicos, mediante programación evolutiva (Veremos estrategias históricas de la evolución, como Darwin o Mendel), para resolver problemas del campo de la Inteligencia Artificial.
	\item Se estudiarán y analizarán distintas técnicas de computación evolutiva que se basan en distintos paradigmas biológicos.
	\item Lo que buscamos es ser capaces de entender cómo funciona la inteligencia de las personas o animales para ver cómo se desarrollan los procesos y poder programarlos para que hagan una tarea o aprendan a hacerla.
\end{itemize}

Un ejemplo de aplicación consiste en desarrollo de antenas que se generan según distintas mediciones de señal de manera que se extenderá en aquellas direcciones que la maximicen, también evitando las interferencias. Este ejemplo se basa en las plantas, en la manera que tienen de ir buscando la luz para poder obtener
energía y seguir creciendo.

También se emplean muchas basadas en insectos por su corta vida, que permiten observar la evolución de una manera más rápida.

Inteligencia artificial (no hay una única definición): Disciplina que consiste en diseñar programas que son capaces de resolver problemas de manera similar a un humano (razonamiento, aprendizaje o creatividad), pero sin haberles programado el proceso que les permite resolverlo, mediante una serie de casos base. De manera que dándole unos datos sea capaz de resolverlo, aprende a resolverlos.

\section{Evaluación continua}
\begin{itemize}
	\item 3 prácticas a lo largo del curso, que podrían ser 2 prácticas y una extensión de la primera.
	      \begin{itemize}
		      \item Consistirán en la entrega de una memoria siguiendo unos puntos dados y el código.
		      \item Individuales, excepto la última que es en equipo. 1 punto, 1.5 y 2.5 puntos
	      \end{itemize}
	\item 2 pruebas de evaluación, 2,5 puntos cada una.
	\item Examen final no obligatorio, si no se presenta a este la nota final será la de evaluación continua. Si te presentas ser pondera la nota final tal que 0.25 evaluación continua y 0.75 la nota final.
	      \begin{itemize}
		      \item Sin hacer el examen final:
		            \begin{itemize}
			            \item 25\% Parcial 1
			            \item 25\% Parcial 2
			            \item 10\% Practica 1
			            \item 15\% Practica 2
			            \item 25\% Practica 3
		            \end{itemize}
		      \item Haciendo el examen final
		            \begin{itemize}
			            \item 6,3\% Parcial 1
			            \item 6,3\% Parcial 2
			            \item 2,5\% Practica 1
			            \item 3,8\% Practica 2
			            \item 6,3\% Practica 3
			            \item 75\% Examen final
		            \end{itemize}
	      \end{itemize}
\end{itemize}

\chapter{Tema 1: XXX}


\end{document}