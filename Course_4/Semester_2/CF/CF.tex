\documentclass[12pt, twoside, openright]{report} % Fuente a 12pt, formato doble página y chapter a la derecha
\raggedbottom % No ajustar el contenido con un salto de página

% MÁRGENES: 2,5 cm sup. e inf.; 3 cm izdo. y dcho.
\usepackage[
a4paper,
vmargin=2.5cm,
hmargin=3cm
]{geometry}

% INTERLINEADO: Estrecho (6 ptos./interlineado 1,15) o Moderado (6 ptos./interlineado 1,5)
\renewcommand{\baselinestretch}{1.15}
\parskip=6pt

% DEFINICIÓN DE COLORES para portada y listados de código
\usepackage[table]{xcolor}
\definecolor{azulUC3M}{RGB}{0,0,102}
\definecolor{gray97}{gray}{.97}
\definecolor{gray75}{gray}{.75}
\definecolor{gray45}{gray}{.45}

% Soporte para GENERAR PDF/A
\usepackage{etoolbox}
\makeatletter
\@ifl@t@r\fmtversion{2021-06-01}%
 {\AddToHook{package/after/xmpincl}
   {\patchcmd\mcs@xmpincl@patchFile{\if\par}{\ifx\par}{}{\fail}}}{}
\makeatother
\usepackage[a-1b]{pdfx}

% ENLACES
\usepackage{hyperref}
\hypersetup{colorlinks=true,
  linkcolor=black, % enlaces a partes del documento (p.e. índice) en color negro
  urlcolor=blue} % enlaces a recursos fuera del documento en azul

% Añadir pdfs como partes del documento
\usepackage{pdfpages}

% Quitar la indentación de principio de los párrafos
\setlength{\parindent}{0em}

% EXPRESIONES MATEMÁTICAS
\usepackage{amsmath,amssymb,amsfonts,amsthm}

\usepackage{txfonts} 
\usepackage[T1]{fontenc}
\usepackage[utf8]{inputenc}

% Insertar gráficas y fotos
\usepackage{tikz}
\usetikzlibrary{cd}
\usepackage{pgfplots}

\usepackage[spanish, es-tabla]{babel} 
\usepackage[babel, spanish=spanish]{csquotes}
\AtBeginEnvironment{quote}{\small}

% diseño de PIE DE PÁGINA
\usepackage{fancyhdr}
\pagestyle{fancy}
\fancyhf{}
\renewcommand{\headrulewidth}{0pt}
\fancyfoot[LE,RO]{\thepage}
\fancypagestyle{plain}{\pagestyle{fancy}}

% DISEÑO DE LOS TÍTULOS de las partes del trabajo (capítulos y epígrafes o subcapítulos)
\usepackage{titlesec}
\usepackage{titletoc}
\titleformat{\chapter}[block]
{\large\bfseries\filcenter}
{\thechapter.}
{5pt}
{\MakeUppercase}
{}
\titlespacing{\chapter}{0pt}{0pt}{*3}
\titlecontents{chapter}
[0pt]                                               
{}
{\contentsmargin{0pt}\thecontentslabel.\enspace\uppercase}
{\contentsmargin{0pt}\uppercase}                        
{\titlerule*[.7pc]{.}\contentspage}                 

\titleformat{\section}
{\bfseries}
{\thesection.}
{5pt}
{}
\titlecontents{section}
[5pt]                                               
{}
{\contentsmargin{0pt}\thecontentslabel.\enspace}
{\contentsmargin{0pt}}
{\titlerule*[.7pc]{.}\contentspage}

\titleformat{\subsection}
{\normalsize\bfseries}
{\thesubsection.}
{5pt}
{}
\titlecontents{subsection}
[10pt]                                               
{}
{\contentsmargin{0pt}                          
  \thecontentslabel.\enspace}
{\contentsmargin{0pt}}                        
{\titlerule*[.7pc]{.}\contentspage}  

\usepackage{adjustbox}

% DISEÑO DE TABLAS.
\usepackage{multirow} % permite combinar celdas 
\usepackage{caption} % para personalizar el título de tablas y figuras
\usepackage{floatrow} % utilizamos este paquete y sus macros \ttabbox y \ffigbox para alinear los nombres de tablas y figuras de acuerdo con el estilo definido. Para su uso ver archivo de ejemplo 
\usepackage{array} % con este paquete podemos definir en la siguiente línea un nuevo tipo de columna para tablas: ancho personalizado y contenido centrado
\newcolumntype{P}[1]{>{\centering\arraybackslash}p{#1}}
\DeclareCaptionFormat{upper}{#1#2\uppercase{#3}\par}

% Diseño de tabla para ingeniería
\captionsetup[table]{
  format=hang,
  name=Tabla,
  justification=centering,
  labelsep=colon,
  width=.75\linewidth,
  labelfont=small,
  font=small,
}

% DISEÑO DE FIGURAS.
\usepackage{graphicx}
\graphicspath{{img/}} %ruta a la carpeta de imágenes

% Diseño de figuras para ingeniería
\captionsetup[figure]{
  format=hang,
  name=Fig.,
  singlelinecheck=off,
  labelsep=colon,
  labelfont=small,
  font=small    
}

% NOTAS A PIE DE PÁGINA
\usepackage{chngcntr} % Para numeración continua de las notas al pie
\counterwithout{footnote}{chapter}

% LISTADOS DE CÓDIGO
% soporte y estilo para listados de código. Más información en https://es.wikibooks.org/wiki/Manual_de_LaTeX/Listados_de_código/Listados_con_listings
\usepackage{listings}

% definimos un estilo de listings
\lstdefinestyle{estilo}{ frame=Ltb,
  framerule=0pt,
  aboveskip=0.5cm,
  framextopmargin=3pt,
  framexbottommargin=3pt,
  framexleftmargin=0.4cm,
  framesep=0pt,
  rulesep=.4pt,
  backgroundcolor=\color{gray97},
  rulesepcolor=\color{black},
  %
  basicstyle=\ttfamily\footnotesize,
  keywordstyle=\bfseries,
  stringstyle=\ttfamily,
  showstringspaces = false,
  commentstyle=\color{gray45},     
  %
  numbers=left,
  numbersep=15pt,
  numberstyle=\tiny,
  numberfirstline = false,
  breaklines=true,
  xleftmargin=\parindent
}

\captionsetup[lstlisting]{font=small, labelsep=period}
% fijamos el estilo a utilizar 
\lstset{style=estilo}
\renewcommand{\lstlistingname}{\uppercase{Código}}

\pgfplotsset{compat=1.17} 
%-------------
% DOCUMENTO
%-------------

\begin{document}
\pagenumbering{roman} % Se utilizan cifras romanas en la numeración de las páginas previas al cuerpo del trabajo

%----------
% PORTADA
%---------- 
\begin{titlepage}
	\begin{sffamily}
		\color{azulUC3M}
		\begin{center}
			\begin{figure}[H] % Incluimos el logotipo de la Universidad
				\makebox[\textwidth][c]{\includegraphics[width=16cm]{Portada_Logo.png}}
			\end{figure}
			\vspace{2.5cm}
			\begin{Large}
				Grado en Ingeniería Informática\\
				2021-2022\\
				\vspace{2cm}
				\textsl{Apuntes}\\
				\bigskip
			\end{Large}
			{\Huge Ciencia y Ficción}\\
			\vspace*{0.5cm}
			\rule{10.5cm}{0.1mm}\\
			\vspace*{0.9cm}
			{\LARGE Jorge Rodríguez Fraile\footnote{\href{mailto:100405951@alumnos.uc3m.es}{Universidad: 100405951@alumnos.uc3m.es}  |  \href{mailto:jrf1616@gmail.com}{Personal: jrf1616@gmail.com}}}\\
			\vspace*{1cm}
		\end{center}
		\vfill
		\color{black}
		\includegraphics[width=4.2cm]{img/creativecommons.png}\\
		Esta obra se encuentra sujeta a la licencia Creative Commons\\ \textbf{Reconocimiento - No Comercial - Sin Obra Derivada}
	\end{sffamily}
\end{titlepage}

%----------
% ÍNDICES
%---------- 

%--
% Índice general
%-
\tableofcontents
\thispagestyle{fancy}

%----------
% TRABAJO
%---------- 

\pagenumbering{arabic} % numeración con números arábigos para el resto de la publicación  

%----------
% COMENZAR A ESCRIBIR AQUÍ
%---------- 

\chapter{Información}\label{ch:informacion}
\section{Profesores}\label{sec:profesores}
\begin{quote}
	Magistral: Montserrat Iglesias. Humanidades: Filosofía, Lengua y Literatura.

	Magistral: Juan José Vaquero. Bioingeniería e Ingeniería Aeroespacial.
\end{quote}

\section{Sistema de evaluación}\label{sec:sistema-de-evaluación}
\begin{itemize}
	\item 20 \% Asistencia y participación.
	\item 30 \% Comentario/Ensayo de 900 palabras sobre una película o libro tratada en clase. \\ A raíz de lo comentado en clase hablar de un aspecto no tratado o una perspectiva diferente. \\ Fecha de entrega máxima: 9 de mayo.
	\item 50 \% Ensayo de 3000 palabras como examen final. \\ Análisis contrastado entre 2 obras (películas o libros) no comentadas en clase, relacionándolas entre sí y con uno de los temas del programa de la asignatura. \\ Fecha de entrega máxima: 9 de mayo.
\end{itemize}

\chapter{Introducción}\label{ch:introduccion}
Ciencia Ficción es una mala traducción del inglés, traducida palabra a palabra.

Lo más apropiado seria: Science Fiction como Ficción Científica. Aunque también se la puede llamar Ficción Especulativa.

\chapter{Clase 1: Extraterrestres y Robots}
Semiótica: Ciencia que estudia el significado, los de símbolos

Otro // yo, nosotros

Identidad // Alteridad (nos diferencia)

Se utiliza el extraterrestre como el otro, el diferente a los humanos y en la otra se le dará sentido para tratar determinados temas.

Uno de los primeros ejemplos es \underline{\enquote{La guerra de los mundos}, 1897, H. G. Wells}
\begin{itemize}
	\item Revolución industrial / colonialismos
	\item Arrogancia humana con la tecnología y fue vencido por una bacteria
\end{itemize}

\underline{\enquote{Cronicas Marcianas}, 1950, Ray Bradbury}

Durante la guerra fría: \underline{\enquote{La invasión de los ladrones de cuerpos}, 1956, Don Siegels}
\begin{itemize}
	\item Sospechar de los vecinos, podrían ser alienígenas / comunistas
\end{itemize}

Caza de comunismo $\rightarrow$ macartismo

\underline{\enquote{La invasión de los ultracuerpos}, 1978, John Carpenter}
\begin{itemize}
	\item Mismo planteamiento que la anterior, pero en este caso gran ciudad
	\item No conocemos al resto en la gran ciudad, estamos sometidos y sin sentimientos
\end{itemize}

\underline{\enquote{El pueblo de malditos}, 1960, Way Rilla}
\begin{itemize}
	\item En un pueblo, todos se desmayan y vuelven a la normalidad, pero en 9 meses dan a luz las mujeres
	\item Una mente conjunta (comunismo), contra el pueblo, para matarlos, sin sentimientos (pueden leer la mente y controlarlos)
\end{itemize}

Hay otra versión de Carpenter, en la que un niño tiene sentimientos y salvan la situación.

\underline{\enquote{Ultimatum a la Tierra}, 1951, Robert Wise}
\begin{itemize}
	\item Es un alegato pacifista, en plena guerra fría, promueve el diálogo internacional
	\item Hay un robot, Clatus, que funciona como mesías (Jesús) valorando si somos capaces de no destruirnos y no nos tiene que matar
\end{itemize}


En la versión del 2008 la temática gira en torno al cambio climático.

\underline{\enquote{Encuentros en la tercera fase}, 1977, Steven Spielberg}
\begin{itemize}
	\item Se comunica con los alienígenas con música y hay gente predispuesta a percibirla
	\item Representación amable, como niños, humanoides, no son amenazas
	\item Optimismo, los que vienen no tienen malas intenciones.
\end{itemize}

\underline{\enquote{E. T.}, 1985, Wolfgram Petersen}
\begin{itemize}
	\item Una parábola sobre el racismo y la xenofobia.
	\item Un humano y un alien enemigo colaboran para sobrevivir y se van conociendo, viendo que no son tan distintos y se pueden llevar bien
\end{itemize}

\underline{\enquote{Alien: El octavo pasajero}, 1979, Ridley Scott}
\begin{itemize}
	\item Lo imaginado da más miedo que lo conociendo
	\item La mujer como protagonista
\end{itemize}

\underline{\enquote{Independence Day}, 1996, Roland Emmerich}
\begin{itemize}
	\item Busca la repulsión al alien.
	\item Son como langostas que vienen a llevarse nuestros recursos.
\end{itemize}

\underline{\enquote{Avatar}, 2009, James Cameron}
\begin{itemize}
	\item Colonialismo, planeta explotado e indígenas
	\item Empática, los científicos en cuerpos de alienígenas
	\item Ecologismo de los alienígenas, respeto a la naturaleza y su árbol sagrado
\end{itemize}

\underline{\enquote{Planet 51}}
\begin{itemize}
	\item Se giran las tornas, el humano llega a un planeta lleno de alienígenas
\end{itemize}

\underline{\enquote{Solaris}, 1972, Torkosky}

\underline{\enquote{Solaris}, 1961, Lem}

\underline{\enquote{Arrival}, 2016, Denis Villeneuve} adaptación de \underline{\enquote{La historia de tu vida}, Ted Chiang}

\chapter{Clase 2: Humanos y Robots}
El valle inquietante (Uncanny Valley), fenómeno que cuanto más se parece un robot a un humano mas rechazo nos produce. Esto se acentúa con el movimiento, cuando apreciamos fallos o cosas no naturales.

\underline{\enquote{Wall-E}, 2008, Andrew Stanton}
\begin{itemize}
	\item Robot adorable, tiene una mirada, brazos cortos con pinzas, etc.
\end{itemize}

\underline{\enquote{El planeta prohibido}, 1956, Fred M. Wilcox}
\begin{itemize}
	\item Robbie de robot
\end{itemize}

\underline{\enquote{Blade Runner}, Philip K. Dick}

\underline{\enquote{Blade Runner}, 1982, Ridley Scott}
\begin{itemize}
	\item Cazador de robots rebeldes, replicantes
	\item Robots casi indistinguibles, solo con una prueba de empatía y respuesta del iris
	\item Su base es una novela de detective clásica, llevada al futuro, como Philip Marlow.
	\item Se encuentra en una gran ciudad como una gran masa anónima de gente en la que buscar a estos robots
\end{itemize}

Si los robots se parecen tanto a los humanos, y trabajan como esclavo, los humanos pueden llegar a ter esclavos.

Cibernético: Ciencia que estudia las analogías entre los sistemas de control y comunicación de los seres vivos y las maquinas.

Robot: Aparato electromecánico que puede realizar una serie de funciones, puede reprogramarse, es sensible al entorno.

\underline{\enquote{Big}, 1988, Penny Marshall}
\begin{itemize}
	\item El genio de la máquina
\end{itemize}

\underline{\enquote{Robocop}, 1987, Paul Verhoeven}
\begin{itemize}
	\item Un cíborg, tiene capacidades humanas con la potencia / ventajas de un robot
\end{itemize}

\underline{\enquote{Total recall}, 1990, Philip K. Dick}

\underline{\enquote{Stership Troopers}, 1997, Robert A. Heinlein}

\underline{\enquote{Hollow Man}, 2000, H. G. Wells}

\underline{\enquote{Juegos de Ender}, 2013, Gavin Hood}

\underline{\enquote{Juegos de Ender}, 1985, Orson Scott Card}

Robot: Máquina controlada por ordenador con algunos grados de libertad que\dots
\begin{itemize}
	\item capacidad de moverse por su entorno
	\item tiene sensores para detectar su entorno
	\item tiene que cumplir una misión
	\item hace una planificación
	\item programan sus movimientos
	\item siguen una secuencia de operaciones
\end{itemize}

\chapter{Clase 3: Robots}
Origen de la palabra robot: \underline{\enquote{Robots Univesales Rossum}, Karel Capek}. Proviene de robota: servidumbre, trabajo forzoso.

Cuando se justifica el uso de robots?
\begin{itemize}
	\item \enquote{unicamente en aquellas tareas en las que el hermano actua como un robot y no el robot como  el humano} N. W. Clapp, 1982
	\item \enquote{(a menudo) sobre la base de conseguir una ganancia neta en productividad que redunda en beneficio de toda la sociedad} M. Minsky, 1985
\end{itemize}

Habitualmente en la ficción se suelen rebelar los robots, por el sentimiento revolucionario de luchar contra la esclavitud.

\underline{\enquote{War wit the newts}, 1936, Karel Capek}
\begin{itemize}
	\item No solo un protagonista, analiza el desarrollo de los tritones
	\item 3 personajes centrales: Marines, el industrial del desarrollo y el ciudadano de industria Newtz
	\item Solo los últimos 4 de 27 capítulos hablan de la guerra
	\item El resto es del descubrimiento y desarrollo de los tritones
\end{itemize}

\underline{\enquote{2001: Una odisea en el espacio}, 1984, Stanley Kubrick}
\begin{itemize}
	\item Basado en \underline{\enquote{The sentinel}, C. Clark}
	\item HAL 9000, sale de la siguiente letra de IBM.
	\item Robot superhumanizado, siente, canta y tiene miedo
\end{itemize}

Las tres leyes de la robótica (\underline{\enquote{Circulo vicioso}, 1942, Isaac Asimov}):
\begin{enumerate}
	\item Un robot no hará daño a un ser humano, no por inacción permitirá que un ser humano sufra daño
	\item Un robot debe cumplir las órdenes dadas por los seres humanos, mientas no incumpla la primera ley
	\item Un robot debe proteger su propia integridad en la medida en que esta protección no entre en conflicto con la primera o con la segunda ley
\end{enumerate}

\chapter{Clase 4: Robots}
\underline{\enquote{Frankstein}, 1818, Mary Shelley}

\underline{\enquote{Metropolis}, 1926, Fritz Lang}
\begin{itemize}
	\item Robot, Maria (símbolo religioso), remplaza a una mujer para provocar disturbios y tirar la sociedad mediante sus encantos
	\item Sociedad dividida, una soporta a la otra
	\item Razón, trabajo y corazón
	\item El mediador entre el cerebro y la mano ha de ser el corazón
\end{itemize}

\underline{\enquote{Robots}, Carlos Saldanha}

\underline{\enquote{Un amigo para Frank}, 2012, Jake Schrier}

\chapter{Clase 5: Zombis y Robots}
\underline{\enquote{La noche de los muertos vivientes}, 1973, George A. Romero}
\begin{itemize}
	\item En una granja, resurgen los muertos
	\item Surgen de Venus, no Marte (guerra)
	\item Queman los cadáveres, impactante
	\item Usa en planos del KKK cazando negros, como miedo al otro y se quiere acabar con ellos
\end{itemize}

\underline{\enquote{A. I.}, 2001, Steven Spielberg}
\begin{itemize}
	\item Como pinocho, quiere ser como un niño normal
	\item Es abandonado
	\item Se vuelve posesivo con el amor de la madre
\end{itemize}

\chapter{Clase 6: Inteligencia Artificial}
Inteligencia Natural
\begin{itemize}
	\item Sistema Nervioso Central
	\item Neuronas, impulsos nerviosos, sensores
	\item Génesis del cerebro
	\item Memoria, sueños
\end{itemize}

Lógica de Boole y eléctrica digital

El perceptrón y redes neuronales

Aprendizaje máquina

\chapter{Clase 7: La condición Humana}
\underline{\enquote{Yo, robot}, 2004, Alex Proyas}
\begin{itemize}
	\item Migas de pan (pistas para descubrir todo)
	\item Robot con conciencia
	\item Vicky ordenador malo mujer
	\item Los malos rojos y los buenos azules (simbólico)
	\item Humano odia robots le piden  investigar suicidio
\end{itemize}

\underline{\enquote{Ex-machina}, 2014, Alex Garland}

\underline{\enquote{Neuromancer}, 1984, William Gibson}
\begin{itemize}
	\item Nace el término ciberespacio
	\item Ser eternos, el transhumanismo
\end{itemize}

\underline{\enquote{Altered Carbon}, 2002, Richard Morgan }

\underline{\enquote{El hombre bicentenario}, 1976, Isaac Asimov}

\chapter{Clase 8: Transhumanismo}
\underline{\enquote{Matrix}, 1999, The Wachowski Brothers}

\underline{\enquote{Terminator}, 1984, James Cameron}

\underline{\enquote{Her}, 2013, Spike Jonze}

\underline{\enquote{Transcendence}, 2014, Wally Pfister}

\underline{\enquote{La ciudad y las estrellas}, 1956, Arthur C. Clarke}

\underline{\enquote{Homo sapiens}, 2014, Yuval Noah Harari}

\underline{\enquote{Homo deus}, 2016, Yuval Noah Harari}

\underline{\enquote{21 lecciones para el siglo XXI}, 2018, Yuval Noah Harari}

\chapter{Clase 9}

\underline{\enquote{Burn-in}, 2020, August Cole}

\underline{\enquote{Teknolust}, 2002, Lynn Hershman Leeson}

\chapter{Clase 10}
\underline{\enquote{In Time}, 2011, Andrew Niccol}

\underline{\enquote{La amenaza de Andromeda}, 1071, Robert Wise}

\underline{\enquote{The island}, 2005, Michael Bay}

\underline{\enquote{Biohackers}, 2020}

\underline{\enquote{Un mundo feliz}, 1932, Aldous Huxley}

\underline{\enquote{Gattaca}, 1997, Andrew Niccol}

\underline{\enquote{Mr. Humble and Dr. Butcher}, 2021, Brandy Schillace}

\underline{\enquote{12 monos}, 1996, Terry Gilliam}

\chapter{Clase 11: Distopías y Utopias}

\underline{\enquote{Juegos de Guerra}, 2983, John Badham}

\underline{\enquote{Nunca me abandones}, 2005, Kazuo Ishiguro}

\underline{\enquote{El corazon de las tinieblas}, 1899, Joseph Conrad}

\underline{\enquote{El senor de los nillos}, 1954, J. R. R. Tolkien}

\underline{\enquote{Matar aun ruisenor}, 1960,  Harper Lee}

\underline{\enquote{Idiocracia}, 2006, Mike Judge}

\underline{\enquote{Lista de Schindler}, 1993, Steven Spielberg}

\underline{\enquote{Anna Karenina}, 1878, Leo Tolstoy}

\underline{\enquote{Contagio}, 2011, Steven Soderbergh}

\underline{\enquote{The stepford wives}, 1972, Ira Levin}

\underline{\enquote{La carrera}, 2006, Cormac McCarthy}

\underline{\enquote{El planeta de los simios}, 1968, Franklin J. Schaffner}


\end{document}