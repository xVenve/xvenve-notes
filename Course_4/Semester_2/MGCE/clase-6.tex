\chapter{Clase 6}\label{ch:clase6}
Siguiendo con Teseo\dots

El reconocimiento de los atenienses y otras partes de la alianza ateniense es lo que dimensiona ese mito local y lo engloba junto al resto del mito de Teseo, haciendo que se sepa históricamente, está puesto en los libros de historia griegos que los huesos fueron puestos en el Teseion.

\section{El Teseion}
\begin{itemize}
	\item Culto heroico
	\item Culto cívico ateniense a Teseo
	\item Sustituto de ancestros
	\item Huesos traídos desde Esciro en el año 475 a.C.
	\item Es el lugar donde se concentra toda la moral cívica mezclada con religión y tradición local.
	\item Es el gran templo dedicado al culto a Teseo que se sitúa en la acrópolis ateniense, aunque en escritos de la época bizantina mucho después no se sabe específicamente donde estaba.
\end{itemize}

Teseo de Creta se convirtió en el elemento por excelencia para justificar la singularidad ateniense, como también ocurrió a los romanos con Rómulo y Remo para hablar de su característica especial y porque dominan el mundo. Los estadounidenses hablando del american way of life donde se crearan una serie de herramientas de libertad democrática tratadas casi como elemento religiosos, casi míticos, de alegoría a una moral más que a uno fundacional de una ciudad.

Igual que ocurriría en el mundo cristiano, donde algunas ciudades se relacionan muy fuertemente con su santo patrón, donde la gente tiene una relación sentimental con este santo patrón.

\section{Perseo}
\begin{itemize}
	\item Zeus padre, Acrisio abuelo
	\item Destino profetizado
	\item Niño abandonado
	\item Encargo mortal
	\item Dones de los dioses
\end{itemize}

El rey de Argos, que moriría a manos de su propio nieto, para evitarlo adquirió e hizo que su hija Dánae fuera encerrada en una mazmorra de bronce o una torre alta de bronce (según las versiones) para impedir que tuviera ningún tipo de trato con el exterior, especialmente con ningún varón que la dejara embarazada.

Pero el rey de los dioses Zeus sí que la vio y sí que pudo acceder, se transformó en lluvia dorada y cayó a través de la ventana sobre Dánae preñándola. De este embarazo milagroso nace Prometeo.

El rey Acrisio que sabe que es un semidiós, tuvo miedo de matarlo directamente, por lo que lo coloca en una caja de madera a él y a su madre y los echa al mar, donde llegan a la isla de Sefilos y son regidos por un amable rey. Perseo creció en el exilio como un príncipe bastardo, pero el mismo rey se enamoró de Dánae y ante el rechazo de esta que solo quería criar a su hijo decidió deshacerse de Perseo al ser un obstáculo para conseguir su amor, por lo que le pide una tarea imposible que le traiga la cabeza de medusa y ante la desesperación del héroe sabiendo que es casi imposible este reza los dioses.

Atenea y Hermes bajan de los cielos para ayudar al joven semidiós, le dan un yelmo de Hades que lo hacía invisible, Hermes le prestó sus sandalias aladas, una espada especial muy afilada irrompible para cortar la cabeza de medusa y por último le entregó una bolsa donde guardar con seguridad la cabeza del monstruo.

Atenea le aconseja volar al oeste con estos regalos, hacia la oscuridad lejana, donde podrá encontrar a tres brujas, las Greas (no confundir con las moiras) que comparten un mismo ojo.

Perseo para convencerlas lo que hace es acercarse a ellas, hablarles amablemente y les roba el ojo, por tanto, les obliga a cambio del ojo mostrarle el camino para encontrar a la gorgona medusa. Cuando llegó a la tierra de las gorgonas también eran un grupo de 3 y solo una de ellas será mortal. Decide matarla esperando a que estén dormidas, pero Atenea aparece de nuevo en ese momento para evitar que las despierte y le dice que espere que se acerque sigilosamente y le corte la cabeza a Medusa mirando su cara a través de la imagen que se refleja en su escudo para evitar que la cara de medusa lo volviese en piedra.

La cabeza tenía víboras como pelo, la corta y la mete en la bolsa alejándose, pero de las gotas de sangre que sale en el cuello medusa surge Pegaso, el caballo alado.

En su vuelta pasa por Etiopia y ve a un dragón enviado por Poseidón que trata de devorar a la princesa Andrómeda, que había sido ofrecida a este por haber ofendido a Poseidón diciendo que era más bella que las nereidas. Perseo llega volando, mata al monstruo (con la espada de los dioses o la cabeza de medusa, según la versión) y rescata a la princesa. Entonces queda prendado de ella y decide casarse con Andrómeda.

En la boda tratan de recuperar a la princesa, pero usa la cabeza de medusa y convierte a la mitad en piedra, por la situación se ve obligado a huir.

En Argos como héroe, en unos juegos en su honor, compitiendo, tira un disco (el que más lejos) y cumple la profecía, matando de un golpe en la cabeza a Crisio sin querer. Se le ofrece el trono, pero por vergüenza de este asesinato no rechaza, en la época aunque no fuera intencional es culpable.

Se irá y funda Micenas, fundamental en los ciclos troyanos.

\section{Atenea Minerva}
\begin{itemize}
	\item Hija de Zeus y variantes
	\item Diosa de la sabiduría y olivo
	\item Sabia consejera
	\item Portadora de la Egida
\end{itemize}

Atenea de Atenas visitó el lugar y de allí trajo ideas como la justicia. De hecho, también se dice que esta diosa trajo sus ropajes extraños de libia, por eso aparece con un aspecto de pieles de cabra adornadas con serpientes, que será la vestimenta original de las mujeres antiguas de roma.

En el mito famoso el más famoso de todos es hija de Zeus y Metis, la titán, la primera mujer de Zeus que había sido profetizada que engendraría un hijo que castigaría a Zeus y lo vencería. Y como Cronos había derrotado a su padre, Urano, alarmado por esto, lo que hace es devorar a su esposa Metis. Por esta razón Zeus recibe la razón de la razón justiciera, que es lo que significa Metis y ella le aconseja desde sus tripas. A la vez empieza a experimentar unas migrañas terribles y decide apartarse del olimpo para intentar descansar y relajarse, pero estas migrañas no se van. Hermes entonces acude en su ayuda y adivina qué es lo que le pasa, Metis está embarazada y la hija ha nacido y está intentando salir. Entonces llamada a Hefesto y junto con él abren una brecha en el cráneo de Zeus y de allí sale ya lista para la batalla y armada la diosa Atenea.

El dios se come a en este caso Zeus no mata Metis y acabó con embarazo, sino que se la come pensando lo que acabará con el embarazo, es una parte simbólica de absorber la fuerza del vencido.

Los símbolos son el búho mochuelo de la sabiduría, el árbol del olivo por el mito de Atenas y la serpiente de la gorgona.

Se ha relacionado con Minerva, pero Minerva realmente era originalmente una diosa itálica distinta, una mecánica común que hay júpiter no es el mismo que Zeus. El conjunto de iconografía de los griegos, pero también este caso el mitológico y el religioso, de esta forma encontramos que Atenea y Minerva acaban siendo progresión de la misma diosa, se vuelven equiparables.

\section{Troya}
\begin{itemize}
	\item Combaten griegos y troyanos
	\item Caballo de Troya
	\item Muere Héctor a manos de Aquiles
\end{itemize}

Es el elemento fundacional de la cultura griega y por esa razón una de las semillas principales de la cultura occidental, no solo europea sino también estadounidense y de Latinoamérica directa e indirectamente.

En la mitología, en este caso, Troya engloba numerosos episodios de los cuales uno, el más famoso, La Guerra de Troya, es el nombre general y la Ilíada de Homero es la obra particular que no es lo mismo. La Ilíada es la obra principal para los griegos, es una obra cultural de referencia

\section{Tetis y Peleo}
Se obliga a esta diosa menor a casarse con un mortal para evitar que nazca un dios peligroso y el mortal seleccionado no fue ni más ni menos que peleó el rey de los mil millones en tesalia.

La boda
\begin{itemize}
	\item Tetis y dioses
	\item Peleo tesalio y reyes
	\item Excusión de Eris
	\item Juicio de Paris
\end{itemize}
La diosa no quería casarse y huyó constantemente y Tetis se transformaba en animales para escapar igual que su padre Nereo.

Un matrimonio no correspondido del que nacerá Aquiles y en este momento cuando se consolida la unión a un momento antes del nacimiento de Aquiles se celebra la boda.

Lanza una manzana con la inscripción para la más bella, donde las diosas Minerva (Atenea), Hera y Afrodita (Venus) se quedan solas en la disputa por la manzana, en esta situación se recurre a París un joven y apuesto príncipe troyano que aparece siempre como pastor y Hermes le lleva la manzana para elegir la más bella. Cada una de las diosas le ofrece algo, Hera le ofrece el dominio sobre los hombres, Minerva le ofrece la sabiduría mayor que a cualquier hombre y Afrodita le ofrece el amor de la mujer más bella.

Afrodita le dará el amor de Helena a París, aquí también donde pasamos a hablar del rapto de Helena, aunque hay diferentes versiones algunos dicen que no fue llevada por la fuerza, sino que se dejó enamorar o quedó prendidamente enamorada por París. Helena, que era hija de Zeus y de le da, por tanto, era también semidiosa.

El rey de Micenas (ciudad que ha sido fundada por Perseo) y señor de todos los reyes griegos son llamados a la guerra para recuperar a Helena y para evitar el agravio de los troyanos.

Aquiles es para ese momento ya había empezado a crecer como uno de los héroes más importantes del mundo griego, de hecho se dice que Peleo, su padre, lo entregó a los centauros que vivían en las fronteras de tesalia y fue educado por ni más ni menos que Quirón, el centauro sabio que formaba a los héroes.

Le dan a elegir entre una vida larga o una vida gloriosa, y decide la muerte gloriosa para alcanzar la inmortalidad a través de su fama.

Lo que hace la Ilíada es mencionar la guerra de Troya a mitad de la guerra, es un conflicto gigantesco, lo que buscaba este relato era acabar con la hiperabundancia de humanos en el mundo por todo lo que había ocurrido desde el matrimonio de Peleo y Tetis, la manzana de la discordia y el enfrentamiento de las tres diosas. Era una artimaña de Zeus para que los humanos se enfrentarán entre sí y los dos grandes grupos de población en mundo griego y Asia se enfrentasen y redujesen la población y su fuerza en gran medida para calmar la hiperpoblación del mundo.

La cólera de Aquiles que es el dinamizador de la mayoría de relatos, este criado por Quirón se enfada y consigue grandes proezas, pero también grandes daños que es lo que ellos pretenden. La cólera sin control que prácticamente destruye el mundo narra la guerra de Troya, todo cumpliendo la voluntad de Zeus.

Los troyanos se acercan a la victoria definitiva y acorralando contra la playa a los griegos, entre ellos está el gran amigo o amante (una constante en el mundo griego, relación entre la parte poderosa y la parte querida más débil) Patroclo decide unirse a las fuerzas en algunas explicaciones contra y otras con los deseos de Aquiles toma sus armas y sin la participación de Aquiles intenta el último defender el campamento griego y detiene el avance troyano, pero a costa de su vida, ya que se enfrentará con el gran héroe troyano Héctor y recibirá muerte. En este momento es cuando la cólera de Aquiles alcanza una sobredimensión, una cólera divina, Aquiles pide ayuda a su madre y a los dioses, estos le regalan una nueva armadura y lleno de furia vence a los héroes de distintas ciudades de Troya.

Acorrala contra las murallas de Troya a los troyanos y empieza a buscar al asesino de su querido Patroclo, Héctor, es aquí el destino definitivo de uno de los grandes héroes y la cúspide del relato de la Ilíada.

Héctor es el héroe cívico por antonomasia, luchado por su ciudad, por su familia, por sus gentes y además se le muestra con paciencia. Todo se sale de su sitio cuando finalmente se tiene que enfrentar o piensa que está enfrentando Aquiles y mata a Patroclo, entonces Aquiles lo perseguirá. Héctor decide evitar el conflicto porque sabe que no está bajo una rabia normal, sino que está bajo una rabia divina e intenta huir, la Ilíada cuenta como dan vueltas a la ciudad de Troya, esperando a que perdiera su furia inicial para combatir.

Engañado por Atena que se hace pasar por su hermano que le pasa las lanzas, es derrotado por Aquiles.

El héroe troyano lacera sus tendones, los ata a su carro y desde los tobillos lo arrastran desde la ciudad de Troya extramuros durante doce días expuesto el cadáver al sol y los animales.

En una pira al que acuden todos los héroes a la vez en una tregua sagrada donde observando como el cuerpo de Héctor se va deshaciendo pasto de las llamas, acaba el relato épico de la Ilíada.