\chapter{Clase 7}\label{ch:clase7}

\section{Otros relatos del ciclo troyano}
Tras la muerte de Héctor se trae a los bárbaros y las amazonas (hasta que derrotan a su líder) para combatir a los troyanos.

\section{El destino de los héroes}
El fin de la Ilíada es la muerte de Aquiles de un flechazo en el tendón, el talón de Aquiles. Lo hace Paris tras la muerte de su hermano, mediante varias flechas.

No se sabe en qué momento se redactan sus muertes, así como la de Paris u otros personajes.

Se dice que es débil en el talón porque no lo sumergió cuando lo hicieron inmortal o Tetis lo alimenta con ambrosía.

\section{El final de La Guerra de Troya}
Ulises convence a los reyes de que hagan una treta para hacerse con la victoria, ya que no avanzaban en la guerra, simulan una huida a unas islas de Turquía, pero unos pocos hombres se esconden en una ofrenda a los troyanos. Por la noche salen y masacran a los troyanos.

\section{Jason y los Argonautas}
\begin{itemize}
	\item Ciclo pre-troyano. Estos héroes viajan por un mundo que se va abriendo lugar y van descubriendo elementos de los que se hablara más adelante en la Odisea y los ciclos.
	\item Nefele y Atamante: el vellocino de otros. Toisón de oro (tomado por la corona francesa). Es un carnero alado que rescata a los herederos al trono que van a ser sacrificados para salvar la hambruna, se dice que es enviado por los dioses. Por el camino en el mar negro se cae y muere uno de ellos y el otro llega con éxito.
	\item Colquida (actual Georgia) y Eetes
	\item Jason de Yolco, Tesalia. Es criado por Quirón y es probado por Hera (vieja que cruza un río) que le ayudará.
	\item Hermanastro Pelias y se le pone una tarea heroica, traer el vellocino, de esta manera se gana el puesto al trono (aunque era legítimo). Se le pone por la profecía que tenía el rey de que un forastero con una sola sandalia le destronara.
	\item Argonautas y viaje mítico. Atenea ayuda a construir el barco Argo con un tablón, con el que emprenderá el viaje, como tripulación se unen los argonautas, una serie de héroes y personajes relevantes (Padre de Aquiles, Heracles, hijo de Hermes, ...). Es un viaje con muchas etapas, que se cuentan como historias y mitos relacionados con las etapas.
	\item Algunas veces son recibidos como aliados y otra como enemigos, como en Micos, el hijo de Poseidón que era boxeador (los hijos de Poseidón son monstruos o agresivos)
	\item Se le entregaría el vellocino si Jason hace 3 tareas, pero cae en depresión y no la completa, por lo que con ayuda de los dioses enamora a la princesa (Medeas) quien le ayuda y logra sus tareas.
\end{itemize}

Hay una representación en el cine de uno de estos viajes míticos.

La historia de Medea: la mala mujer. La naturaleza más oscura del amor, fertilidad, la brujería y los sentimientos derivados: engaño, celos y resentimiento. Mata a la mujer de Jason con un vestido maldito, que se pega a la piel y la quema. Además, mata a sus hijos.

\section{Afrodita (Venus)}
\begin{itemize}
	\item Hija de la espuma. La más hermosa elegida por París.
	\item Diosa del amor y erotismo. Es una personificación.
	\item Popular en la mitología y alegoría, más que en los cultos.
	\item Eros es hijo suyo junto a Hades.
	\item Esposa de Hefesto, que al enterarse de su aventura con Hades los atrapa y expone en la cama.
	\item Guerra de Troya y otros mitos
	\item Festival Afrodisias, tiene una representación más carnal en algunos lugares y en otro más sentimental. Aparecían en varias ciudades griegas, pero eran especialmente relevantes en la isla de Chipre, lugar de origen de la diosa según la mitología.

	      Era muy popular en especial entre las prostitutas, que la consideraban su patrona.
\end{itemize}

Tiene relación con dos mortales, ...  y Adonis del que se enamora de verdad. A la muerte de Adonis en una cacería por el ataque de un animal, Afrodita corre descalza a recoger su moribundo cuerpo y de las heridas de sus pies nace el rojo de las rosas o las anémonas.

\section{Odiseo (Ulises) y la Odisea}
Se caracteriza por su sabiduría e ingenio. Criado por Quirón y casado con Penélope.
\begin{itemize}
	\item Telemaquia, nos cuenta la historia de los héroes que se van repartiendo por el mundo.
	\item Viajes de Odiseo.
	\item La Odisea comienza en mitad del viaje, in media res.
	\item La venganza de Odiseo
\end{itemize}

El viaje es de regreso a Itaca, su hogar.

Odiseo llega a la isla de los cíclopes donde en una cueva encuentran alimento y se dan un banquete, entonces llega Polifemo, hijo de Zeus y se los empieza en comer en venganza. Ulises con sus hombres entonces lo emborrachan y le ciegan con una lanza. Por lo que pueden escapar cuando saca a pastar a sus ovejas. Zeus le maldice y deberá navegar otros 10 años.

En la isla de Circe, bruja semidiosa, les convierte en cerdos, pero gracias a un antídoto de Hermes se vuelve humano y la obligan a devolverlos a su forma. Llega a vivir allí un año.

Por el viaje encuentran sirenas mitologías y son atados a los mástiles, aquellos que no reman para que no se tiren al mar y sean devorados por estas bestias.

Ulises desciende al inframundo en busca de Tiresias para que le aconseje e ilumine acerca de su regreso a Ítaca. Nada más llegar, ve acercarse una multitud, la de los que no son personas ni tienen rostro, no son visibles, no son nada. Entre ellas distingue el espectro de Aquiles, al que da de beber sangre para devolverle algo de vitalidad y pueda expresarse. El héroe aqueo le dice que los muertos están privados de sentidos y que son las imágenes de los hombres que ya fallecieron. Añade que preferiría ser el último servidor del hombre más pobre del mundo, pero vivo bajo la luz del sol, que ser el rey de ese mundo de tinieblas que es el Hades.

Tras 20 años regresa a Itaca, pero nadie le reconoce más que su perro. Cuando ve a Penélope, esta no le reconoce y le dice que había anunciado que quien consiguiera armar el arco se casaría con ella. Penélope sabia que solo sería capaz de hacerlo su marido, Ulises, porque no quería casarse con otro hombre. Ayudado por sus hijos y sirvientes, acaba con los pretendientes.

Es uno de los pocos relatos en los que el héroe acaba bien.

Esta obra tiene mucha influencia en la cultura.