\chapter{Clase 3}\label{ch:clase3}
El mito es atemporal, no sigue un hilo argumental fijo.

La relación entre los dioses/lo sobrenatural se relaciona con los mortales/lo natural.

Ganímedes, fue raptado por Zeus y se convierte en el copero de los dioses. Zeus usa sus atributos, como águila, para conseguir un nuevo efebo. En la época el amor y el matrimonio van por separado, muchas veces se representa con cadenas, que te obligan a hacer cosas que no quieres.

Cuando hay una relación entre dos hombres, se muestra una clara diferencia social, la sumisión es un elemento importante. Por ejemplo, Adriano que tras el ahogo de su amado construye estatuas y lo trata de divinizar, como paso con Ganímedes.

\section{Zeus - Júpiter}
\begin{itemize}
    \item Señor, rey y padre
    \item Señor del Cielo y tormenta
    \item Dios de la Justicia (con Metis = Prudencia)
    \item Zeus Olímpico, Horkios, Xenios\dots
    \item Faceta mitológica
    \item Faceta de dios supremo
    \item Evolución cronológica
    \item Sus símbolos son el águila, el trono, el rayo y el cetro (que sería usado por los reyes).
\end{itemize}

Zeus se come a su primera mujer Metis para que no le pase lo mismo que a su padre y abuelo, pero le provoca gran dolor y pedirá que le rompan la cabeza, donde nace Atenea.

Se les designan epítetos, conjuntos de adjetivos, a los dioses, reyes y héroes para definirlos, su grandeza. Además, sirven para mostrar respeto. Atributos como "el Grande", "el Sabio"\dots

Se habla tanto de los amoríos como manera de tejer la genealogía y relación entre los dioses.

Uno de los santuarios más importantes es: el Santuario de Olimpia, donde tiene lugar las olimpiadas clásicas, se iba ampliando con los años para realizar más juegos. En el Templo de Zeus era permanente y se encontraron estatuillas del neolítico. Además, el monte más próximo es el Kronos. Se celebraron los juegos desde 700 a.C. hasta 300 d.C. que con la religión se detuvieron.

Tras la victoria de los dioses se dividen los poderes entre ellos, Hades el Inframundo (mundo subterráneo), Poseidón el señor de los océanos (, caballos y terremotos) y Zeus el dios del cielo. Y todos mandan sobre la tierra, donde viven los mortales.
El Panteón está formando por todos los dioses, 12 olímpicos y otros que varían según las escrituras.

\section{Más allá de Tifón}
Tras esto tenemos a las diosas del destino, seres divinos (no diosas) de poder primordial, las Moiras (Nona, Decima y Morta). Representan las diferentes fases de la vida.

Las Furias (nacen de la sangre de Urano), las Harpías y Seres variados salvajes (como Centauros).

Centauros, representan las pasiones humanas sin control; lujuriosos, violentos y borrachos. Quirón es un ejemplo de centauro civilizado, que entreno a grandes héroes y dioses para la batalla, muere de manera accidental con una flecha envenenada que le hiere eternamente (renunciando por su dolor a la inmortalidad).

\section{Prometeo}
\begin{itemize}
    \item Titán benévolo
    \item Dador de fuego (civilización)
    \item Por él, el sacrificio es como es 
    \item Engañó con astucia a Zeus y roba el fuego a los dioses.
\end{itemize}

\section{La Caja de Pandora}
\begin{itemize}
    \item Uno de los mitos de origen del género humano
    \begin{itemize}
        \item Deucalión y primera
        \item Barro
        \item Pandora (significa "la que da todas las cosas" fue dotada por los dioses de sus virtudes
    \end{itemize}
    \item Originalmente era una jarra y no una caja.
    \item Epimeteo, hermano de Prometeo al que le regalan a Pandora.
    \item Construcción de una mujer, es la primera mujer.
    \item Visión misógina
    \item Origen de los males, al estar dotada de curiosidad no se puede resistir a abrirla y libera a todos los males
\end{itemize}

Un mitema muy habitual es crear al hombre a partir del barro.

\section{El Castigo de Prometeo}
\begin{itemize}
    \item Encadenado a una montaña en Escitia (Cáucaso), el lugar más lejano. Donde un águila, el símbolo de Zeus el que le castiga, se comería sus vísceras cada día, como ser inmortal las regenera, un castigo brutal.
    \item Esquilo, Prometheia
    \begin{itemize}
        \item Prometero Encadenado
        \item Prometero liberado
        \item Prometeo portador del fuego
    \end{itemize}
    \item Prometeo capturado, de Rubens
    \item Diferentes interpretaciones > diferentes
    \begin{itemize}
        \item Hesíodo
        \item Platón
        \item Esquilo
        \item Pausanias: cultos
    \end{itemize}
\end{itemize}

La liberación de Prometeo, Hércules/Heracles es el encargado de liberarle y vencer al águila.

Heracles es el héroe por antonomasia (buen ejemplo para trabajo final), pasa de héroe a dios, era hijo de Zeus y Almenada. Al ser hijo bastardo de Zeus, es mitad dios mitad mortal. Destaca por su fortaleza y buen corazón, aunque no era muy listo y se metía en problemas.

Es la base para los héroes modernos, tiene que realizar proezas para compensar su poder.

Zeus engaña a Hera para que amante a Heracles, pero este al morder tan fuerte le hace daño y nota que va ganando fuerza e inmortalidad, entonces Hera se lo quita de encima derramando la leche y creando la Vía Láctea.