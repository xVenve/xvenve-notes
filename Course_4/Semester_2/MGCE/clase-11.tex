\chapter{Clase 11}\label{ch:clase11}
\section{La influencia del Teatro Clásico}
Mitemas → topos teatrales → cine moderno:
\begin{itemize}
	\item Ciudadano Kane (1941): el desastre del héroe.
	\item Adaptaciones directas como Médee (2011) o The Killing of a Sacred Deer (2017). Y autores de culto como Pasolini.
	\item Obras de directores como Yorgos Lanthimos: Canino (2009), La favorita (2018).
\end{itemize}

\section{El mito clásico y la transformación}

\subsection{De Héroe a Dios: Asclepio-Esculapio}
\begin{itemize}
	\item Hijo de Apolo y Coronis.
	\item Aparece desde época arcaica.
	\item De mito heroico a culto.
	\item Dios de la Salud junto a Higía (Salus romana).
\end{itemize}

Asclepio se forma con Quirón, se especializa en medicina, además, de las artes heroicas. Emplea  pociones, fármacos, e intervenciones clínicas para su función de curandero. Era capaz de curar a los enfermos e incluso llego a revivir.

Por el gibiris es matado, por alterar el orden del mundo. Pero su padre Apolo, le asciende al Olimpo, se comienza a hacer culto a el. Comienza como héroe y termina deificado con su propio culto.

Sale en el símbolo de la OMS, como serpiente.

\subsection{El Asclepeion}
Va sustituyendo con el tiempo las funciones de Apolo, y se le edificarán muchos templos, más que a muchos dioses.

Estaban orientados a la sanación, alejados de las grandes poblaciones y los peregrinos recurrían a ellos. Estaban en lugares con naturaleza y manantiales.

Se tenían que hacer tributos y sacrificios, además, de unos óbolos a los sacerdotes del templo para curarse.

\subsubsection{Artemisa-Diana}
\begin{itemize}
	\item Hermana melliza de Apolo.
	\item Diosa de la caza, asesora del parto.
	\item Diosa sincrética
	      \begin{itemize}
		      \item Éfeso, Creta
		      \item Hécate y Selene
	      \end{itemize}
	\item Diosa virgen de las jóvenes
\end{itemize}

Su símbolo es la cierva.

\subsection{Mitos de Tranformación}

\subsubsection{Transformaciones Parciales}
Es una inversión incompleta o simbólica
\begin{itemize}
	\item Heracles y Aquiles, cuando son forzados a convertirse en mujeres. Hércules tras matar a un inocente es obligado a servir como una mujer y a vistiéndose como tal. Aquiles, para evitar ir a la batalla, convive con ellas como si fuera una más.
	\item Zeus
	\item Amazonas, mezclan lo masculino con la batalla y lo femenino de la belleza.
	\item Otros
\end{itemize}

\subsubsection{Transformación completa, de género o sexual}
Tiresias
\begin{itemize}
	\item Es el adivino por antonomasia del ciclo troyano. Se hizo famoso por ser el consejero de reyes tebanos. Se le representa como un viejo, hijo de una ninfa y un dios, un semidiós.
	\item Tiresias es particular, porque de joven vio a dos serpientes copulando y mato a la serpiente mujer. En ese momento se enfada Hera por interrumpir la relación y le castiga a convertirse en mujer.
	\item Se dice que pudo darse a la prostitución.
	\item Pasados 7 años de mujer, ve otra vez a serpientes copulando y las deja, por el que se le quita el maleficio.
	\item Una vez, los dioses le preguntaron a él quién era el que más disfrutaba de las relaciones sexuales, este dijo que de 10, 9 la mujer y 1 el hombre. Por el que Hera le castiga con ceguera y Zeus con la visión del futuro. Hay otra versión de este suceso por el que se queda ciego.
\end{itemize}

Cénide-Ceneo y Hermafrodito
\begin{itemize}
	\item Hermafrodito, era hijo de Hermes y Afrodita, pero fue abandona en el monte Ida al cuidado de las ninfas.
	\item Mientras se bañaba en un lado, Salmácide se enamora de Hermafrodito, por lo que pidió a los dioses mientras le abrazara que les fusionara.
	\item Un dios no se sabe cuál, los fusiona y pasa a ser una mujer con genitales masculinos.
	\item Esto era una manera de explicar el suceso de los problemas hormonales que provocaban un desarrollo intermedio.
	\item Cenide-Ceneo.
\end{itemize}

Otros: Iphis, Siproites y Zeus.