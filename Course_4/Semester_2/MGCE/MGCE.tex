\documentclass[12pt, twoside, openright]{report} % Fuente a 12pt, formato doble página y chapter a la derecha
\raggedbottom % No ajustar el contenido con un salto de página

% MÁRGENES: 2,5 cm sup. e inf.; 3 cm izdo. y dcho.
\usepackage[
a4paper,
vmargin=2.5cm,
hmargin=3cm
]{geometry}

% INTERLINEADO: Estrecho (6 ptos./interlineado 1,15) o Moderado (6 ptos./interlineado 1,5)
\renewcommand{\baselinestretch}{1.15}
\parskip=6pt

% DEFINICIÓN DE COLORES para portada y listados de código
\usepackage[table]{xcolor}
\definecolor{azulUC3M}{RGB}{0,0,102}
\definecolor{gray97}{gray}{.97}
\definecolor{gray75}{gray}{.75}
\definecolor{gray45}{gray}{.45}

% Soporte para GENERAR PDF/A
\usepackage{etoolbox}
\makeatletter
\@ifl@t@r\fmtversion{2021-06-01}%
 {\AddToHook{package/after/xmpincl}
   {\patchcmd\mcs@xmpincl@patchFile{\if\par}{\ifx\par}{}{\fail}}}{}
\makeatother
\usepackage[a-1b]{pdfx}

% ENLACES
\usepackage{hyperref}
\hypersetup{colorlinks=true,
  linkcolor=black, % enlaces a partes del documento (p.e. índice) en color negro
  urlcolor=blue} % enlaces a recursos fuera del documento en azul

% Añadir pdfs como partes del documento
\usepackage{pdfpages}

% Quitar la indentación de principio de los párrafos
\setlength{\parindent}{0em}

% EXPRESIONES MATEMÁTICAS
\usepackage{amsmath,amssymb,amsfonts,amsthm}

\usepackage{txfonts} 
\usepackage[T1]{fontenc}
\usepackage[utf8]{inputenc}

% Insertar gráficas y fotos
\usepackage{tikz}
\usetikzlibrary{cd}
\usepackage{pgfplots}

\usepackage[spanish, es-tabla]{babel} 
\usepackage[babel, spanish=spanish]{csquotes}
\AtBeginEnvironment{quote}{\small}

% diseño de PIE DE PÁGINA
\usepackage{fancyhdr}
\pagestyle{fancy}
\fancyhf{}
\renewcommand{\headrulewidth}{0pt}
\fancyfoot[LE,RO]{\thepage}
\fancypagestyle{plain}{\pagestyle{fancy}}

% DISEÑO DE LOS TÍTULOS de las partes del trabajo (capítulos y epígrafes o subcapítulos)
\usepackage{titlesec}
\usepackage{titletoc}
\titleformat{\chapter}[block]
{\large\bfseries\filcenter}
{\thechapter.}
{5pt}
{\MakeUppercase}
{}
\titlespacing{\chapter}{0pt}{0pt}{*3}
\titlecontents{chapter}
[0pt]                                               
{}
{\contentsmargin{0pt}\thecontentslabel.\enspace\uppercase}
{\contentsmargin{0pt}\uppercase}                        
{\titlerule*[.7pc]{.}\contentspage}                 

\titleformat{\section}
{\bfseries}
{\thesection.}
{5pt}
{}
\titlecontents{section}
[5pt]                                               
{}
{\contentsmargin{0pt}\thecontentslabel.\enspace}
{\contentsmargin{0pt}}
{\titlerule*[.7pc]{.}\contentspage}

\titleformat{\subsection}
{\normalsize\bfseries}
{\thesubsection.}
{5pt}
{}
\titlecontents{subsection}
[10pt]                                               
{}
{\contentsmargin{0pt}                          
  \thecontentslabel.\enspace}
{\contentsmargin{0pt}}                        
{\titlerule*[.7pc]{.}\contentspage}  

\usepackage{adjustbox}

% DISEÑO DE TABLAS.
\usepackage{multirow} % permite combinar celdas 
\usepackage{caption} % para personalizar el título de tablas y figuras
\usepackage{floatrow} % utilizamos este paquete y sus macros \ttabbox y \ffigbox para alinear los nombres de tablas y figuras de acuerdo con el estilo definido. Para su uso ver archivo de ejemplo 
\usepackage{array} % con este paquete podemos definir en la siguiente línea un nuevo tipo de columna para tablas: ancho personalizado y contenido centrado
\newcolumntype{P}[1]{>{\centering\arraybackslash}p{#1}}
\DeclareCaptionFormat{upper}{#1#2\uppercase{#3}\par}

% Diseño de tabla para ingeniería
\captionsetup[table]{
  format=hang,
  name=Tabla,
  justification=centering,
  labelsep=colon,
  width=.75\linewidth,
  labelfont=small,
  font=small,
}

% DISEÑO DE FIGURAS.
\usepackage{graphicx}
\graphicspath{{img/}} %ruta a la carpeta de imágenes

% Diseño de figuras para ingeniería
\captionsetup[figure]{
  format=hang,
  name=Fig.,
  singlelinecheck=off,
  labelsep=colon,
  labelfont=small,
  font=small    
}

% NOTAS A PIE DE PÁGINA
\usepackage{chngcntr} % Para numeración continua de las notas al pie
\counterwithout{footnote}{chapter}

% LISTADOS DE CÓDIGO
% soporte y estilo para listados de código. Más información en https://es.wikibooks.org/wiki/Manual_de_LaTeX/Listados_de_código/Listados_con_listings
\usepackage{listings}

% definimos un estilo de listings
\lstdefinestyle{estilo}{ frame=Ltb,
  framerule=0pt,
  aboveskip=0.5cm,
  framextopmargin=3pt,
  framexbottommargin=3pt,
  framexleftmargin=0.4cm,
  framesep=0pt,
  rulesep=.4pt,
  backgroundcolor=\color{gray97},
  rulesepcolor=\color{black},
  %
  basicstyle=\ttfamily\footnotesize,
  keywordstyle=\bfseries,
  stringstyle=\ttfamily,
  showstringspaces = false,
  commentstyle=\color{gray45},     
  %
  numbers=left,
  numbersep=15pt,
  numberstyle=\tiny,
  numberfirstline = false,
  breaklines=true,
  xleftmargin=\parindent
}

\captionsetup[lstlisting]{font=small, labelsep=period}
% fijamos el estilo a utilizar 
\lstset{style=estilo}
\renewcommand{\lstlistingname}{\uppercase{Código}}

\pgfplotsset{compat=1.17} 
%-------------
% DOCUMENTO
%-------------

\begin{document}
\pagenumbering{roman} % Se utilizan cifras romanas en la numeración de las páginas previas al cuerpo del trabajo

%----------
% PORTADA
%---------- 
\begin{titlepage}
  \begin{sffamily}
    \color{azulUC3M}
    \begin{center}
      \begin{figure}[H] % Incluimos el logotipo de la Universidad
        \makebox[\textwidth][c]{\includegraphics[width=16cm]{Portada_Logo.png}}
      \end{figure}
      \vspace{2.5cm}
      \begin{Large}
        Grado en Ingeniería Informática\\
        2021-2022\\
        \vspace{2cm}
        \textsl{Apuntes}\\
        \bigskip
      \end{Large}
      {\Huge Mitos Griegos y Cultura Europea}\\
      \vspace*{0.5cm}
      \rule{10.5cm}{0.1mm}\\
      \vspace*{0.9cm}
      {\LARGE Jorge Rodríguez Fraile\footnote{\href{mailto:100405951@alumnos.uc3m.es}{Universidad: 100405951@alumnos.uc3m.es}  |  \href{mailto:jrf1616@gmail.com}{Personal: jrf1616@gmail.com}}}\\
      \vspace*{1cm}
    \end{center}
    \vfill
    \color{black}
    \includegraphics[width=4.2cm]{img/creativecommons.png}\\
    Esta obra se encuentra sujeta a la licencia Creative Commons\\ \textbf{Reconocimiento - No Comercial - Sin Obra Derivada}
  \end{sffamily}
\end{titlepage}

%----------
% ÍNDICES
%---------- 

%--
% Índice general
%-
\tableofcontents
\thispagestyle{fancy}

%--
% Índice de figuras. Si no se incluyen, comenta las líneas siguientes
%-
\listoffigures
\thispagestyle{fancy}

%--
% Índice de tablas. Si no se incluyen, comenta las líneas siguientes
%-
\listoftables
\thispagestyle{fancy}

%----------
% TRABAJO
%---------- 

\pagenumbering{arabic} % numeración con números arábigos para el resto de la publicación  

%----------
% COMENZAR A ESCRIBIR AQUÍ
%---------- 

\chapter{Información}\label{ch:informacion}
\section{Profesores}\label{sec:profesores}
\begin{quote}
	Magistral: Lorena Pérez Yarza, loperezy@inst.uc3m.es
\end{quote}

\section{Sistema de evaluación}\label{sec:sistema-de-evaluación}
Hay evaluación continua, que si se sigue no hace falta ir a un examen final.
\begin{enumerate}
    \item Visita del Prado (17 de febrero) y Práctica
    \begin{itemize}
      \item Visita corta con la Uni (jueves por la mañana/sábado por la mañana)
      \item Hacer un trabajo de 2-3 páginas (3 páginas máximos), con imágenes, explicando 3 obras de arte con contenido mitológico.
    \end{itemize}
  \item Seminario online: 17 de febrero o 3 de marzo (es una alternativa al museo).
  \item 1 lectura obligatoria: Selección de obras en aula global.
    \begin{itemize}
      \item Extensión de 3 páginas (máximo 3 páginas), pero no solo copiar y pegar.
      \item Recensión: Decir en que consiste la obra y que impresión nos ha causado, como reflexión personal, teniendo en cuenta el ámbito mitológico.
      \item Decir más o menos como se organiza, cuál es la temática y como aborda esa temática (irónica, burlona, poética, etc.). Explicar el vehículo, el libro, el cómo se explican las cosas es lo que interesa de una recensión. Cuanto es de carácter científico se evalúa si es bueno o malo el libro según si se adecua a la cuestión, expresando que entendemos el texto y que sabemos explicarlo de forma resumida, pero sin incluir todos los hechos, hacer una selección de ejemplos importantes para que quede racional.
    \end{itemize}
  \item (Examen final) Presentaciones finales: oral en grupo y trabajo individual (18 de mayo, límite)
    \begin{itemize}
      \item Un tema mitológico que aparezca en la actualidad, ya sea libro, película, comic o contexto social.
      \item Sobre ese tema hay que explicar de que va el mito antiguo y su origen, para después mostrar en que ámbitos o lugares se da en la actualidad.
      \item Nos podemos inspirar en las Transformaciones de Ovidio, salen historias muy famosas que podemos usar.
      \item Grupos de 2 personas máximo para la presentación.
      \item Presentación oral de 5 minutos, online.
      \item Trabajo individual de 5-8 páginas (5 páginas mínimo) sobre este tema, entre compañeros será parecido. Además, debe incluir una reflexión personal tanto del mito como de su uso en la actualidad (con ejemplos) y una bibliografía. Puede incluir imágenes.
    \end{itemize}
\end{enumerate}

\chapter{Introducción}\label{ch:introduccion}
\textbf{Lord Byron}, poeta romántico británico, combatió en la Guerra de independencia griega y murió en esta.

La mitología era un parte cultural oral y viva que se usaba para explicar el mundo que les rodea, servían para enseñar. Entender la mitología griega nos puede ayudar a entender el mundo antiguo y su historia, la historia de los museos, edificios históricos o incluso analizar arte del cine o los cuadros.

\textbf{Heinrich Schliemann (1822-1890)} se inspiró en los relatos míticos y participo en los relatos míticos y participó activamente en el descubrimiento arqueológico de Troya y Micenas, lugares de la Ilíada. Dando lugar a que se tuviera más en cuenta los mitos y que la historia no solo se reconstruye con hechos fácticos.

Los historiadores y filósofos consideraban que eran la pura negación de la historia, ya que no tiene verosimilitud y son muy fantásticos, carecen de una lógica formal. Se pensaba que la ciencia tenía que explicar todo, pero no es así, las personas somos irracionales que actúan por impulsos y no todo tiene que ser fáctico.

En muchos casos se dan hechos que no son probados, pero que se han mitificado hasta el punto de que todo el mundo piense que es cierto. Es el caso del Cid o la Guerra de Troya.

En principio eran historias sagradas de la tribu que explicaban el mundo mediante relatos de hechos memorables donde dioses y héroes de antaño actuaban. Historias de otro tiempo, de los orígenes del mundo, separado del nuestro mentalmente, que por acciones divinas y heroicas en una época primordial cambiaban, lo ordenaban o lo rediseñaban hasta ser el que tenemos hoy.

\enquote{Los mitos \enquote{clásicos} representan ya el triunfo de la obra literaria sobre la creencia religiosa} \textbf{(Mircea Eliade, 1994: 166)}

Eran reescritos y modificados por escribas, dando lugar a veces a confusión y prejuicios de los historiadores, dando el significado que ellos piensan. Por ejemplo, la eliminación de algunos dioses menores, como el Dios Sol, fue eliminado durante el renacimiento por los escribas al pensar que era un dios oriental y no romano, que era una confusión y sustituyéndolo por Júpiter.

\enquote{Los mitos viven en el país de la memoria} \textbf{(Marcel Detienne)}

Se transmite de manera oral normalmente de generación en generación, aunque también se hace de manera literaria en instituciones especializadas. Cuando los griegos aprendían a escribir/leer lo primero que aprendían era la Ilíada, era conocido por todos.

Son ejes culturales, tanto políticamente como ligados a las creencias, tienen el mismo objetivo, van de la mano, explican ciertos ritos y acontecimientos.

La cultura occidental no es pagana, bebe de la memoria cultural, romana y griega, cultura clásica, pero en muchas zonas se ha expandido y adaptado a las diferentes culturas. Por ejemplo, los campos elíseos, como lugar bonito tras la muerte, bebe de la cultura egipcia.

\textbf{Mitema (tópico literario):} En el estudio de la mitología es la parte más pequeña que conforma un mito (equivale al fonema de la lingüística), es un elemento constante que siempre aparece intercambiado y reensamblado con otros mitemas o contextos. Son partes estructurales. Puede ser la idea de un viaje de regreso por el mar o un tipo de nacimiento/hecho milagroso (ejem. echado a un río, sumerios con Sargón de Acad o cristianismo con Moisés), el amor, un diluvio (ejem. sumerios y cristianismo). Son buenas ideas, que sirven para articular la obra y que cuando pasan a estar por escrito son tópicos literarios.

El mito se transmite a partir de ideas exitosas, pueden ser mitemas o ideas de contexto generales que gustan, que se usan para formar nuevas palabras o seres fantásticos como quimera que significaba cabra y se interpreta como una bestia ceremonial. Por ejemplo, capricornio parte del calendario. Pasa con las sirenas, grifos, etc.

Los mitos eran parte de la cultura viva e iban evolucionando constantemente, hay que tener muy presente que tienen una larga historia y variaciones. Tienen muchas raíces y es difícil llegar hasta el original, diferenciando obras literarias, teatrales o religiosas.

Hay muchos animales a los que se les dota de significado, como el Pegaso, Minotauro, Hidra, Paloma, León, Toro, Esfinge, etc.

El mito es fundamentalmente la historia oral, más que la escrita y literaria.

\chapter{Tema 2}

Un ejemplo de mito son Las Tres Gracias, que representan la alegria, encanto y ..., suelen ser acompanantes de Venus. Sus nombres representan el concepto que representan. Salen en el mito de Marte destruyendo todo incluidas las Gracias. Se han representado de muchas manera a lo largo de la historia.

Es muy habitual la representacion en trios como las moiras, las Horas o las hilanderas del ?futuro?, que representan ciertos conceptos alegoricos y se relacionan con los dioses.

Tambien estan las musas que son 9 y representan seres divinos que inspiran al arte.

?Donde encontramos el mito?
\begin{enumerate}
  \item En la alegoria filofica, desde la cosmogonia (el inicio del universo) hasta temas filosoficos.
  \item La explicación racional de los mitos (libros erudicos).
  \item En satira o parodia.
  \item Fabula sentimental.
  \item En el teatro (Ejem. Edipo Rey o Antigona)
  \item Relatos historicos y geograficos, que indirectamente se refieren a los hechos de los mitos o en las propias religiones.
  \item Propaganda politica, se apropian un dios o heroe que les represente.
  \item Sagas heroicas y epopeyas.
  \item Explicaciones complementarias en ficción realista, como La Ilíada.
  \item En moralejas antiguas y cuentos moralizantes, como La Odisea, emplean partes de los mitos para hacer una moraleja, tal como Icaro con aspirar a algo demasiado alto.
  \item Anecdotas curiosas o humiristicas en obras variadas, como Luciano de Samosata que habla de los mitos con un tono cinico.
  \item En representaciones artisticas, las mas amplia, al haber tantas nos permite saber cuales eran las mas populares y mas detalles al haber tanto escrito.
\end{enumerate}

Ejem. de mitos en la actualidad: Diosa de la justicia en tribunales, maraton, juegos olimpicos, astronomia, zodiaco, logo nike, nombres de los planetas (larga historia), continente Europa, ...

En muchos casos se cuentan historias uniendo y relacionando los mitos, se reutilizan de manera constante por las diferentes culturas.

Mitopoiesis la union de \dots, la creacion de un mundo fantasticos/ficticios basando en hechos o una referencia.

Sabemos de los mitos por los Aedos, cantantes populares que creaban canciones con ritos faciles de memorizar con un baston, y mitógrafos, redactores de mitos en epocas populares (poetas, filosofos) y recolectrores de estos.

Hay autores que crearon sus propios mitos y otros se encargaban de crear un corpus comun de las historias.

900-600 a.C. Epoca ... , anterior a que se pusiera la historia por escrito.

Cubrian la zona de Grecia y Turquia, posteriormente se expandieron, algunos los utilizaban los mitos como inspiración y otros los escribian propiamente.

323 - 31 a.C. Epoca Helenistica, la consolidación escrita de la mitologia..

La primera epopeya que se tiene es la Gilgamesh, que se ha conseguido reproducir en la actulidad de la manera que se hacia en la epoca. Narraciones lentas para remarcar los hechos epicos, con mucha repetición para facilitar que la gente lo recuerde.



Olimpo, el hogar de los dioses.

Se pensaba que los dioses habitaban en lugares inaccesibles, ya sea en las montanas (el Olimpo), cielo, subterraneo, etc. De esta manera esban en un lugar que los morales no alcanzaban, en la epoca no tenian los medios, es por esto que en los caminos habia santuarios.

Estos lugares tenian un sentido simbolico, por ejemplo monte Olimpo habia varios.




Cosmogonia: Narración mítica o un modelo, que pretende dar respuesta al origen del universo y de la propia humanidad..

Teogonía: Contiene una de las más antiguas versiones del origen del cosmos y el linaje de los dioses de la mitología griega. Habla de la creación del mundo.

Son temas muy recurrente por los aedos, en sus cantos.

Hay muchas versiones del origen del mundo.

Rea y Cronos son los que dan origen a los dioses, por una profecia de que su hijo acabara con el se los come. Rea arta oculta a Zeus en Creta, cuidado por ... . Un dia se traga una piedra y vomita a todos los dioses dandoles origen.

La teogonia de Hesiodo:
\begin{enumerate}
  \item Mundo en contacto, Koiné fenicia (la puesta en comun fenicia).
  \item Poeta rural de Beocia. \\ Musas.
  \item Competidor en certámenes de poesia.
  \item Cronologia incierta.
  \item Recoge parte del conocimeinto de su tiempo.
  \item En representaciones artisticas.
\end{enumerate}

El origen de Venus, se puede ver en El nacimiento de Venus de Boticelli 1482-1485, Afrodita nace de los genitales de Urano cortado por Cronos y al llegar a tierra.

Las grandes luchas: Titanomaquia
\begin{enumerate}
  \item Caida de la primera dinastia.
  \item Dioses primordiales (conceptuales)
  \item Cambios de dinastias de dioses.
  \item El episodio se reproduce con la Gigantomaquia y lucha contra Tifón.
\end{enumerate}

Se utilizaban los mitos para dar nombre a algunas cosas o ciertas cosas inspiraban para crear mitos.

\end{document}