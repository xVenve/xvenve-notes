\documentclass[12pt, twoside, openright]{report} % Fuente a 12pt, formato doble página y chapter a la derecha
\raggedbottom % No ajustar el contenido con un salto de página

% MÁRGENES: 2,5 cm sup. e inf.; 3 cm izdo. y dcho.
\usepackage[
a4paper,
vmargin=2.5cm,
hmargin=3cm
]{geometry}

% INTERLINEADO: Estrecho (6 ptos./interlineado 1,15) o Moderado (6 ptos./interlineado 1,5)
\renewcommand{\baselinestretch}{1.15}
\parskip=6pt

% DEFINICIÓN DE COLORES para portada y listados de código
\usepackage[table]{xcolor}
\definecolor{azulUC3M}{RGB}{0,0,102}
\definecolor{gray97}{gray}{.97}
\definecolor{gray75}{gray}{.75}
\definecolor{gray45}{gray}{.45}

% Soporte para GENERAR PDF/A
\usepackage{etoolbox}
\makeatletter
\@ifl@t@r\fmtversion{2021-06-01}%
 {\AddToHook{package/after/xmpincl}
   {\patchcmd\mcs@xmpincl@patchFile{\if\par}{\ifx\par}{}{\fail}}}{}
\makeatother
\usepackage[a-1b]{pdfx}

% ENLACES
\usepackage{hyperref}
\hypersetup{colorlinks=true,
  linkcolor=black, % enlaces a partes del documento (p.e. índice) en color negro
  urlcolor=blue} % enlaces a recursos fuera del documento en azul

% Añadir pdfs como partes del documento
\usepackage{pdfpages}

% Quitar la indentación de principio de los párrafos
\setlength{\parindent}{0em}

% EXPRESIONES MATEMÁTICAS
\usepackage{amsmath,amssymb,amsfonts,amsthm}

\usepackage{txfonts} 
\usepackage[T1]{fontenc}
\usepackage[utf8]{inputenc}

% Insertar gráficas y fotos
\usepackage{tikz}
\usetikzlibrary{cd}
\usepackage{pgfplots}

\usepackage[spanish, es-tabla]{babel} 
\usepackage[babel, spanish=spanish]{csquotes}
\AtBeginEnvironment{quote}{\small}

% diseño de PIE DE PÁGINA
\usepackage{fancyhdr}
\pagestyle{fancy}
\fancyhf{}
\renewcommand{\headrulewidth}{0pt}
\fancyfoot[LE,RO]{\thepage}
\fancypagestyle{plain}{\pagestyle{fancy}}

% DISEÑO DE LOS TÍTULOS de las partes del trabajo (capítulos y epígrafes o subcapítulos)
\usepackage{titlesec}
\usepackage{titletoc}
\titleformat{\chapter}[block]
{\large\bfseries\filcenter}
{\thechapter.}
{5pt}
{\MakeUppercase}
{}
\titlespacing{\chapter}{0pt}{0pt}{*3}
\titlecontents{chapter}
[0pt]                                               
{}
{\contentsmargin{0pt}\thecontentslabel.\enspace\uppercase}
{\contentsmargin{0pt}\uppercase}                        
{\titlerule*[.7pc]{.}\contentspage}                 

\titleformat{\section}
{\bfseries}
{\thesection.}
{5pt}
{}
\titlecontents{section}
[5pt]                                               
{}
{\contentsmargin{0pt}\thecontentslabel.\enspace}
{\contentsmargin{0pt}}
{\titlerule*[.7pc]{.}\contentspage}

\titleformat{\subsection}
{\normalsize\bfseries}
{\thesubsection.}
{5pt}
{}
\titlecontents{subsection}
[10pt]                                               
{}
{\contentsmargin{0pt}                          
  \thecontentslabel.\enspace}
{\contentsmargin{0pt}}                        
{\titlerule*[.7pc]{.}\contentspage}  

\usepackage{adjustbox}

% DISEÑO DE TABLAS.
\usepackage{multirow} % permite combinar celdas 
\usepackage{caption} % para personalizar el título de tablas y figuras
\usepackage{floatrow} % utilizamos este paquete y sus macros \ttabbox y \ffigbox para alinear los nombres de tablas y figuras de acuerdo con el estilo definido. Para su uso ver archivo de ejemplo 
\usepackage{array} % con este paquete podemos definir en la siguiente línea un nuevo tipo de columna para tablas: ancho personalizado y contenido centrado
\newcolumntype{P}[1]{>{\centering\arraybackslash}p{#1}}
\DeclareCaptionFormat{upper}{#1#2\uppercase{#3}\par}

% Diseño de tabla para ingeniería
\captionsetup[table]{
  format=hang,
  name=Tabla,
  justification=centering,
  labelsep=colon,
  width=.75\linewidth,
  labelfont=small,
  font=small,
}

% DISEÑO DE FIGURAS.
\usepackage{graphicx}
\graphicspath{{img/}} %ruta a la carpeta de imágenes

% Diseño de figuras para ingeniería
\captionsetup[figure]{
  format=hang,
  name=Fig.,
  singlelinecheck=off,
  labelsep=colon,
  labelfont=small,
  font=small    
}

% NOTAS A PIE DE PÁGINA
\usepackage{chngcntr} % Para numeración continua de las notas al pie
\counterwithout{footnote}{chapter}

% LISTADOS DE CÓDIGO
% soporte y estilo para listados de código. Más información en https://es.wikibooks.org/wiki/Manual_de_LaTeX/Listados_de_código/Listados_con_listings
\usepackage{listings}

% definimos un estilo de listings
\lstdefinestyle{estilo}{ frame=Ltb,
  framerule=0pt,
  aboveskip=0.5cm,
  framextopmargin=3pt,
  framexbottommargin=3pt,
  framexleftmargin=0.4cm,
  framesep=0pt,
  rulesep=.4pt,
  backgroundcolor=\color{gray97},
  rulesepcolor=\color{black},
  %
  basicstyle=\ttfamily\footnotesize,
  keywordstyle=\bfseries,
  stringstyle=\ttfamily,
  showstringspaces = false,
  commentstyle=\color{gray45},     
  %
  numbers=left,
  numbersep=15pt,
  numberstyle=\tiny,
  numberfirstline = false,
  breaklines=true,
  xleftmargin=\parindent
}

\captionsetup[lstlisting]{font=small, labelsep=period}
% fijamos el estilo a utilizar 
\lstset{style=estilo}
\renewcommand{\lstlistingname}{\uppercase{Código}}

\pgfplotsset{compat=1.17} 
%-------------
% DOCUMENTO
%-------------

\begin{document}
\pagenumbering{roman} % Se utilizan cifras romanas en la numeración de las páginas previas al cuerpo del trabajo

%----------
% PORTADA
%---------- 
\begin{titlepage}
  \begin{sffamily}
    \color{azulUC3M}
    \begin{center}
      \begin{figure}[H] % Incluimos el logotipo de la Universidad
        \makebox[\textwidth][c]{\includegraphics[width=16cm]{Portada_Logo.png}}
      \end{figure}
      \vspace{2.5cm}
      \begin{Large}
        Grado en Ingeniería Informática\\
        2021-2022\\
        \vspace{2cm}
        \textsl{Apuntes}\\
        \bigskip
      \end{Large}
      {\Huge Mitos Griegos y Cultura Europea}\\
      \vspace*{0.5cm}
      \rule{10.5cm}{0.1mm}\\
      \vspace*{0.9cm}
      {\LARGE Jorge Rodríguez Fraile\footnote{\href{mailto:100405951@alumnos.uc3m.es}{Universidad: 100405951@alumnos.uc3m.es}  |  \href{mailto:jrf1616@gmail.com}{Personal: jrf1616@gmail.com}}}\\
      \vspace*{1cm}
    \end{center}
    \vfill
    \color{black}
    \includegraphics[width=4.2cm]{img/creativecommons.png}\\
    Esta obra se encuentra sujeta a la licencia Creative Commons\\ \textbf{Reconocimiento - No Comercial - Sin Obra Derivada}
  \end{sffamily}
\end{titlepage}

%----------
% ÍNDICES
%---------- 

%--
% Índice general
%-
\tableofcontents
\thispagestyle{fancy}

%--
% Índice de figuras. Si no se incluyen, comenta las líneas siguientes
%-
\listoffigures
\thispagestyle{fancy}

%--
% Índice de tablas. Si no se incluyen, comenta las líneas siguientes
%-
\listoftables
\thispagestyle{fancy}

%----------
% TRABAJO
%---------- 

\pagenumbering{arabic} % numeración con números arábigos para el resto de la publicación  

%----------
% COMENZAR A ESCRIBIR AQUÍ
%---------- 

\chapter{Información}\label{ch:informacion}
\section{Profesores}\label{sec:profesores}
\begin{quote}
	Magistral: Lorena Pérez Yarza, loperezy@inst.uc3m.es
\end{quote}

\section{Sistema de evaluación}\label{sec:sistema-de-evaluación}
Hay evaluación continua, que si se sigue no hace falta ir a un examen final.

\section{Calendario y Evalución Continua}
\begin{enumerate}
    \item Visita del Prado y Práctica
    \begin{itemize}
      \item Visita corta con la Uni (jueves por la manana/sabado por la manana)
      \item Hacer un trabajo de 2-3 paginas, con imagenes, hablando de 3 obras del museo relacionadas con mitos.
    \end{itemize}
  \item Seminario online: 17 de febrero o 3 de marzo. Es una alternativa al museo.
  \item 1 lectura obligatoria (max. 3 páginas): Selección de obras en aula global.
    \begin{itemize}
      \item Extensión de 3 paginas, pero no solo copiar y pegar.
      \item Decir en que consiste la obra, recensión, OIR
      \item Hacer una reflexion de la lectura personal
    \end{itemize}
  \item (Examen final) Presentaciones finales: oral en grupo y trabajo individual (18 de mayo, límite)
    \begin{itemize}
      \item Grupos de 2 maximo
      \item Presentación oral de 5 minutos, online.
      \item Entre 5-8 páginas de extensión del documento, incluye referencias/bibliografia.
      \item Hablar de una obra, haciendo una reflexion y relacionarla con el mundo real
      \item Ambas personas la deben entregar en aula global
    \end{itemize}
\end{enumerate}

\chapter{Tema 1}
Lord Byron, poeta romantico britanico, combatió en la Guerra de inependencia Griega.
  Muere en la guerra de independencia de \ldots

La mitologia se usaba para explicar el mundo que les rodea, de manera oral y eran populares, servian para ensenar.

Los historiadores y filosofos consideraban que eran lo contrario a la verdad, ya que no tiene verosimilitud y son muy fantasticos.

Se pensaba que la ciencia tenia que explicar todo, pero no es así las personas somos iracionales que actuan por impulsos y no todo tiene que ser factico.

Heinrich Schliemann (1822-1890) se inspiro en los relatos miticos y participo en los relatos miticos y paritcipo activamente en le descubrimiento arqueologico de Troya y Micenas, lugares de la Ilíada.

En muchos casos se dan hechos que no son probados, pero que se han mitificado hasta el punto que todo el mundo piense que es cierto. El caso del Cid o la Guerra de Troya.

Los mitos clasicos representan ya el triunfo de la obra literia sobre la creencia religiosa (M. Eliade, 1994: 166)

  Por ejemplo el Dios Sol fue eliminado al pensar que era un dios oriental y era un error sustituyendose por Jupiter.

  Los mitos viven en el país de la memoria (Marcel Detienne)

  Se transmite de manera oral normalmente de generación en generación, aunque tambien se hace de manera literaria. Cuando los griegos aprendian a escribir/leer lo primero que aprendian era la Ilíada, era conocido por todos.

  Estan ligados a las creencias y tienen el mismo objetivo, van de la mano, explican ciertos acontecimientos. Sirven para explicar ciertos ritos.

  La cultura europea no es pagana, bebe de la romana y griega, cultura clasica, pero en muchas zonas se ha expandido y adaptado a las diferentes culturas. Por ejemplo los campos eliseos, como lugar bonito tras la muerte bebe de la cultura egipcia.

  Mitema (topico literario): La parte mas pequena que conforma un mito (equivale al fonema de la lengua), como puede ser el viaje de regreso de un heroe, un nacimiento/hecho milagroso (ejem. echado a un rio, sumerios con Sargon de Akkad o critianismo con Moises, el amor, un diluvio (ejem. sumerios y critianismo) \ldots. Son buenas ideas, que sirven para articular la obra.


  El mito se transmite a partir de ideas buenas, que se usan para crear nuevas palabras como quimera que significaba cabra y se interpreta como una bestia. Por ejemplo capricornio parte del calendario. Pasa con las sirenas, grifos, etc.

  Los mitos eran parte de la cultura viva e iban evolucionando constantemente, hay que tener muy presente que tienen una larga historia y variaciones. Tienen muchas raices y es dificil llegar hasta el original, diferenciando obras literias, teatrales o religiosas.

Hay muchos animales a los que se les dota de significado, como el Pegaso, Minotauro, Hidra, Paloma, Leon, Toro, Esfinge, etc.

  El mito es fundamentalmente la historia oral, más que la escrita y literaria.




\end{document}