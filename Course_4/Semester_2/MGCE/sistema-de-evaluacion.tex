\section{Sistema de evaluación}\label{sec:sistema-de-evaluación}
Hay evaluación continua, que si se sigue no hace falta ir a un examen final.
\begin{enumerate}
    \item Visita del Prado (17 de febrero) y Práctica
    \begin{itemize}
      \item Visita corta con la Uni (jueves por la mañana/sábado por la mañana)
      \item Hacer un trabajo de 2-3 páginas (3 páginas máximos), con imágenes, explicando 3 obras de arte con contenido mitológico.
    \end{itemize}
  \item Seminario online: 17 de febrero o 3 de marzo (es una alternativa al museo).
  \item 1 lectura obligatoria: Selección de obras en aula global.
    \begin{itemize}
      \item Extensión de 3 páginas (máximo 3 páginas), pero no solo copiar y pegar.
      \item Recensión: Decir en que consiste la obra y que impresión nos ha causado, como reflexión personal, teniendo en cuenta el ámbito mitológico.
      \item Decir más o menos como se organiza, cuál es la temática y como aborda esa temática (irónica, burlona, poética, etc.). Explicar el vehículo, el libro, el cómo se explican las cosas es lo que interesa de una recensión. Cuanto es de carácter científico se evalúa si es bueno o malo el libro según si se adecua a la cuestión, expresando que entendemos el texto y que sabemos explicarlo de forma resumida, pero sin incluir todos los hechos, hacer una selección de ejemplos importantes para que quede racional.
    \end{itemize}
  \item (Examen final) Presentaciones finales: oral en grupo y trabajo individual (18 de mayo, límite)
    \begin{itemize}
      \item Un tema mitológico que aparezca en la actualidad, ya sea libro, película, comic o contexto social.
      \item Sobre ese tema hay que explicar de que va el mito antiguo y su origen, para después mostrar en que ámbitos o lugares se da en la actualidad.
      \item Nos podemos inspirar en las Transformaciones de Ovidio, salen historias muy famosas que podemos usar.
      \item Grupos de 2 personas máximo para la presentación.
      \item Presentación oral de 5 minutos, online.
      \item Trabajo individual de 5-8 páginas (5 páginas mínimo) sobre este tema, entre compañeros será parecido. Además, debe incluir una reflexión personal tanto del mito como de su uso en la actualidad (con ejemplos) y una bibliografía. Puede incluir imágenes.
    \end{itemize}
\end{enumerate}