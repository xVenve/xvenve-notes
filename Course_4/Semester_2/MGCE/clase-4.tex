\chapter{Clase 4}\label{ch:clase4}
Belerofonte es un héroe con relevancia en la época clásica, destaca por su montura Pegaso, aunque en la actualidad no. Pegaso se ha representado en la época Helenística.

El nombre de Belerofonte viene de cuando mato a su hermano, se oculta y el rey le refugio, la reina se enfada de celos y lo manda matar por el asesinato de su propio hermano.

También aparece como que Belerofonte recibe una carta que dice que debe matar al que le haya entregado la carta, típico leitmotiv.

Yóbates le encargan matar a una quimera, cuerpo de león y cola de serpiente. La consigue matar tras pedir ayuda a los dioses, Atenea le ayuda con unas bridas especiales a capturar a Pegaso, con el que crea un fuerte vínculo, este le ayuda a matar al monstruo. Tras esto le nombran rey. 

En el momento culminante de su gloria, trata de subir con Pegaso al Olimpo, pero le mandan un tábano que le pica y tira, muriendo o vagando por la tierra maldito.

\section{Hera - Juno}
\begin{itemize}
    \item Soberana del Olimpo.
    \item Mujer y hermana de Zeus.
    \item No madre de dioses (provincia compartida).
    \item Diosa del hogar matrimonio. Es el símbolo del matrimonio romano, en el que sufre idas y venidas, soportando al marido que está por ahí.
    \item Hera Ziguía, con faceta de soberana, matrona y protectora. Es una figura secundaria.
    \item Faceta mitológica: celos maritales.
    \item Diosa antigua. Algunos de los templos más antiguos estand dedicados a ella.
    \item Sin evolución cronológica clásica.
    \item Se le sacrificaban vacas y a Zeus toros.
\end{itemize}
\pagebreak

\section{Heracles ''el que place a Hera''}
Los trabajos de Hércules (Hera le hace enloquecer, matando a su familia, hijos o mujer dependiendo de la versión) para redimirse, típico de los héroes, los dioses le dan proezas:
\begin{itemize}
    \item Errores
    \item Purga de sus pecados (asesinato de hijos)
    \item Trabaja para Euristeo, rey de Tirinto, que le asigna 12 (número simbólico) trabajo imposibles.
    \item Solo o con ayuda
    \begin{itemize}
        \item Divina (variada)
        \item Heroica (Yolao)
    \end{itemize}
    \item Muerte trágica
    \item Sublimación final
    \begin{itemize}
        \item Cuerpo mortal
        \item Ascenso al Olimpo y reconciliación
    \end{itemize}
\end{itemize}
El primero es la caza del león de Nemea, que lo ahoga por su piel es permeable a las armas por eso la usara y se hará símbolo de él.

Muerte: Heracles se enamora de Yanira y se queda con ella para vivir, tras matar a Neso (un centauro malo, opuesto a Quirón) y regalarle una capa llena de sangre de este. Al tiempo Deianira su esposa, de celos le regala una capa envenenada a Yanira que se la entrega a Heracles, quien al ponérsela empieza a quemarse. Sufre tal dolor que pide a los dioses que acaben con su vida, se lo conceden y su parte de dios sube al Olimpo y se reconcilia con su madre.

Antropofagia y el fuero eran signos de la barbarie.

Apolo y Heracles luchan por el trípode délfico, de Delfos, finalmente gana Apolo, pero él resuelve la duda de como librarse de la maldición de la muerte de Ífito.
\begin{itemize}
    \item La sacerdotisa se sentaba sobre él y entraba en trance.
\end{itemize}

Héroe Dorio Civilizador (Hércules es el dios civilizador por antonomasia, cuanto más bajo caes más puedes subir)
\begin{itemize}
    \item Mitos secundarios
    \item Heráclidas
    \begin{itemize}
        \item Descendencia
    \end{itemize}
    \item La civilización alcanza  los límites del mundo conocido
\end{itemize}

Hércules pide a Atlas que le traiga unas manzanas de sus hijas, mientras él aguantaría el mundo, pero no vuelve... Entonces Heracles le dice que le sujete la esfera un momento y en este momento se zafa de sujetarlo más y le roba las manzanas. 

Es estrecho de Gibraltar, las columnas de Hércules.

La dinastía de Esparta decían ser descendientes de Heracles.

\section{La Apoteosis (sublimación)}
\begin{itemize}
    \item Referencia de sublimación, paso a ser dioses o diosificados, culto al líder y asciende a los cielos para ser dioses. Ascienden como ceniza tras su muerte o llevados por Marte como los del origen de Roma
    \item Heracles - Hércules
    \item Reyes helenísticos
    \item Emperadores romanos
    \item Neoclasicismo: Apoteosis de héroes nacionales.
\end{itemize}

Culto latino

Heracles y Caco (significa el mal), se le erige el Templo Hércules Victor (vencedor)

Foro Boario

Dios popular

Templo púnicos (Melkart)

Cómodo se hizo representar como Hércules a finales de siglo II d.C.

Los trabajos de Hércules ya en el mundo latino se difundieron por todo el Mediterráneo. Como un dios fuerte y heroico. Con el cristianismo baja su popularidad.

\section{Hércules en Hispania: Gerión, Pyrenne y Tubal}
\begin{itemize}
    \item Monstruo Gerión.
    \item Ninfa Pyrenne. Por la pena de Heracles tras la muerte de Pyrenne empieza a amontonar piedra y se da origen a los Pirineos.
    \item Personaje bíblico menor, Túbal, nieto de Noé. Se repartieron el mundo y se encarga de poblar la península ibérica, dando entre otros a los vascos.
\end{itemize}