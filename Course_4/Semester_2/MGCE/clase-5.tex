\chapter{Clase 5}\label{ch:clase5}
Tema clase: Desde los dioses hasta los hombres. Las edades del hombre o de la humanidad.

Los humanos son hijos de la Tierra, Gea, u originarios de diferentes regiones.

\section{Otros orígenes las edades del hombre}
\begin{enumerate}
	\item Hesíodo frente a Ovidio
	      \begin{enumerate}
		      \item \textbf{Oro - Cronos.} Sin sufrimiento, se alimentaban de lo que la naturaleza proporciona
		      \item \textbf{Plata - olímpicos.} 100 años de juventud. Soberbia, condena al Hades, no rendían culto a los dioses. Son castigados con un diluvio universal. Pasan a tener que trabajar y envejecen rápidamente.
		      \item \textbf{Bronce - Zeus.} Guerreros y cultura de bronce (como material), fuertes y valientes para la batalla. Se alimentan con carne y no solo cereales. Autodestrucción. Hades sin memoria.
		      \item \textbf{Héroes - héroes del periodo mítico.} Desaparecidos o en los márgenes. Las hazañas a diferencia que el bronce son recordadas por los hombres. No tiene nombre de metal porque no es reconocida por Ovidio.
		      \item \textbf{Hierro - actual.} Pecadora y codiciosa, acechada por los males para ellos, el medioambiente y los dioses. Material más resistente.
	      \end{enumerate}
	\item Fin del mundo: cuando el bien, juramento y justicia no tengan valor.
\end{enumerate}

Uno de los mitemas es la degeneración de la sociedad, siempre la sociedad previa era mejor, según se van alejando de su vida en el Edén. Ciclos de creación y destrucción. Los mexicas o aztecas usaron este tema como eje de su teología política y los sacrificios se hacían para evitar el fin.

\section{Poseidón / Neptuno}
\begin{itemize}
	\item Soberano de los mares
	\item Agua dulce
	\item Terremotos y caballos
	\item Tridente y carro de hipocampos
\end{itemize}

Muchos de sus descendientes son terribles.

Poseidón tenía los juegos del istmo de Corinto.

Dio lugar a ninfas, pueblan la naturaleza como diosas menores. Supervisoras de lugares sagrados, en Galicia hay muchas representaciones. Ninfeos, ...

No tuvo muchos lugares de culto, además de Corintia. Atenea confronta a Poseidón para hacerse con Atenas.

Las ciudades tienen un dios patrones que tiene un lugar preferente.

Atenas es el centro de la Grecia clásica, donde se conectaba la ciudad con el puerto.

En la Atenas clásica era el centro cultural y político del momento, donde se escriben importantes obras como la Ilíada. Tenía la acrópolis, el ágora, ... Tenían una gran estatua de Atenea.

\section{Teseo: Héroe de Atenas}
\begin{itemize}
	\item Bastardo del rey Egeo. Cuando llega su adolescencia es llevado por su madre a recoger bajo una pesada piedra su armadura y arma, enterrada por su padre para cuando fuese digno.
	\item Fuerte e inteligente (ingenio), no se dejó engañar por el asesino de Corinto.
	\item Héroe jonio
	\item Mitos históricos
	\item Civilizo el camino a Atenas. Venció a múltiples bandidos con nombre, el héroe civiliza las ciudades para que los hombres las puedan poblar.
\end{itemize}

\section{Creta y Monte Ida}
\begin{itemize}
	\item Ida y yacimiento de Zominthos. En el monte Ida hay una cueva en la que se han encontrado ofrendas a Zeus, además era un centro de peregrinaje al lugar donde creció Zeus
	\item Juventud de Zeus
	\item Refugio de Europa
	\item Casada con el rey y 3 reglados: perro, jabalina y automata (el automata Talos, parece se una figura importante, era defensor y tenía su punto deben en el tobillo)
\end{itemize}

\section{Hades}
\begin{itemize}
	\item Hijos de Europa: Radamantis, Sarpedon y Minos.
	\item Regalo de Poseidón: el toro blanco. Se hizo con parte del mediterráneo, pero por su soberbia y no querer sacrificarlo es castigada su mujer.
	\item Relación con la cultura minoica
\end{itemize}

\section{Un castigo divino: Pasífae y el Toro}
\begin{itemize}
	\item Pasífae es asistida por Dédalo para entrar en la vaca de madera y acercarse al Toro de Poseidón, teniendo relaciones con el, dando como resultado el minotauro. Creció rápidamente y demostró su naturaleza salvaje, alimentándose de personas y matando ganado.
\end{itemize}

\section{El laberinto}
\begin{itemize}
	\item Labrys: hacha minoica y objeto ritual.
	\item Palacio y símbolo de identidad posterior en Creta
	\item Dédalo le construye este palacio en Labrys, de ahí laberinto.
	\item El rey Minos cada ciertos años mandaba a 14 jóvenes como alimento, no lo querían matar de hambre.
	\item Teseo es el encargado de vencer al Minotauro, acabando con los sacrificios y convirtiéndose en rey. Previamente, es retado por el rey a recuperar un anillo tirado al centro del océano, llevándose además una corona.
	\item Ariadna se enamora de Teseo  y le cuanta el secreto para atravesar el laberinto. Hilo de Ariadna. Teseo era capaz de vencerlo, pero no de salir.
	\item Egeo (padre Teseo) se queda compungido al ver las velas negras del barco que volvía y se tira al mar. Dado que se suponía que llevaría si gana velas blancas y de la tristeza se tira al mar, así surge el mar Egeo.
	\item En algunas versiones no muere el minotauro, sino que lo ata y lo llevara celebrando su victoria, lo que daría lugar a ciertos cultos dedicados al toro.
\end{itemize}

Ícaro es encerrado con su padre Dédalo, dado que construyeron el laberinto y sabían salir de este y de la isla. En esta torre construyen un par de alas de cera cada uno, ambos huyen, pero advertido por su padre se acerca demasiado al calor de Helios, se le derriten y caen.

Minos estuvo buscando al inventor huido, Dédalo, escondido en Griganto. Pero con un truco, hilar una concha, imposible para el resto de mortales, consiguió que por su orgullo saliera y lo lograse (usa una hormiga que tira de ella).

En un baño Minos es ahogado, siguiendo Dédalo con su vida, pero termina la época.

\section{Minos}
\begin{itemize}
	\item Muerta por Artemisa (Homero)
	\item Abandono en Naxos (después)
	\item Rescate de Dionisio volviendo de la India (lugar lejano y místico, desde donde los dioses traen nuevos conocimientos)
	\item Ariadna se convierte en su esposa
\end{itemize}

\section{Otras aventuras de Teseo }
 (hay muchas historias y variaciones, muchas inconexas)
\begin{itemize}
	\item Con las amazonas con Hércules, rapta a Hipólita y nace Hipólito. Esta será maltratada y posteriormente abandonada.
	\item Pirítoo e Hipodamia y la Centauromaquia
	\item Teseo se casa con Helena de Esparta, que más tarde seria Helena de Troya
	\item Descenso a los infiernos, donde Hades les tiende una trampa y los deja pegados a un banco por artes mágicas (cinto mágico o brazalete, según la versión) como castigo por toda la eternidad. Heracles en su duodécimo baja al Hades y se los encuentra, como son amigos logra liberarlos con su fuerza. Con esa fuerza tiembla la Tierra y abandonan a Pirítoo en el infierno.
	\item La madre de Teseo es vendida como esclava y hacen huir a sus hijos con Fedra. Poniendo en el trono a un enemigo de Atenas.
	\item Teseo en la cúspide de su gloria pierde todo por intentar meterse donde no le llaman, con los dioses.
\end{itemize}
