\chapter{Clase 9}\label{ch:clase9}
\section{Midas}
Un rey de Frigia que todo lo que tocaba se convertía en oro. Se le conoce por ser el primer extranjero en entrar en contacto con el oráculo de Delfos.

Antes de ser rey de Frigia, Asia, conoció a Sileno que le contó que hay un país lleno de gigantes o un lago donde los que se bañaban rejuvenecían. Midas se queda maravillado con estas fábulas y le sigue hasta Dionisio, al que le pide que todo lo que toque se convierta en oro, Baco se lo concede. Pero como convertía todo en oro, se moría de hambre, entonces le pide que le quite el don (otras versiones dicen que es por convertir a su hija en oro) y este le dice que se bañe en el río Pactolo, Anatolia. Esto le quita el don y las piedras del lecho del río se convirtieron en oro, lo que explicaba que hubiera riqueza en esta región.

Pan es un dios de la naturaleza y un sátiro, que tenía gran control de la flauta doble, rivalizando con Apolo, dios de la música. Apolo le da unas orejas de burro para que pudiera apreciar la verdadera música.

\section{Dionisio - Baco}
\begin{itemize}
	\item Dios del vino, la fiesta y el desenfreno. Dios relativo a la fertilidad.
	\item Nacimiento semi-milagroso de Zeus.
	\item Origen y características de nacimiento incierto en la antigüedad.
	\item Dios popular y exótico: Bromios, Eleuteros. Representa lo exótico, pero ninguna lo acepta como suyo.
	\item No aceptado inicialmente en Roma.
\end{itemize}

Es gestado del muslo de Zeus, donde estuvo 3 meses creciendo, tras la muerte de su madre.

Lleva un tirso y llega pieles exóticas
\pagebreak

\subsection{Los milagros del Dios}
\begin{itemize}
	\item Dios milagroso, nacimiento milagroso y crece rápidamente.
	\item Le acompañan silenos y faunos, así como ninfas y ménades. Se dejaban lleva por la pasión, no les gustaba el trabajar, solo la fiesta y bebida. Estos forman el cortejo de Baco.
	\item Danzaba y cantaba.
	\item Es apresado en un barco, que no saben que un dios. Convierte el mar en vino y el mástil en una vid, todos se tiran al mar menos el timonel y estos se convierten en delfines.
	\item Castiga a los que niegan su divinidad
	\item Mediante el alcoholismo los mortales podían acercarse a la divinidad
\end{itemize}

\subsection{Bacanales y Menados}
Desenfreno sexual, bebiendo y realizando orgías. Muestran el extremo a Apolo, con sentimientos y excitación. Se realizaban en lugares naturales y apartados de la luz del sol.

Bacantes: Mujeres que llevaban estos cultos y procesiones, en las que se baila al ritmo de la música.

Ménades: divinidades menores, símbolo del desenfreno. Fieles que entraban en trance y entraban en locura.

\section{Eneas en Troya}
Héroe troyano importado por los romanos. Hijo de Anquises y Afrodita, del que se enamora por una maniobra de Zeus y del que nacerá Eneas. Anquises al desvelar quien es la madre del niño es alcanzado por un rayo.

Áyax pelea con Odiseo para quedarse con la armadura de Aquiles, como gana Odiseo, mata al ganado. Arrepentido se suicida lanzándose sobre su propia espada.

Es representado como herido por Diomedes de la guerra de Troya, pero favorecido y curado por dios, principalmente su madre, Afrodita.

En el templo arcaico de Lavinium hay 13 altares dedicados al culto de Eneas, que se fueron ampliando durante los siglos VI y IV a.C.

\subsection{La adopción de otras culturas}
Los cantares de gesta medievales como el Ciclo Artúrico o la Canción de Roldán. Los autores ingleses reinterpretaban una tradición bretona y propia, con reyes míticos galeses.

Estos son adoptados de los bretones (descienden de una mezcla de romanos y celtas de Britania) y en estos se cuentan el origen de ciertas ciudades o victorias.

\section{El mito político: La Eneida}
\begin{itemize}
	\item Inicio del Imperio de Augusto
	\item Escrita por Virgilio
	\item Encargo político, redacción epopeya, se nota la propagando al nombrar generales como héroes.
	\item 29-19 a.C.
	\item Fue muy famoso, se utilizaba para aprender a leer y escribir, expandiéndose por todo el mundo. Similar a la Ilíada.
	\item 12 libros: Ciclo de viajes y Ciclo de Conquista (fundación del Lacio)
\end{itemize}

\subsection{Dido de Cartago}
La historia comienza con la promesa de Zeus a Odiseo en Troya de que sus descendientes gobernarían el mundo.

El antagonista es Hera, que apoyaba a los griegos y que tratara de retrasar los avances de Eneas, que dará origen a Roma.

La Eneida se escribe una vez se conoce el resultado histórico. Eneas naufraga con sus hombres por las costas africanas y son acogidos por la reina fenicia local (ya cartaginesa) llamada Dido, le cuenta las aventuras de la guerra de Troya.

Dido era la primera reina de Cartago, era su fundadora mitológica según los latinos. Se decía que se llamaba Elisa y que había sido acogida por los caudillos, dejándola gobernar solo sobre una región que abarcara la piel de un buey. Ingeniosamente, mando cortar la piel del buey en finas tiras para abarcar una península pequeña donde se fundó la ciudad de Cartago. Sería un rival de Roma, en las guerras púnicas.

En el relato de Eneas narra el hecho fundacional de Dido, que acoge a los troyanos huidos. Afrodita hace que Dido se enamore de Eneas, haciéndole que les acoja y les dé de todos. Eneas se acomoda y se enamora de esta reina, en ese momento aparece Zeus para recordarle que su destino es seguir viajando al oeste donde poder fundar una ciudad de la que surgirá ese gran imperio y así el héroe cumple con la llamada de los dioses.

Esta trata de frenarle, por lo que huye una noche con ayuda de los dioses y está al ver los barcos irse, rota por la traición, se suicida y jura venganza contra Eneas y sus descendientes. Lo que explica el antagonismo de los cartagineses y los romanos en época histórica.

\subsection{Los viajes de Eneas}
Como paso con Jason y los argonautas recorre los mares, ocurriendo sucesos y contando su paso por los diferentes lugares. Uno de estos lugares en la costa africana es Cartago donde conoce a Dido, tras esto desembarca cerca de Pompeya, donde consulta a la Sibila (divina latina) y esta lo guía al inframundo.

\subsection{La llegada al Lacio}
En el inframundo visita a los héroes de la guerra de Troya, donde visita también a antiguos miembros de su tripulación e incluso con el personaje más importante, Anquises (su padre). Su padre le muestra los futuros héroes latinos y romanos que nacerán de su fundación, entre los que esta Cesar y Augusto.

Estas visiones le animan a seguir hacia el norte y a desembarcar en el Lacio, donde Virgilio narra como es recibido Eneas por la hospitalidad del rey latino, que le da en matrimonio a su hija Lavinia, con la que se casará, para honrarla esta vez su busca donde fundar su ciudad. Se encuentra en una cueva una cerda amamantando a sus hijos y allí encuentra el lugar idóneo para funda Lavinio (ciudad de Lavinia). Su hijo Ascanio (Yolu para los romanos) fundará cerca de allí la ciudad de Alba longa, lugar donde nacerán Rómulo y Remo.