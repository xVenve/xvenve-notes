\chapter{Clase 2}\label{ch:clase2}
Un ejemplo de mito son Las Tres Gracias, que representan Aglaya (‘Belleza’), Eufrósine (‘Júbilo’) y Talia (‘Abundancia’), suelen ser acompañantes de Venus. Sus nombres representan el concepto que representan. Salen en el mito de Marte destruyendo todo incluidas las Gracias. Se han representado de muchas maneras a lo largo de la historia.

Es muy habitual la representación en tríos como las Moiras (hilanderas del destino) o las Horas (naturaleza y estaciones), que representan ciertos conceptos alegóricos y se relacionan con los dioses.

También están las musas que son 9 y representan seres divinos que inspiran al arte.

¿Dónde encontramos el mito?
\begin{enumerate}
  \item En la alegoría filosófica, desde la cosmogonía (el inicio del universo) hasta temas filosóficos.
  \item La explicación racional de los mitos (libros eruditos).
  \item En sátira o parodia.
  \item Fábula sentimental.
  \item En el teatro (Ejem. Edipo Rey o Antígona)
  \item Relatos históricos y geográficos, que indirectamente se refieren a los hechos de los mitos o en las propias religiones.
  \item Propaganda política, se apropian un dios o héroe que les represente.
  \item Sagas heroicas y epopeyas.
  \item Explicaciones complementarias en ficción realista, como La Ilíada.
  \item En moralejas antiguas y cuentos moralizantes, como La Odisea, emplean partes de los mitos para hacer una moraleja, tal como Ícaro con aspirar a algo demasiado alto.
  \item Anécdotas curiosas o humorísticas en obras variadas, como Luciano de Samosata que habla de los mitos con un tono cínico.
  \item En representaciones artísticas, las más amplias, al haber tantas nos permite saber cuáles eran las más populares y más detalles al haber tanto escrito.
\end{enumerate}

Ejem. de mitos en la actualidad: Diosa de la justicia en tribunales, maratón, juegos olímpicos, astronomía, zodiaco, logo Nike, nombres de los planetas (larga historia), continente Europa, etc.

En muchos casos se cuentan historias uniendo y relacionando los mitos, se reutilizan de manera constante por las diferentes culturas.

\textbf{Mitopoiesis} consiste la creación de un mundo fantásticos/ficticios basando en hechos o una referencia. Es el género narrativo en el cual el autor crea todo un conjunto de conceptos, regiones, personajes, sucesos, y arquetipos interrelacionados creando una mitología propia.

Sabemos de los mitos por los Aedos, cantantes populares que creaban canciones con ritos fáciles de memorizar con un bastón, y mitógrafos, redactores de mitos en épocas populares (poetas, filósofos) y recolectores de estos.

Hay autores que crearon sus propios mitos y otros se encargaban de crear un corpus común de las historias.

900-600 a.C. Periodo arcaico de Grecia, anterior a que se pusiera la historia por escrito.

Cubrían la zona de Grecia y Turquía, posteriormente se expandieron, algunos los utilizaban los mitos como inspiración y otros los escribían propiamente.

323 - 31 a.C. Época Helenística, la consolidación escrita de la mitología.

La primera epopeya que se tiene es la Gilgamesh, que se ha conseguido reproducir en la actualidad de la manera que se hacía en la época. Narraciones lentas para remarcar los hechos épicos, con mucha repetición para facilitar que la gente lo recuerde.

Olimpo, el hogar de los dioses.

Se pensaba que los dioses habitaban en lugares inaccesibles, ya sea en las montañas (el Olimpo), cielo, subterráneo, etc. De esta manera están en un lugar que los morales no alcanzaban, en la época no tenían los medios, es por esto que en los caminos había santuarios.

Estos lugares tenían un sentido simbólico, por ejemplo monte Olimpo había varios.

\textbf{Cosmogonía:} Narración mítica o un modelo, que pretende dar respuesta al origen del universo y de la propia humanidad.

\textbf{Teogonía:} Contiene una de las más antiguas versiones del origen del cosmos y el linaje de los dioses de la mitología griega. Habla de la creación del mundo.

Son temas muy recurrente por los aedos, en sus cantos.

Hay muchas versiones del origen del mundo.

Rea y Cronos son los que dan origen a los dioses, por una profecía de que su hijo acabara con él se los come. Rea arta oculta a Zeus en Creta. Un día se traga una piedra y vomita a todos los dioses dándoles origen.

La teogonía de Hesíodo:
\begin{enumerate}
  \item Mundo en contacto, Koiné fenicia (la puesta en común fenicia).
  \item Poeta rural de Beocia. \\ Musas.
  \item Competidor en certámenes de poesía.
  \item Cronología incierta.
  \item Recoge parte del conocimiento de su tiempo.
  \item En representaciones artísticas.
\end{enumerate}

El origen de Venus, se puede ver en El nacimiento de Venus de Botticelli 1482-1485, Afrodita nace de los genitales de Urano cortado por Cronos y al llegar a tierra.

Las grandes luchas: Titanomaquia
\begin{enumerate}
  \item Caída de la primera dinastía.
  \item Dioses primordiales (conceptuales)
  \item Cambios de dinastías de dioses.
  \item El episodio se reproduce con la Gigantomaquia y lucha contra Tifón.
\end{enumerate}

Se utilizaban los mitos para dar nombre a algunas cosas o ciertas cosas inspiraban para crear mitos.
