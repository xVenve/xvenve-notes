\chapter{Clase 10}\label{ch:clase10}
\section{Mitos griegos y el teatro}
Rapto de Perséfone
\begin{itemize}
    \item Rey solitario
    \item Ya en los himnos homéricos
    \item Rapto (violación) y transformación.
    \item Sentido alegórico (orden del mundo)
\end{itemize}

Perséfone es hija de Zeus y Deméter. Para los romanos es Proserpina y como doncella por Kore.

Se la considera la diosa infernal, tras ser raptada por Hades. Su vuelta es el resultado del cambio de estaciones.

Perséfone come 6 granos de una granada, que era el fruto prohibido.

\section{Hades-Plutón}
\begin{itemize}
    \item Dios del Inframundo y los muertos, al principio no era tan terrible, era otro de los dioses tragado por Cronos.
    \item Uno de los tres grandes dioses. Pero sin culto público
    \item Dios ctonico. Da riquezas del suelo: Plutón (significa rico)
    \item Señor de la corte infernal. Había una serie de ríos que delimitaban este lugar que se consideraba el final de la vida, cada río tiene un nombre representativo.
    \item Dioses infernales como Caronte, el barquero del Styx o Laguna Estigia, al que había que hacerle el pago de un óbolo que se le ponía en la boca a los muertos. Si no se podía pagar se vagaba hasta que se apiada de su alma.
    \item También se le puede llamar al Hades, Tántalo u Orco, en el centro de este lugar está la casa de Hades custodiada por Cerbero (perro de 3 cabezas y cola de serpiente). Este animal solo fue burlado 2 veces por Hércules y con la música de Orfeo.
    \item No moraba en el olimpo, tenía su propia corte. Cuando salía del inframundo lleva su casco que le hace invisible, hecho de piel de perro.
    \item El ciprés por su perennidad se asocian con los dioses del inframundo.
\end{itemize}

No es el dios de la muerte, es el dios que guarda a los muertos, el dios de los muertos es Tánatos.

Se les hacía llamadas a estos dioses del inframundo para comunicarse con los difuntos, y se le llama necromancia, pero sin resucitarlos.

\section{Theatron (Teatro griego)}
Dos tipos de celebraciones que se realizaban de maneras muy diferentes, dionisias:
\begin{itemize}
    \item Orgías y ritos públicos, ritos sensuales similares a las bacanales.
    \item Celebraciones de recolección de la vid o de la fertilidad.
\end{itemize}

Del ditirambo a la tragedia popular VII → VI a.C. Fue ganando éxito y los poetas de la época se animaron a crear guiones para ganar concursos. Esto evolucionaría para ser el Theatron.

Hay tres tipos de personajes, el narrado, el personaje principal y los secundarios. Estaban acompañados por una orquesta y se hacían sobre un escenario.

El teatro empezó a representarse en el siglo VI a.C.

Se empieza a cobrar por acceder a estos recintos teatrales y se acude a la gente adinerada para que inviertan para ganar fama y tener buenos sitios.

Evolución: del coro a la escena.

Con él los primeros edificios de piedra, primeras estructuras monumentales no religiosas de la antigüedad.

Tipos de teatro: (que vivieron en paralelo y en honor a Dionisio)
\begin{itemize}
    \item Tragedia, la conocida tragedia griega, el desarrollo del héroe que va evolucionando para tener un final heroico o trágico.
    \item Comedia, situaciones absurdas que ocurrían a los humanos y personajes caricaturescos del momento. Gente que se caía o absurdas, con una representación sexual y fálica.
    \item Satírico
\end{itemize}

También está la tragicomedia, que se hizo muy popular.

Empleaban máscara para representar a los personajes y al comienzo de la obra sacrificaban habitualmente una cabra en honor a Dionisio.

\section{El Ciclo Tebano}
Inspiro muchas obras de las que todavía se conservan, tres poemas épicos que narran la historia de la ciudad de Tebas en la época clásica. Habla de la ascensión del rey Edipo y de las guerras posteriores guerras de sus hijos.

Ejem. Los siete contra Tebas. Aparecen Eteocles y Polinices enfrentándose por Tebas.

Es cuando se introducen en las obras batallas épicas (junto con la guerra de Troya), venganzas y ultranzas. Hablan de héroes sin nombres, centrándose en el sufrimiento de la hermana y los sentimientos de la batalla. Un ejemplo es Antígona, que muestra el deber cívico contra el moral, que quiere enterrar a su hermano Polinices, pero estaba prohibido.

\section{Edipo Rey (Sófocles)}
Transmitida la historia gracias a esta obra, la historia original está perdida.

Es un mito antiguo que se transforma para mostrar a un rey trágico que está al límite y puede llegar a perderlo todo. Layo y Yocasta son avisados de que su hijo acabara con él y se casaría con su madre, tras nacer es llevado a la naturaleza para que muera, pero es rescatado por un pastor que lo lleva al rey de Corinto que no tenía hijos. Cuando crece se sabe que no es su hijo, consulta al oráculo y le dice que se apareara con su madre y derramara la sangre de su padre. Por lo que huye a Tebas.

En el camino a Tebas, Edipo, sin saberlo, mata al rey de Tebas, su padre. Al tirar de su carro.

Hay una esfinge en la puerta de Tebas con un acertijo, que quien lo resuelva podrá entrar, mucha gente muere y deja de recibir personas, pero Edipo es capaz de resolverlo (era el hombre) y se convierte en rey. Enamorándose de su madre.

Al entrar se busca al asesino del anterior rey, pero él no sabe que es él, Tiresias le dice que fue él en el camino. Siguen buscando y se va viendo que los hechos apuntan hacia Edipo, incluso los oráculos, este momento es la catarsis.

Su mujer se suicida por la culpa, cuando la encuentra Edipo, era lo único que él quedaba y era la única que podía resolver sus preguntas, se clava alfileres en los ojos quedándose ciego y dice al pueblo que es el culpable y que le exilien.

Esta obra ha sido muy psicoanalizada por mostrar el límite del hombre.

El complejo de Edipo, el amor por tu madre y odio por tu padre, amor sentimental. Entre otras cosas, también se relaciona con el parricidio.