\chapter{Introducción}\label{ch:introduccion}
\textbf{Lord Byron}, poeta romántico británico, combatió en la Guerra de independencia griega y murió en esta.

La mitología era un parte cultural oral y viva que se usaba para explicar el mundo que les rodea, servían para enseñar. Entender la mitología griega nos puede ayudar a entender el mundo antiguo y su historia, la historia de los museos, edificios históricos o incluso analizar arte del cine o los cuadros.

\textbf{Heinrich Schliemann (1822-1890)} se inspiró en los relatos míticos y participo en los relatos míticos y participó activamente en el descubrimiento arqueológico de Troya y Micenas, lugares de la Ilíada. Dando lugar a que se tuviera más en cuenta los mitos y que la historia no solo se reconstruye con hechos fácticos.

Los historiadores y filósofos consideraban que eran la pura negación de la historia, ya que no tiene verosimilitud y son muy fantásticos, carecen de una lógica formal. Se pensaba que la ciencia tenía que explicar todo, pero no es así las personas somos irracionales que actúan por impulsos y no todo tiene que ser fáctico.

En muchos casos se dan hechos que no son probados, pero que se han mitificado hasta el punto de que todo el mundo piense que es cierto. Es el caso del Cid o la Guerra de Troya.

En principio eran historias sagradas de la tribu que explicaban el mundo mediante relatos de hechos memorables donde dioses y héroes de antaño actuaban. Historias de otro tiempo, de los orígenes del mundo, separado del nuestro mentalmente que por acciones divinas y heroicas en una época primordial cambiaban, lo ordenaban o lo rediseñaban hasta ser el que tenemos hoy.

"Los mitos 'clásicos' representan ya el triunfo de la obra literaria sobre la creencia religiosa" \textbf{(Mircea Eliade, 1994: 166)}

Eran reescritos y modificados por escribas, dando lugar a veces a confusión y prejuicios de los historiadores, dando el significado que ellos piensan. Por ejemplo la eliminación de algunos dioses menores, como el Dios Sol fue eliminado durante el renacimiento por los escribas al pensar que era un dios oriental y no romano, que era una confusión y sustituyéndolo por Júpiter.

"Los mitos viven en el país de la memoria" \textbf{(Marcel Detienne)}

Se transmite de manera oral normalmente de generación en generación, aunque también se hace de manera literaria en instituciones especializadas. Cuando los griegos aprendían a escribir/leer lo primero que aprendían era la Ilíada, era conocido por todos.

Son ejes culturales, tanto políticamente como ligados a las creencias, tienen el mismo objetivo, van de la mano, explican ciertos ritos y acontecimientos.

La cultura occidental no es pagana, bebe de la memoria cultural romana y griega, cultura clásica, pero en muchas zonas se ha expandido y adaptado a las diferentes culturas. Por ejemplo los campos elíseos, como lugar bonito tras la muerte bebe de la cultura egipcia.

\textbf{Mitema (tópico literario):} En el estudio de la mitología es la parte más pequeña que conforma un mito (equivale al fonema de la lingüística), es un elemento constante que siempre aparece intercambiado y reensamblado con otros mitemas o contextos. Son partes estructurales. Puede ser la idea de un viaje de regreso por el mar o un tipo de nacimiento/hecho milagroso (ejem. echado a un río, sumerios con Sargón de Acad o cristianismo con Moisés), el amor, un diluvio (ejem. sumerios y cristianismo). Son buenas ideas, que sirven para articular la obra y que cuando pasan a estar por escrito son tópicos literarios.

El mito se transmite a partir de ideas exitosas, pueden ser mitemas o ideas de contexto generales que gustan, que se usan para formar nuevas palabras o seres fantásticos como quimera que significaba cabra y se interpreta como una bestia ceremonial. Por ejemplo capricornio parte del calendario. Pasa con las sirenas, grifos, etc.

Los mitos eran parte de la cultura viva e iban evolucionando constantemente, hay que tener muy presente que tienen una larga historia y variaciones. Tienen muchas raíces y es difícil llegar hasta el original, diferenciando obras literarias, teatrales o religiosas.

Hay muchos animales a los que se les dota de significado, como el Pegaso, Minotauro, Hidra, Paloma, León, Toro, Esfinge, etc.

El mito es fundamentalmente la historia oral, más que la escrita y literaria.