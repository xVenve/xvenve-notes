\chapter{Clase 8}\label{ch:clase8}
\section{Rómulo y Remo}
Es uno de los clásicos mitos fundacionales, que tiene origen con Alba longa fundada por Eneas. En Alba longa reinaron dos hermanos, para que solo reinara otros le expulsa.

Son los hijos gemelos de Rea Silvia y Marte (descendientes directos de Eneas) que fueron rescatados en las orillas del río Tiber (zona pantanosa) por una loba (loba capitolina/luperca), que se los llevo, como esta había perdido sus lobeznos, los acoge y los amamanta como a sus propios hijos. Se dice que originalmente podrían haber sido por una prostituta dado que la palabra significa lo mismo que loba.

Los orígenes de Romas son el fango y la loba.

Un pastor, Lumitor, les explica su origen de alta cuna y estos volverán a Alba longa para vengarse junto a un conjunto de pastores. Una vez con esta venganza se van para fundar su propia ciudad, su localización seria donde la loba los rescata. Como no se ponen de acuerdo, el que viera más buitres decidiría el nombre.

En una pelea entre los hermanos acaba muriendo Remo, que fue enterrado en el punto de origen de la ciudad de Roma. Fue ayudado por los patricios.

El rapto de las Sabinas, como había pocas mujeres para poblar roma, cuando los sabinos vinieron a la ciudad raptaron a sus mujeres. Con el tiempo las mujeres raptadas se adaptan y se les da ciudadanía, como uno más, teniendo a sus propios hijos y maridos. Hubo una gran lucha entre los romanos y los sabinos, pero fue detenida por estas mujeres dado que ganara quien ganara perderían (morirían sus maridos o sus familiares).

Cuando Rómulo muere comienza la monarquía, a partir de aquí hay varias versiones, alguna de ellas dice que es matado por los patricios y otra que asciende a los dioses.

\section{Marte - Ares}
\begin{itemize}
    \item Dios de la Guerra
    \item Papel menor en Grecia, naturaleza distinta en Roma
    \item Relación con Venus (Afrodita)
    \item Unido a Bellona (diosa de la guerra) en Roma, en la mitología griega no tiene una mujer asociada
    \item Hijos varios, destacan Fobos y Dimos
    \item Se le considera muy importante por su vigor bélico, también como dios que ayuda a Rómulo a ascender
    \item Sus símbolos son la lanza y el fuego
\end{itemize}

\section{Los dioses de la Luz}
\subsection{Apolo}
Es joven y representaba la belleza, masculinidad y es la luz sin ser Helios. Representa también la sanación, asociándosele peanos (un peán). Se le asocia como animal el ratón y como instrumento la lira, que le es entregada por un dios.
Hijo de Zeus y Leto (titanide), de ahí nace Apolo y Artemisa. Fue un parto muy duro al no contar con la presencia de Hera y la representación del parto.

Dafne significa laurel, que se usaba como símbolo de la victoria.

\subsection{Delfos}
\begin{itemize}
    \item Es el templo más importante de la griega occidental
    \item Templo de Apolo en Delfos, sede del Oráculo
    \item Pitonisa daba oráculos sobre el trípode. Las pitonisas mascaban laurel y daban repuesta a las cuestiones que les plantean
    \item Delfos era el ombligo religioso y geográfico del mundo griego (ónfalos), Apolo lanza dos buitres opuestos y se cruzan ahí
\end{itemize}

Zeus y Apolo son los principales dioses oraculares.

\subsection{Apolo y las Artes}
Las musas son divinidades inspiradoras de las artes: cada una de ellas está relacionada con ramas artísticas y del conocimiento. Son hijas de Zeus y de Mnemósine, compañeras del séquito de Apolo.
\begin{enumerate}
    \item Calíope - elocuencia y poesía explica
    \item Clío - historia
    \item Erató - poesía
    \item Euterpe - música
    \item Melpomene - tragedia
    \item Polimnia - himno
    \item Talía - comedia
    \item Terpsicore - danza
    \item Urania - astronomía, didáctica y las ciencias (ciencias exactas)
\end{enumerate}

\subsection{Helios (Sol)}
Es un titán más que un dios, es hijo de Hiperion y Eurifae (Tea). Es el grecolatino Sol. Se caracteriza por su carro, cuadriga solar, que le permitía moverlo por el cielo.
 
Suele aparecer con un limbo en la cabeza o luz, que será utilizada también por el cristianismo. Ese halo, además de estar en la divinidad cristiana, deriva en la corona que se utiliza en la monarquía.

Da origen a muchos personajes que hemos visto, como a Circe o Medeas, que les da poder para poder, en caso de Circe para convertir en cerdos. También Pasifae la madre del minotauro de Minos y Serena/Luna. Otro destacado son las Horas (4 para estaciones otras veces por 12 por las horas de día y de noche), aunque muchas veces son atribuidas a Zeus, así como las Tres Gracias.

\section{Mitraísmo}
Culto que tiene su propia mitología copiada del culto Persa. Narran en unos relatos una alianza entre Mitra y Sol para derrotar al toro cósmico. Hay diferentes niveles de pertenencia a este culto, desde iniciado hasta páter.

\section{Faetón}
\begin{itemize}
    \item Mito trágico del recorrido solar
    \item Evolución del mito y su estructura
    \item Mito moralista (fábula)
\end{itemize}
Joven, bello, fuerte y apuesto. Se burlaban de él diciendo que no era hijo de Helios, pero su madre le dice que es cierto, que vaya hasta el final mundo para verle. Faetón le pide una prueba de paternidad, Helios dice que le dará lo que sea, entonces le deja montar en la cuadriga. Se arrepiente, pero como se lo promete le deja advirtiéndole, ni los dioses querían conducirla.

Faetón lo monta y pierde el descontrol, subiendo demasiado revolviendo las galaxias, entonces baja demasiado y quema la superficie de la Tierra, secando el suelo, ríos y quemando a los etiopes (volviéndose negros). Por este descontrol, los dioses le tiran un rayo para sacarle de la cuadriga y muere al caer. Entonces es enterrado por sus hermanas, la heliades  que se convierten en álamos.

Al día siguiente, Helios no salió y hubo un día de oscuridad (eclipse).