\chapter{Modelos de Computación}\label{ch:modelos-de-computación}
Modelos que se utilizan para resolver problemas.
\section{Máquinas de Turing}
\subsection{Origen}
Formulada por Turing como un medio para formalizar el proceso sistemático de resolver problemas de un matemático.

En sentido más amplio: Para representar cualquier Algoritmo, por precaución hablamos de Procedimientos Efectivos para resolver un problema.

Actualmente, es una máquina evolucionada en base a contribuciones de Post, Turing y otros.

\subsection{Utilidad}
Turing se sirvió de ella para desarrollar diversas cuestiones: 
\begin{itemize}
    \item Máquina de Turing Universal.
    \item Problema de la parada (demostración computacional) en equivalencia con el Entscheidungsproblem de Gödel.
    \item Modelo Universal de Computación.
\end{itemize}

\subsection{Modelo de Computación Universal}
Las Máquinas de Turing definen un Modelo de Computación que es la base de los Lenguajes Imperativos y Procedimentales.

Suposiciones previas:
\begin{itemize}
    \item MT de cinta infinita
    \item Alfabeto$=\{b,1\}$ siendo b o $\#$ representan celda en blanco
    \item Alfabeto auxiliar con símbolos de marcado $\{\$, X, etc\}$
    \item Se representan Números Naturales en base 1
\end{itemize}

\subsubsection{Aritmética de Peano} 
Conjunto axiomático que sirve para definir los Números naturales
\begin{itemize}
    \item R0 - Objeto inicial. 0 es un objeto especial y es un Número Natural.
    \item R1 - Sucesor. Para cada n NN existe otro NN $succ(n)$.
    \item R2 - Predecesor. Si a es $succ(b)$ entonces b es $pred(a)$.
    
    Ojo con $pred(k=1)$, no está definido si no incluimos 0
    \item R3 - Única. No hay dos NN diferentes con el mismo sucesor.
    \item R4 - Igualdad. Comparar a y b NN en cuanto a igualdad.
    \begin{itemize}
        \item Reflexiva $Eq(a,a)$
        \item Simétrica $Eq(a,b) \Leftrightarrow Eq(b,a)$
        \item Transitiva $Eq(a,b), Eq(b,c) \Leftrightarrow Eq(a,c)$
    \end{itemize}
    \item R5 - Inducción. Predicado $P(n)$ es cierto $\forall n$ si cumple: 
    
    $P(0)$ cierta.
    
    $P(n)$ cierta para algún n y se puede demostrar que $P(succ(n))$ es cierta.
\end{itemize}

\subsection{Eficiencia}
No, no busca la eficiencia.

La MT es un modelo de computación abstracta.

Igual sabe hacer cosas que podríamos calcular de otra manera. No obstante, vamos a estudiar la eficiencia $\Rightarrow$ Coste Computacional.

\subsection{Coste Computacional con MT}
Uso para el Análisis del Coste Computacional de Algoritmos.

\subsubsection{Ventajas}
\begin{itemize}
    \item Propone un modelo único, al contrario que con el uso de un Lenguaje X y una CPU Y
    \item No hay dudas sobre el coste del Paso Base
\end{itemize}

\subsubsection{Desventaja}
\begin{itemize}
    \item Es muy pesado para algoritmos complejos $\Rightarrow$ Limitaremos el tamaño de las MT
\end{itemize}

\subsubsection{Propósito}
\begin{itemize}
    \item Relacionar conceptos vistos en la asignatura.
    \item En base a un modelo muy sencillo
    \item Abstraer de construcciones de muy alto nivel
\end{itemize}

\subsubsection{Marco necesario para el Análisis}
\begin{itemize}
    \item MT de cinta infinita hacia ambos lados
    \item Alfabeto determinado, en principio con pocos símbolos, pero pueden ser más
    \item Una instancia de un Problema se describirá con una cinta de Entrada
    \item El número de símbolos de la Entrada define el Tamaño de la Instancia
    \item Coste de Paso Base: Una Transición con su Lectura/Escritura y Desplazamiento
\end{itemize}

\subsubsection{Coste computacional empírico: Aplicando diferencias finitas}
\begin{table}[H]
    \begin{tabular}{|l|l|l|l|l|l|l|}
    \hline
    Tamaño n                  & \textbf{1} & \textbf{3} & \textbf{5} & \textbf{7} & \textbf{9} & \textbf{11} \\ \hline
    \textbf{Pasos}            & 2          & 9          & 20         & 35         & 54         & 77          \\ \hline
    \textbf{A: T(n) - T(n-2)} &            & 7          & 11         & 15         & 19         & 23          \\ \hline
    \textbf{B: A(n) - A(n-2)} &            &            & 4          & 4          & 4          & 4           \\ \hline
    \textbf{C: B(n) - B(n-2)} &            &            &            & 0          & 0          & 0           \\ \hline
    \end{tabular}
    \caption{Diferencias finitas}
\end{table}

Si $T(n)$ \textbf{exponencial}, las diferencias finitas no se anulan, pero aparecen comportamientos particulares.

Si $T(n)$ \textbf{polinómico} de orden k, las diferencias finitas de orden $k=cte$. En este caso orden 2.

Sabiendo el orden se plantea un polinomio genérico de ese orden: $T(n)=xn^2+bn+c$.

Se resuelve la ecuación y ya tenemos la complejidad.

$$a+b+c=2;\; 9a+3b+c=9;\; 25a+5b+c=20 \Rightarrow a=\frac{1}{2},\; b=\frac{3}{2},\; c=0$$

$$T(n)=\frac{1}{2}n^2+\frac{3}{2}n$$

\subsubsection{Coste computacional empírico: Recursivamente}
Se plantean las observaciones y observan los patrones de los datos.

\begin{itemize}
    \item $n=1 \Rightarrow 2$ pasos
    \item $n=3 \Rightarrow 4$ pasos a derecha $+ 3$ a izquierda $+ T(1)$
    \item $n=5 \Rightarrow 6$ pasos a derecha $+ 6$ a izquierda $+ T(3)$
    \item $n \Rightarrow n+1$ pasos a derecha $+ n+1$ a izquierda $+ T(n-2) = \Sigma i \Rightarrow T(n)=\frac{(n+1)(n+2)}{2}-1=\frac{1}{2}n^2+\frac{3}{2}n$ 
\end{itemize}

\subsection{Variantes del Modelo de Máquina de Turing}
Abordamos:
\begin{itemize}
    \item Máquinas de Turing de dos o más cintas (multicinta)
    \item Máquinas de Turing No Deterministas
    \item Automata de Dos Pilas
    \item Máquinas de Turing 2D
    \item Aplicaciones
    \item Costes Computacionales
\end{itemize}

\subsubsection{Máquinas de Turing de Dos Cintas}
La misma estructura que una MT convencional, pero:
\begin{itemize}
    \item Tiene dos cintas y sus respectivos cabezales
    \item Pueden operar de forma independiente
\end{itemize}

Implica cambios en el funcionamiento, las transiciones se definen:
\begin{itemize}
    \item Cambio de estado
    \item Para cabezales
    \begin{itemize}
        \item Una Lectura
        \item Una Escritura
        \item Un Desplazamiento
    \end{itemize}
\end{itemize}

Las operaciones de cada cabezal son independientes entre sí, pero están vinculados en cada transición.

\subsubsection{Máquinas de Turing No Deterministas}
La misma estructura que una MT convencional, pero
\begin{itemize}
    \item Admiten transiciones No Deterministas, es decir, para un mismo símbolo leído define al menos dos transiciones que:
    \begin{itemize}
        \item Escriben símbolos diferentes
        \item Desplazan el cabezal de distinta forma
        \item Transitan a estados diferentes
    \end{itemize}
\end{itemize}

Cuando se encuentra una transición ND, entendemos que la MT bifurca en dos o más instancias de la misma que adoptaran estados diferentes.

Si alguna de las instancias alcanza un Estado Final $\Rightarrow$ MT acepta