\chapter{Análisis de la complejidad de los Algoritmos}\label{ch:análisis-de-la-complejidad-de-los-algoritmos}
\adjustbox{\textwidth,center}{%
    \begin{tikzcd}
                                   &                                                             & \text{Complejidad Computacional} \arrow[d] &                                       \\
                                   &                                                             & \text{tipos} \arrow[rd] \arrow[ld]         &                                       \\
                                   & \text{Complejidad temporal} \arrow[d]                       &                                            & \text{Complejidad espacial} \arrow[d] \\
                                   & S(n) \arrow[d]                                              &                                            & T(n)                                  \\
                                   & \text{Número de pasos base} \arrow[ld] \arrow[d] \arrow[rd] &                                            &                                       \\
\text{Máquina de Turing} \arrow[d] & \text{Ensamblador} \arrow[d]                                & \text{Lenguaje alto nivel*} \arrow[d]      &                                       \\
\text{transición}                  & \text{instrucción}                                          & \text{operación elemental} \arrow[d]       &                                       \\
                                   &                                                             & \text{OE}                                  &
\end{tikzcd}
}
* Estructuras de los algoritmos siempre se pueden considerar como estructuras de tipo Secuencial, Iterativo y Condicional, esto determina la operación elemental.

\section{Introducción}\label{sec:introduccion}
\textbf{Complejidad Computacional:} Recursos requeridos durante la ejecución de un algoritmo que da respuesta a un problema. Hay 2 tipos:
\begin{itemize}
  \item Tiempo: Número de pasos base de ejecución de un algoritmo para resolver un problema.
  \item Espacio: Memoria utilizada para resolver un problema. Registros, Memoria caché y Memoria RAM, etc.
\end{itemize}
Espacial y Temporal.

\textbf{Problema:} Una función entre el espacio de instancias y el espacio de respuestas, $X \rightarrow P(X)$.
\begin{itemize}
  \item Ejem. La ecuación de grado 2 tiene raíces cuadradas? $P(x^2+1=0)=\text{No}$
\end{itemize}

\textbf{Problema de decisión:} Tiene 2 respuestas: Si o No.

\textbf{Instancia de un problema:} Es la especificación exacta de los datos de un problema para un caso particular.
\begin{itemize}
  \item \textbf{Tamaño de instancia:} Número de datos de la instancia.
  \item Ejem. Independiente, $x^2$ y $x$ son los 3 parámetros de una instancia de un problema de ecuación de segundo grado.
\end{itemize}

\textbf{Algoritmo:} Conjunto de instrucciones que garantiza encontrar una solución correcta para cualquier instancia en un número finito de pasos.

La complejidad puede hacer referencia a un algoritmo o un problema.
\begin{itemize}
  \item \textbf{Algoritmo:} Resolver las instancias de un problema. \\ Medida de número de pasos base requeridos para la PEOR instancia de tamaño n. \\ Relación entre n y el número de pasos.
  \item \textbf{Problema:} Es el objeto de estudio de la Teoría de la Complejidad Computacional.
\end{itemize}

\section{Paso Base}\label{sec:paso-base}
Se pueden modelar los algoritmos mediante:
\begin{enumerate}
    \item Construcción matemática denominada Máquina de Turing $\rightarrow$ Transición / Tabla de transiciones.
  \item Código Ensamblador $\rightarrow$ Instrucción \ Conjunto estructurado de instrucciones.
  \item Lenguaje de alto nivel $\rightarrow$ Operación Elemental / Conjunto estructurado de instrucciones.
    \begin{itemize}
      \item Suma, Restar División, Multiplicación, etc.
    \end{itemize}
\end{enumerate}

\section{Complejidad de los Algoritmos}\label{sec:complejidad-de-los-algoritmos}
Se expresa como: $T(n)$, número de pasos peor caso con entrada n.

La complejidad computacional puede referirse también a los recursos de espacio necesarios según el tamaño de la entrada. Así tenemos que:
\begin{itemize}
  \item Tiempo de ejecución (complejidad temporal) $T(n)$
  \item Espacio necesario para los datos (complejidad espacial) $S(n)$
\end{itemize}

Todo algoritmo puede ser implementado con 3 estructuras: Secuencial, Condicional e Iterativa.

\textbf{Operación Elemental (OE)}
\begin{itemize}
  \item Comparación
  \item Asignación variable
  \item Acceso a estructura básica
  \item Llamada función
  \item Retorno función
\end{itemize}

\textbf{Estructura Secuencial:} Cada I es una instancia o bloque, se calcula:
\begin{itemize}
  \item El de todos: $T(I_1; I_2; \dots; I_n) = T(I_1) + T(I_2) + \dots + T(I_n)$
  \item General: $T= \max \{T(I_1), T(I_2), \dots, T(I_n)\}$ El T es el más alto, el del peor caso/bloque.
  \item Para los programas es la suma de todas las estructuras: $T = T_1 + T_2 + \dots + T_k = \max \{ T_1, T_2, \dots, T_k \}$
\end{itemize}

\textbf{Estructura Condicional:}
\begin{itemize}
  \item if (Condición) then BloqueThen else BloqueElse
  \item $T = T(Condicion) + \max \{T(BloqueThen), T(BloqueElse)\}$
\end{itemize}

\textbf{Estructura Iterativa:}
\begin{itemize}
  \item while(C) S - $T = C + n_{iter} (C+S)$
  \item for(A C B) S - $T = A + C + n_{iter} (C+B+S)$
  \begin{itemize}
    \item A: Inicialización
    \item C: Condición
    \item B: Incremento
    \item S: Bloque
  \end{itemize}
\end{itemize}
\pagebreak

\textbf{Llamada a Procedimiento, Función o Método:}
\begin{itemize}
  \item $funcion(p_1, p_2, \dots, p_k) f$
  \item 1 (por la llamada) + tiempo de evaluación parámetros + tiempo que tarde en ejecutarse el ''cuerpo'' de la función.
  \item $T = 1 + T(p_1) + T(p_2) + \dots + T(p_n) + T(f)$
\end{itemize}

No se contabiliza la copia de los argumentos a la pila de ejecución salvo estructuras complejas (vector, registros) que se pasen por valor.

El paso por referencia y paso de punteros, no se contabiliza.

\subsection{Ejemplo}\label{subsec:ejemplo}
\begin{lstlisting}[language=Java,label={lst:lstlisting}]
for (int i = 0, acum = 0; i < n; i++)
  acum += i;
\end{lstlisting}

$T(n) = A + C + n ( C + B + S ) = 2 + 1 + n ( 1 + 2 + 2 ) = 3 + 5 n$

\begin{lstlisting}[language=Java,label={lst:lstlisting2}]
if (x < 5)
  for (int i = 0; i < n; i++)
    acum += i;
else
  acum = 3;
\end{lstlisting}

$T(n) = 1 + \max \{ 1 + 1 + n(1+2+2) ; 1 \} = 1 + \max \{ 2 + 5 n ; 1 \} = 3 + 5n$

\begin{lstlisting}[language=Java,label={lst:lstlisting3}]
for (int i = 0, acum = 0; i < n; i++)
  for (int j = 0; j < i/2; j++)
    acum += i;
\end{lstlisting}

$T(n) = 2 + 1 + n ( 1 + 2 + \[ 1 + 2 + \frac{n}{2} ( 2 + 2 + 2) \]) = 3 + n (3 + 3 + 3 n) = 3 + 6 n + 3 n^2$

\begin{lstlisting}[language=Java,label={lst:lstlisting4}]
x = 5;
while (x < N)
  x = 3 * x;
\end{lstlisting}

$T(n) = 1 + 1 + p ( 1 + 2 ) = 2 + 3 p$

$5 \cdot 3^p < n; p \log 3 < \log \frac{n}{7}; p = \frac{\log \frac{n}{5}}{\log 3}$

\section{Algoritmos Recursivos}\label{sec:algoritmos-recursivos}
El coste (T(n)) de un algoritmo recursivo sera recursivo.

$T(n)= E(n)$ En la expresión E aparece la propia función T.

El objetivo es encontrar $T(n)= f(n)$, sin funciones t

Algoritmo recursivo
| Tiene coste T(n)
v
Ecuaciones de recurrencia E(n)
| - > Cmabio de variable
v
Ecuacion caracteristica (Polinomica P(n)=0)
|
v
Obtenemos $x_1, x_2, x_3, ..., x_n$
|
v
Obtenemos de T(n) no recursiva

Para resolver E(n) tambien podemos recurrir a Despliegue (despliegue: T(n) no recursiva).

Dos metodos:
  Ecuación caracteritica.
  Despliegue.]


Despligue de Recurrencia

Aplicar varias veces la formaula recurrente hasta obtener una formula general que relaciona la función para el tamano original con otros tamano menores. A partir de esta formula se obtiene...

No debe aplicarse este metodo si aparecen varios terminos recurrentes, como es el caso de la serie de Fibonacci.

Se parte de un caso base y se va sustituyendo dentro de la ecuación el propio termino, de manera que cuando vamos sustituyendo y resolviendo la ecuación hasta llegar a un caso en que podamos aplicar el caso base. En ese punto podremos sustituir nuestro caso base en la ecuación y quitarnoos completamente la recurrencia, t().

Lo que buscamos es que al desdoblar los terminos constantemente encontremos una regularidad. En ese momento hacemos que una varible valga un valor que podamos utilizar para usar el caso base, normalmente t(1).



Resolución General

Expresión en n

Ecuación de Recurrencia: $c_n t(n)+ c_{n-1}+ ... + c_{n-k} +- b$

Si b=0, se dice que es Ecuación de Recurrencia Lineal (de orden k) HOMOGENEA.

Los terminos de las ecuaciónes de recurrencia estan afectados por coeficientes.

Solo se trataran ecuaciones de recurrencia lineales con coeficientes constantes.



Ecuacion de Recurrencia Lineal Homogenea (ERLH)

Metodo de Ecacuión Caracteristica

A una ecuación de recurrencia de orden k, mediante un cmabio de variable, se le acocia una ecuación polinomica de grado k (la ecuación caracteristica)

Las soluciones ...

Se cambia el t(n), t(n-1), etc. por variables x que dependan del n.

El grado de la variable siempre es un grado menos que el numero de terminos recurrentes y cada uno de valor menor reduce el grado.

$T(n)=T(n-1)+T(n-1) \leftarrow x^2= x + 1$

Sus raices son $r = \frac{1+- \sqrt{5}}{2}$ esto terminos se sustituirian en las ecuaciones de arriba quedando:

Sabiendo $T(0) = 0$ y $T(1)=1$
$T(n)= b_1(\frac{1+ \sqrt{5}}{2})^n + b_2(\frac{1- \sqrt{5}}{2})^n$

$T(0)= b_1(\frac{1+ \sqrt{5}}{2})^0 + b_2(\frac{1- \sqrt{5}}{2})^0 = b_1 + b_2$

$T(1)= b_1(\frac{1+ \sqrt{5}}{2})^1 + b_2(\frac{1- \sqrt{5}}{2})^1 = 1$

Despejamos con ambas ecuaciones: $b_1 = -b_2 = \frac{1}{5}$.


Lo que hacemos tras la sustitución es obtener las raices, que pueden ser distintas x=2 y x=3 o iguales x=1 [(x-1)(x-1)].
\begin{itemize}
  \item Para las raics distintas se aplica: T(n) =
  \item Una raiz con multiplicidad mayor que 1: T(n) =
\end{itemize}





\section{Tratabilidad}\label{sec:tratabilidad}


\section{Análisis Asintótico}\label{sec:análisis-asintótico}