\chapter{Fractales}
Figuras que se autorrepiten, es decir, la figura está formada de sí misma.

Esta estructura se da en la naturaleza, en los vasos sanguíneos, cuenca del Amazonas o las vías romanas. Se pueden dar en forma de ramificación, en espiral o en conos.
\begin{itemize}
    \item Ejem. Cálculo de la longitud de la costa de Gran Bretaña, que se trata de estimar con un segmento de x metros, cuanto más pequeño lo vayamos haciendo más grande sale la longitud. Por lo que siempre hay un segmento más pequeño. 

\end{itemize}

\section{Definición}
El concepto de fractal aparece en 1975 con el libro Benoit Mandelbrot - ''Los objetos Fractales: forma, azar y dimensión''.

Aunque se habían descubierto antes y se llamaron ''monstruos matemáticos''

Un objeto fractal tiene dos características básicas:
\begin{itemize}
    \item \textbf{Infinito detalle} en cada punto 
    \item \textbf{Auto similitud entre las partes} del objeto y su totalidad
\end{itemize}
Veremos \textbf{siempre la misma figura}, independientemente de lo que nos acerquemos al mismo (fractal auténtico).

Su geometría requiere \textbf{gráficos por ordenador} para visualizarse e investigarlos.

\section{Caracterización}
\begin{table}[H]
    \begin{tabular}{|c|c|}
    \hline
    \rowcolor[HTML]{BFBFBF} 
    \begin{tabular}[c]{@{}c@{}}Geometría\\ Euclídea\end{tabular}                                   & \begin{tabular}[c]{@{}c@{}}Geometría\\ Fractal\end{tabular}                         \\ \hline
    Ecuaciones                                                                                     & \begin{tabular}[c]{@{}c@{}}Procedimientos\\ (sin ecuaciones)\end{tabular}           \\ \hline
    Objetos fabricados                                                                             & Objetos naturales                                                                   \\ \hline
    \begin{tabular}[c]{@{}c@{}}Diferenciable,\\ localmente suave\end{tabular}                      & \begin{tabular}[c]{@{}c@{}}No diferenciable,\\ localmente rugoso\end{tabular}       \\ \hline
    \begin{tabular}[c]{@{}c@{}}Aumenta el detalle\\ y definición a \\ distancia corta\end{tabular} & \begin{tabular}[c]{@{}c@{}}Mismo detalle desde\\  todas las distancias\end{tabular} \\ \hline
    \end{tabular}
    \caption{Comparación Geometría Fractal y Euclídea}
\end{table}

\section{Aplicaciones}
\begin{itemize}
    \item Antenas fractales (ancho de banda)
    \item Compresión de imágenes
    \item Sistemas dinámicos (cotización de valores bursátiles)
    \item Modelado de formas naturales (helechos, montañas o relieve)
    \item Arte fractal.
\end{itemize}

\section{Procedimiento de Generación}
Aplicar recursivamente una función de transformación a los puntos de una región del espacio:
\begin{itemize}
    \item Tenemos una función de transformación $F(X)$ y partimos de punto inicial seleccionado $P_0 =(x_0 , y_0 , z_0 )$
    \item Cada vez que aplicamos F, generamos niveles sucesivos de detalle, como $P_1 =F(P_0 ), P_2 =F(P_1 ), ... P_{k+1} =F(P_k ), ...$
    \item Sistema dinámico realimentado
\end{itemize} 

$F(X)$ puede aplicarse a: Conjunto específico de puntos, o a un Conjunto inicial de primitivas (líneas rectas, áreas de color, superficies o sólidos). 

$F(X)$ puede definirse en términos de transformaciones: Geométricas (escalado, traslación, rotación) o Coordenadas no lineales y parámetros de decisión.

No podemos desplegar variaciones de detalle inferiores a un pixel $\rightarrow$ $F(X)$ se aplica un número finito de veces. Para ver más detalle acercamos la imagen y aplicamos $F(X)$ de nuevo.

Los procedimientos a utilizar pueden ser Deterministas o Aleatorios / estocásticos.

\section{Clasificación de Fractales}
\begin{itemize}
    \item \textbf{Autosimilares:} Sus partes son versiones a escala de su totalidad. Se aplica un parámetro de escala (S) toda la forma inicial.
    
    \textbf{Autosimilar estadísticamente} si aplicamos variaciones aleatorias.
    \item \textbf{Autoafines:} Un factor de escala en cada eje, $S_x, S_y, S_z$.
    
    \textbf{Autoafines estadísticamente} si aplicamos variaciones aleatorias.
    \item \textbf{Invariantes:} Transformaciones no lineales. Fractales autocuadráticos (Mandelbrot), autoinversos...
\end{itemize}

\section{Dimensión Fractal}
Medida de la variación de detalle de un objeto fractal.

En la geometría euclídea los objetos tienen dimensión entera. Los objetos en la geometría fractal tiene dimensión fraccionaria (o dimensión fractal).

La dimensión fractal de un objeto siempre es mayor que su dimensión euclídea (o topológica).

Dada una dimensión, podemos generar objetos fractales mediante procedimientos recursivos.

Podríamos calcular la dimensión fractal de un objeto a partir de las propiedades del mismo, mediante otros procedimientos (tarea difícil de realizar).

Cálculo de la dimensión de un fractal autosimilar determinista (un solo factor de escala) por analogía a las subpartes de un objeto euclidiano.

$NS^{DE}=1 \rightarrow NS^{D}=1$
\begin{itemize}
    \item DE = Dimensión euclídea
    \item D = Dimensión fractal
    \item N = n. piezas semejantes
    \item S = factor de escala
\end{itemize}
$$N=\frac{1}{S^D}=\left(\frac{1}{S}\right)^D; \ln N = D \ln \frac{1}{S}; D = \frac{\ln N}{\ln \frac{1}{S}}$$