\chapter{Fractales}
Figuras que se autorrepiten, es decir, la figura está formada de sí misma.

Esta estructura se da en la naturaleza, en los vasos sanguíneos, cuenca del Amazonas o las vías romanas. Se pueden dar en forma de ramificación, en espiral o en conos.
\begin{itemize}
    \item Ejem. Cálculo de la longitud de la costa de Gran Bretaña, que se trata de estimar con un segmento de x metros, cuanto más pequeño lo vayamos haciendo más grande sale la longitud. Por lo que siempre hay un segmento más pequeño. 

\end{itemize}

\section{Definición}
El concepto de fractal aparece en 1975 con el libro Benoit Mandelbrot - ''Los objetos Fractales: forma, azar y dimensión''.

Aunque se habían descubierto antes y se llamaron ''monstruos matemáticos''

Un objeto fractal tiene dos características básicas:
\begin{itemize}
    \item \textbf{Infinito detalle} en cada punto 
    \item \textbf{Auto similitud entre las partes} del objeto y su totalidad
\end{itemize}
Veremos \textbf{siempre la misma figura}, independientemente de lo que nos acerquemos al mismo (fractal auténtico).

Su geometría requiere \textbf{gráficos por ordenador} para visualizarse e investigarlos.

\section{Caracterización}
\begin{table}[H]
    \begin{tabular}{|c|c|}
    \hline
    \rowcolor[HTML]{BFBFBF} 
    \begin{tabular}[c]{@{}c@{}}Geometría\\ Euclídea\end{tabular}                                   & \begin{tabular}[c]{@{}c@{}}Geometría\\ Fractal\end{tabular}                         \\ \hline
    Ecuaciones                                                                                     & \begin{tabular}[c]{@{}c@{}}Procedimientos\\ (sin ecuaciones)\end{tabular}           \\ \hline
    Objetos fabricados                                                                             & Objetos naturales                                                                   \\ \hline
    \begin{tabular}[c]{@{}c@{}}Diferenciable,\\ localmente suave\end{tabular}                      & \begin{tabular}[c]{@{}c@{}}No diferenciable,\\ localmente rugoso\end{tabular}       \\ \hline
    \begin{tabular}[c]{@{}c@{}}Aumenta el detalle\\ y definición a \\ distancia corta\end{tabular} & \begin{tabular}[c]{@{}c@{}}Mismo detalle desde\\  todas las distancias\end{tabular} \\ \hline
    \end{tabular}
    \caption{Comparación Geometría Fractal y Euclídea}
\end{table}

\section{Aplicaciones}
\begin{itemize}
    \item Antenas fractales (ancho de banda)
    \item Compresión de imágenes
    \item Sistemas dinámicos (cotización de valores bursátiles)
    \item Modelado de formas naturales (helechos, montañas o relieve)
    \item Arte fractal.
\end{itemize}

\section{Procedimiento de Generación}
Aplicar recursivamente una función de transformación a los puntos de una región del espacio:
\begin{itemize}
    \item Tenemos una función de transformación $F(X)$ y partimos de punto inicial seleccionado $P_0 =(x_0 , y_0 , z_0 )$
    \item Cada vez que aplicamos F, generamos niveles sucesivos de detalle, como $P_1 =F(P_0 ), P_2 =F(P_1 ), ... P_{k+1} =F(P_k ), ...$
    \item Sistema dinámico realimentado
\end{itemize} 

$F(X)$ puede aplicarse a: Conjunto específico de puntos, o a un Conjunto inicial de primitivas (líneas rectas, áreas de color, superficies o sólidos). 

$F(X)$ puede definirse en términos de transformaciones: Geométricas (escalado, traslación, rotación) o Coordenadas no lineales y parámetros de decisión.

No podemos desplegar variaciones de detalle inferiores a un pixel $\rightarrow$ $F(X)$ se aplica un número finito de veces. Para ver más detalle acercamos la imagen y aplicamos $F(X)$ de nuevo.

Los procedimientos a utilizar pueden ser Deterministas o Aleatorios / estocásticos.

\section{Clasificación de Fractales}
\begin{itemize}
    \item \textbf{Autosimilares:} Sus partes son versiones a escala de su totalidad. Se aplica un parámetro de escala (S) toda la forma inicial.
    
    \textbf{Autosimilar estadísticamente} si aplicamos variaciones aleatorias.
    \item \textbf{Autoafines:} Un factor de escala en cada eje, $S_x, S_y, S_z$.
    
    \textbf{Autoafines estadísticamente} si aplicamos variaciones aleatorias.
    \item \textbf{Invariantes:} Transformaciones no lineales. Fractales autocuadráticos (Mandelbrot), autoinversos...
\end{itemize}

\section{Dimensión Fractal}
Medida de la variación de detalle de un objeto fractal.

En la geometría euclídea los objetos tienen dimensión entera. Los objetos en la geometría fractal tiene dimensión fraccionaria (o dimensión fractal).

La dimensión fractal de un objeto siempre es mayor que su dimensión euclídea (o topológica).

Dada una dimensión, podemos generar objetos fractales mediante procedimientos recursivos.

Podríamos calcular la dimensión fractal de un objeto a partir de las propiedades del mismo, mediante otros procedimientos (tarea difícil de realizar).

Cálculo de la dimensión de un fractal autosimilar determinista (un solo factor de escala) por analogía a las subpartes de un objeto euclidiano.

$NS^{DE}=1 \rightarrow NS^{D}=1$
\begin{itemize}
    \item DE = Dimensión euclídea
    \item D = Dimensión fractal
    \item N = n. piezas semejantes
    \item S = factor de escala
\end{itemize}
$$N=\frac{1}{S^D}=\left(\frac{1}{S}\right)^D; \ln N = D \ln \frac{1}{S}; D = \frac{\ln N}{\ln \frac{1}{S}}$$

Dimensión de una \textbf{curva fractal en un plano 2-dim} (litorales)
\begin{itemize}
    \item Habitualmente $1<D \leq 2$ (más suave cuanto más cerca de 1). 
    
    $D = 2$ cuando llena una región finita del plano
    \item $2<D<3$ cuando la curva se autointersecta y el área puede cubrirse infinitas veces
\end{itemize}

Dimensión de una \textbf{curva fractal espacial}
\begin{itemize}
    \item En general $1 < D$ y puede ser $2 < D \leq 3$ sin autointersectarse
    
    $D = 3$ cuando llena un volumen de espacio
    \item $3 < D < 4$ cuando la curva se autointersecta y el volumen de espacio puede cubrirse
    infinitas veces
\end{itemize}

Dimensión de una \textbf{superficie fractal} (tierra, nubes, agua)
\begin{itemize}
    \item Habitualmente $2 < D \leq 3$ (más suave cuanto más cerca de 1)
    
    $D = 3$ cuando llena un volumen de espacio
    \item $3 < D < 4$ cuando la superficie se autointersecta y el volumen de espacio puede y cubrirse infinitas veces
\end{itemize}

Dimensión de un \textbf{solido fractal} (densidad de vapor de agua, temperatura en
una región del espacio)
\begin{itemize}
    \item En general $3 < D \leq 4$
    \item $4 < D$ cuando el sólido se autointersecta
\end{itemize}

\section{Fractales autosimilares}
Construir un fractal autosimilar:
\begin{itemize}
    \item Iniciador. Forma geométrica determina.
    \item Generador. Patrón que siguen las subpartes del iniciador.
    \item Reglas básicas de construcción.
    \begin{itemize}
        \item Infinito detalle en cada punto.
        \item Auto similitud entre las partes del objeto y su totalidad.
    \end{itemize}
\end{itemize}

Patrones muy diversos. Añadir/sustraer partes o características del generador.

Se puede añadir carácter aleatorio en la construcción de un fractal autosimilar.
\begin{itemize}
    \item Seleccionando generador al azar.
    \begin{itemize}
        \item Rotado aleatorio del generador o alguna de sus partes.
        \item Escalado aleatorio del generador o alguna de sus partes.
        \item Selección aleatoria del color.
        \item Selección aleatoria de bultos.
    \end{itemize}
    \item Desplazamiento de coordenadas al azar. Traslación aleatoria del generador
\end{itemize}

\section{Fractales autosimilares deterministas}
\textbf{Conjunto de Cantor} (1883) - subconjunto fractal de [0,1] que elimina, en cada paso, el segmento correspondiente al tercio central en cada intervalo. Línea que se parte en 3 y se elimina la del medio. $D=\frac{\ln 2}{\ln 3}= 0,63=\frac{\ln (partes)}{\ln (escala)}$.
\begin{itemize}
    \item Autosimilar
    \item Invariante respecto de la escala
    \item No puede ser descrito analíticamente
\end{itemize}

\textbf{Curva de Koch} (1904) - Línea que se divide en 4 partes, formando un pico, que ocupa el tercio central. $D=\frac{\ln 4}{\ln 3}=1,26$.
\begin{itemize}
    \item Continua en todos los puntos
    \item No derivable en ningún punto
\end{itemize}

\textbf{Copo de Koch} (1904) - Triángulo que se divide cada lado en 4 partes cada cara, formando un pico, que ocupa el tercio central. $D=\frac{\ln 4}{\ln 3}=1,26$.

\textbf{Triángulo de Sierpinski} - Triángulo que divide su área en 4 trozos desechando el central, cada uno de estos triángulos tiene de lado la mitad. $D=\frac{\ln 3}{\ln 2}=1,58$.

\textbf{Alfombra de Sierpinski} - Cuadrado que se divide su área en 9 trozos desechando el central, cada uno de estos cuadrados tiene un tercio de lado. $D=\frac{\ln 8}{\ln 3}=1,89$.

\textbf{Curva de Peano} - S en la que cada lado se divide en 3, pero el espacio total se divide, por tanto, en 9 partes. $D=\frac{\ln 9}{\ln 3}=2$.
\begin{itemize}
    \item En el límite, recubre todo el plano
    \item Similar a la curva de Hilbert
\end{itemize}

\section{Fractales autoafines}
Construcción de fractales autoafines es igual que la de los fractales autosimilares pero con distintos factores de escala aplicados a cada una de las dimensiones del generador

Podemos añadir aleatoriedad mediante el movimiento browniano
\begin{itemize}
    \item Modelamos una curva fractal - desde posición inicial $(x, y)$
    \begin{itemize}
        \item Generamos \textbf{dirección} aleatoria
        \item Generamos \textbf{longitud} aleatoria
    \end{itemize}
    \item Matriz bidimensional de saltos del browniano
    \begin{itemize}
        \item Sobre una \textbf{cuadrícula de plano} (terreno)
        \item Sobre una \textbf{esfera} (planeta)
    \end{itemize}
\end{itemize}

Una trayectoria browniana se define como la sucesión de puntos (variables aleatorias) $s_0, s_1, \ldots, s_n$ definidas por $s_n = s_{n-1} + \textbf{salto} X_n$ en sus respectivos periodos de tiempo $0, \Delta t, 2\Delta t, \ldots, n\Delta t$ donde $\textit{salto}=\sqrt{\Delta t}$ permite controlar la varianza de la trayectoria y las $X_1, \ldots, X_n$ con variables aleatorias con una distribución determinada. $X_j \sim U(0,1), N(0,1), \ldots$.

Ajustar dimensión fractal en cálculos del browniano para dar más realismo modelado.

Escalar elevaciones $\rightarrow$ incrementar/diminuir ''saltos'' browniano.

\section{Modelos Gramaticales}
Smith presenta un método para describir la estructura de ciertas plantas.

Se utiliza lenguajes gramaticales (gramáticas-L) de grafos paralelos. Lenguajes descritos con una gramática que consiste en una colección de \textbf{producciones que se aplican todas a la vez}.

El lenguaje generado por la gramática, interpretado de forma apropiada, nos da la sucesión de operaciones que debemos realizar para generar el objeto.

En general, más de una regla para un mismo símbolo no terminal. Podemos seleccionar la regla a aplicar de forma aleatoria

Lindenmayer extendió los lenguajes para que incluyeran corchetes (L-Systems). Podemos añadir a las gramáticas un operador de pila, de manera que lo que hay en los corchetes se meta en la pila. También podemos añadir paréntesis para enriquecer construcción

Especificación:
\begin{itemize}
    \item Palabra en el lenguaje $\rightarrow$ secuencia de segmentos en una estructura gráfica
    \item Porciones entre corchetes representan porciones que se ramifican desde el símbolo anterior
\end{itemize}

Obtención de gramáticas que representen con precisión la biología de plantas durante el desarrollo

Variaciones
\begin{itemize}
    \item Las gramáticas se han enriquecido para permitir llevar un registro de la “edad” de la letra en una palabra. Las letras “viejas” y “jóvenes” se transforman de distinta manera
    \item Se puede ajustar la longitud correspondiente al movimiento hacia delante de tal forma que en cada iteración las ramas se vayan haciendo cada vez más pequeñas
    \item Aplicar de forma aleatoria el número de llamadas recursivas de la expansión, lo que hace que se generen árboles de diferente tipo
\end{itemize}

\section{Fractales autocuadráticos}
Aplicación recursiva de función de transformación F(z) a puntos en el
espacio complejo (función compleja).

Representamos un punto y su módulo en el plano complejo como: $z=x+iy$ $|Z|=(x^2+y^2)^{\frac{1}{2}}$.

Definimos dicha función $F:Z \rightarrow Z$ como: $z_{k+1}=F(z_k)=z^2_k + c$.

Otra familia de transformación rica en fractales ($\lambda$ complejo) es: $F(z_k)=\lambda z^2_k(1-z_k)$.

Al aplicar recursivamente cuadrados a un número complejo $F(x)=x^2+c$, este:
\begin{itemize}
    \item Tiende a infinito si $|z|>1$
    \item Tiende a cero si $|z|<1$
    \item Permanece en $|z|=1$ si $|z|=1$
\end{itemize}

Dependiendo del punto inicial, $z_0 = c$, la sucesión de transformaciones:
\begin{itemize}
    \item Divergen a infinito si $|z|>1$
    \item Convergen a un punto de atracción (atractor) si $|z|<1$
    \item Son periódicas si $|z|=1$
\end{itemize}

\subsection{Conjunto de Julia}
Para cada valor de $c = a + bi$, hay un conjunto de Julia diferente

$$F(z)=z^2+c$$

Frontera fractal que separa
\begin{itemize}
    \item Los puntos que divergen a infinito
    \item Los que convergen a un punto de atracción
\end{itemize}

\subsection{Conjunto de Mandelbrot}
\begin{itemize}
    \item El rey de los monstruos matemáticos
    \item Es conexo
    \item Contiene infinitas copias de sí mismo. Las copias están conectadas al cuerpo principal por “cadenas”
    \item Tiene dimensión fractal 2
\end{itemize}

Calculamos el conjunto de Julia para cada valor de c y coloreamos:
\begin{itemize}
    \item El punto de negro cuando el conjunto de Julia está conectado (continuo)
    \item El punto de blanco cuando el conjunto de Julia no está conectado (discontinuo)
\end{itemize}

Tenemos la función de transformación: $z_{k+1}=F(z_k)=z^2_k + c$

Elegimos el punto inicial $z_0 = 0 + 0i$ y le aplicamos la función de transformación, para
cada valor de c (de la forma $a + bi$.) Un punto c está en el conjunto de Mandelbrot si y solo si todos los puntos generados por esta función tienen como módulo un valor finito

Fijamos un radio máximo n como límite de para las posiciones sucesivas

En la práctica si $|z_k| > n$, valores sucesivos de $|z_k|$ serán más y más grandes. Por tanto,
podemos detener el cálculo para valores siguientes a $z_k$

Fijamos un número máximo de iteraciones k
\begin{itemize}
    \item Si $|z_k| > n$ antes de alcanzar k (la función diverge) $\rightarrow$ colorear el punto c de blanco
    \item Si $|z_k| < n$ antes de alcanzar k (la función converge) $\rightarrow$ colorear el punto c de negro (c pertenece al conjunto)
\end{itemize}

Consideramos cuanto tarda la función en diverger para un cierto c  representar con diferentes colores

\section{Fractales y caos}
El juego del caos
\begin{itemize}
    \item Dado tres puntos, A, B, C y un punto inicial arbitrario $z_0$
    \item Seleccionar aleatoriamente A, B o C y calcular $z_1 = \frac{P+ z_0}{2}, z_n = \frac{P+ z_{n-1}}{2}$
    \item Seleccionar
\end{itemize}

Algoritmo de Midpoint Displacement
\begin{enumerate}
    \item Dividir la escena en regiones
    \item Cambiar la altura de las esquinas de cada región aleatoriamente
    \item Dividir cada región en regiones más pequeñas
    \item Cambiar la altura de las esquinas de cada región aleatoriamente, pero en menor cantidad
    \item Volver a 3
\end{enumerate}
