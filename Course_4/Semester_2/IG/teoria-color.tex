\chapter{Teoría del Color}
\section{Introducción}
La percepción del color de los objetos depende de:
\begin{itemize}
    \item Propiedades del objeto
    \item Fuente de luz
    \item Color del entorno
    \item Sistema visual humano (no solo el ojo, también el cerebro)
\end{itemize}

La realidad física, son longitudes de onda, pero la percepción son los colores.

Existen diversas teorías, técnicas de medida, estándares..., pero no hay una teoría universalmente aceptada sobre la percepción del color que realizamos los humanos.

La ciencia del color es una mezcla de componentes subjetivos (percepción) con componentes físicos (longitudes de onda, espectros)

\section{La Luz}
La luz (gama de colores) es la banda de longitudes (o frecuencias) de onda del espectro electromagnético, que es visible para nosotros. Son de longitud 750-400 nm.

\subsection{Historia}
Siglo XIX aceptaban que la luz tenía naturaleza ondulatoria, aunque no se relacionaba con la electricidad ni con el magnetismo.
\begin{itemize}
    \item Pero todas las ondas conocidas necesitaban un medio para propagarse y la luz viajaba en el vacío con más velocidad que en medios como el aire o el agua. Se proponía la existencia del éter luminífero, que se desecha en el s. XIX.
\end{itemize}

En 1845 Faraday (aproximación práctica) descubrió que el magnetismo altera el plano de polarización de la luz (relación entre la luz y el magnetismo). Y especuló que la luz podía ser resultado de la vibración de las líneas de fuerza eléctricas y magnéticas. 
\pagebreak

Nadie le hizo caso, hasta que Maxwell (aproximación teórica) publica las ecuaciones del electromagnetismo. Deduce que deben existir perturbaciones electromagnéticas que se autopropagan en el vacío. Mediante una ecuación consigue sin necesidad de punto de referencia aproximar la velocidad de la luz mediante campos.

Quedó aceptado que la luz es una radiación electromagnética.

A finales del siglo XIX se encontraron nuevos efectos que no se podían explicar con la naturaleza ondulatoria de la luz.
\begin{itemize}
    \item Radiación del cuerpo negro
    \item Efecto fotoeléctrico
\end{itemize}

Solo se podían explicar si la luz consistía en partículas: cuantos de energía, fotones. Plank $E=hv$

La luz se puede manifestar como onda o como partícula (dualidad ondapartícula)

\subsection{Radiación de cuerpo negro}
La energía que emite un cuerpo depende de su temperatura. Esta temperatura les confiere un color que puede parecer contradictorio, son:

Temperatura más alta $\rightarrow$ Temperatura de color frío (azulado)

Temperatura más baja $\rightarrow$ Temperatura de color cálido (rojizo)

El sol actúa como una especie de cuerpo negro, cuya radiación solar en el camino se ve afectada, de manera que lo que se ve fuera de la tierra es diferente que en la superficie.

\section{Sistema visual humano}
Tres tipos de conos: S (corto), M (medio),L (largo). Cada uno responde a diferentes rangos de longitudes de onda. Estos se alinean bastante bien con el azul, verde y rojo.

Nuestro cerebro con 3 señales interpreta cualquier color que podamos percibir.

Las aves tienen cuatro tipos de conos, uno de ellos detecta la luz UV

Puede ocurrir que percibamos igual diferentes espectros, mezclas de colores.
\begin{itemize}
    \item La luz espectral es monocromática, y no se puede descompones en otros.
    \item Luz resultado de mezclar colores, que se puede con un prisma descomponer en los colores que lo componen.
\end{itemize}

\section{Caracterización del Color}
Cada color del espectro puede describirse en función de su frecuencia de onda ($f$, medida en hercios) o su longitud de onda ($\lambda$, medida en nanómetros), que están relacionadas por $c = \lambda f$.

\textbf{Luz monocromática o Color espectral:} está formada por una sola longitud de onda (colores del arcoíris). Ejem. El rosa no existe como onda.

\textbf{Fuente de luz blanca (sol)} emite todas las frecuencias. Al incidir sobre un objeto, este absorbe algunas longitudes y refleja otras. 

\textbf{La combinación de longitudes de onda reflejadas determina el color (tono, matiz) del objeto}

Desde el punto de vista de nuestra percepción, un color se caracteriza con 3 parámetros (\textbf{modelo HSB}/HSL/HSV):
\begin{itemize}
    \item \textbf{Hue} (Tono, matiz, color): El color, longitud de onda predominante, se puede discriminar aproximadamente entre 150 tonos.
    \item \textbf{Saturation} (Saturación): Es el grado de pureza del color observado: va desde el color puro al gris. Los menos saturados contienen más luz blanca.
    \item \textbf{Brightness}, Lightness, Value (Brillo): Intensidad de la luz que se percibe. Cuanta energía emite.
    
    Intensidad - energía radiante que se emite por unidad de tiempo
\end{itemize}

Podemos distinguir del orden de 7 millones de colores

\textbf{Colores primarios:} Aquellos que no pueden obtenerse mediante la mezcla de ningún otro.
\begin{itemize}
    \item Colores aditivos - al añadirse generan el blanco. Azul, verde y rojo.
    \item Colores substractivos - filtros que absorben la luz. Cian, amarillo y magenta.
\end{itemize}

\textbf{Colores complementarios:} Combinados producen luz blanca. Rojo-cian, verde-magenta, azul-amarillo

Se pueden combinar fuentes de luz de color con distinta intensidad para, seleccionando adecuadamente las intensidades, generar un determinado rango de colores.

\textbf{No se puede} combinar un conjunto finito o real de colores primarios para \textbf{generar todos los colores visibles} posibles

Modelos de color emplean \textbf{3 colores primarios para obtener la gama de colores (gamut)} asociada a ese modelo.

\section{Estandarización}
Necesidad de estandarizar los colores percibidos. En principio, con tres valores podríamos representar cualquier color.

\textbf{Color matching experiment:} Mezcla de tres colores primarios (ej. azul, verde y rojo del modelo RGB) para generar todos los colores del espectro visual.
\begin{itemize}
    \item Combinar colores puros RGB para obtener un cierto color espectral.
    \item Se hace para todos los colores del espectro y con diferentes personas.
    \item Lo que si varía de los colores base es la saturación de los colores (ya que el matiz es propio de la longitud de RGB)
    \item No todos los colores del espectro pueden ser igualados mediante la mezcla de R, G y B como el cian de 500 nm, pero si se le puede restar rojo.
    \item \textbf{Resultado:} Para cada color del espectro, obtenemos los valores RGB necesarios.
\end{itemize}

Estos experimentos son la base de la ciencia del color. Dieron lugar al espacio de color \textbf{CIE 1931} establecido por la \textbf{Comission Internationale de l'Eclairage (CIE)}.

\section{Colores Primarios Estándar - Modelo CIE XYZ}
\textbf{La Comisión Internacional sobre Iluminación, CIE (1931)} define un estándar de tres colores imaginarios aditivos que no necesiten valores negativos. 

Se hace una transformación lineal del espacio R,G,B en el espacio X,Y,Z (vectores en espacio 3D de color aditivo) para que todos los valores de estos nuevos colores primarios sean positivos. $\left(\begin{matrix} X \\ Y \\ Z \end{matrix}\right)=M\left(\begin{matrix} R \\ G \\ B \end{matrix}\right)$. Combinando los colores imaginarios XYZ se puede generar todo el espectro.

Se eligen los primarios de forma que Y es igual a la luminosidad.

\section{CIE 1931: diagrama de cromaticidad (2D)}
\textbf{Colores imaginarios XYZ:} no son RGB, XYZ son combinaciones lineales de los valores de RGB para que todos los valores sean positivos.

Para hacerlo más manejable, se proyectará el espacio XYZ (que es 3D) en un plano xy $\rightarrow$ esto es el diagrama de cromaticidad (también llamado CIE xyY)

Esto se consigue normalizando los valores X, Y, Z, para que su suma sea 1 ($x+y+z=1$). $x=\frac{X}{Z+Y+Z}$ $y=\frac{Y}{Z+Y+Z}$ $z=\frac{Z}{Z+Y+Z}$.

El color puede dividirse en dos partes: brillo y cromaticidad.
\begin{itemize}
    \item $Y$ es una medida del brillo o luminosidad de un color. Máxima en el amarillo verdoso
    \item La cromaticidad se determina a través de los dos parámetros derivados $x$ e $y$ (dos
    de los valores de X,Y, Z)
\end{itemize}

\section{Diagrama de Cromaticidad}
Proyección del plano $x+y+z=1$ en el plano XY.
\begin{itemize}
    \item A partir de la cromaticidad $x$ e $y$ (fija relación entre colores) y la luminancia $Y$, se obtienen los valores mediante: $X=x\frac{Y}{y}$ y $Z=z\frac{Y}{y}$
\end{itemize}

El diagrama representa toda la cromaticidad percibida por el ojo humano
\begin{itemize}
    \item Aplicaciones: Nombrar y normalizar colores, Mezcla de colores, Identificar colores complementarios, Determinar la frecuencia dominante y Comparar gamut de dispositivos.
    \item Propiedades
    \begin{itemize}
        \item Los colores del borde son: \textbf{Colores espectrales puros (spectral locus)}
        \item La línea recta es: \textbf{La línea del púrpura (purple boundary)}
        \item La suma de dos colores se encuentra en la línea que los une
        \item El blanco se encuentra en $x=1/3$, $y=1/3$. La línea que une dos colores complementarios pasa por ese punto.
    \end{itemize}
    \item Se puede representar cualquier color con:
    \begin{itemize}
        \item los distintos valores del \textbf{triestímulo XYZ}
        \item o con los valores de \textbf{luminancia Y y de cromaticidad x,y}
    \end{itemize}
\end{itemize}

Mezcla de colores
\begin{itemize}
    \item Mezclando los \textbf{colores I,J} se pueden obtener toda la \textbf{gama} situada en el \textbf{segmento $\overline{IJ}$}
    \item Con los colores \textbf{I,J,K se obtienen los colores del triángulo inscrito (gamut)}
\end{itemize}
\pagebreak

Color blanco
\begin{itemize}
    \item El blanco es una mezcla de todos los colores.
    \item Se define un blanco estándar, blanco de igual energía.
    \item Pero hay muchos colores que pueden ser considerados blancos.
    \item CIE estandariza los diferentes blancos.
    \item El blanco D65 es el que emitiría un cuerpo negro a 6500 K.
\end{itemize}