\documentclass[12pt, twoside, openright]{report} % Fuente a 12pt, formato doble página y chapter a la derecha
\raggedbottom % No ajustar el contenido con un salto de página

% MÁRGENES: 2,5 cm sup. e inf.; 3 cm izdo. y dcho.
\usepackage[
a4paper,
vmargin=2.5cm,
hmargin=3cm
]{geometry}

% INTERLINEADO: Estrecho (6 ptos./interlineado 1,15) o Moderado (6 ptos./interlineado 1,5)
\renewcommand{\baselinestretch}{1.15}
\parskip=6pt

% DEFINICIÓN DE COLORES para portada y listados de código
\usepackage[table]{xcolor}
\definecolor{azulUC3M}{RGB}{0,0,102}
\definecolor{gray97}{gray}{.97}
\definecolor{gray75}{gray}{.75}
\definecolor{gray45}{gray}{.45}

% Soporte para GENERAR PDF/A
\usepackage{etoolbox}
\makeatletter
\@ifl@t@r\fmtversion{2021-06-01}%
 {\AddToHook{package/after/xmpincl}
   {\patchcmd\mcs@xmpincl@patchFile{\if\par}{\ifx\par}{}{\fail}}}{}
\makeatother
\usepackage[a-1b]{pdfx}

% ENLACES
\usepackage{hyperref}
\hypersetup{colorlinks=true,
  linkcolor=black, % enlaces a partes del documento (p.e. índice) en color negro
  urlcolor=blue} % enlaces a recursos fuera del documento en azul

% Añadir pdfs como partes del documento
\usepackage{pdfpages}

% Quitar la indentación de principio de los párrafos
\setlength{\parindent}{0em}

% EXPRESIONES MATEMÁTICAS
\usepackage{amsmath,amssymb,amsfonts,amsthm}

\usepackage{txfonts} 
\usepackage[T1]{fontenc}
\usepackage[utf8]{inputenc}

% Insertar gráficas y fotos
\usepackage{tikz}
\usetikzlibrary{cd}
\usepackage{pgfplots}

\usepackage[spanish, es-tabla]{babel} 
\usepackage[babel, spanish=spanish]{csquotes}
\AtBeginEnvironment{quote}{\small}

% diseño de PIE DE PÁGINA
\usepackage{fancyhdr}
\pagestyle{fancy}
\fancyhf{}
\renewcommand{\headrulewidth}{0pt}
\fancyfoot[LE,RO]{\thepage}
\fancypagestyle{plain}{\pagestyle{fancy}}

% DISEÑO DE LOS TÍTULOS de las partes del trabajo (capítulos y epígrafes o subcapítulos)
\usepackage{titlesec}
\usepackage{titletoc}
\titleformat{\chapter}[block]
{\large\bfseries\filcenter}
{\thechapter.}
{5pt}
{\MakeUppercase}
{}
\titlespacing{\chapter}{0pt}{0pt}{*3}
\titlecontents{chapter}
[0pt]                                               
{}
{\contentsmargin{0pt}\thecontentslabel.\enspace\uppercase}
{\contentsmargin{0pt}\uppercase}                        
{\titlerule*[.7pc]{.}\contentspage}                 

\titleformat{\section}
{\bfseries}
{\thesection.}
{5pt}
{}
\titlecontents{section}
[5pt]                                               
{}
{\contentsmargin{0pt}\thecontentslabel.\enspace}
{\contentsmargin{0pt}}
{\titlerule*[.7pc]{.}\contentspage}

\titleformat{\subsection}
{\normalsize\bfseries}
{\thesubsection.}
{5pt}
{}
\titlecontents{subsection}
[10pt]                                               
{}
{\contentsmargin{0pt}                          
  \thecontentslabel.\enspace}
{\contentsmargin{0pt}}                        
{\titlerule*[.7pc]{.}\contentspage}  

\usepackage{adjustbox}

% DISEÑO DE TABLAS.
\usepackage{multirow} % permite combinar celdas 
\usepackage{caption} % para personalizar el título de tablas y figuras
\usepackage{floatrow} % utilizamos este paquete y sus macros \ttabbox y \ffigbox para alinear los nombres de tablas y figuras de acuerdo con el estilo definido. Para su uso ver archivo de ejemplo 
\usepackage{array} % con este paquete podemos definir en la siguiente línea un nuevo tipo de columna para tablas: ancho personalizado y contenido centrado
\newcolumntype{P}[1]{>{\centering\arraybackslash}p{#1}}
\DeclareCaptionFormat{upper}{#1#2\uppercase{#3}\par}

% Diseño de tabla para ingeniería
\captionsetup[table]{
  format=hang,
  name=Tabla,
  justification=centering,
  labelsep=colon,
  width=.75\linewidth,
  labelfont=small,
  font=small,
}

% DISEÑO DE FIGURAS.
\usepackage{graphicx}
\graphicspath{{img/}} %ruta a la carpeta de imágenes

% Diseño de figuras para ingeniería
\captionsetup[figure]{
  format=hang,
  name=Fig.,
  singlelinecheck=off,
  labelsep=colon,
  labelfont=small,
  font=small    
}

% NOTAS A PIE DE PÁGINA
\usepackage{chngcntr} % Para numeración continua de las notas al pie
\counterwithout{footnote}{chapter}

% LISTADOS DE CÓDIGO
% soporte y estilo para listados de código. Más información en https://es.wikibooks.org/wiki/Manual_de_LaTeX/Listados_de_código/Listados_con_listings
\usepackage{listings}

% definimos un estilo de listings
\lstdefinestyle{estilo}{ frame=Ltb,
  framerule=0pt,
  aboveskip=0.5cm,
  framextopmargin=3pt,
  framexbottommargin=3pt,
  framexleftmargin=0.4cm,
  framesep=0pt,
  rulesep=.4pt,
  backgroundcolor=\color{gray97},
  rulesepcolor=\color{black},
  %
  basicstyle=\ttfamily\footnotesize,
  keywordstyle=\bfseries,
  stringstyle=\ttfamily,
  showstringspaces = false,
  commentstyle=\color{gray45},     
  %
  numbers=left,
  numbersep=15pt,
  numberstyle=\tiny,
  numberfirstline = false,
  breaklines=true,
  xleftmargin=\parindent
}

\captionsetup[lstlisting]{font=small, labelsep=period}
% fijamos el estilo a utilizar 
\lstset{style=estilo}
\renewcommand{\lstlistingname}{\uppercase{Código}}

\pgfplotsset{compat=1.17} 
%-------------
% DOCUMENTO
%-------------

\begin{document}
\pagenumbering{roman} % Se utilizan cifras romanas en la numeración de las páginas previas al cuerpo del trabajo

%----------
% PORTADA
%---------- 
\begin{titlepage}
  \begin{sffamily}
    \color{azulUC3M}
    \begin{center}
      \begin{figure}[H] % Incluimos el logotipo de la Universidad
        \makebox[\textwidth][c]{\includegraphics[width=16cm]{Portada_Logo.png}}
      \end{figure}
      \vspace{2.5cm}
      \begin{Large}
        Grado en Ingeniería Informática\\
        2021-2022\\
        \vspace{2cm}
        \textsl{Apuntes}\\
        \bigskip
      \end{Large}
      {\Huge Informática Gráfica}\\
      \vspace*{0.5cm}
      \rule{10.5cm}{0.1mm}\\
      \vspace*{0.9cm}
      {\LARGE Jorge Rodríguez Fraile\footnote{\href{mailto:100405951@alumnos.uc3m.es}{Universidad: 100405951@alumnos.uc3m.es}  |  \href{mailto:jrf1616@gmail.com}{Personal: jrf1616@gmail.com}}}\\
      \vspace*{1cm}
    \end{center}
    \vfill
    \color{black}
    \includegraphics[width=4.2cm]{img/creativecommons.png}\\
    Esta obra se encuentra sujeta a la licencia Creative Commons\\ \textbf{Reconocimiento - No Comercial - Sin Obra Derivada}
  \end{sffamily}
\end{titlepage}

%----------
% ÍNDICES
%----------

%--
% Índice general
%-
\tableofcontents
\thispagestyle{fancy}

%--
% Índice de figuras. Si no se incluyen, comenta las líneas siguientes
%-
\listoffigures
\thispagestyle{fancy}

%--
% Índice de tablas. Si no se incluyen, comenta las líneas siguientes
%-
\listoftables
\thispagestyle{fancy}

%----------
% TRABAJO
%----------

\pagenumbering{arabic} % numeración con números arábigos para el resto de la publicación

%----------
% COMENZAR A ESCRIBIR AQUÍ
%----------

\chapter{Información}\label{ch:informacion}
\section{Profesores}\label{sec:profesores}
\begin{quote}
    Coordinador: Antonio Berlanga

	Magistral: Yago Sáez, yago.saez@uc3m.es, 2.1.C.13

    Magistral: José María Valls

    Prácticas: David Quintana
\end{quote}

\section{Sistema de evaluación}\label{sec:sistema-de-evaluación}
\begin{itemize}
  \item 50 \% Teoría
  \begin{itemize}
    \item 10 \% Test 1. 4 temas
    \item 10 \% Test 2. 4 temas
    \item 30 \% Examen Final. Todos los contenidos teóricos de la asignatura \\ Nota mínima: 3
  \end{itemize}
  \item 50 \% Práctica. La 1, 2 y 3 son por parejas y la última en grupo de 3
  \begin{itemize}
    \item 12,5 \% Práctica 1. Modelado 3D
    \item 10 \% Práctica 2. Fractales
    \item 12,5 \% Práctica 3. Paisajes y terreno
    \item 15 \% Práctica Final. Modelado 3D, Iluminación y sombreado y Animación
  \end{itemize}
\end{itemize}

\chapter{Introducción a la Informática Gráfica}\label{ch:introduccion-a-la-informática-gráfica}
  \section{Que es una Imagen}
\begin{itemize}
  \item Una vista en un monitor.
  \item Un fichero en un camara.
  \item Numeros de un RAM
\end{itemize}
  Definición
\begin{itemize}
  \item Una distribución 2D de intensidad o color.
  \item Una función definida en un plano bidimensional.
  \item !No hay mención a pixeles.
\end{itemize}
Necesidad
\begin{itemize}
  \item Representar la imagen (codificarla en numeros)
  \item Mostrar la imagen (aplicar transformaciónes, intensidades de corriente, cantidades de tinta)
\end{itemize}

  \section{Areas de Conocimiento}
  Procesado de imagenes, se reciben solamente imagenes, se hacen modificaciónes y analisis de imagenes.
\begin{itemize}
  \item Analisis de escenas.
  \item Reconstruccion de modelos en 2D o 3D de las imagenes.
  \item Compresión de imagenes, mejora de nitidez, posterizado (convertir imagenes en dibujos o posteres)\ldots
\end{itemize}

  Visión Artificial, dadas imagenes y datos sobre estas se obtiene información mas completa.
\begin{itemize}
  \item Construcción de sistemas que obtengan información automaticamente a partir de las imagenes.
  \item Clasificación de imagenes en base a conocimiento previo o información estadistica.
\end{itemize}

Informatica Grafica, sistesis de obejetos reales o imaginarios a partir de modelos basados en computación.

\section{Aplicaciones}
\subsection{Diseno Asistido por Computador (CAD)}
Herramientas grafica para disenar prototipos y evaluarlos antes de construirlos.

Areas importantes: Diseno industrial, Arquitectura, Circuitos electronicos y electronicos.

Tecnicas: Diseno basado en primitivas constructivas y Superficies curvas.

Posibilidades: Realidad virtual, Presentación realista, Analisis del diseno y Conexión con el sistema de fabricación.
\subsection{Gráficos de presentación}
Uso de los graficos para producción de ilustraciones de soporte a informes y trabajos.

Areas importantes: Economia, Estadistica, Matematicas y Administración y gestión.

Tecnicas: Graficos de lineas, Graficos de barra, Graficos de tarta y Superficies 3D.
\subsection{Creaciones artisticas}
Producción de imagenes con un fin artistico o comercial.

Areas importantes: Diseno de logotipos, Bellas artes y Animaciones publicitarias.

Tecnicas: Diseno vectorial, Soporte a la animación, Tratamiento de imagen y Rendering.

\subsection{Entretenimiento}
Producción de videos y videojuegos con efectos realistas.

Areas importantes: Cine (peliculas y animación), Televisión (cortinillas, cabeceras, AR) y Videojuegos.

Tecnicas: Animación, Visualización realista, Efectos especiales y Interactividad.

Se utilizan API y SDK: Truevision 3D, XNA, OGRE, IrrLicht, CryEngine 3, Unreal Engine y Unity.

\subsection{Simulación y entrenamiento}
Desarrollo de sistemas para simular tareas realistas y procesos.

Areas importantes: Simulación de conducción, Simulación de procesos industriales, Entrenamiento y Educación.

Tecnicas: Tiempo real, Interactividad y Realidad virtual.

\subsection{Visualización cientifica}
Visualización en dominios especificos y de grandes cantidades de datos.

Areas importante: Medicina (resonancias, tomografias), Ingenieria, Fisica (campos, dinamica de fuerza), Quimica (interacción moleculas), Matematica y Topologico (terrenos y corrientes).

Tecnicas: Codificación por color, Curvas de nivel, Visualización de volumenes.

\section{Historia}
\subsection{Anos 50 (Comienzos)}
Whirlwind y SAGE.

Se empezo utilizando la informatica grafica para detercar aviones o submarinos, con radares, en estos tiempos se utilizaban ordenadores del tamano de ordenadores. En esta visualizacion se podia marcar puntos.

\subsection{Anos 60}
\begin{itemize}
  \item Sketchpad: el primer programa grafico interactivo.
  \item Russell (MIT) desarrolla Spacewar en un PDP-1.
  \item El primer juego <"Tennis for two>" (Pong) se jugaba en un osciloscopio.
  \item Primer algoritmo de superficies ocultas (Catmull)
  \item Realismo mediante sombreado de superficies con color.
\end{itemize}

\subsection{Anos 70}
\begin{itemize}
  \item Suavizado de superficies poligonales (Gouraud)
  \item Comercialización del microprocesador.
  \item Fundación de Atari.
  \item Primeros itentos de informatica grafica en el cine.
  \item Disenos de la tetera (Newell, Utah, Univ.)
  \item Introducción de texturas y Z-buffer.
  \item Suavizado de superficies poligonales (Phong) que se utiliza en el 3D
  \item Fundación de Apple y Microsoft.
  \item Lucasfilm crea la división de graficos por computador.
  \item Westworld: primera pelicula en emplear graficos por ordenador.
  \item La Guerra de la Galaxias.
\end{itemize}

  \subsection{Anos 80}
\begin{itemize}
  \item Popularización de SIGGRAPH como evento de referencia en el area.
  \item Publicación sobre el Raytracing (Bell Labs)
  \item Construcción del primer motor de rendering (REYES), precursor de Renderman (Carpenter)
  \item Ecuación de rendering (Kajiya)
  \item Pelicula de TRON de Disney (Lisberger y Kushner)
  \item Venta masiva de terminales graficas: IBM, Tektronix...
  \item GKS se constituye como estandar ISO y ANSI de construcción de librerias graficas.
  \item IBM crea el ordenador personal (PC)
  \item Introducción de la radiosidad (Goral, Torrance, Cohen)
  \item Cornell Box: prueba para comprobar la adecuación de un sistema de renderizado a una fotografia real.
  \item
\end{itemize}
  \section{Procesado de Imagen 2D}
  La secuencia de pasos por los que pasa el procesado de imagen en 2D puede representarse como un pipeline (no todos los pasos tienen que estar presentes)

  \begin{enumerate}
      \item Adquisición: Muchas tecnicas, divisibles en dos categorias: Sinstesis y Captura.
    \item Preprocesado: Modificación de formas, tamano y propiedades de color.
    \item Mapeado: Varias imagenes se combinan con transformaciónes.
    \item Postprocesado: Se utiliza para aplicar efectos globales a toda o parte de la imagen.
    \item Salida: El dispositivo de salida puede afectar a su visualización.
      \begin{itemize}
        \item El mapa de color de una impresora puede <"falsear>" o acentuar colores.
        \item Se pueden mapear los colores entre la pantalla y la impresora.
      \end{itemize}
  \end{enumerate}

  \section{Procesado Grafico 3D}
  Se encarga de generar las imagenes que se pueden ver en un dispositivo.
\begin{itemize}
  \item Vista del sistema operativo.
  \item Imágenes (frames) de un videojuego.
  \item Pantalla de un telefono.
  \item Imágenes  de una pelicula en el cine.
\end{itemize}
\subsection{Pipeline 3D}
Varias módulos van realizando diferentes tareas para la generación (render) de la imagen.
\begin{itemize}
  \item Tranformación
  \item Iluminación
  \item Texturado
  \item Shaders: Unidades de procesado que se compilan de forma independiente, modificanado localmente las propiedades de los objetos de la escena. Vertices, Geometricos y Fragmentos.
\end{itemize}
Diagrama de flujo
\begin{itemize}
  \item Entrada de los datos de la geometria de objetos y atributos de sus materiales.
  \item Definición de la condiciones de iluminación.
  \item Salida de las imganes.
\end{itemize}
DIAGRAMA #D Y @D

Framebuffer: Memoria local para las imagenes que se mapean en un dispositivo. El hardware de video convierte el contenido en una senal para el dispositivo de salida.

  COMPLETAR
\end{document}