\chapter{Iluminación y Sombreado}
\section{Rendering}
Generar una imagen 2D a partir de un modelo 3D,
\begin{itemize}
    \item \textbf{Analíticas}, mediante ecuaciones.
    \item Aproximaciones mediante \textbf{mallas de triángulos}.
\end{itemize}

Dos maneras de convertir el modelo 3D en una imagen 2D:
\begin{itemize}
    \item \textbf{Ray tracing (trazado de rayos):} simula el proceso físico de la propagación y reflejo de la luz desde la \textbf{fuente hasta la cámara} (aunque \textbf{lo hace al revés}, de la \textbf{cámara hasta la fuente de luz})
    \begin{itemize}
        \item Para cada pixel: Envía un rayo y Comprueba intersección.
        \item Más realista, pero más lento.
    \end{itemize}
    \item \textbf{Graphics pipeline:} proyecta vértices/triángulos ''en bloque'' (de manera paralela) sobre los píxeles de la imagen. Tarjetas gráficas.
    \begin{itemize}
        \item Para cada objeto/triángulo: Proyectarlo a la pantalla.
        \item Utiliza GPU, es más rápido, pero menos realista.
    \end{itemize}
\end{itemize}

\section{Ray-tracing}
\subsection{Algoritmo}
Para cada pixel
\begin{enumerate}
    \item Lanza un rayo a la escena
    \item Encuentra el punto de intersección p más cercano que interseccione con el rayo.
    \item Calcula lighting/sombra para p, $[A+D+S]$, colp.
        
    $$I= R_a \cdot L_a + \Sigma_i s_i \left( \frac {R_d \cdot \max (n\cdot l,0) \cdot L_d}{a+bd+cd^2} + \frac {R_s \cdot \max (cos(r,v)^n,0) \cdot L_s}{a+bd+cd^2} \right)$$
    \item Si la superficie es un espejo, se lanza un rayo de reflexión, $(1-r)colp+ r\cdot refCol$, refCol.
    \item Si además la superficie es refractante, se lanza un rayo de refracción, $(1-(r_1-r_2))\cdot colp + r_1\cdot refCol +r_2\cdot refrCol$, refrCol.
\end{enumerate}
\subsection{Ray-casting (primera iteración)}
En el mundo real:
\begin{itemize}
    \item Las fuentes luminosas emiten fotones
    \item Las superficies reflejan y absorben fotones
    \item Las cámaras registran fotones
\end{itemize}

Consideraremos que los fotones son rayos (líneas) de luz. Ignoraremos la naturaleza ondulatoria de la luz

\subsubsection{Rayo}
Se definirá a un rayo como un punto del que parte un vector
\begin{itemize}
    \item $Ray = (Point, Vector) = (S, V)$
    \item Point es el punto de vista (cámara), Vector es la dirección en la que mira la cámara.
    \item La forma paramétrica de un rayo lo expresa como una función escalar t. Para cada t se proporciona un punto por el que el rayo pasa: $P(t)=S+tV, 0 \leq t \leq \infty$
\end{itemize}

\textbf{Rayos primarios:} Aquellos que parten del observador (V).

\textbf{Rayos secundarios:} Aquellos rayos generadores para simular sombra, reflexión (E: espejos) y refracción (T: transmisión) en el punto de intersección del rayo con la superficie.

\subsubsection{Fuentes de luz} 
Cada tipo emite fotos de manera diferente.
\begin{itemize}
    \item \textbf{Fuentes de punto (Bombillas):} Emiten luz en todas las direcciones.
    \item \textbf{Fuentes direccionales (el Sol):} Están muy lejos y se puede suponer que los rayos llegan en paralelo.
    \item \textbf{Spot Lights (Focos):} Como las bombillas, pero la intensidad varía con la dirección. Hay zonas donde no se emite luz.
    \item \textbf{Luces de techo (o de área):} Emiten no desde un punto sino desde un área, conjunto de puntos de luz.
\end{itemize}
\pagebreak

\textbf{Superficies}
\begin{itemize}
    \item Reaccionan a los rayos: Absorbiéndolos, Reflejándolos, Refractándolos, Atenuándolos, Fluorescencia, etc.
    \item Tipos materiales: Mate, Brillante, Rugoso, Suave, Transparente, Translúcido, etc.
\end{itemize}

\textbf{Cámaras:} Transformación de una imagen 3D a una 2D. El punto por el que pasa la luz, como un ojo o una cámara fotográfica.

\subsubsection{Tipos de Iluminación}
\textbf{Iluminación indirecta:} Aquella que proviene de otro objeto que no emite luz. En trazado de rayos, nos ceñimos a la iluminación directa.

\textbf{Iluminación directa:} La que viene de una fuente que emite luz.

El problema de implementar el modelo directamente, es que el programa de rendering sería muy inestable, porque muchos de los rayos emitidos no alcanzarían la cámara. Es por esta razón por la que se hace al revés, en vez de emitir los rayos desde la fuente de luz, hagamos que la cámara los emita.

Para cada pixel
\begin{enumerate}
    \item Lanza un rayo a la escena.
    \item Encuentra el punto de intersección con el objeto más cercano que interseccione con el rayo.
    \item Colorea/ilumina (lighting) el pixel con las propiedades de color del punto de intersección teniendo en cuenta la fuente de luz. Para mostrar espejo se necesitan varias iteraciones.
\end{enumerate}

\subsubsection{La esfera} 
Solo consideraremos un tipo de superficie, la esfera (hay otras: planos, conos, ...)
\begin{itemize}
    \item Ecuación implícita de una esfera centrada en $(x_0, y_0, z_0)$.
    \item $r^2=(x-x_0)^2+(y-y_0)^2+(z-z_0)^2$
    \item Si está centrada en el origen: $r^2=x^2+y^2+z^2$
\end{itemize}

\subsubsection{Intersección rayo-esfera} 
Utilizando las ecuaciones de esfera ($r^2=x^2+y^2+z^2$) y rayo ($P(t)=S+tV$) tenemos $r^2=(S_x+tV_x)^2+(S_y+tV_y)^2+(S_z+tV_z)^2$, lo que queremos es encontrar el t que haga toda la expresión 0.
\begin{itemize}
    \item $0=t^2(V_x^2+V_y^2+V_z^2)+t(S_xV_x+S_yV_y+S_zV_z)+(S_x^2+S_y^2+S_z^2-r^2)$ y resolvemos con $t=\frac{-b \pm \sqrt{b^2-4ac}}{2a}$.
    \item Tres casos: Sin intersección ($b^2-4ac <0$), 1 intersección en t ($b^2-4ac =0$) o 2 intersecciones t más cercanas ($b^2-4ac >0$).
\end{itemize}

Este proceso se puede hacer para otros tipos de objetos si conocemos sus ecuaciones implícitas.
\begin{itemize}
    \item En el caso del triángulo, calculamos la intersección entre el rayo y el plano que contiene el triángulo, comprobando si la intersección p está dentro del triángulo.
    \item Coordenadas baricéntricas: Cualquier triángulo de vértices A, B y C se puede expresar como $P=wA+uB+vC$. Si $0\leq w, u, v \leq 1$, P está dentro del triángulo. Si no está fuera.
\end{itemize}

\subsubsection{Iluminación (Lighting)} 
Interacción luz-superficie.
\begin{itemize}
    \item \textbf{Reflexión difusa:} la luz se difunde en todas direcciones.
    \item \textbf{Reflexión especular:} la luz se difunde en una dirección preferente. El efecto son brillos
    \item Se excluye de momento: espejos, refracción y translúcidos.
\end{itemize}

\textbf{Espectrograma:} La luz contiene energía en muchas frecuencias y la interacción luz-superficie podría ser distinta para cada frecuencia. 
\begin{itemize}
    \item \textbf{Simplificación:} usaremos un modelo de color RGB, cada color se puede definir como una mezcla de rojo, verde y azul. $L_m=R+G+B$
\end{itemize} 

\textbf{Simplificaciones}, para hacer los cálculos tratables.
\begin{itemize}
    \item Se utiliza el color RGB.
    \item Solo luz directa.
    \item En principio la fuente ''de punto'', aunque se puede extender a otros.
    \item Superficies con reflexión especular y difusa.
\end{itemize}

\subsubsection{Modelo de reflexión de Phong} 
$I=I_a(R_a,L_a)+I_d(n, l, R_d, L_d, a, b, c, d)+I_s(r,v, R_s, L_s, n, a, b, c, d)$

$R_x=(R_R, R_G, R_B)$ es la reflectividad de la superficie y $L_x=(L_R, L_G, L_B)$ la intensidad de la luz.

Usa 4 vectores para computar el color de cada punto p
    \begin{itemize}
        \item Normal (n), Luz (l), Punto de vista (v) y Reflexión (r).
    \end{itemize}

3 componentes. 
    \begin{itemize}
        \item \textbf{Ambiente (uniforme):} es una manera de incluir una pseudo-iluminación indirecta. Misma luz de todas las partes, uniforme. 
        
        $I_a(R_a, L_a)=R_a \cdot L_a$
        \item \textbf{Reflexión difusa:} Sigue la \textbf{Ley del coseno de Lambert}. 
        
        El ángulo de incidencia del rayo de luz afecta la intensidad perceptible por el ojo. Si la \textbf{superficie es perpendicular al rayo, captura más cantidad de luz} que si es oblicua. La ley de Lambert afirma que es proporcional al coseno del vector de luz y la normal. 
        
        La luz parece atenuarse con la distancia d entre la fuente de luz y la superficie. Se usan las constantes a, b y c, que son definidas por el usuario.
        
        $I_d(n, l, R_d, L_d, a, b, c, d)= \frac {R_d \cdot \max (n\cdot l,0) \cdot L_d}{a+bd+cd^2}$, con $n \cdot l = |n|*|l|*\cos(n,l)$, \textbf{producto escalar} en ambas reflexiones.
        \item \textbf{Reflexión especular:} Se usa r y v, porque la reflexiones más fuerte en la dirección del vector de reflexión. Es máximo cuando coinciden ambos vectores.
        
        Se utiliza un factor $n$ de intensidad de reflexión (hardness), cuanto mayor es menos intensidad de reflexión, puesto que el coseno es un valor entre 0 y 1.
        
        $I_s(r,v, R_s, L_s, n, a, b, c, d)= \frac {R_s \cdot \max (cos(r,v)^n,0) \cdot L_s}{a+bd+cd^2}$
    \end{itemize}

$L_d$ suele ver igual a $L_s$. Y la atenuación suele ser la misma (a, b, c)

La suma de esos tres componentes define el color e intensidad en un punto de la superficie.

Múltiples fuentes de luz: 

$$I=I_a(R_a,L_a)+\Sigma_i (I_d(n, l_i, R_d, L_{di}, a, b, c, d_i)+I_s(r_i,v, R_s, L_{Si}, n, a, b, c, d_i))$$
\pagebreak

\subsubsection{Sombras} 
Son causadas por un objeto interpuesto entre la superficie y la fuente de luz.

Hemos hecho ray-casting, y hemos visto que el \textbf{rayo intersecciona con la superficie}.

Ahora queremos iluminar el punto. Lo cual incluye \textbf{comprobar si está en la sombra}.

Podemos comprobarlo \textbf{emitiendo un rayo desde el punto hasta la fuente de luz}.
\begin{itemize}
    \item Si se detecta intersección, puede que el punto esté en sombra.
    \item El punto está en sombra si el valor t de la intersección es menor que el valor t del punto de luz.
    \begin{itemize}
        \item Luces direccionales, al estar en el infinito, cualquier intersección genera una sombra (t positivo).
        \item Focos, hay que asegurarse que el rayo de sombra está dentro del cono de luz del foco, pero como la bombilla (fuente de punto)
        \item Luces de área, las más complicadas. Hay que trazar, no uno, sino infinitos rayos de sombra, y sumar. Aparecen zonas de penumbra, sombra para una parte pero para otra no.
    \end{itemize}
\end{itemize}

\subsubsection{Modelo de Phong con sombra} 
$I=I_a(R_a,L_a)+\Sigma_i s_i \cdot (I_d(n, l_i, R_d, L_{di}, a, b, c, d_i)+I_s(r_i,v, R_s, L_{Si}, n, a, b, c, d_i))$
\begin{itemize}
    \item $s_i$ es 0 si luz bloqueada en ese punto y 1 si luz no bloqueada en ese punto. Si el objeto interpuesto es semitransparente y deja pasar algo de luz $0 \leq s_i \leq 1$.
\end{itemize}

\subsection{Reflexión}
El lighting de Phong ya tenía un componente para la luz especular (el cálculo de brillos/reflejos), usando el vector de reflexión.

Para el reflejo en espejos, añadiremos un nuevo término a la fórmula de lighting, y usaremos la reflexión del vector del punto de vista para la nueva dirección del rayo.

e = vector reflexión del vector punto de vista v con respecto a la normal n. Mismo ángulo entre v y n que n con e.

Los puntos de un espejo no tienen propiedades de color intrínsecas, sino que son las del objeto reflejado.

$I=(1-r)\cdot[\textit{Ambiente}+\textit{Difusión}+\textit{Especular}]+r\cdot(refColor)$
\pagebreak

\begin{itemize}
    \item Se añade el término $r\cdot(refColor)$.
    \begin{itemize}
        \item r es un valor en [0,1] que mide la reflectividad.
        \item refColor es el color del rayo de reflexión e.
        \item Se repite el cálculo para el rayo e, es recursivo el método, de manera que se calcula el color con los sucesivos cálculos.
    \end{itemize}
\end{itemize}

El valor de refColor
\begin{itemize}
    \item Se lo trata como cualquier rayo de la cámara.
    \item Comprueba si interseca con algún objeto
    \begin{itemize}
        \item SI: iluminar normalmente y si es necesario, reflejarlo otra vez.
        \item NO: devuelve el color de fondo.
    \end{itemize}
\end{itemize}

Cuando solo se hace un paso de cálculo de color es Ray-casting, pero cuando sobre ese primer cálculo se repiten teniendo en cuenta los previos de manera recursiva es Ray-tracing.

Cuantas veces se debe hacer esta recursión o profundidad de algoritmo lo determina:
\begin{itemize}
    \item Establecer un límite de profundidad (depth)
        \begin{itemize}
            \item Los rayos desde la cámara tienen una profundidad de 0.
            \item Cada rayo hijo incrementa la profundidad en 1, por lo que no se lanzan más rayo si se supera el límite.
        \end{itemize}
    \item No lanzar más rayos si el coeficiente de reflexión es 0.
\end{itemize}

\subsection{Refracción}
La refracción es parecida a la reflexión

Cuando un rayo choca con una superficie, la ilumina según el algoritmo estándar y comprueba si hay que emitir un rayo de
refracción.
\begin{itemize}
    \item SI: calcula el nuevo rayo. Ilumínalo de la manera estándar y añade un término adicional a la ecuación de iluminación
\end{itemize}

Hay que almacenar índices de refracción para los materiales

Si el objeto es transparente, lanza un nuevo rayo usando la ley de Snell: $n_1 \sin \alpha_1 = n_2 \sin \alpha_2$
\begin{itemize}
    \item Las n son los coeficientes del medio y $\alpha$ los ángulos de intersección con la normal.
\end{itemize}

\subsection{Trazado de rayos recursivo (Whitted)}
Ventajas:
\begin{itemize}
    \item Gran realismo
    \item Lighting por iluminación directa (iluminación local)
    \item Cálculo de Sombras
    \item Incluye cierta iluminación indirecta (espejos y refracción), no en objetos difusos (paredes, techas)
    \item Incluso eliminación de superficies ocultas
\end{itemize}

Limitaciones:
\begin{itemize}
    \item Lento (1 rayo por pixel)
    \item Es susceptibles a problemas de precisión numérica, provocando diferencia entre píxeles contiguos
    \item Rayos de luz L no se refractan en su trayectoria hacia la luz (pasan a través de los objetos sin modificar el rayo, solo se modifica al volverse al lanzar desde la superficie)
    \item No incluye iluminación indirecta (o iluminación global) aparte de espejos y refracción. No incluye la luz que rebota en objetos de reflexión difusa.
\end{itemize}

No tiene iluminación indirecta (luz reflejada por superficies), utiliza el componente de ambiente.

\subsection{Modelos de rénder para iluminación global}
\subsubsection{Path tracing}
Similar a Ray Tracing, pero trazando todos los caminos, no solo los de espejos y refracción, sino también cuando la luz choca y rebota en superficies difusas o especulares.

\subsubsection{Radiosidad (Radiosity)}
\textbf{Radiosidad:} Tasa con la que la energía (emitida + reflejada) parte de una superficie. $\textit{tasa}= \frac{\textit{energía por unidad de tiempo}}{\textit{unidad de area}}$

Discretiza la \textbf{escena en patches (trozos)} y supone que \textbf{cada patch puede ser fuente de luz y/o reflejar luz de cualquier otro patch} en la escena. \textbf{Calcula la contribución de cada porción} a la iluminación de las restantes.

Tiene en cuenta \textbf{solo la reflexión difusa} (superficies mates). La cantidad de luz reflejada por un reflector difuso no depende de la dirección desde la que la observemos. Por tanto, el procesamiento que hace \textbf{radiosity es independiente del punto de vista} (a diferencia de Ray Tracing).

Shading por \textbf{interpolación para que no haya cambios demasiado bruscos} en patches adyacentes

Después habrá que generar la escena desde algún punto de vista y determinar cuáles son las superficies visibles

\textbf{Ecuación de Radiosidad:} $B_i = E_i + \rho_i \Sigma_{i \leq j \leq n} F_{i-j} B_j $
\begin{itemize}
    \item $B_i$ radiosidad del path $B_i$
    \item $E_i$ tasa con que $B_i$ emite luz (si $B_i$ es una fuente de luz directa)
    \item $\rho_i$ reflectividad del path $B_i$
    \item $F_{i-j} B_j$ radiosidad que alcanza $B_j$ desde $B_i$. Fracción de energía que parte de j y alcanza i por completo.
    \item $F_{i-j}$ factor de forma que conecta los patches i y j. Se calcula para cada pareja de patches, que depende de la orientación relativa entre ellos. Será 0 si no llega la luz.
\end{itemize}

Equivalente: $B_i - \rho_i \Sigma_{i \leq j \leq n} F_{i-j} B_j=E_i$.

Matricialmente: $(I - \rho F)B=E$, luego $B=(I-\rho F)^{-1} E$

Habiendo computado los factores de forma, se puede resolver el sistema. Es \textbf{muy lento cuando hay muchos patches}.

Para agilizar el proceso se puede usar otra estrategia, mediante soluciones parciales y aplicando un refinamiento progresivo. 
\begin{enumerate}
    \item Se colecta la energía del ambiente sobre el path i. 
    
    $B_i^{k+1} = E_i + \rho_i \Sigma_{i \leq j \leq n} F_{i-j} B_j^{k}$
    \item Se irradia la energía del patch i a todo el ambiente. 
    
    $For (j= i:n)\;\; B_j^{k+1} = B_j^{k} + \rho_j F_{j-i} \Delta B_i$
\end{enumerate}
\pagebreak

\section{Graphics pipeline}
Se ejecuta en hardware GPU (Graphics Processing Unit), hace una ejecución paralela masiva de cada una de las etapas. Cada triángulo o vértices, se proyecta sobre los píxeles.

Se llama tubería porque, por un lado, \textbf{entra el modelo} como geométrica, y por el otro, \textbf{salen los píxeles} de la imagen.

El proceso que ocurre entre medias es:
\begin{enumerate}
    \item \textbf{Procesamiento de vértices (Vertex shader)}
    \begin{itemize}
        \item Transformaciones (translación, rotación, escalado, ...) de sus coordenadas originales a las que tenga que ocupar.
        \item Lighting solo en vértices.
        \item Cambio al sistema de coordenadas de la cámara (x, y), para que se pueda ver lo que queremos antes de proyectar.
        \item Proyección a 2D (pantalla).
    \end{itemize}

    Es un programa escrito por el usuario que toma un vértice y devuelve un vértice transformado (proyectado sobre la pantalla). Se ejecuta en paralelo para cada vértice en la geometría.

    \textbf{Vector normal en el vértice, la malla de polígonos} aproxima una superficie, se calcula el vector normal unitario promedio en cada vértice del polígono como el \textbf{promedio de las normales de polígonos con ese vértice}. $n= \frac{\Sigma_i n_i}{|\Sigma_i n_i|}$
    \item \textbf{Rasterización}
    
    Convierte los vértices en la figura completa, en la pantalla, en forma de píxeles.

    Hay que \textbf{calcular que píxeles cubre un polígono}. Aunque en lugar de llamarlos \textbf{píxeles, les llamamos fragmento}, porque un pixel es un fragmento que se puede ver y pueden estar ocultos los fragmentos.

    Por fuerza bruta:
    \begin{itemize}
        \item Para cada pixel:
        \begin{itemize}
            \item Calcula las ecuaciones de la línea para el centro del pixel.
            \item Si todas dan > 0, el pixel está dentro (en sentido de estar dentro).
        \end{itemize}
        \item Si el triángulo es pequeño, se pierde tiempo.
        \item Mejora: 
        \begin{itemize}
            \item Hacer los cálculos solo para los puntos dentro de la caja que contiene el triángulo. Aunque todavía pierde tiempo en píxeles externos.
            \item Rasterizar los lados y rellenar.
        \end{itemize}
    \end{itemize}
    \item \textbf{Procesado de fragmentos (Fragment shader)}
    
    Shading/Interpolado. Colorea/ilumina cada uno de los fragmentos de un polígono.

    \textbf{Método de shading:} Constante (rápido), Gouraud (medio) y Phong (lento).
    \begin{itemize}
        \item \textbf{Constante (flat/facet shading):} Usa el lighting de un vértice (o centroide) para todos los fragmentos del triángulo. 
        
        Es muy rápido, pero para conseguir escenas realistas, las caras de los objetos deben estar formadas por polígonos muy pequeños.

        Funciona razonablemente bien cuando: Fuente luminosa muy lejos, Observador muy lejos o Polígono representa la superficie real modelada (no aproximación a una superficie curva).
        \item \textbf{Gouraud shading:} Interpolación de intensidades. El color de cada punto p se puede calcular como la interpolación de los colores de los vértices (I1, I2, I3). Cada vértice tiene una componente RGB.
        \begin{itemize}
            \item \textbf{Interpolar bi-linealmente} las intensidades de los vértices para hallar intensidades sobre cada pixel. Hace la media ponderada según la cercanía a los vértices.

            $I_a = (1-\alpha) \cdot I_1 + \alpha \cdot I_2 = I_1 - (I_1 - I_2) \cdot \alpha$. $\alpha$ es la proporción de distancia entre nuestro punto a un vértice y la distancia entre los dos vértices sobre uno de los ejes.
    
            Si queremos un punto interior calculamos los $I$ de los extremos en un eje y los usamos para el otro eje. Línea horizontal con $I_a$ e $I_b$ hallados con la y, y mediante estos dos en la x calculamos el punto entre $I_a$ e $I_b$, $I_p$.
            
            $$I_a =  I_1 - (I_1 - I_2) \cdot \frac{y_1-y_s}{y_1 - y_2};
            I_b =  I_1 - (I_1 - I_3) \cdot \frac{y_1-y_s}{y_1 - y_3};
            I_p =  I_b - (I_b - I_a) \cdot \frac{x_b - x_p}{x_b-x_a}$$
    
            \item\textbf{Interpolación baricéntrica}, computa las coordenadas ($\alpha, \beta, \gamma$) del punto p, y las usa para interpolar $I_1$, $I_2$ y $I_3$. 
            
            Los valores indican la influencia de cada vértice para calcular nuestro punto. 
            
            $0\leq \alpha, \beta, \gamma \leq 1$
    
            $$I_p=\alpha I_1 + \beta I_2 + \gamma I_3$$
        \end{itemize}
        \item \textbf{Phong shading:} Para que los objetos con superficies curvas aproximadas por triángulos se visualicen bien, se pueden interpolar las normales de los vértices.
        
        De esta manera, las normales de las caras tienen influencia por su proximidad a los vértices, que son el resultado de interpolarse con las de las caras previamente.

        Computacionalmente, más costoso, ya que tiene que recalcular las normales y posteriormente normalizarlas.

        \textbf{Interpolar bi-linealmente las normales} de los vértices en la superficie del polígono (también \textbf{interpolación baricéntrica}).

        $$n_a =  n_1 - (n_1 - n_2) \cdot \frac{y_1-y_s}{y_1 - y_2};
        n_b =  n_1 - (n_1 - n_3) \cdot \frac{y_1-y_s}{y_1 - y_3};
        n_p =  n_b - (n_b - n_a) \cdot \frac{x_b - x_p}{x_b-x_a}$$

        $$n_p=\alpha n_1 + \beta n_2 + \gamma n_3$$
    \end{itemize}


    \item \textbf{Testing and blending}
    
    Eliminación de superficies ocultas.
\end{enumerate}

Los shaders son las partes programables, las escritas por el usuario en lenguajes como OpenGL y ejecutados en la GPU. Son el Procesamiento de vértices y de fragmentos.

Las otras 2 partes, Rasterización y Test and blending, son fijas.