\chapter{Percepción Visual}
La percepción visual se divide en:
\begin{itemize}
	\item \textbf{Percepción física:} Cuando los fotones llegan a nuestro nervio óptico y se transforman en impulsos electromagnéticos que estimulan el cerebro.
	\item \textbf{Procesado e interpretación:} Se transforman los estímulos/impulsos en imagen.
\end{itemize}

Por las características físicas, los humanos no podemos verlo todo.

El sistema de interpretación puede construir imágenes a partir de información incompleta.

El procesamiento compensa el Movimiento y Variaciones en la luminosidad, pero el contexto es fundamental. Las expectativas resuelven ambigüedades.

\textbf{Movimiento:} Mínimo de 16 fotogramas/segundo para percibirlo. Al recibir las imágenes tan rápido sentimos que se mueven.
\begin{itemize}
	\item Cine mudo 16-18 Hz, Cine 24 Hz y TV 25-29 Hz.
\end{itemize}

\section{Ilusiones Ópticas}
El procesamiento visual puede crear ilusiones ópticas.
\begin{itemize}
	\item Nombres de colores de diferente color.
	\item Tamaño de objetos iguales, pero en una posición diferente, escaleras, flechas o platos.
	\item Objetos que se mueven o cambian al mirarlos.
\end{itemize}

\section{Percepción de la Luminosidad}
\textbf{Luminosidad:} Cantidad de luz que recibimos (subjetiva) y depende de:
\begin{itemize}
	\item \textbf{Luminancia:} Cantidad de luz emitida por un objeto.
	\item \textbf{Contraste:} Relación entre la luminancia de un objeto y la luminancia de su entorno.
\end{itemize}

\section{Percepción de Tamaño y Profundidad}
\textbf{Ángulo de visión:} Ángulo definido por los bordes del objeto percibido y el ojo. Los objetos familiares se perciben a tamaño constante.

\textbf{Agudeza visual:} Capacidad para percibir detalles. El límite está en un ángulo visual de 1/60 grados.

\textbf{Pistas visuales:} Estas ayudas permiten tener percepción del tamaño y profundidad.
\begin{itemize}
	\item Tamaño y posición relativa de los objetos.
	\item Más/menos definidos parecen estar más cerca o más lejos.
\end{itemize}

Percepción de tamaño con el mismo ángulo visual.

Tamaños iguales percibimos como diferentes - diferente tamaño relativo.

Tamaño iguales percibimos como diferentes - diferente profundidad.

Tamaño diferentes percibidos como iguales - diferente profundidad.

\section{Características de la Visión}
Percepción del tamaño y la profundidad
\begin{itemize}
	\item \textbf{Ángulo de visión:} Cantidad que un objeto ocupa el área visual. Permite determinar tamaño y distancia.
	\item \textbf{Agudeza visual:} Capacidad para percibir detalles.
	\item \textbf{Pistas:} Ayudas visuales que permiten tener percepción del tamaño y la profundidad.
\end{itemize}

Pistas visuales
\begin{itemize}
	\item \textbf{Contraste, claridad y brillo:} Objetos más definidos parecen más cercanos.
	\item \textbf{Sombras:} Pistas adicionales de la posición relativa.
	\item \textbf{Texturas:} Menos definidas cuando más alejados del observador.
\end{itemize}
