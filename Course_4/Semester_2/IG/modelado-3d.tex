\chapter{Proyecciones, Objetos 3D y GSD}
\section{Proyecciones}
\textbf{Proyección:} Transformación matemática que convierte un conjunto de puntos n-dimensional en un conjunto de puntos k-dimensional, siendo $k \leq n$. Reduce la dimensionalidad, normalmente 3D a 2D.

La proyección se define por unas \textbf{líneas de proyección} (proyectores) que, dirigidos hacia el \textbf{centro de proyección}, pasan a través del objeto e intersecan el \textbf{plano de proyección/superficie de proyección} para formar la proyección del objeto en el mismo.

Proyecciones Geométricas Planas
\begin{itemize}
    \item Paralela
    \begin{itemize}
        \item Ortográfica
        \begin{itemize}
            \item Elevaciones, las proyecciones ortográficas en tres planos.
            \item Axonométrica, el objeto es rotado respecto a uno o más ejes del plano de proyección.
            \begin{itemize}
                \item Isométrica
            \end{itemize}
        \end{itemize}
        \item Oblicua
        \begin{itemize}
            \item Cabinet, eje de profundidad (z) forma 45 grados con el eje x. Profundidad escalada (1/2, 2/3, etc.).
            \item Caballera, eje de profundidad (z) forma 45 grados con el eje x. Profundidad real.
        \end{itemize}
    \end{itemize}
    \item perspectiva
    \begin{itemize}
        \item 1 punto
        \item 2 puntos
        \item 3 puntos
    \end{itemize}
\end{itemize}

Métodos de generar vistas de objetos:
\begin{itemize}
    \item \textbf{Proyección paralela:} El centro de proyección en el infinito, los proyectores son paralelos hacia el centro.
    \begin{itemize}
        \item Mantiene la forma y dimensión de los objetos.
        \item Representación no realista de los objetos, se usan para el diseño.
        \item Conservan las propiedades relativas.
    \end{itemize}
    \begin{figure}[H]
        \ffigbox[\FBwidth]
        {\caption{Proyección perspectiva}}
        {\def\svgwidth{.5\textwidth}
            \input{./img/perspectiva.eps_tex}}
    \end{figure}
    \item \textbf{Proyección perspectiva:} Hay un centro de proyección cercano.
    \begin{itemize}
        \item Altera la forma y dimensión de los objetos.
        \item Representación realista de los objetos, simula la visión humana.
        \item No conserva las proporciones
    \end{itemize}
    \vspace{-5cm}
    \begin{figure}[H]
        \ffigbox[\FBwidth]
        {\caption{Proyección paralela}}
        {\def\svgwidth{.5\textwidth}
            \input{./img/paralela.eps_tex}}
    \end{figure}
\end{itemize}

\subsection{Visión Estereográfica}
Se obtiene realizando dos proyecciones en perspectiva, ambas imágenes se combinan para dar una tridimensional. Es el caso de los ojos.

\subsection{Proyecciones no planas}
\begin{itemize}
    \item Proyección esférica: Imagen proyectada sobre una esfera.
    \item Proyección fisheye (ojo de pez, media esfera): Proyección esférica con un ángulo de 180 grados (estándar) o 360 grados (ultra). La proyección sobre un círculo.
    \item Proyección ultra wide angle: Fisheye en el que la proyección es un rectángulo.
    \item Proyección Omnimax: Fisheye con un ángulo de visión reducido en el eje vertical.
    \item Proyección cilíndrica: Imagen proyectada sobre un cilindro.
    \item Proyección panorámica: Tipo de proyección cilíndrica capaz de ver ángulos superiores a 180 grados.
\end{itemize}

\section{Representación de Objetos 3D}
Definir estructuras de datos 3D capaces de representar las propiedades geométricas y físicas.

Debemos ser capaces de representar el interior, el exterior y el comportamiento de la superficie.

Aproximaciones al problema:
\begin{itemize}
    \item Definición de la superficie exterior: Modelos de superficie, solo interesa el contorno (exterior).
    \item Definición del espacio ocupado por el objeto: Modelado Sólido, para saber exterior, interior y propiedad de los objetos.
\end{itemize}

Requisitos:
\begin{itemize}
    \item Facilidad en la generación de objetos.
    \item Calidad en el aspecto final de la representación, según las restricciones.
    \item Recursos computacionales de la aplicación.
\end{itemize}

Representación de Objetos 3D
\begin{itemize}
    \item Modelos de Superficie
    \begin{itemize}
        \item Poligonales
        \item Matemáticos
        \begin{itemize}
            \item Superficies Implícitas
            \item Superficies Paramétricas
        \end{itemize}
    \end{itemize}
    \item Modelo Sólido
    \begin{itemize}
        \item Modelos de Barrido
        \item Geometría Sólida Constructiva (GSC)
    \end{itemize}
    \item Modelos de Partición Espacial
\end{itemize}

\subsection{Modelos de Superficie}
Se emplea cuándo
\begin{itemize}
    \item El objeto a representar es muy similar a una superficie.
    \item Solo interesa el aspecto externo del objeto.
    \item El tipo de biblioteca gráfica solo soporta este modelo (no hay más remedio).
\end{itemize}

Para modelar superficies debemos asegurarnos de que las superficies no se intersecan consigo mismas.

Cuando la superficie es cerrada se denominan a los objetos modelos de frontera $\rightarrow$ posibilidad de cambio con modelos sólidos.

Aproximaciones al modelado de superficies:
\begin{itemize}
    \item Discreta y aproximada $\rightarrow$ modelo poliédrico
    \item Continua y exacta $\rightarrow$ modelo matemático
\end{itemize}

\subsubsection{Modelos de Superficie: Poligonales}
