\chapter{Proyecciones, Objetos 3D y GSD}
\section{Proyecciones}
\textbf{Proyección:} Transformación matemática que convierte un conjunto de puntos n-dimensional en un conjunto de puntos k-dimensional, siendo $k \leq n$. Reduce la dimensionalidad, normalmente 3D a 2D.

La proyección se define por unas \textbf{líneas de proyección} (proyectores) que, dirigidos hacia el \textbf{centro de proyección}, pasan a través del objeto e intersecan el \textbf{plano de proyección/superficie de proyección} para formar la proyección del objeto en el mismo.

Proyecciones Geométricas Planas
\begin{itemize}
    \item Paralela
    \begin{itemize}
        \item Ortográfica
        \begin{itemize}
            \item Elevaciones, las proyecciones ortográficas en tres planos.
            \item Axonométrica, el objeto es rotado respecto a uno o más ejes del plano de proyección.
            \begin{itemize}
                \item Isométrica
            \end{itemize}
        \end{itemize}
        \item Oblicua
        \begin{itemize}
            \item Cabinet, eje de profundidad (z) forma 45 grados con el eje x. Profundidad escalada (1/2, 2/3, etc.).
            \item Caballera, eje de profundidad (z) forma 45 grados con el eje x. Profundidad real.
        \end{itemize}
    \end{itemize}
    \item perspectiva
    \begin{itemize}
        \item 1 punto
        \item 2 puntos
        \item 3 puntos
    \end{itemize}
\end{itemize}

Métodos de generar vistas de objetos:
\begin{itemize}
    \item \textbf{Proyección paralela:} El centro de proyección en el infinito, los proyectores son paralelos hacia el centro.
    \begin{itemize}
        \item Mantiene la forma y dimensión de los objetos.
        \item Representación no realista de los objetos, se usan para el diseño.
        \item Conservan las propiedades relativas.
    \end{itemize}
    \begin{figure}[H]
        \ffigbox[\FBwidth]
        {\caption{Proyección perspectiva}}
        {\def\svgwidth{.5\textwidth}
            \input{./img/perspectiva.eps_tex}}
    \end{figure}
    \item \textbf{Proyección perspectiva:} Hay un centro de proyección cercano.
    \begin{itemize}
        \item Altera la forma y dimensión de los objetos.
        \item Representación realista de los objetos, simula la visión humana.
        \item No conserva las proporciones
    \end{itemize}
    \vspace{-5cm}
    \begin{figure}[H]
        \ffigbox[\FBwidth]
        {\caption{Proyección paralela}}
        {\def\svgwidth{.5\textwidth}
            \input{./img/paralela.eps_tex}}
    \end{figure}
\end{itemize}

\subsection{Visión Estereográfica}
Se obtiene realizando dos proyecciones en perspectiva, ambas imágenes se combinan para dar una tridimensional. Es el caso de los ojos.

\subsection{Proyecciones no planas}
\begin{itemize}
    \item Proyección esférica: Imagen proyectada sobre una esfera.
    \item Proyección fisheye (ojo de pez, media esfera): Proyección esférica con un ángulo de 180 grados (estándar) o 360 grados (ultra). La proyección sobre un círculo.
    \item Proyección ultra wide angle: Fisheye en el que la proyección es un rectángulo.
    \item Proyección Omnimax: Fisheye con un ángulo de visión reducido en el eje vertical.
    \item Proyección cilíndrica: Imagen proyectada sobre un cilindro.
    \item Proyección panorámica: Tipo de proyección cilíndrica capaz de ver ángulos superiores a 180 grados.
\end{itemize}

\section{Representación de Objetos 3D}
Definir estructuras de datos 3D capaces de representar las propiedades geométricas y físicas.

Debemos ser capaces de representar el interior, el exterior y el comportamiento de la superficie.

Aproximaciones al problema:
\begin{itemize}
    \item Definición de la superficie exterior: Modelos de superficie, solo interesa el contorno (exterior).
    \item Definición del espacio ocupado por el objeto: Modelado Sólido, para saber exterior, interior y propiedad de los objetos.
\end{itemize}

Requisitos:
\begin{itemize}
    \item Facilidad en la generación de objetos.
    \item Calidad en el aspecto final de la representación, según las restricciones.
    \item Recursos computacionales de la aplicación.
\end{itemize}

Representación de Objetos 3D
\begin{itemize}
    \item Modelos de Superficie
    \begin{itemize}
        \item Poligonales
        \item Matemáticos
        \begin{itemize}
            \item Superficies Implícitas
            \item Superficies Paramétricas
        \end{itemize}
    \end{itemize}
    \item Modelo Sólido
    \begin{itemize}
        \item Modelos de Barrido
        \item Geometría Sólida Constructiva (GSC)
    \end{itemize}
    \item Modelos de Partición Espacial
\end{itemize}

\subsection{Modelos de Superficie}
Se emplea cuándo
\begin{itemize}
    \item El objeto a representar es muy similar a una superficie.
    \item Solo interesa el aspecto externo del objeto.
    \item El tipo de biblioteca gráfica solo soporta este modelo (no hay más remedio).
\end{itemize}

Para modelar superficies debemos asegurarnos de que las superficies no se intersecan consigo mismas.

Cuando la superficie es cerrada se denominan a los objetos modelos de frontera $\rightarrow$ posibilidad de cambio con modelos sólidos.

Aproximaciones al modelado de superficies:
\begin{itemize}
    \item Discreta y aproximada $\rightarrow$ modelo poliédrico
    \item Continua y exacta $\rightarrow$ modelo matemático
\end{itemize}

\subsubsection{Modelos de Superficie: Poligonales}
Describir una superficie a partir de un conjunto de polígonos conectados mediante vértices y aristas.

Los polígonos se intersecan únicamente en las artistas
\begin{itemize}
    \item Una \textbf{arista} es compartida únicamente por dos polígonos.
    \item Un \textbf{vértice} pertenece a dos o más aristas
\end{itemize}

Cualquier forma 2D o superficie 3D puede aproximarse por polígonos, pero la calidad de la representación mejora a medida que aumentamos el número de polígonos.

\textbf{Ventajas}
\begin{itemize}
    \item Simplicidad 
    \item Fácil y rápido de renderizar 
    \item Ocupa poca memoria
\end{itemize}

\textbf{Limitaciones}
\begin{itemize}
    \item La naturaleza no es poligonal.
    \item Aliasing de polígonos, alisamiento para quitar bordes de sierra, debido a la discretización.
    \item No es sencillo modelar objetos complejos.
\end{itemize}

Normalmente, se trabaja a partir de primitivas poligonales optimizadas, por su reducción de la trasferencia de datos, el problema es la perdida de flexibilidad.
\begin{itemize}
    \item Tira de cuadrados / Tira de triángulos
    \item Matriz de cuadrado / Abanico de triángulos
\end{itemize}

La primitiva poligonal más habitual es el \textbf{Triangulo}, es siempre convexo, matemáticamente simple y siempre coplanares. Además, cualquier polígono puede ser descompuesto en triángulos.

PovRay utiliza triángulos para definir superficies: Mesh almacena eficientemente un número de triángulos y Mesh2 utilizado para la conversión entre formatos gráficos. 

\subsubsection{Modelos de Superficie: Matemáticos}
Ecuaciones matemáticas describen las superficies
\begin{itemize}
    \item \textbf{Superficies Implícitas} $F( x, y, z) = 0$. 
    \begin{itemize}
        \item \textbf{Cuádricas}
        
        Primitivas matemáticas que responden a la ecuación, que dependiendo de los coeficientes será una forma u otra:
        $$a\cdot x^2+b\cdot y^2+c\cdot z^2+2d\cdot xy+2e\cdot yz+2f\cdot xz+2g\cdot z+2h\cdot y+2j\cdot z+k=0$$
        Limitada variedad de las formas (no es simple extenderlo a representación de sólidos) y no sirven para modelar elementos naturales.

        \item \textbf{Superficies equipotenciales o Isosuperficies}
        
        Útiles para representación de formas suaves

        Cada superficie queda definida por el conjunto de puntos con un determinado campo, se modela según el equilibrio entre campos de potencia (equipotencial).

        Al interactuar objetos se crean superficies suaves. El modelado se basa en mover los puntos o elementos que generan el campo, habitualmente los campos son esféricos o cilíndricos.

        \item \textbf{Blobs:} Globulares de formas abultadas.
        $$f(x,y,z)=\Sigma_k b_k e^{-r_k^2a_k}-T=0$$
        T = punto iniciar. a = abultamiento (positivo). b = hendidura (negativos).
        \item \textbf{Metaballs y Softobjects:} Bultos gaussianos
        $$f(r)=\begin{cases}b(1-3r^2/d^2) & si\; r \in [0,d/3]\\ \frac 3 2 b(1-r/d)^2 & si\; r \in [d/3,d] \\ 0 & si\; r>d\end{cases}$$
    \end{itemize} 
    \item \textbf{Superficies Paramétricas} $( x, y, z) = f( u, v)$. Bezier, B-splines y NURBS.
    
    Características
    \begin{itemize}
        \item Facilidad de modelado de formas libres: Control local de deformaciones y Posibilidades de cálculo de tangencias y curvatura
        \item Buenas prestaciones en transformaciones a modelos poligonales 
        \item Buenas prestaciones en cuanto a almacenamiento
    \end{itemize}
    Dos aproximaciones
    \begin{enumerate}
        \item Con superficie que pasa por un conjunto de puntos $\rightarrow$ \textbf{Superficies de Interpolación}
        \item Un conjunto de puntos controlan la forma de la superficie, que no pasa necesariamente por ellos $\rightarrow$ \textbf{Superficies de aproximación}
    \end{enumerate}
    Podemos aproximar cualquier curva unidimensional
    \begin{itemize}
        \item La más básica se basa en el uso de funciones polinómicas $C(u)=\Sigma^n_{i=0} a_iu^i$
        \item La curva depende del grado del polinomio: para definir una curva con n puntos es necesario un polinomio de grado n (problema!!!)
        \item Otra posibilidad consiste en definir la curva como combinación de trozos de curvas: splines. Cuidando los grados de continuidad.
    \end{itemize}
    Tipos:
    \begin{itemize}
        \item \textbf{Splines naturales} 
        
        Normalmente se basan en polinomios de grado cúbico (n=3 en la ecuación), donde se exige continuidad y diferenciabilidad doble.

        Por tanto, la curva consiste en n-1 trozos de grado 3, donde n es el número de puntos por los que pasa la curva.
        \item \textbf{Curvas de Bézier}
        
        Solamente pasan por los puntos extremos y cumplen la propiedad de cierre convexo del polígono de control.
        
        Se definen a partir de la combinación de una familia especial de polinomios: los polinomios de Berstein

        Dos aproximaciones: Curvas de Bezier basadas en polinomios de grado n y de las definidas a trozos.

        Curva de Bezier generales: \\$C(u)=\Sigma^n_{i=0}B_{i,n}(u)P_i \;\; 0 \leq u \leq 1$ con \\$B_{i,n}(u)=\frac{n!}{i!(n-1)!} u^i(1-u)^{n-i}$  con $\Sigma^n_{i,n}(u)=1 \forall u$
        \item \textbf{B-splines}
        
        Similar a las splines naturales, pero sin interpolación. Se une el espacio paramétrico de los diferentes trozos. Esto se realiza definiendo un vector de nudos.
        
        Ventajas
        \begin{enumerate}
            \item Se puede seleccionar el grado de los polinomios base para controlar la suavidad de la curva (grado d)

            \item Permiten control local sobre la forma de la superficie
        \end{enumerate}
        $C(u)=\Sigma^n_{i=0}N_{i,d}(u)P_i$ $u_{min}<u<u_{max}, 2 \leq d \leq n+1$

        $N_{i,1}(u)=$ 1 si $u_i < u < u_{i+1}$ en caso contrario $0$.

        $N_{i,d}(u)= \frac{u-u_i}{u_{i+d-1}-u_i}N_{i,d-1}(u)+\frac{u_{i+d}-u}{u_{i+d}-u_{u+1}}N_{i+1,d-1}(u)$

        Según la distribución de vectores de nudos:
        \begin{itemize}
            \item \textbf{B-splines Racionales} Se definen a partir de la razón o promedio de los polinomios base. Se pueden representar cuádricas y otras superficies de manera exacta.
            $$C(u)=\frac{\Sigma_{i=0}^n w_i N_{i,d}(u)P_i}{\Sigma_{i=0}^n w_i N_{i,d}(u)}$$
            \item \textbf{NURBS-Non Uniform Rational B-Splines:} Cuando podemos tener cualquier distribución del vector de nudos.
        \end{itemize}     
    \end{itemize}
    \item \textbf{Superficies Explícitas} $Z = f( x, y)$, se tratan como superficies paramétricas
\end{itemize}

\subsection{Modelado Sólido}
\subsubsection{Modelos de Barrido}
\begin{enumerate}
    \item Definición de una figura bidimensional (por sus bordes)
    \item Desplazamiento de la figura bidimensional a lo largo de un camino y los bordes generan la superficie.
    \begin{itemize}
        \item Rotación respecto de un eje “Superficie de revolución” 
        \item Desplazamiento a lo largo de una línea recta “Extrusión”
    \end{itemize}
\end{enumerate}
\subsubsection{Geometría Sólida Constructiva (GSC)}
La combinación de primitivas u objetos mediante operaciones booleanas, produciendo objetos más complejos. 

Los objetos diseñados con CSG se representa mediante un árbol binario con nodos y operaciones 

Las operaciones básicas son: Unión, Diferencia, Intersección y Fusión.