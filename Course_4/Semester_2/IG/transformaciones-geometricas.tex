\chapter{Transformaciones Geométricas}
\section{Introducción}
\textbf{Transformación geométrica:} Variación del tamaño, forma, posición y orientación de un objeto dentro de una escena.
\begin{itemize}
    \item Transformaciones geométricas afines en 2D y 3D.
    \begin{itemize}
        \item Transformaciones lineales y traslaciones:
        \begin{itemize}
            \item Traslación
            \item Escalamiento 
            \item Rotación
        \end{itemize}
    \end{itemize}
    \item Transformaciones no afines.
\end{itemize}

\section{Aplicaciones en Informática Gráfica}
Para modelar objetos.
\begin{itemize}
    \item Ejem. Blender se puede partir de un cubo, escalarlo en una dirección y rotarlo para formar una columna.
\end{itemize}

Para crear objetos complejos.

Para animar objetos, traslaciones y rotaciones son los elementos básicos del movimiento.

\section{Transformaciones lineales}
Una transformación es una función: $v(x,y,z) \rightarrow v'(x',y',z')$.

Una transformación es lineal, si se cumple:
\begin{itemize}
    \item $T(u+v) = T(u) + T(v)$ La transformación del vector suma es la suma de las transformaciones.
    \item $T(x\cdot u) = x\cdot T(u)$, siendo x un escalar.
\end{itemize}

\textbf{Son lineales:} Rotación, Escalado y Estiramiento (shear).

En las transformaciones lineales se cumple que las líneas paralelas siguen siendo paralelas (los ángulos no tienen por qué mantenerse).

El origen de coordenadas se mantiene.

\textbf{No es lineal:} Traslación.

En la traslación se mantiene el paralelismo, pero el origen de coordenadas no se mantiene.

La traslación no es una transformación lineal, pero sí es una transformación afín.

\textbf{Transformación afín:} transformación lineal y traslación.

\section{Transformaciones 2D}
\subsection{Forma General}
Las transformaciones lineales se pueden representar con matrices cuadradas. La traslación también, pero usando coordenadas homogéneas (una dimensión más).

$$\left[\begin{matrix}
x' \\ y'
\end{matrix}\right] =
\left[\begin{matrix}
a & b \\ c & d
\end{matrix}\right]
\left[\begin{matrix}
x \\ y
\end{matrix}\right]=
\left[\begin{matrix}
ax+by \\ cx+dy
\end{matrix}\right]$$
\begin{itemize}
    \item $a$ es eje x.
    \item $d$ es eje y.
    \item $b$ es deformación en x.
    \item $c$ es deformación en y.
\end{itemize}

\subsection{Transformación lineal 2D}
Vector como combinación lineal de los vectores base $i$, $j$. Cuando se haga una transformación cambia el vector base.
\begin{itemize}
    \item Ejem. $v (-1,2) = -1 i + 2 j$
\end{itemize} 
$i$ es la primera columna de la matriz de transformación y $j$ la segunda.

\subsection{Traslación}
Cambio de la posición de un objeto a lo largo de una línea recta.

Desplaza cada punto de una figura o espacio la misma cantidad en una determinada dirección, vector de traslación $(d_x,d_y)$. $d_x$ unidades paralelas al eje x y $d_y$ unidades paralelas al eje y.

$$\left[\begin{matrix}
x' \\ y'
\end{matrix}\right] =
\left[\begin{matrix}
x \\ y
\end{matrix}\right]+
\left[\begin{matrix}
d_x \\ d_y
\end{matrix}\right]=
\left[\begin{matrix}
x+d_x \\ y+d_y
\end{matrix}\right]$$

NO es una transformación lineal, aunque si afín. La transformación de la suma de dos vectores no es la suma de las transformaciones.

\subsection{Escalado}
Alteración de las dimensiones de un objeto:
\begin{itemize}
    \item Segmentos paralelos al eje x se multiplican por $s_x$.
    \item Segmentos paralelos al eje y se multiplican por $s_y$.
\end{itemize}

Cuando $s_x = s_y$ es un escalado uniforme.

$$\left[\begin{matrix}
x' \\ y'
\end{matrix}\right] =
\left[\begin{matrix}
s_x & 0 \\ 0 & s_y
\end{matrix}\right] \cdot
\left[\begin{matrix}
x \\ y
\end{matrix}\right]=
\left[\begin{matrix}
x\cdot s_x \\ y\cdot s_y
\end{matrix}\right]$$

\subsection{Rotación}
Cambiar la orientación de un objeto $\theta$ respecto del origen.

$\phi$ es la posición angular del punto $P(x, y)$ respecto al origen.

Los vectores base pasarán a ser:
\begin{itemize}
    \item $i'=(1 \cos \theta , 1 \sin \theta)$
    \item $j'=(-1 \sin \theta, 1 \cos \theta)$
\end{itemize}

$$\left[\begin{matrix}
x' \\ y'
\end{matrix}\right] =
\left[\begin{matrix}
\cos \theta & - \sin \theta \\ \sin \theta & \cos \theta
\end{matrix}\right] \cdot
\left[\begin{matrix}
x \\ y
\end{matrix}\right]=
\left[\begin{matrix}
x \cos \theta - y \sin \theta \\ x \sin \theta + y \cos \theta
\end{matrix}\right]$$

\subsection{Cuidado con el orden de las transformaciones}
Secuencia Rotación y Escalado no es equivalente a Escalado y Rotación.

Se multiplica de Derecha a Izquierda.

No se cumple la propiedad conmutativa, pero si la asociativa.
\begin{itemize}
    \item $A2 = S \cdot R \cdot A$
    \item $A2 = (S \cdot R) \cdot A$
\end{itemize}

Aunque existen casos especiales donde al aplicar transformaciones de traslación, escalado y rotación, el producto de matrices es conmutativo, como:
\begin{itemize}
    \item Traslación seguida de traslación.
    \item Escalado seguido de escalado
    \item Rotación seguida de rotación
    \item Rotación y escalado uniforme
\end{itemize}

Se busca ejecutar la secuencia de transformaciones más eficientemente, para esto necesitamos un sistema de referencia que sea homogéneo y entre la traslación.

\section{Sistema de Coordenadas Homogéneas 2D}
Un punto $( x , y )$ en el sistema cartesiano se representa por $( xW , yW , W )$ en el sistema homogéneo, donde $W\neq 0$ puede ser cualquier número real (geometría proyectiva).

Cuando $W\neq 1$, para simplificar cálculos, dividimos todas las coordenadas por W, obteniendo $( x , y , 1 )$ (W suele utilizarse en el eje z profundidad).

Cuando $W = 0$, estamos realizando transformaciones del punto $( x , y )$ sobre el plano $z = 1$.

Las matrices que representan puntos en 2D serán de 3x3.

\subsection{Traslación en coordenadas homogéneas}
Ahora la traslación podrá hacerse representarse como una matriz.

$$\left[\begin{matrix}
x' \\ y' \\ 1
\end{matrix}\right] =
\left[\begin{matrix}
1 & 0 & d_x \\ 0 & 1 & d_y \\ 0 & 0 & 1
\end{matrix}\right] \cdot
\left[\begin{matrix}
x \\ y \\ 1
\end{matrix}\right]=
\left[\begin{matrix}
x + d_x \\ y + d_y \\ 1
\end{matrix}\right]$$

Es como el bias de las redes de neuronas, el 1 se multiplica por un desplazamiento.

Ahora se puede hacer composición de traslaciones.

$P'=T(d_{x1},d_{y1}) \cdot P; \;\;P''=T(d_{x2},d_{y2}) \cdot P'$

$P''=T(d_{x2},d_{y2}) \cdot (T(d_{x1},d_{y1}) \cdot P)=(T(d_{x2},d_{y2}) \cdot T(d_{x1},d_{y1})) \cdot P$

$$\left[\begin{matrix}
1 & 0 & d_{x1} \\ 0 & 1 & d_{y1} \\ 0 & 0 & 1
\end{matrix}\right] \cdot
\left[\begin{matrix}
1 & 0 & d_{x2} \\ 0 & 1 & d_{y2} \\ 0 & 0 & 1
\end{matrix}\right]=
\left[\begin{matrix}
1 & 0 & d_{x1}+d_{x2} \\ 0 & 1 & d_{y1}+d_{y2} \\ 0 & 0 & 1
\end{matrix}\right]$$

\subsection{Escalado en coordenadas homogéneas}
$$\left[\begin{matrix}
x' \\ y' \\ 1
\end{matrix}\right] =
\left[\begin{matrix}
s_x & 0 & 0 \\ 0 & s_y & 0 \\ 0 & 0 & 1
\end{matrix}\right] \cdot
\left[\begin{matrix}
x \\ y \\ 1
\end{matrix}\right]=
\left[\begin{matrix}
x\cdot s_x \\ y\cdot s_y \\ 1
\end{matrix}\right]$$

$$\left[\begin{matrix}
    s_{x2} & 0 & 0 \\ 0 & s_{y2} & 0 \\ 0 & 0 & 1
\end{matrix}\right] \cdot
\left[\begin{matrix}
    s_{x1} & 0 & 0 \\ 0 & s_{y1} & 0 \\ 0 & 0 & 1
\end{matrix}\right]=
\left[\begin{matrix}
s_{x1} \cdot s_{x2} & 0 & 0 \\ 0 & s_{y1} \cdot s_{y2} & 0 \\ 0 & 0 & 1
\end{matrix}\right]$$

\subsection{Rotación en coordenadas homogéneas}
$$\left[\begin{matrix}
x' \\ y' \\ 1
\end{matrix}\right] =
\left[\begin{matrix}
\cos \theta & - \sin \theta & 0 \\ \sin \theta & \cos \theta & 0 \\ 0 & 0 & 1
\end{matrix}\right] \cdot
\left[\begin{matrix}
x \\ y \\ 1
\end{matrix}\right]=
\left[\begin{matrix}
x \cos \theta - y \sin \theta \\ x \sin \theta + y \cos \theta \\ 1
\end{matrix}\right]$$

$$\left[\begin{matrix}
\cos \alpha & - \sin \alpha & 0 \\ \sin \alpha & \cos \alpha & 0 \\ 0 & 0 & 1
\end{matrix}\right] \cdot
\left[\begin{matrix}
\cos \beta & - \sin \beta & 0 \\ \sin \beta & \cos \beta & 0 \\ 0 & 0 & 1
\end{matrix}\right]=
\left[\begin{matrix}
\cos (\alpha + \beta ) & - \sin (\alpha + \beta ) & 0 \\ \sin (\alpha + \beta ) & \cos (\alpha + \beta ) & 0 \\ 0 & 0 & 1
\end{matrix}\right]$$

\subsection{Transformaciones inversas}
\textbf{Traslación} $T( d_x, d_y)$ es: $T^{-1} ( d_x , d_y ) = T( -d_x , -d_y )$

\textbf{Escalado} $S( s_x , s_y )$ es: $S^{-1} ( s_x , s_y ) = S(1/s_x ,1/s_y )$

\textbf{Rotación} $R(T)$ es: $R^{-1}(\theta) = R(-\theta)$ que equivale a cambiar los signos de los senos de la transformación normal.

\subsection{Transformaciones generales afines}
Producto de una secuencia arbitraria de matrices de rotación, traslación y escalamiento.

Propiedad de conservar el paralelismo de las líneas, pero no longitudes ni ángulos. Rotaciones, escalamientos y traslaciones subsiguientes no podrían hacer que las líneas dejen de ser paralelas (nótese que sí permiten deformar el objeto).

\subsection{Estiramiento (Shear)}
El estiramiento puede realizarse respecto de cualquier eje.

$$SH_x = \left[\begin{matrix}
1 & a & 0 \\ 0 & 1 & 0 \\ 0 & 0 & 1
\end{matrix}\right] \;\;\;
SH_y = \left[\begin{matrix}
1 & 0 & 0 \\ b & 1 & 0 \\ 0 & 0 & 1
\end{matrix}\right]$$

\subsection{Composición de transformaciones}
El objetivo es ganar eficiencia aplicando una sola transformación compuesta a un punto, como es la aplicación de una rotación a un objeto sobre un punto P, que no es el origen.
\begin{enumerate}
    \item Traslación al origen $T(-x_1, -y_1)$
    \item Rotación $R(\theta)$
    \item Traslación al punto P $T(x_1, y_1)$
\end{enumerate}

\subsection{Operaciones respecto a un punto $P_1(x_1,y_1)$}
Recordar que se hace las operaciones de derecha a izquierda.

\subsubsection{Rotación}
$$T(x_1, y_1)\cdot R(\theta) \cdot T(x_1, y_1) = 
\left[\begin{matrix}
\cos \theta & - \sin \theta & x_1(1-\cos \theta)+y_1 \sin \theta \\ \sin \theta & \cos \theta & y_1(1-\cos \theta)-x_1 \sin \theta \\ 0 & 0 & 1
\end{matrix}\right]$$

\subsubsection{Escalado}
$$T(x_1, y_1)\cdot S(s_x, s_y) \cdot T(x_1, y_1) = 
\left[\begin{matrix}
s_x & 0 & x_1(1-s_x) \\ 0 & s_y & y_1(1-s_y) \\ 0 & 0 & 1
\end{matrix}\right]$$

\subsubsection{Escalado y Rotación}
$M=T(x_1, y_1) \cdot R(\theta) \cdot S(s_x, x_y) \cdot T(-x_1, -y_1)$

\section{Sistema de Coordenadas Homogéneas 3D}
Un punto $(x, y, z)$ en el sistema cartesiano se representa por $( xW , yW , zW, W )$ en el sistema homogéneo, donde $W\neq 0$ puede ser cualquier número real.

Cuando $W\neq 1$, para simplificar cálculos, dividimos todas las coordenadas por W, obteniendo $( x , y , z, 1 )$

\subsection{Traslación}
$$T(d_x, d_y, d_z) \cdot
\left[\begin{matrix}
x \\ y \\ z \\ 1
\end{matrix}\right] = 
\left[\begin{matrix}
1 & 0 & 0 & d_x \\ 0 & 1 & 0 & d_y \\ 0 & 0 & 1 & d_z \\ 0 & 0 & 0 & 1
\end{matrix}\right] \cdot
\left[\begin{matrix}
x \\ y \\ z \\ 1
\end{matrix}\right] = 
\left[\begin{matrix}
x+ d_x \\ y+d_y \\ z+d_z \\ 1
\end{matrix}\right]$$

\subsection{Escalado}
$$S(s_x, s_y, s_z) \cdot
\left[\begin{matrix}
x \\ y \\ z \\ 1
\end{matrix}\right] = 
\left[\begin{matrix}
s_x & 0 & 0 & 0 \\ 0 & s_y & 0 & 0 \\ 0 & 0 & s_z & 0 \\ 0 & 0 & 0 & 1
\end{matrix}\right] \cdot
\left[\begin{matrix}
x \\ y \\ z \\ 1
\end{matrix}\right] = 
\left[\begin{matrix}
x\cdot s_x \\ y\cdot s_y \\ z\cdot s_z \\ 1
\end{matrix}\right]$$

\subsection{Rotación}
$$R_x(\theta)=\left[\begin{matrix}
1 & 0 & 0 & 0 \\ 
0 & \cos \theta & - \sin \theta & 0 \\ 
0 & \sin \theta & \cos \theta & 0 \\ 
0 & 0 & 0 & 1
\end{matrix}\right]
R_y(\theta)=\left[\begin{matrix}
\cos \theta & 0 & \sin \theta & 0 \\ 
0 & 1 & 0 & 0 \\ 
- \sin \theta & 0 & \cos \theta & 0 \\ 
0 & 0 & 0 & 1
\end{matrix}\right]
R_z(\theta)=\left[\begin{matrix}
\cos \theta & -\sin \theta & 0 & 0 \\ 
\sin \theta & \cos \theta & 0 & 0 \\ 
0 & 0 & 1 & 0 \\
0 & 0 & 0 & 1
\end{matrix}\right]$$

\subsection{Rotación en coordenadas homogéneas}
La composición de una secuencia arbitraria de rotaciones con respecto a los ejes x, y, z como submatriz 3x3 ortogonal. La inversa de una matriz ortogonal es su transpuesta.
$$R_z(\theta)=\left[\begin{matrix}
r11 & r12 & r13 & 0 \\ 
r21 & r22 & r23 & 0 \\ 
r31 & r32 & r33 & 0 \\
0 & 0 & 0 & 1
\end{matrix}\right]$$
\pagebreak

\section{Rotación alrededor de cualquier eje}
\begin{itemize}
    \item Primero movemos el eje hasta el eje z.
    \begin{enumerate}
        \item Rotamos alrededor de x $R_x$
        \item Rotamos alrededor de y $R_y$
    \end{enumerate}
    \item Después, aplicamos la rotación deseada seguida de las inversas usadas para colocarlo sobre el eje z.
    \begin{enumerate}
        \item Rotamos lo deseado $R_z$
        \item Deshacemos la rotación $R_x^{-1}R_y^{-1}$
    \end{enumerate}
\end{itemize}
$$R_{final}=R_x^{-1}R_y^{-1}R_zR_yR_x$$

\section{Rotación alrededor de cualquier eje y punto de origen}
\begin{enumerate}
    \item Traslación al origen deseado $T_{xyz}$
    \item Rotamos $R_x^{-1}R_y^{-1}R_zR_yR_x$
    \item Invertir traslación $T_{xyz}^{-1}$
\end{enumerate}
$$R_{final}=T_{xyz}^{-1}R_x^{-1}R_y^{-1}R_zR_yR_xT_{xyz}$$

\section{Matrices en PovRay}
Las matrices se definen con matrix <v00, v01, v02, v10, ...>, en el que la última columna se asume que es 0, 0, 0, 1. De manera que la matriz está traspuesta, siendo la última fila la de traslación 0, 0, dy, dx (está rotada).