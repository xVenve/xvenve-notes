\documentclass[12pt, twoside, openright]{report} %fuente a 12pt, formato doble pagina y chapter a la derecha
\raggedbottom % No ajustar el contenido con un salto de pagina

% MÁRGENES: 2,5 cm sup. e inf.; 3 cm izdo. y dcho.
\usepackage[
a4paper,
vmargin=2.5cm,
hmargin=3cm
]{geometry}

% INTERLINEADO: Estrecho (6 ptos./interlineado 1,15) o Moderado (6 ptos./interlineado 1,5)
\renewcommand{\baselinestretch}{1.15}
\parskip=6pt

% DEFINICIÓN DE COLORES para portada y listados de código
\usepackage[table]{xcolor}
\definecolor{azulUC3M}{RGB}{0,0,102}
\definecolor{gray97}{gray}{.97}
\definecolor{gray75}{gray}{.75}
\definecolor{gray45}{gray}{.45}

% Soporte para GENERAR PDF/A
\usepackage[a-1b]{pdfx}

% ENLACES
\usepackage{hyperref}
\hypersetup{colorlinks=true,
  linkcolor=black, % enlaces a partes del documento (p.e. índice) en color negro
  urlcolor=blue} % enlaces a recursos fuera del documento en azul

% Añadir pdfs como partes del documento
\usepackage{pdfpages}

% Quitar la indentación de principio de los parrafos
\setlength{\parindent}{0em}

% EXPRESIONES MATEMATICAS
\usepackage{amsmath,amssymb,amsfonts,amsthm}

\usepackage{txfonts} 
\usepackage[T1]{fontenc}
\usepackage[utf8]{inputenc}

% Insertar graficas y fotos
\usepackage{tikz}
\usepackage{pgfplots}

\usepackage[spanish, es-tabla]{babel} 
\usepackage[babel, spanish=spanish]{csquotes}
\AtBeginEnvironment{quote}{\small}

% diseño de PIE DE PÁGINA
\usepackage{fancyhdr}
\pagestyle{fancy}
\fancyhf{}
\renewcommand{\headrulewidth}{0pt}
\fancyfoot[LE,RO]{\thepage}
\fancypagestyle{plain}{\pagestyle{fancy}}

% DISEÑO DE LOS TÍTULOS de las partes del trabajo (capítulos y epígrafes o subcapítulos)
\usepackage{titlesec}
\usepackage{titletoc}
\titleformat{\chapter}[block]
{\large\bfseries\filcenter}
{\thechapter.}
{5pt}
{\MakeUppercase}
{}
\titlespacing{\chapter}{0pt}{0pt}{*3}
\titlecontents{chapter}
[0pt]                                               
{}
{\contentsmargin{0pt}\thecontentslabel.\enspace\uppercase}
{\contentsmargin{0pt}\uppercase}                        
{\titlerule*[.7pc]{.}\contentspage}                 

\titleformat{\section}
{\bfseries}
{\thesection.}
{5pt}
{}
\titlecontents{section}
[5pt]                                               
{}
{\contentsmargin{0pt}\thecontentslabel.\enspace}
{\contentsmargin{0pt}}
{\titlerule*[.7pc]{.}\contentspage}

\titleformat{\subsection}
{\normalsize\bfseries}
{\thesubsection.}
{5pt}
{}
\titlecontents{subsection}
[10pt]                                               
{}
{\contentsmargin{0pt}                          
  \thecontentslabel.\enspace}
{\contentsmargin{0pt}}                        
{\titlerule*[.7pc]{.}\contentspage}  


% DISEÑO DE TABLAS.
\usepackage{multirow} % permite combinar celdas 
\usepackage{caption} % para personalizar el título de tablas y figuras
\usepackage{floatrow} % utilizamos este paquete y sus macros \ttabbox y \ffigbox para alinear los nombres de tablas y figuras de acuerdo con el estilo definido. Para su uso ver archivo de ejemplo 
\usepackage{array} % con este paquete podemos definir en la siguiente línea un nuevo tipo de columna para tablas: ancho personalizado y contenido centrado
\newcolumntype{P}[1]{>{\centering\arraybackslash}p{#1}}
\DeclareCaptionFormat{upper}{#1#2\uppercase{#3}\par}

% Diseño de tabla para ingeniería
\captionsetup[table]{
  format=hang,
  name=Tabla,
  justification=centering,
  labelsep=colon,
  width=.75\linewidth,
  labelfont=small,
  font=small,
}

% DISEÑO DE FIGURAS.
\usepackage{graphicx}
\graphicspath{{img/}} %ruta a la carpeta de imágenes

% Diseño de figuras para ingeniería
\captionsetup[figure]{
  format=hang,
  name=Fig.,
  singlelinecheck=off,
  labelsep=colon,
  labelfont=small,
  font=small    
}

% NOTAS A PIE DE PÁGINA
\usepackage{chngcntr} %para numeración continua de las notas al pie
\counterwithout{footnote}{chapter}

% LISTADOS DE CÓDIGO
% soporte y estilo para listados de código. Más información en https://es.wikibooks.org/wiki/Manual_de_LaTeX/Listados_de_código/Listados_con_listings
\usepackage{listings}

% definimos un estilo de listings
\lstdefinestyle{estilo}{ frame=Ltb,
  framerule=0pt,
  aboveskip=0.5cm,
  framextopmargin=3pt,
  framexbottommargin=3pt,
  framexleftmargin=0.4cm,
  framesep=0pt,
  rulesep=.4pt,
  backgroundcolor=\color{gray97},
  rulesepcolor=\color{black},
  %
  basicstyle=\ttfamily\footnotesize,
  keywordstyle=\bfseries,
  stringstyle=\ttfamily,
  showstringspaces = false,
  commentstyle=\color{gray45},     
  %
  numbers=left,
  numbersep=15pt,
  numberstyle=\tiny,
  numberfirstline = false,
  breaklines=true,
  xleftmargin=\parindent
}

\captionsetup[lstlisting]{font=small, labelsep=period}
% fijamos el estilo a utilizar 
\lstset{style=estilo}
\renewcommand{\lstlistingname}{\uppercase{Código}}

\pgfplotsset{compat=1.17} 
%-------------
% DOCUMENTO
%-------------

\begin{document}
\pagenumbering{roman} % Se utilizan cifras romanas en la numeración de las páginas previas al cuerpo del trabajo

%----------
% PORTADA
%---------- 
\begin{titlepage}
	\begin{sffamily}
		\color{azulUC3M}
		\begin{center}
			\begin{figure}[H] %incluimos el logotipo de la Universidad
				\makebox[\textwidth][c]{\includegraphics[width=16cm]{Portada_Logo.png}}
			\end{figure}
			\vspace{2.5cm}
			\begin{Large}
				Grado en Ingeniería Informática\\
				2019-2020\\
				\vspace{2cm}
				\textsl{Apuntes}\\
				\bigskip
			\end{Large}
			{\Huge Fundamentos de gestión empresarial}\\
			\vspace*{0.5cm}
			\rule{10.5cm}{0.1mm}\\
			\vspace*{0.9cm}
			{\LARGE Jorge Rodríguez Fraile\footnote{\href{mailto:100405951@alumnos.uc3m.es}{Universidad: 100405951@alumnos.uc3m.es}  |  \href{mailto:jrf1616@gmail.com}{Personal: jrf1616@gmail.com}}}\\
			\vspace*{1cm}
		\end{center}
		\vfill
		\color{black}
		\includegraphics[width=4.2cm]{img/creativecommons.png}\\
		Esta obra se encuentra sujeta a la licencia Creative Commons\\ \textbf{Reconocimiento - No Comercial - Sin Obra Derivada}
	\end{sffamily}
\end{titlepage}

%----------
% ÍNDICES
%---------- 

%--
% Índice general
%-
\tableofcontents
\thispagestyle{fancy}

%----------
% TRABAJO
%---------- 

\pagenumbering{arabic} % numeración con múmeros arábigos para el resto de la publicación  


%----------
% COMENZAR A ESCRIBIR AQUI
%---------- 



\chapter{Tema 1. La empresa- naturaleza y tipos}

\paragraph{¿Qué es la empresa?}
Es una institución en la que un conjunto de personas transforma diversos recursos en
bienes y/o servicios que satisfagan necesidades humanas.

Las empresas organizan y estructuran sus recursos y el trabajo de los empleados con el fin de obtener bienes
y servicios que aporten valor añadido, y de esta forma obtener beneficios para distribuir entre sus
propietarios.

Inputs: El capital de recursos financieros, la mano de obra y materias primas. Outputs: Bienes y servicios.

Las empresas se entienden como una unidad técnico-económica porque transforman un conjunto de
recursos, mediante el uso de determinada tecnología en un producto o en un servicio que tiene valor
económico.

Unidad técnica: Porque transforma un conjunto de bienes (inputs) en un conjunto de productos y servicios
(outputs).

Unidad económica: Porque genera valor añadido y por tanto valor económico mediante el proceso de
transformación técnico.

Unidad sociopolítica: Unidad social porque una empresa funciona porque trabajan, arriesgan, deciden y
colaboran personas. Y es unidad política porque los intereses de las personas no suelen coincidir con los
intereses empresariales.

Unidad de decisión: Alguien se preocupa (y toma decisiones) de que los diferentes elementos y personas
que integran la empresa guarden una serie de relaciones preestablecidas, que permitan dar coherencia a sus
actuaciones.

\paragraph{¿Qué es una organización?}
Un conjunto de personas que, con los medios o recursos adecuados, funciona
mediante un conjunto de procedimientos y reglas establecidos para alcanzar un fin determinado.

Todas las empresas son organizaciones, pero no todas las organizaciones son empresas.

La empresa se diferencia del resto de organizaciones porque sus propietarios tienen ánimo de lucro, mientras
que las organizaciones tienen otros fines al ser creadas: fines educativos, culturales, ecológicos…

La continua interrelación de los mercados de bienes y servicios y de factores determina una doble corriente,
la monetaria y la real, mediante la cual la economía de un país responde a las cuestiones básicas de qué
producir, cómo y para quién.

Naturaleza de la empresa: La economía estudia cómo lograr satisfacer necesidades humanas ilimitadas
partiendo de recursos escasos, lo cual obliga a elegir qué producir, cómo y para quién. ¿De qué formas se
puede lograr cubrir esta necesidad? Con la especialización (división del trabajo).

La Sociología y la Psicología entienden la empresa como una unidad sociopolítica, y no como una unidad
técnico-económica.

Estas ciencias entienden y analizan las organizaciones como grupos de personas interdependientes, donde
el objeto de atención son los intereses y objetivos individuales y grupales y el uso de diferentes mecanismos
(por ejemplo: autoridad, liderazgo, motivación) para la resolución de los conflictos.

El comportamiento humano, individual y grupal, y su papel en la obtención de los objetivos de las empresas
y de las organizaciones, son aportaciones esenciales para entender y explicar los resultados de la empresa.

La Dirección de Empresas se preocupa por la eficiencia y la eficacia de las empresas y organizaciones.

Entienden que para que la empresa funcione deben mejorar la eficacia y la eficiencia en las organizaciones y
en las empresas. Por ello, desde la Dirección de Empresas se analizan y se estudian los determinantes para
que una empresa alcance la eficacia y la eficiencia.

El comportamiento humano, individual y grupal, y su papel en la obtención de los objetivos de las empresas
y de las organizaciones, son aportaciones esenciales para entender y explicar los resultados de la empresa.

Funciones reales: Inputs $\rightarrow$ Actividades reales $\rightarrow$ Outputs

Actividades relacionadas con: el aprovisionamiento, la producción, la comercialización y otras actividades o
funciones de soporte.

Funciones financieras: Pago adquisición de inputs $\leftarrow$ Actividades financieras $\leftarrow$ Ingresos de ventas

Captación recursos fros. $\rightarrow$ Actividades financieras $\rightarrow$ Devolución por el pago de ingresos fros.

Actividades relacionadas con la financiación y la inversión.

Funciones de la dirección: Actividades de planificación, control y organización.

Procesos empresariales. Definición: Conjunto de actividades interrelacionadas que, con un conjunto de
inputs recibidos, permite crear ciertos resultados (outputs).

Las actividades están relacionadas entre sí e integradas de forma coherente.

Las funciones empresariales pueden alimentar a varios procesos.

Un proceso comprende una serie de actividades realizadas en diferentes áreas de la organización, que
deberán agregar valor, proporcionando así un servicio a su cliente. Este cliente podrá ser un cliente interno
o un cliente externo…
\pagebreak

Procesos del Negocio: se orientan a satisfacer al cliente. Por ejemplo, un proceso de venta que incluye
registrar el pedido, enviarlo a producción, fabricar el producto, despacharlo y cobrar; Proceso de diseño del
producto que incluye la creación del modelo, prototipos, experimentos, evaluación de resultados…

Procesos de Apoyo: dan servicios a los procesos del negocio. Por ejemplo, el pago de sueldos a los empleados
o reparación de una maquinaria.

La ingeniería en la gestión empresarial: Se ha potenciado desde la revolución industrial aportando eficiencia
y productividad empresarial.

Así, las ciencias de la ingeniería forman profesionales promotores del desarrollo y cambio, generadores de
conocimiento y aplicadores de técnicas en los diferentes sectores de actividad.

Con ello, la ingeniería es capaz de generar valor a las personas, empresa y sociedad.

El mundo de la empresa requiere la toma continua de decisiones. El ingeniero, a partir de las percepciones
de los fenómenos del mundo real y teniendo en cuenta su conocimiento y especialización personal, es capaz
de definir situaciones problemáticas, analizarlas y proponer soluciones.

La superior perspectiva y capacidad de diseñar modelos e introducir mejoras ha permitido el replanteamiento
del ingeniero en el diseño, la gestión y las decisiones alrededor de las organizaciones, incluso superando la
eficacia en este ámbito que podrían alcanzar otros profesionales.



\chapter{Tema 2. Creación de valor- entorno y ventaja competitiva}
\paragraph{¿Qué es un empresario?} Es un factor básico de desarrollo económico, creador de riqueza y de puestos de
trabajo, fundamentalmente para mejorar la competitividad de la empresa.

\paragraph{¿Quién es el empresario?} No existen resultados concluyentes sobre quién es esta figura y qué papel cumple
el empresario de una empresa. No está clara la función del empresario, sobre todo en las grandes empresas
donde están propietarios, directivos y prestamistas financieros, entre otros, que parecen compartir algunas
de las características atribuidas a una sola persona: el empresario.

Las funciones que puede cumplir un empresario son: Persona que coordina y controla el proceso productivo,
que asegura la renta al resto de factores, que toma decisiones arriesgadas, que innova y que detecta
oportunidades bajo incertidumbre.

Organización interna de la empresa = Trabajo en equipo

La producción en equipos requiere de un monitor.

¿Cuál es el incentivo para realizar esta función? Recibir la renta o el excedente residual, controlar y reasignar
el resto de factores productivos, ser parte central en los contratos con los inputs y vender los derechos
anteriores. El empresario se identifica con aquel que posee estos derechos.

\section{CRITERIOS DE CLASIFICACIÓN.}
Según criterios económicos: Sector (primario, secundario y terciario), tamaño (grandes, medianas y
pequeñas) y ámbito de competencia (monomercado/multimercado, nacional/multinacional).

Según criterios jurídicos: Titularidad de los medios de producción (empresas privadas o públicas) y
personalidad (empresas individuales o sociales).

-La empresa individual: Empresario individual es el que posee su propio negocio. Asume todos los riesgos en
la empresa, toma decisiones y busca oportunidades para la empresa. Por tanto, posee todos los derechos.

Ventajas: Tiene autonomía para regir su propio negocio y recibe la totalidad de la renta residual.

Problemas: Financiación, supervivencia y concentración de riesgos.
\pagebreak
-La empresa social: Dividida en sociedades personalistas (sociedad colectiva y sociedad comanditaria simple),
sociedades de capital (sociedad anónima, sociedad de responsabilidad limitada y sociedad comanditaria por
acciones) y cooperativas.

La sociedad anónima (S.A.): Tiene su capital dividido en partes alícuotas o acciones, transmisibles libremente
en los mercados y que otorgan a sus propietarios derechos económicos (reparto de beneficios, patrimonio
acumulado) y políticos (coordinación).

La S.A. tiene unas características que le han permitido favorecer y financiar el proceso de crecimiento de la
empresa moderna, proporcionándole fondos y capacidad directiva.

Ventajas frente a la empresa individual: Limitación de la responsabilidad, diversificación de riesgos, facilidad
de deshacerse de la inversión, mayor capacidad de obtener financiación, favorece la especialización de
funciones y el crecimiento de la empresa y su existencia está separada de su fundador.

Problemas: Los directivos solo buscan sus propios beneficios y el accionista no controla la actuación del
directivo, las empresas están dirigidas por directivos asalariados que toman las decisiones y asumen el riesgo
profesional, los directivos pueden tomar decisiones beneficiosas para ellos y perjudiciales para el accionista,
los accionistas y directivos tienen asimetría de información y divergencia de objetivos, los accionistas no
tienen información suficiente para evaluar el comportamiento de los directivos, y tampoco tienen incentivos
para recabar información sobre la marcha de la empresa. No tienen incentivos para intentar mejorar la
gestión de la empresa.

Todo esto genera un conflicto de intereses entre propiedad y control, y, por tanto, problemas de agencia.
Solución: El accionista puede controlar la actuación de los directivos y así estos no buscan solo su propio
beneficio.

Mecanismos de control de la actuación de los directivos: Órganos de Gobierno de la empresa (Junta de
accionistas y Consejo de Administración, Comisión Ejecutiva o consejeros delegados), relaciones
contractuales con los directivos, auditorias, indicadores y controles externos de los mercados (Mercado de
empresas (OPAS), mercado del trabajo directivo y mercado de productos y factores).

Cada vez es mayor la coincidencia entre los que tienen muchas acciones (paquetes de acciones) y los que
toman decisiones.

Las Cooperativas. Tipos según sus miembros:

Cooperativas de Producción: Sus miembros son los trabajadores, ponen en común los bienes de producción
con los que elaboran los outputs que después venden en el mercado.

Cooperativas de Venta: Sus miembros actúan como proveedores de otras empresas, facilitan la labor de
distribución de productos. Comercializan productos con el fin de obtener mejores condiciones de venta.

Cooperativas de consumo y servicios diversos: Los socios son los clientes de la empresa cooperativa,
proporcionan a sus miembros productos en condiciones favorables, los socios compran a través de la
cooperativa más barata, consiguiendo productos en condiciones más ventajosas. Son las cooperativas de
consumo, vivienda, educación y sanidad, y cooperativas financieras.

Principios de: puerta abierta, democrático, retribución limitada del capital aportado por los socios, reparto
del excedente en función de la actividad desarrollada en la cooperativa, fomento de cooperación y fomento
de educación y ayuda mutua.

Problemas: Horizonte temporal de la inversión (No motivación para invertir más allá de su permanencia en
la empresa, preferirán reparto de excedentes), carácter compartido de los derechos de los socios (Los nuevos
socios tendrán los mismos derechos que los antiguos, son reacios a emprender nuevas actividades que
supongan nuevos socios) y concentración de riesgo para los socios (Los socios de la cooperativa tienen un
patrimonio poco diversificado).

Sociedad formada por un conjunto de socios que tienen responsabilidad ilimitada y el derecho a participar
en la dirección de la empresa.

Características: Responsabilidad ilimitada, no se pueden transmitir la participación de cada socio (salvo que
los demás socios lo autoricen) y la sociedad debe disolverse si falta alguno de los socios.

Se trata de una forma jurídica muy utilizada por los profesionales: Trabajos muy cualificados en los que la
reputación es su principal activo.

\section{Objetivo de la empresa = Objetivo del empresario}

Según la teoría clásica, el objetivo de la empresa era maximizar el beneficio, por lo que se provocan
limitaciones. Dificultad para la definición y cuantificación del término beneficio, no tiene en cuenta el riesgo
asociado a la obtención de beneficios, noción de maximización y excesivo protagonismo del empresario
(separación entre propiedad y dirección).

La dirección y los objetivos: Para evitar el conflicto de intereses se alinean los objetivos de los directivos con
los de los accionistas. Con sistemas de incentivos, mercados de empresas, mercado de trabajo y mercados
de factores.

La dirección intentará maximizar su utilidad, pero condicionada a la obtención de un nivel mínimo de
beneficios.

El papel de los grupos de interés en la fijación de objetivos. ¿Quiénes son los grupos de interés de la
empresa? Accionistas, bancos y otros agentes financieros, consumidores, trabajadores/sindicatos, estado y
administraciones públicas y proveedores. Todos los grupos bajo el mando de la dirección de la empresa.

Teoría de la organización: Resultado de un proceso de negociación entre los diferentes grupos de intereses
de la empresa, restringido por el objetivo de supervivencia.

Objetivo de la empresa: Aquel que satisface al grupo con más poder negociador, manteniendo como
restricción los objetivos del resto de los grupos y la supervivencia de la empresa.

\section{Responsabilidad social corporativa}

Cada vez se exige mayor responsabilidad social a las empresas para limitar su impacto negativo sobre la
sociedad y el medioambiente.

Las empresas se enfrentan a crecientes exigencias de mayor compromiso social impulsadas por la presión de
los medios de comunicación y colectivos organizados (consumidores, sindicatos, grupos ecologistas…).

La responsabilidad social es un comportamiento de la empresa que va más allá del cumplimiento de sus
obligaciones legales. Supone responder a las demandas de los diferentes grupos sociales que afectan a las
actividades de la empresa o que se vean afectados por estas, es decir, internalizar lo que es bueno para la
sociedad, respondiendo a lo que la sociedad requiere de la empresa.

Para lograr el objetivo de la empresa se debe establecer una estrategia, lo que implica decisiones
estratégicas, estas a su vez, decisiones tácticas, y estas, decisiones operativas.

Las estrategias pueden ser competitivas (liderazgo en costes, diferenciación) y corporativas
(internacionalización, integración vertical, diversificación).

Entorno genérico de la empresa: Factores económicos (variables macroeconómicas, fase del ciclo
económico, tipo de política económica, recursos que dispone el país), factores político-legales (políticas
monetarias y fiscales, regulación del mercado laboral, regulación de las industrias), factores socioculturales
(sistema de valores, nivel educativo, estructura de clases, distribución de la renta y la población) y factores
tecnológicos (tecnologías básicas, claves e incipientes).

El entorno específico de la empresa.

Análisis del sector – 5 fuerzas de Porter
\begin{figure}[H]
	\ffigbox[\FBwidth]
	{\caption{5 fuerzas de Porter}}
	{\def\svgwidth{.9\textwidth}
		\input{./img/Porter.eps_tex}}
\end{figure}


\chapter{Tema 3. La función financiera}
Contabilidad: Procedimientos estandarizados para todas las empresas que tienen como finalidad reflejar la
actividad económica de la empresa.

Permite observar la situación y condiciones en que se encuentra la empresa en cualquier momento para
tomar decisiones adecuadas sobre su rendimiento económico.

Plan General Contable: normas que fijan los procedimientos y métodos a seguir en el registro de la actividad
económica.

Obligados: SL, SA, Sociedad Laboral, Sociedad Cooperativa.

\section{CUENTAS ANUALES}
-Balance de situación: Refleja bienes, derechos y obligaciones que tiene la empresa en un momento del
tiempo.

Estructura económica: Definida por las inversiones que hace la empresa. Representa el empleo de los
recursos financieros en bienes y derechos.

Se materializa en el ACTIVO (lo que tiene). Se divide en corriente y no corriente.

Estructura financiera: Definida por las fuentes de financiación de la empresa. Recoge los recursos financieros
utilizados para desarrollar las inversiones (deudas y obligaciones).

Se materializa en el PASIVO (lo que debe). Se divide en patrimonio neto y pasivo. El pasivo, a su vez, se divide
en corriente y no corriente.

-Cuenta de pérdidas y ganancias: contiene información sobre los ingresos y gastos en el ejercicio económico,
reflejando el resultado del mismo, con pérdidas o beneficios.

Ingresos: de explotación (venta de bienes y servicios) y financieros.

Gastos: de explotación (costes de las ventas) y financieros

Resultado de explotación: Beneficio antes de intereses e impuestos (BAIT)

Resultados antes de impuestos: Beneficios antes de impuestos (BAT)

Resultados consolidados: Beneficio neto (BN)

-Memoria: recoge la información del resto de documentos contable de forma ampliada y detallada.

Las cuentas tienen dos zonas para anotar movimientos: DEBE y HABER.

Las cuentas de activo (bienes y derechos) aumentan su valor mediante anotaciones en el “debe” y
disminuyen con anotaciones en el “haber”.

Las cuentas de pasivo (obligaciones y deudas) aumentan mediante anotaciones en el “haber” y disminuyen
con anotaciones en el “debe”.

Para realizar apuntes contables es necesario tener en cuenta una estructura de “CONTRAPARTIDA” de forma
que no se puede anotar ninguna partida del debe sin anotar la variación generada en el haber.

Todos estos movimientos se anotarán en el libro diario (refleja todas las operaciones económicas producidas,
ordenadas cronológicamente) y en el libro mayor (para cada cuenta se anotarán los movimientos reflejando
fecha, número de asiento y conceptos).

Lo que vemos en la contabilidad de la empresa nos permite considerar cuál es la estructura económica financiera de la empresa y tomar decisiones de inversión y financiación.

Las decisiones de la empresa se centran en:

-Decisiones de inversión: Consiste en la colocación de capital en proyectos de inversión de los que se espera
obtener un beneficio futuro. Se invertirá en aquellos proyectos que permitan incrementar el valor de la
empresa.

-Decisiones de financiación: Se refiere a la elección entre medios de financiación propios y ajenos. Los
recursos que puede utilizar la empresa pueden ser internos y externos (propios y ajenos). Comprenden dos
tipos de cuestiones (decisiones sobre la estructura del capital y decisiones de dividendos).

ESTRUCTURA FINANCIERA: Se divide en propia y ajena.

PROPIA:

-Capital propio externo (recursos propios externos: capital inicial y acciones).

Aportaciones de los socios de la empresa en el momento de constitución de la sociedad (capital inicial) y
posteriores ampliaciones (la emisión de acciones) de capital para financiar las inversiones de la empresa.

No tienen fecha de vencimiento para devolver al inversor su aportación. Una participación de capital se hace
líquida únicamente vendiéndola o transmitiéndola.

Acciones: Parte proporcional del capital de la empresa. Otorga al poseedor (accionista) derechos económicos
y políticos: responsabilidad Limitada. Son títulos de renta variable. Diferentes valoraciones de las empresas:
valor nominal, contable, intrínseco y de mercado. Ampliaciones de capital. No obligación del pago de
dividendos.

-Capital propio interno (autofinanciación).

Los beneficios retenidos (recursos generados por la empresa como consecuencia de su ciclo de explotación),
y las dotaciones para amortizaciones y provisiones, constituyen la autofinanciación de la empresa. Es decir,
son aquellos fondos que la empresa obtiene por sí misma sin necesidad de acudir a los mercados o a las
instituciones financieras.

+Dotación para amortizaciones. +Dotación para provisiones. -Impuestos. -Dividendos.

AJENA (financiación externa con recursos ajenos):

-A largo plazo (emisión de títulos, préstamos y créditos bancarios a l/p, leasing).

Emisión de títulos (obligaciones y bonos): Se trata de un crédito contra la sociedad emisora que se subdivide
en un número de participaciones, títulos, y que dan lugar al pago de intereses y a la devolución del principal
en una fecha contenida y en una forma estipulada.

Tipos de obligaciones: Cupón americano (se paga el interés de forma periódica), cupón cero (se pagan los
intereses de una vez y al final del vencimiento) y al descuento (los intereses se pagan al momento de realizar
la inversión: desembolso = nominal-intereses).

Préstamos: Se recibe una cantidad que habrá que devolver en un determinado periodo de tiempo y pagar
los intereses correspondientes.

Créditos: La institución financiera pone a disposición del cliente una cantidad de dinero hasta un límite
máximo, el cliente utilizará la cantidad que estime oportuno en cada momento, pagando intereses solamente
por la cantidad utilizada (esta mayor flexibilidad se traduce mayores tipos de interés).

Formas de amortización: Sistema Americano (cada periodo se abonan los intereses, el principal al
vencimiento), Francés (cada periodo se devuelve una cantidad constante del principal e intereses de forma
decreciente) y de Cuota fija (cada periodo se paga la misma cuota: intereses+principal)

Leasing (o arrendamiento financiero): Es un contrato de arrendamiento donde el arrendador alquila un bien
a una empresa o arrendatario, que se compromete a pagar unas cuotas por el alquiler, pudiendo optar a
comprar el bien al finalizar el contrato. Tipos: Financiero y operativo.

-A corto plazo (proveedores, préstamos y créditos bancarios a c/p, factoring).

Crédito comercial: La financiación de los proveedores (Crédito Comercial) surge cuando el pago de las
mercancías se realiza en un momento posterior a su recepción. Los proveedores ofrecen a la empresa la
posibilidad de aplazar el pago de las mercancías.

Préstamos y créditos: Mismas características que en l/p.

Descuento Comercial: Operación mediante la cual una entidad financiera pone a disposición de su cliente el
importe de un efecto comercial (letra, pagaré...) antes de su vencimiento, previa deducción de una serie de
gastos (tipo de interés, comisión…).

Factoring: Consiste en que la empresa vende, antes de su vencimiento, un crédito concedido a sus clientes a
una sociedad factoring, que se encarga de su cobro, asumiendo el riesgo de insolvencia.

Objetivo general: Maximizar el valor de la empresa Æ Maximizar la riqueza de los que poseen derecho sobre
los activos de la empresa Æ Maximizar la rentabilidad de los activos.

$$\begin{gathered}\textit{Variables } \\ \textit{explicativas}\end{gathered}
	\begin{cases}
		\textit{Beneficio} \begin{cases}
			\begin{gathered}\textit{Beneficio contable = Incremento de los fondos propios} \\  \textit{+ dividendos repartidos}\end{gathered} \\
			\begin{gathered}\textit{Beneficio economico = Incremento del valor de mercado} \\  \textit{de los fondos propios + dividendos repartidos} \end{gathered}
		\end{cases}       \\
		\begin{gathered}\textit{Coste de oportunidad}\\ \textit{de los inversores}\end{gathered} \begin{cases}
			\textit{Coste del dinero} \\
			\textit{Inflación}        \\
			\textit{Nivel de riesgo}
		\end{cases}
	\end{cases}$$

Criterios de clasificación de proyectos de inversión:

-En función del tamaño de la inversión: Inversión alta y pequeña.

-En función de la finalidad en el seno de la empresa: Variado (posicionamiento estratégico, crecimiento, etc.).

-En función de la duración temporal: Corto plazo (activo circulante) y medio y largo plazo (activo fijo o
inmovilizado).

No se debe realizar ningún proyecto de inversión que no genere la rentabilidad suficiente para atender, como
mínimo, a la remuneración de los recursos que la empresa emplea para su realización.

El valor del dinero en el tiempo: El análisis de las decisiones de inversión exige, en primer lugar, examinar
las relaciones de equilibrio y criterios para la asignación de recursos en el tiempo. Ello requiere conocer los
criterios que permitan comparar las cantidades de dinero recibidas o consumidas en momentos distintos de
tiempo.

Capitalizar es obtener el equivalente futuro de una cantidad disponible en el momento actual. La operación
inversa es determinar la cantidad de dinero actual que equivale a una cantidad disponible con certeza en el
futuro. A esta operación se la llama descontar.

La evaluación de un proyecto de inversión se realizará en función de los siguientes parámetros:

-Desembolso inicial o tamaño de la inversión

-Cobros o entradas de dinero originadas por la inversión en cada periodo de tiempo

-Pagos o salidas de dinero generadas por la inversión en cada periodo de tiempo

-Horizonte temporal o vida de proyecto
\pagebreak

Plazo de recuperación (payback): Se obtiene a partir de la comparación de los flujos de caja y el desembolso
inicial de la inversión, es decir, el número de años necesario para que los flujos de caja permitan recuperar el
desembolso inicial o el tamaño de la inversión.
\begin{table}[H]
	\begin{tabular}{lll}
		V: Volumen de ventas        & $\textit{Rotación de activos} = V / TA$   \\
		TA: Total de activos        & $\textit{Endeudamiento} = FA / FP$        \\
		FP: Fondos propios          & $BAT = BAIT - (Ki * FA)$                  \\
		FA: Fondos ajenos           & $BN = BAT (1 - T)$                        \\
		Ki: Coste del capital ajeno & $\textit{Margen sobre ventas} = BAIT / V$ \\
	\end{tabular}
\end{table}

RE: Rentabilidad económica. Mide la rentabilidad desde el punto de vista del activo.

$RE = BAIT / TA$ = Rotación de activos * margen de ventas

RF: Rentabilidad financiera o de los fondos propios

$RF = BAT / FP$ (sin efecto impositivo) o $RF = BN / FP$ (incluye efecto impositivo)

Apalancamiento financiero: Efecto palanca del endeudamiento sobre la rentabilidad económica.
\begin{table}[H]
	\begin{tabular}{lll}
		$Af = RE - Ki$ & $Af > 0 \rightarrow$ Amplificador & $Af < 0 \rightarrow$ Reductor
	\end{tabular}
\end{table}


Valor actual neto (VAN): Suma algebraica de los valores actualizados de todos los flujos de caja asociados a
la posesión del activo, menos el desembolso inicial necesario para la realización del mismo.
$$VAN = -A + \sum\limits_{t=1}^n \frac{Q_t}{(1-i)^t}$$

Tasa interna de retorno (TIR) (o eficiencia marginal): Es la tasa de descuento que equilibra el valor actual de
los flujos de caja esperados de una determinada inversión y su desembolso inicial.
$$TIR = -A + \sum\limits_{t=1}^n \frac{Q_t}{(1-r)^t}=0$$

Relación entre VAN y TIR: La tasa de Fisher

Tasa de interés en la que VAN (A) = VAN (B)



\chapter{Tema 4. La función de producción}
La función de producción es la parte de la empresa que se dedica a crear un producto o servicio que cubre una necesidad.

Sus principales objetivos son el tiempo de envío, la calidad, el coste y la flexibilidad.

Las decisiones estratégicas de la función de producción son:
\begin{itemize}
	\item ¿Qué vender?
	\item ¿Cómo hacerlo?
	\item ¿Dónde hacerlo?
	\item ¿Cuánto hacer?
\end{itemize}

Podemos diferenciar diferentes tipos de producción:
\begin{enumerate}
	\item PRODUCCIÓN POR PROYECTO: Los proyectos son personalizados y únicos, por lo que cubren necesidades muy específicas.
	      \begin{itemize}
		      \item Volumen de producción muy bajo
		      \item Variabilidad muy alta
		      \item Altos costes variables
		      \item Trabajadores muy cualificados
	      \end{itemize}
	      El tipo de distribución de esta producción es de posición fija, todos los trabajadores e inputs se mueven al punto donde se realizara el proyecto.
	\item PRODUCCIÓN FLEXIBLE: Este sistema de producción cubre las necesidades de un número pequeño de consumidores dispuestos a pagar mucho dinero por productos distintivos.
	      \begin{itemize}
		      \item Volumen de producción medio
		      \item Variabilidad alta
		      \item Trabajadores polivalentes
		      \item Altos costes variables
	      \end{itemize}
	      \pagebreak
	\item PRODUCCIÓN EN MASA: Sistema de producción que intenta satisfacer a un largo número de clientes con productos indiferenciados a precio bajo.
	      \begin{itemize}
		      \item Volumen de producción alto
		      \item Variabilidad muy baja
		      \item Trabajadores poco cualificados
		      \item Maquinaria muy especializada
		      \item Altos costes fijos
		      \item Precios bajos
	      \end{itemize}
	      En la producción en masa es común la distribución por producto, donde un conjunto de máquinas se predisponen a otro, conformando el proceso de montaje del producto.

	      Dependiendo de la situación será más conveniente una producción flexible o una en masa.
	\item PRODUCCIÓN JUST IN TIME: Esta producción es una mezcla de la producción en masa y la flexible, que busca producir las cantidades justas con la máxima calidad, mediante un proceso que evite cualquier despilfarro.
\end{enumerate}

\section{DECISIONES DEL SISTEMA PRODUCTIVO:}

Capacidad: Es la cantidad máxima de bienes y servicios que podemos obtener de una unidad productiva en un tiempo determinado.
\begin{itemize}
	\item Es la capacidad máxima, no efectiva
	\item Depende de un período de tiempo
	\item En condiciones normales, no a capacidad punta.
\end{itemize}
Localización: Es la fuente de muchas ventajas competitivas, por reducción de costes, acceso a materias primas, cercanía al cliente…
\begin{itemize}
	\item Se deben tener en cuenta los inputs, la demanda y el entorno en el que se lleva a cabo la producción.
\end{itemize}

\section{PUNTO MUERTO}

El punto muerto es el estado de producción en el que los ingresos cubren exactamente los costes.

\section{GRÁFICO DE GANT}
Su objetivo es planificar la ejecución de las actividades que componen un proyecto. Su utilidad es muy grande, ya que permite visualizar si hay algún grado de adelanto o retraso en el proyecto.


\section{LOGÍSTICA}
La logística está conformada por todas las actividades de aprovisionamiento, fabricación, almacenaje y distribución de productos. La logística también crea sistemas de información y control para conseguir un flujo continuo de materiales. Se divide en 4 principios:
\begin{enumerate}
	\item RESPONSABILIDAD INTEGRAL: Control del flujo de productos desde el proveedor hasta el punto de venta.
	\item EQUILIBRIO DE CAPACIDADES: Desarrollo de políticas de personal, inversiones y recursos, acorde con las previsiones de venta.
	\item CONTROL PROACTIVO DEL FLUJO DE MATERIALES: Lanzamiento de programas de fabricación y compra (solo si se tiene la capacidad)
	\item PLANIFICACIÓN TOP-DOWN: De lo más general a lo más específico en cuanto a lo planificado, y de lo más lejano a lo más cercano en cuanto a horizonte temporal.
	      \begin{itemize}
		      \item Nivel estratégico: fábricas y almacenes, stocks, transporte…
		      \item Nivel táctico: equipos de fabricación, manutención, volumen de inventarios…
		      \item Nivel operativo: Programación de aprovisionamiento, asignación de cargas…
	      \end{itemize}
\end{enumerate}

La incorporación de las TIC a la logística hace que se tenga un mayor control del flujo de información, esto permite una automatización de los procesos de aprovisionamiento y selección. (e-procurement y e-sourcing respectivamente)

\section{METODOLOGÍA AGILE}
Se basa en la división del proceso de producción en sprints, y adaptarse a una nueva forma de organizarse y trabajar mediante las TIC, ya que las empresas se encuentran en un entorno cambiante, si una empresa no se adapta lo suficientemente rápido, seguramente acabe en bancarrota, con la metodología agile se intenta asegurar una adaptación rápida y segura de la empresa.



\chapter{Tema 5. Gestión comercial y marketing}
La misión del marketing es lograr intercambios con los mercados deseados.

El marketing mantiene el contacto con los consumidores y desarrolla productos para cubrir sus
necesidades, además de como desarrollarlos, producirlos, y expresar los propósitos de la
organización.

\section{FUNCIONES DEL MARKETING}
Marketing activo: El objetivo es conocer al cliente y hacer que el producto se venda por sí mismo
mediante la adaptación.

Marketing de organización: El objetivo es crear una organización comercial que permita buscar y
planificar salidas para los productos.

Marketing pasivo: Se centra en la producción, no se le da importancia al marketing.
\begin{itemize}
	\item Evolución del concepto de marketing
\end{itemize}

Marketing social: Determinar las necesidades del mercado objetivo y mejorar el bienestar de los
consumidores.

El marketing es un proceso de doble sentido entre la empresa y el cliente

\section{PROCESO DE DIRECCIÓN Y GESTIÓN DEL MARKETING}
Marketing estratégico: Investigación del cliente y las variables no controlables de la empresa.

Marketing operativo: Se consigue mediante la combinación del producto, el precio, la distribución y
la promoción.

\section{MARKETING ESTRATÉGICO}
El objetivo es estudiar las amenazas y oportunidades del mercado, para combinarlas con los recursos
y capacidades de la empresa para obtener una ventaja competitiva.
\pagebreak

Etapas:
\begin{enumerate}
	\item Análisis de situación
	\item Diagnóstico de la situación
	\item Definición del objetivo del marketing
	\item Formulación de estrategias.
\end{enumerate}

Una de las variables más importantes del marketing estratégico es la segmentación del mercado,
que consiste en identificar posibles compradores de acuerdo a grupos con alguna característica
común.

Algunos de las características comunes pueden ser el factor geográfico, el socio demográfico, el
psicológico, y el comportamiento de compra.

Una vez que se segmenta el mercado, se elige una estrategia de segmentación:
\begin{itemize}
	\item Estrategia diferenciada
	      \begin{itemize}
		      \item Marketing diferenciado: Para cada sector se lanzan unos productos
	      \end{itemize}
	\item Estrategia indiferenciada
	      \begin{itemize}
		      \item Marketing diferenciado: Para cada sector se lanzan unos productos
		      \item Marketing concentrado: Nos concentramos en un único segmento del mercado.
		      \item Marketing masivo: Un único producto para todo el mercado.
	      \end{itemize}
\end{itemize}

Al definir la estrategia, se pasa a posicionar el producto, que se basa en crear una imagen del
producto que se va a presentar.

Por último, se hace un análisis y previsión de la demanda, si una demanda es elástica, se sabe que si
se cambia el precio, la demanda variará bastante, en cambio, si es inelástica, apenas variará.

\section{MARKETING OPERATIVO(MARKETING MIX)}
\subsection{PRODUCTO}
\begin{enumerate}
	\item Marca: La marca es toda señal o símbolo que se colocan en el producto para que el cliente lo sepa distinguir los demás.

	      En una marca se distinguen el nombre y el logotipo.
	      \pagebreak
	      Hay de diferentes tipos, marcas de distribuidores (Alcampo), marcas únicas (Loewe), marcas
	      individuales (P\&G, se distribuye en Pantene, Head and shoulders…), marca de familia (Panrico
	      tiene Donut, Qe!...), marca paraguas (Sony tiene Xperia para móviles AIO para sonido…)
	\item Diseño y forma: Hay que tener en cuenta tanto la estética del producto como su funcionalidad
	\item Packaging: Se debe estudiar cómo envolver el producto para facilitar su transporte, ajustarse a
	      las normas legales, y aceptar la información complementaria.
	\item Calidad: Grado de adecuación del producto
	\item Otros atributos: Garantía, servicio posventa…
	      Gama de productos: Cuál va a ser la longitud y profundidad de nuestros productos, y si
	      tenemos pensado añadir más líneas de productos o extenderlas.
	\item Ciclo de vida: ¿Va a durar la popularidad de nuestro producto?
\end{enumerate}

\subsection{DISTRIBUCIÓN}
El objetivo de la distribución es poner a disposición del consumidor o del distribuidor el producto, en
la cantidad de la demanda que se necesite, y en el momento y lugar adecuado.

La distribución tiene dos facetas: Los canales de distribución y la distribución física.
\begin{enumerate}
	\item Canales de distribución: Pueden ser directos (el propio fabricante los vende al consumidor) o
	      indirectos (en el canal de distribución existen mayoristas y minoristas), se deben tener en
	      cuenta el tipo y la longitud que va a tener el canal de distribución, además del n.º de
	      intermediarios.

	\item Intermediarios: Son los representantes (trabajan a comisión sobre ventas y la mercancía no es
	      suya, ponen en contacto al fabricante y al consumidor), los mayoristas (asumen el riesgo de su
	      actividad y el producto no es suyo, venden al por mayor) y los minoristas (los productos son
	      suyos y afrontan un mayor riesgo, venden al por menor)

	      Las funciones de los intermediarios es reducir el número de contactos, adecuar la oferta a la
	      demanda, mover el producto…
\end{enumerate}

\subsection{PRECIO}
Las razones de la importancia del precio es que influye en el nivel de la demanda y en la percepción
del producto.

Para fijar un precio se debe tener en cuenta la elasticidad, los costes del producto, y la referencia del
precio del líder del mercado.
\subsection{PROMOCIÓN}
Se basa en estimular la demanda mediante la transmisión de las características del producto.
\begin{enumerate}
	\item Publicidad: Es una forma de comunicación en forma de masa pagada, puede ser informativa o
	      persuasiva.
	\item Promoción en ventas: Crear ofertas y acciones comerciales para que el consumidor se vea más
	      atraído a la hora de elegir el producto.
	\item Relaciones públicas: Ser sponsor de un equipo y actividades similares hacen que se le den una
	      imagen positiva a la empresa y a la marca
	\item Marketing directo: Son el conjunto de actividades que ofrecen el producto o servicio a
	      segmentos del mercado previamente definidos, un teleoperador, por ejemplo.
	\item Fuerza de ventas: Un ejemplo sería un vendedor que va puerta por puerta vendiendo un
	      producto.
\end{enumerate}



\chapter{Tema 6. La dirección de la empresa}
\section{LA FUNCIÓN DIRECTIVA}
La tarea directiva se basa en conseguir una actuación conjunta de las personas que componen la
organización, dándoles objetivos y valores comunes, para poder responder a los cambios.
\begin{description}
	\item[PLANIFICACION] Es el proceso en el que se decide que se va a hacer y cómo. Implica definir los
	      objetivos y las estrategias de la empresa. La planificación facilita la dirección, incrementa el
	      éxito y ayuda a mejorar la adaptación de la empresa.
	\item[ORGANIZACIÓN] Se basa en definir las tareas de cada uno, así como las vinculaciones entre
	      ellos. La organización se basa en dos procesos, la diferenciación y la integración
	\item[DIRECCIÓN DE RRHH] Es el proceso en el que se decide que se va a hacer y cómo. Implica definir los
	      objetivos y las estrategias de la empresa. La planificación facilita la dirección, incrementa el
	      éxito y ayuda a mejorar la adaptación de la empresa.
	\item[CONTROL] Es la fase que vigila la actuación a nivel individual y grupal y determina el grado de
	      cumplimiento de los planes. Con el control, se modifican las actividades de la empresa para
	      que se puedan desarrollar correctamente los planos definidos.
\end{description}

\section{TOMA DE DECISIONES}
\begin{enumerate}
	\item Se detecta el problema
	\item Criterios de evaluación
	\item Se buscan alternativas
	\item Se evalúan dichas alternativas
	\item Se elige una
	\item Se pone en práctica
	\item Se vuelve a aplicar el proceso de control para detectar otros problemas
\end{enumerate}

\section{NIVELES DIRECTIVOS}
Los hay de alta dirección, de nivel medio y supervisores de primera línea.

\section{ESTRUCTURA ORGANIZATIVA}
Para diseñar esta estructura hay que tener en cuenta:
\begin{itemize}
	\item Especialización: Existen la horizontal (n.º de departamentos y n.º de puestos definidos) y
	      vertical (n.º de niveles jerárquicos)
	\item Estandarización: Limitación a la hora de que tareas se harán, cómo y cuándo.
	\item Centralización: Grado de concentración de la autoridad para tomar decisiones.
	\item Configuración estructural: Vinculación entre los puestos.
	\item Componentes comunes: Lo son el ápice estratégico, la línea media, el núcleo de operaciones,
	      la tecno estructura, y el staff de apoyo.
\end{itemize}

\section{TIPOS DE ESTRUCTURA:}
\begin{itemize}
	\item Simple: Solo tiene un ápice estratégico y un núcleo de operaciones. Mucha centralización y poca especialización. No hay departamentos. Solo existe una supervisión directa.
	\item Burocracia mecánica: Posee todos los elementos comunes. Tiene una alta centralización y formalización, además de mucha especialización. Posee procesos estandarizados y tareas rutinarias.
	\item Burocracia profesional: Posee lo mismo que la mecánica, pero tiene menos influencia la línea media, la tecno estructura y el staff de apoyo. Tiene una alta especialización horizontal. El núcleo de operaciones es clave y puede formar parte de la línea media. La selección de profesionales es muy importante para la coordinación y control. Descentralizada.
	\item Multidivisional: Formada por muchas sedes, la central asigna recursos y las demás funcionan como burocracias. Los resultados son controlados por la sede central.
	\item Adhocrática: No posee tecnoestructura ni staff de apoyo, la línea media es clave. La especialización horizontal es alta, pero la vertical muy baja. La creatividad e innovación son muy importantes, desaparece el componente administrativo, y además hay un alto nivel de departamentalización.
\end{itemize}
\pagebreak
\section{DIRECCIÓN DE RR. HH.}
La dirección de recursos humanos tiene una importancia básica para conseguir los objetivos de la
empresa.

El problema principal es que la empresa y los individuos tengan el mismo objetivo, para ello se
aplican unos sistemas de evaluación e incentivos, y se motiva a los individuos.
\begin{itemize}
	\item Motivación: para dar motivación a los individuos de la empresa se crea un ciclo el cual empieza
	      con una necesidad de la empresa, se resuelve con un comportamiento que da lugar a la
	      resolución de la necesidad o a la frustración al fracasar. Existen diversas teorías de motivación,
	      las más importantes son la Pirámide Maslow y los factores de motivación de Herzberg.
	\item Liderazgo y comunicación: El liderazgo es una forma de poder, la capacidad de afectar en el
	      comportamiento de los demás. La comunicación es un proceso de transmisión de información,
	      puede ser descendente, ascendente o cruzada.
	\item Reclutamiento: Es un proceso mediante el cual la organización intenta conseguir empleados
	      potenciales, que cumplan los requisitos deseados. Existen el interno y el externo. Las ventajas
	      del interno es que aumenta la motivación, evita sorpresas y tiene menos costes, pero genera
	      conflictos internos y hay pocos candidatos. En cuanto al externo, hay muchos candidatos,
	      evitan problemas entre el personal y se aprovechan las ayudas públicas, por otro lado, tiene
	      costes mayores y no se conoce al 100 \% al candidato.
	\item Selección y formación: El proceso de selección se basa en elegir cuál de los candidatos
	      reclutados es mejor para cada posición. La formación puede ser por necesidad de formación
	      inicial o por necesidad de mantener el conocimiento.
	\item Evaluación de rendimiento: Son herramientas para medir el cumplimiento de los objetivos a
	      los individuos y/o grupos.
	\item Sistemas de retribución: Es la acción de remunerar a los trabajadores, juntos con las leyes,
	      normas y principios que rigen dichas acciones.
\end{itemize}



\chapter{Tema 7. Creación empresarial e innovación. Las empresas de base tecnológica}
La innovación se basa en la comercialización de un invento.

Tipos:
\begin{itemize}
	\item Por grado de novedad:
	      \begin{itemize}
		      \item Radical: son las más novedosas y difíciles de imitar (internet)
		      \item Incremental
	      \end{itemize}
	\item Por el resultado innovador:
	      \begin{itemize}
		      \item Producto
		      \item Proceso
		      \item etc.
	      \end{itemize}
\end{itemize}

\section{Modelo interactivo de la innovación}
\begin{itemize}
	\item Innovación I+D: Se basan en investigaciones de trabajos teóricos o prácticos en los que se les intentan buscar diferentes usos y aplicarlos en la producción de productos o procesos.
	\item Innovación learning by doing: Surge al repetir las operaciones de producción
	\item Innovación learning by using: Información que proporcionan usuarios o clientes.
	\item Innovación learning by failing: Los fracasos ayudan a rectificar y rediseñar o cambiar diferentes elementos de producción.
\end{itemize}


Las transformaciones dadas por estas innovaciones han creado nuevas oportunidades para las
empresas, como un constante desarrollo de la ciencia y tecnología.

Estas innovaciones se protegen mediante patentes, derechos de autor, secretos industriales…
\pagebreak
\section{EBT`s}
Las empresas de base tecnológica son aquellas que hacen el desarrollo la base de su negocio.
\begin{itemize}
	\item Vigilan el entorno fijándose en los avances tecnológicos.
	\item El conocimiento táctico y explicito es su recurso fundamental.
\end{itemize}

Las nuevas EBT`s (NBET`s) tienen escasos recursos financieros, pero se esfuerzan enormemente en el
marketing, sus socios fundadores tienen un alto nivel educativo y se suelen centrar en los sectores
tecnológicos.

\end{document}