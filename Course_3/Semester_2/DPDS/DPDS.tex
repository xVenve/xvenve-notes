\documentclass[12pt, twoside, openright]{report} %fuente a 12pt, formato doble pagina y chapter a la derecha
\raggedbottom % No ajustar el contenido con un salto de pagina

% MÁRGENES: 2,5 cm sup. e inf.; 3 cm izdo. y dcho.
\usepackage[
a4paper,
vmargin=2.5cm,
hmargin=3cm
]{geometry}

% INTERLINEADO: Estrecho (6 ptos./interlineado 1,15) o Moderado (6 ptos./interlineado 1,5)
\renewcommand{\baselinestretch}{1.15}
\parskip=6pt

% DEFINICIÓN DE COLORES para portada y listados de código
\usepackage[table]{xcolor}
\definecolor{azulUC3M}{RGB}{0,0,102}
\definecolor{gray97}{gray}{.97}
\definecolor{gray75}{gray}{.75}
\definecolor{gray45}{gray}{.45}

% Soporte para GENERAR PDF/A
\usepackage[a-1b]{pdfx}

% ENLACES
\usepackage{hyperref}
\hypersetup{colorlinks=true,
	linkcolor=black, % enlaces a partes del documento (p.e. índice) en color negro
	urlcolor=blue} % enlaces a recursos fuera del documento en azul

% Añadir pdfs como partes del documento
\usepackage{pdfpages}

% Quitar la indentación de principio de los parrafos
\setlength{\parindent}{0em}

% EXPRESIONES MATEMATICAS
\usepackage{amsmath,amssymb,amsfonts,amsthm}

\usepackage{txfonts} 
\usepackage[T1]{fontenc}
\usepackage[utf8]{inputenc}

% Insertar graficas y fotos
\usepackage{tikz}
\usepackage{pgfplots}

\usepackage[spanish, es-tabla]{babel} 
\usepackage[babel, spanish=spanish]{csquotes}
\AtBeginEnvironment{quote}{\small}

% diseño de PIE DE PÁGINA
\usepackage{fancyhdr}
\pagestyle{fancy}
\fancyhf{}
\renewcommand{\headrulewidth}{0pt}
\fancyfoot[LE,RO]{\thepage}
\fancypagestyle{plain}{\pagestyle{fancy}}

% DISEÑO DE LOS TÍTULOS de las partes del trabajo (capítulos y epígrafes o subcapítulos)
\usepackage{titlesec}
\usepackage{titletoc}
\titleformat{\chapter}[block]
{\large\bfseries\filcenter}
{\thechapter.}
{5pt}
{\MakeUppercase}
{}
\titlespacing{\chapter}{0pt}{0pt}{*3}
\titlecontents{chapter}
[0pt]                                               
{}
{\contentsmargin{0pt}\thecontentslabel.\enspace\uppercase}
{\contentsmargin{0pt}\uppercase}                        
{\titlerule*[.7pc]{.}\contentspage}                 

\titleformat{\section}
{\bfseries}
{\thesection.}
{5pt}
{}
\titlecontents{section}
[5pt]                                               
{}
{\contentsmargin{0pt}\thecontentslabel.\enspace}
{\contentsmargin{0pt}}
{\titlerule*[.7pc]{.}\contentspage}

\titleformat{\subsection}
{\normalsize\bfseries}
{\thesubsection.}
{5pt}
{}
\titlecontents{subsection}
[10pt]                                               
{}
{\contentsmargin{0pt}                          
	\thecontentslabel.\enspace}
{\contentsmargin{0pt}}                        
{\titlerule*[.7pc]{.}\contentspage}  


% DISEÑO DE TABLAS.
\usepackage{multirow} % permite combinar celdas 
\usepackage{caption} % para personalizar el título de tablas y figuras
\usepackage{floatrow} % utilizamos este paquete y sus macros \ttabbox y \ffigbox para alinear los nombres de tablas y figuras de acuerdo con el estilo definido. Para su uso ver archivo de ejemplo 
\usepackage{array} % con este paquete podemos definir en la siguiente línea un nuevo tipo de columna para tablas: ancho personalizado y contenido centrado
\newcolumntype{P}[1]{>{\centering\arraybackslash}p{#1}}
\DeclareCaptionFormat{upper}{#1#2\uppercase{#3}\par}

% Diseño de tabla para ingeniería
\captionsetup[table]{
	format=hang,
	name=Tabla,
	justification=centering,
	labelsep=colon,
	width=.75\linewidth,
	labelfont=small,
	font=small,
}

% DISEÑO DE FIGURAS.
\usepackage{graphicx}
\graphicspath{{img/}} %ruta a la carpeta de imágenes

% Diseño de figuras para ingeniería
\captionsetup[figure]{
	format=hang,
	name=Fig.,
	singlelinecheck=off,
	labelsep=colon,
	labelfont=small,
	font=small		
}

% NOTAS A PIE DE PÁGINA
\usepackage{chngcntr} %para numeración continua de las notas al pie
\counterwithout{footnote}{chapter}

% LISTADOS DE CÓDIGO
% soporte y estilo para listados de código. Más información en https://es.wikibooks.org/wiki/Manual_de_LaTeX/Listados_de_código/Listados_con_listings
\usepackage{listings}

% definimos un estilo de listings
\lstdefinestyle{estilo}{ frame=Ltb,
	framerule=0pt,
	aboveskip=0.5cm,
	framextopmargin=3pt,
	framexbottommargin=3pt,
	framexleftmargin=0.4cm,
	framesep=0pt,
	rulesep=.4pt,
	backgroundcolor=\color{gray97},
	rulesepcolor=\color{black},
	%
	basicstyle=\ttfamily\footnotesize,
	keywordstyle=\bfseries,
	stringstyle=\ttfamily,
	showstringspaces = false,
	commentstyle=\color{gray45},     
	%
	numbers=left,
	numbersep=15pt,
	numberstyle=\tiny,
	numberfirstline = false,
	breaklines=true,
	xleftmargin=\parindent
}

\captionsetup[lstlisting]{font=small, labelsep=period}
% fijamos el estilo a utilizar 
\lstset{style=estilo}
\renewcommand{\lstlistingname}{\uppercase{Código}}

\pgfplotsset{compat=1.17} 
%-------------
%	DOCUMENTO
%-------------

\begin{document}
\pagenumbering{roman} % Se utilizan cifras romanas en la numeración de las páginas previas al cuerpo del trabajo
	
%----------
%	PORTADA
%----------	
\begin{titlepage}
	\begin{sffamily}
	\color{azulUC3M}
	\begin{center}
		\begin{figure}[H] %incluimos el logotipo de la Universidad
			\makebox[\textwidth][c]{\includegraphics[width=16cm]{Portada_Logo.png}}
		\end{figure}
		\vspace{2.5cm}
		\begin{Large}
			Grado en Ingeniería Informática\\			
			2020-2021\\
			\vspace{2cm}		
			\textsl{Apuntes}\\
			\bigskip
		\end{Large}
		 	{\Huge Dirección de Proyectos de Desarrollo de Software}\\
		 	\vspace*{0.5cm}
	 		\rule{10.5cm}{0.1mm}\\
			\vspace*{0.9cm}
			{\LARGE Jorge Rodríguez Fraile\footnote{\href{mailto:100405951@alumnos.uc3m.es}{Universidad: 100405951@alumnos.uc3m.es}  |  \href{mailto:jrf1616@gmail.com}{Personal: jrf1616@gmail.com}}}\\ 
			\vspace*{1cm}
	\end{center}
	\vfill
	\color{black}
		\includegraphics[width=4.2cm]{img/creativecommons.png}\\
		Esta obra se encuentra sujeta a la licencia Creative Commons\\ \textbf{Reconocimiento - No Comercial - Sin Obra Derivada}
	\end{sffamily}
\end{titlepage}

%----------
%	ÍNDICES
%----------	

%--
% Índice general
%-
\tableofcontents
\thispagestyle{fancy}

%----------
%	TRABAJO
%----------	

\pagenumbering{arabic} % numeración con números arábigos para el resto de la publicación	


%----------
%	COMENZAR A ESCRIBIR AQUI
%----------	

\part{Introducción}

\chapter{Información}

\begin{quote}
Teorías: Ángel García Crespo acrespo@ia.uc3m.es

Prácticas: Israel González Carrasco igcarras@ia.uc3m.es
\end{quote}

\section{Presentación}

Entregas prácticamente semanales.

Grupos de 7 personas, entre 83 y 84

Consultar continuamente a Israel.

A las revisiones deben asistir todos los participantes, se debe redactar
un acta que se le enviara a él para que lo valide(él actúa como cliente).

Trabajo continuo, siempre hay un hito más.

Si se sigue evaluación continua no hay examen final.

50\% Trabajo en equipo, según las entregas, en las que se permiten 2
revisiones de cada entrega.

50\% Trabajo individual, 25 lo evalúan los compañeros y el otro 25 lo
evalúa el profesor.

Todos los trabajos puntuables hay que aprobarlos (5).

Crear un correo de equipo para contactar con el profesor.

USAR el formato de Asunto indicado en las Normas.

Todas las entregas con un formato de documento fijo, con todas las
partes básicas de un documento.

\chapter{Métrica 3}

El cliente solo busca el producto final, no se hace a la idea de todo lo
que hay detrás y que se debe tener en cuenta.

Resolver problemas y conflictos, colaborar con el cliente en la
definición y consecución de objetivos, planificar el proyecto en todos
sus aspectos, dirigir y coordinar recursos, tomar decisiones, etc.

\section{Proyectos de Desarrollo Software}

\textbf{Proyecto}: Tareas interrelacionadas, cosas que debe hacerse y
completarse con tiempo determinado, con cierto coste y otras
restricciones que pueden aplicarse para llegar a un resultado
específico.

Empiezan con una idea u objetivo, que según se progresa se va
detallando.

Son temporales, tienen un inicio y final.

Cada proyecto es único, pero siguen un conjunto de procesos.

Crea algo nuevo, una entrega tangible o intangible que es única.

\section{Metodología de Desarrollo
Software}

Hay muchos participantes y stakeholders.

\textbf{Se necesita:}

\begin{itemize}

\item
  \textbf{Orden}: Definir un orden y establecer distintas prioridades
  para que el proyecto sea un éxito.
\item
  \textbf{Azar}: Evitar que este condicionado por factores aleatorios,
  reducir el azar o suerte.
\item
  \textbf{Eficiencia}: Realizar las tareas de modo racional, para
  lograrlas con en un menor tiempo y esfuerzo.
\end{itemize}

\textbf{Para conseguirlo}:

\textbf{Metodología}: Marco de trabajo usado para estructurar,
planificar y controlar el proceso de desarrollo. Herramientas, Métricas
y Modelos.

\textbf{Marco regulador}: Establecer, ante problemas reales o
potenciales, disposiciones destinadas a obtener un nivel de orientación
y operación óptima. Normas y Estándares.

Hace énfasis al entorno en el cual se plantea y estructura el desarrollo
de un sistema.

\section{Metodología basadas en planificación}

Énfasis en la \textbf{planificación de tareas y control del proyecto},
en especificación precisa de requisitos y modelado. Se centra en el
control del proceso.

\textbf{Principios}:

\begin{itemize}

\item
  \textbf{Modelo de ciclo de vida}: Secuencia estructurada y bien
  definida de las etapas necesarias.
\item
  \textbf{Ejecutar etapas solo una vez:} Lo que se define en cada etapa
  debería ser inamovible y hasta que no se completa con éxito no se pasa
  a la siguiente.
\item
  \textbf{Diferentes profesionales}: Profesionales especializados pueden
  participar en diferentes etapas bien diferenciadas.
\item
  \textbf{Diferentes roles}: Se definen más adelantes.
\end{itemize}

\textbf{El problema}s: No está bien definido el volver a una etapa
anterior, el usuario puede que no vea el producto hasta el final, puede
no adaptarse bien a los cambios y se necesita tener unos buenos
requisitos iniciales (que no siempre es fácil).

\textbf{No existe una metodología universal} para hacer frente a con
éxito a cualquier proyecto de desarrollo de software.

La metodología debe ser adaptada al contexto del proyecto, los recursos
técnicos y humanos.

\section{Métrica V3}

\textbf{Perfiles}: Permite definir que rol deben tomar los
participantes.

\textbf{Proceso}:

\begin{enumerate}
\def\labelenumi{\arabic{enumi}.}

\item
  Planificación de Sistema de Información
\item
  Procesos de desarrollo:

  \begin{itemize}
  
  \item
    Estudio de Viabilidad del Sistema.
  \item
    Análisis del Sistema de Información.
  \item
    Diseño del Sistema de Información.
  \item
    Construcción del Sistema de Información.
  \item
    Implantación y Aceptación del Sistema.
  \end{itemize}
\item
  Mantenimiento de Sistema de Información.
\end{enumerate}

Descompone cada uno de los procesos en actividades, y estas a su vez en tareas.

Cada tarea se describe su contenido haciendo referencia a sus
principales acciones, productos, técnicas, prácticas y participantes.

\textbf{Interfaces}:

\begin{itemize}

\item
  Gestión de Proyectos: Planificación, el seguimiento y control de las
  actividades, los recursos humanos y materiales que intervienen en el
  desarrollo.
\item
  Aseguramiento de la calidad: Garantizar que el sistema resultante
  cumpla con unos requisitos de calidad.
\item
  Seguridad: Gestión de riesgos y planes de contingencia, dota de
  mecanismos de seguridad adicionales.
\item
  Gestión de configuración: Procesos de procedimientos administrativos y
  técnicos durante el desarrollo. Su finalidad es identificar, definir,
  proporcionar información y controlar los cambios en la configuración
  del sistema.
\end{itemize}

\textbf{Participantes}:

\begin{itemize}

\item
  Perfil Directivo: En nuestro proyecto no habrá uno.
\item
  Perfil Jefe de Proyecto: Solo uno en el proyecto.
\item
  Perfil de consultor: Consultor, Especialista en Comunicaciones,
  Técnico de sistemas y Técnico de Comunicaciones.
\item
  Perfil de Analista: Analista, DBA, Equipo de proyecto, Equipo de
  Formación, etc.
\item
  Perfil de Programador.
\item
  Cliente y entidades externas: Clientes, Proveedores y Stakeholders.
\end{itemize}

\section{Oferta}

\begin{quote}
Lo más importante del proyecto, es que nos lo den, para ello hay que
hacer una buena oferta. Una vez nos lo asignen ya nos preocupamos de los
demás factores.
\end{quote}

Ser capaz de \textbf{demostrar que se ha entendido lo que el cliente
pretende}, el problema y como se va a solucionar.

Debe ser un \textbf{documento agradable a la vista}, tiene que demostrar
el interés por el proyecto.

Sin faltas de ortografía, tipografía fija, facilitar la lectura\ldots{}

Se debe \textbf{incluir el precio del proyecto}, como diferencia entre
el coste y el precio.

\textbf{Coste}: Cuanto nos va a costar hacerlo a nosotros. Personal,
equipo, viajes, \ldots{}

\textbf{Costes directos}: Si hay proyecto, el gasto que supone.

\textbf{Costes indirectos}: Los que tiene la empresa haya o no proyecto.
Luz, internet, limpieza, jefe, comercial, gestoría, coche de jefe,
comida cliente, RR. HH., etc. Se suman todos estos costes y se saca un
porcentaje con respecto a los directos. De esta manera aplicamos sobre
los costes directos este extra.

\textbf{Precio}: Lo que el cliente está dispuesto a pagar, no solo
cubrirá el coste, sino que debe haber un margen de beneficio y un marco
de riesgos.

\textbf{EL CLIENTE SOLO CONOCE EL PRECIO, NO EL COSTE, NI BENEFICIO.}

\textbf{Beneficio}: Lo que se gana, pero debe haber un extra por si
aumentan los cotes, para no tener perdidas.

El \textbf{precio se da SIN IVA}, por si cambia antes de que nos acepten
la oferta, de esta manera ya se aplicará más adelante.

Hay que tener en cuenta el concepto de beneficio y el IVA.

Debe darse un \textbf{periodo de garantía} al cliente (en software no
hay periodo obligatorio establecido)

La oferta tiene un \textbf{periodo de validez}, de 1, 2 o 6 meses, para
que en un futuro no nos den ese proyecto. Puede haber cambiado la
situación o estar desactualizado.

Hay que establecer la \textbf{forma de pago}, cuando se van a realizar
los pagos. Cuanto es el adelanto, como se pagará durante el proceso y el
pego final a la entrega.

\part{Oferta}
\href{https://drive.google.com/file/d/1LUPCojygmk0RfA8fO2AoGw3kITsFOROj/view?usp=sharing}{Acta I}

\href{https://drive.google.com/file/d/1LV5uRdDUyES3aMeB0S8oAvGKNwPv9xuK/view?usp=sharing}{OFE}

\part{Documento de Cálculo de costes}
\href{https://drive.google.com/file/d/1LVQf5fXYCUpaM7lFV_R0g4kbWvOobmPp/view?usp=sharing}{DCC}

\part{Plan de Gestión de la Configuración}
\href{https://drive.google.com/file/d/1KpgBc6DvZjsW_vxBUBsAw_BblaD7-pcu/view?usp=sharing}{Acta II}

\href{https://drive.google.com/file/d/1KrtKpw7KcYm6gr0jh-eglSh6dq1oXHfI/view?usp=sharing}{PGC}

\part{Plan de Aseguramiento de la Calidad}
\href{https://drive.google.com/file/d/1LP9jGjzFrMspRAP3BBgR9bi5v_nQg9X2/view?usp=sharing}{Acta III}

\href{https://drive.google.com/file/d/1LQIohiDeDX5awZ9tX5o3Yh5BKFGhDo4e/view?usp=sharing}{PGCal}

\part{Estudio de Viabilidad del Sistema}
\href{https://drive.google.com/file/d/1KUi0SQfIuWXERD3Xd0mY-P5zsGO2RHk8/view?usp=sharing}{Acta IV}

\href{https://drive.google.com/file/d/1K4W4-toIsbZd6jCFqFi4iKWK81FgYl7w/view?usp=sharing}{EVS}

\part{Analisis del Sistema de Información}
\href{https://drive.google.com/file/d/1KI0-A_pAgCdRma0QBC47FAzh9HgQYYxG/view?usp=sharing}{Acta VII}

\href{https://drive.google.com/file/d/1KJitQFAY8ON-x-KGKTZglPeCWCTn3NIa/view?usp=sharing}{DAS}

\part{Diseño del Sistema de Información}
\href{https://drive.google.com/file/d/1KXCoMFY1-TX_T6m1UsY10ZL_tCNOoZBP/view?usp=sharing}{Acta IX}

\href{https://drive.google.com/file/d/1KX1y1D9sbK7g2UXwcDyabhDIhQTO_bc6/view?usp=sharing}{DDS}

\part{Documento de Pruebas}
\href{https://drive.google.com/file/d/1KSNvBWSO13F0WFxPlh84Zz5VClIcziVL/view?usp=sharing}{Acta X}

\href{https://drive.google.com/file/d/1KTolKlLgJPkj9m8FrqmaaNf1BuWHrPTa/view?usp=sharing}{DPS}

\part{Documento de Implantación del Sistema}
\href{https://drive.google.com/file/d/1L9NI0xoDJH_OpDe8Jrl_FbjDfOHkmwqy/view?usp=sharing}{Acta XI}

\href{https://drive.google.com/file/d/1L8zoipQtBPPKfyTNuAEOnztPKpgmIxxu/view?usp=sharing}{DIS}

\part{Documento Historico del Proyecto}
\href{https://drive.google.com/file/d/1KYNfd-q9CbjTFt7WkpDa8RgoX1HNI4-7/view?usp=sharing}{Acta XII}

\href{https://drive.google.com/file/d/1KYXAAvO7gWNyC9FYG57YTcEGIzl4M3Y-/view?usp=sharing}{DHP}

\part{Informe Quincenal de Seguimiento}
\href{https://drive.google.com/file/d/1L1xHATSFQSF5jk4TPlm9KfxG3O9AqxXL/view?usp=sharing}{Acta V}

\href{https://drive.google.com/file/d/1L0n2uRXvH9kIWdEFp10S9zacsIA-4fzl/view?usp=sharing}{IQS}

\href{https://drive.google.com/file/d/1KzZcasG58pMZ9TsOfw4EUvxpW_hrV4vn/view?usp=sharing}{Acta VI}

\href{https://drive.google.com/file/d/1KzBE7g3gP0J8iQwsn1guw5SNbN2lWtcD/view?usp=sharing}{IQS 2}

\href{https://drive.google.com/file/d/1L4dBHTJEHqkG2Z-W_vecDaYAFV_Vc7-b/view?usp=sharing}{Acta VII}

\href{https://drive.google.com/file/d/1KtQgpGa3H_rBSl-UJuhqK73zwbFCsNZZ/view?usp=sharing}{IQS 3}

\part{Registros del Cambio}
\href{https://drive.google.com/file/d/1Jnz0MW4VxhtsBUqZ1VSbHajtMjnYr_H_/view?usp=sharing}{RC 1}

\href{https://drive.google.com/file/d/1Jng5kkHZad_YXAW_xT9Y9qURlIcrOjBR/view?usp=sharing}{RC 2}

\href{https://drive.google.com/file/d/1JnnDv2CSdKSPheIHd8XCanEg3D2tV3oV/view?usp=sharing}{RC 3}

\part{Libro del profesor: Métrica V3}
\href{https://drive.google.com/file/d/12oqnWg1X6irCLH1gw0SLQ52XkxQkfujR/view?usp=sharing}{Métrica V3}

\end{document}
