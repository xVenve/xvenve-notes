\documentclass[12pt]{report} %fuente a 12pt

% MÁRGENES: 2,5 cm sup. e inf.; 3 cm izdo. y dcho.
\usepackage[
a4paper,
vmargin=2.5cm,
hmargin=3cm
]{geometry}

% INTERLINEADO: Estrecho (6 ptos./interlineado 1,15) o Moderado (6 ptos./interlineado 1,5)
\renewcommand{\baselinestretch}{1.15}
\parskip=6pt

% DEFINICIÓN DE COLORES para portada y listados de código
\usepackage[table]{xcolor}
\definecolor{azulUC3M}{RGB}{0,0,102}
\definecolor{gray97}{gray}{.97}
\definecolor{gray75}{gray}{.75}
\definecolor{gray45}{gray}{.45}

% Soporte para GENERAR PDF/A
\usepackage[a-1b]{pdfx}

% ENLACES
\usepackage{hyperref}
\hypersetup{colorlinks=true,
	linkcolor=black, % enlaces a partes del documento (p.e. índice) en color negro
	urlcolor=blue} % enlaces a recursos fuera del documento en azul

\usepackage{pdfpages}
\setlength{\parindent}{0em}

% EXPRESIONES MATEMATICAS
\usepackage{amsmath,amssymb,amsfonts,amsthm}

\usepackage{txfonts} 
\usepackage[T1]{fontenc}
\usepackage[utf8]{inputenc}

\usepackage{tikz}
\usepackage{pgfplots}

\usepackage[spanish, es-tabla]{babel} 
\usepackage[babel, spanish=spanish]{csquotes}
\AtBeginEnvironment{quote}{\small}

% diseño de PIE DE PÁGINA
\usepackage{fancyhdr}
\pagestyle{fancy}
\fancyhf{}
\renewcommand{\headrulewidth}{0pt}
\rfoot{\thepage}
\fancypagestyle{plain}{\pagestyle{fancy}}

% DISEÑO DE LOS TÍTULOS de las partes del trabajo (capítulos y epígrafes o subcapítulos)
\usepackage{titlesec}
\usepackage{titletoc}
\titleformat{\chapter}[block]
{\large\bfseries\filcenter}
{\thechapter.}
{5pt}
{\MakeUppercase}
{}
\titlespacing{\chapter}{0pt}{0pt}{*3}
\titlecontents{chapter}
[0pt]                                               
{}
{\contentsmargin{0pt}\thecontentslabel.\enspace\uppercase}
{\contentsmargin{0pt}\uppercase}                        
{\titlerule*[.7pc]{.}\contentspage}                 

\titleformat{\section}
{\bfseries}
{\thesection.}
{5pt}
{}
\titlecontents{section}
[5pt]                                               
{}
{\contentsmargin{0pt}\thecontentslabel.\enspace}
{\contentsmargin{0pt}}
{\titlerule*[.7pc]{.}\contentspage}

\titleformat{\subsection}
{\normalsize\bfseries}
{\thesubsection.}
{5pt}
{}
\titlecontents{subsection}
[10pt]                                               
{}
{\contentsmargin{0pt}                          
	\thecontentslabel.\enspace}
{\contentsmargin{0pt}}                        
{\titlerule*[.7pc]{.}\contentspage}  


% DISEÑO DE TABLAS.
\usepackage{multirow} % permite combinar celdas 
\usepackage{caption} % para personalizar el título de tablas y figuras
\usepackage{floatrow} % utilizamos este paquete y sus macros \ttabbox y \ffigbox para alinear los nombres de tablas y figuras de acuerdo con el estilo definido. Para su uso ver archivo de ejemplo 
\usepackage{array} % con este paquete podemos definir en la siguiente línea un nuevo tipo de columna para tablas: ancho personalizado y contenido centrado
\newcolumntype{P}[1]{>{\centering\arraybackslash}p{#1}}
\DeclareCaptionFormat{upper}{#1#2\uppercase{#3}\par}

% Diseño de tabla para ingeniería
\captionsetup[table]{
	format=upper,
	name=TABLA,
	justification=centering,
	labelsep=period,
	width=.75\linewidth,
	labelfont=small,
	font=small,
}

% DISEÑO DE FIGURAS.
\usepackage{graphicx}
\graphicspath{{img/}} %ruta a la carpeta de imágenes

% Diseño de figuras para ingeniería
\captionsetup[figure]{
	format=hang,
	name=Fig.,
	singlelinecheck=off,
	labelsep=period,
	labelfont=small,
	font=small		
}

% NOTAS A PIE DE PÁGINA
\usepackage{chngcntr} %para numeración contínua de las notas al pie
\counterwithout{footnote}{chapter}

% LISTADOS DE CÓDIGO
% soporte y estilo para listados de código. Más información en https://es.wikibooks.org/wiki/Manual_de_LaTeX/Listados_de_código/Listados_con_listings
\usepackage{listings}

% definimos un estilo de listings
\lstdefinestyle{estilo}{ frame=Ltb,
	framerule=0pt,
	aboveskip=0.5cm,
	framextopmargin=3pt,
	framexbottommargin=3pt,
	framexleftmargin=0.4cm,
	framesep=0pt,
	rulesep=.4pt,
	backgroundcolor=\color{gray97},
	rulesepcolor=\color{black},
	%
	basicstyle=\ttfamily\footnotesize,
	keywordstyle=\bfseries,
	stringstyle=\ttfamily,
	showstringspaces = false,
	commentstyle=\color{gray45},     
	%
	numbers=left,
	numbersep=15pt,
	numberstyle=\tiny,
	numberfirstline = false,
	breaklines=true,
	xleftmargin=\parindent
}

\captionsetup[lstlisting]{font=small, labelsep=period}
% fijamos el estilo a utilizar 
\lstset{style=estilo}
\renewcommand{\lstlistingname}{\uppercase{Código}}

\pgfplotsset{compat=1.17} 
%-------------
%	DOCUMENTO
%-------------

\begin{document}
\pagenumbering{roman} % Se utilizan cifras romanas en la numeración de las páginas previas al cuerpo del trabajo
	
%----------
%	PORTADA
%----------	
\begin{titlepage}
	\begin{sffamily}
	\color{azulUC3M}
	\begin{center}
		\begin{figure}[H] %incluimos el logotipo de la Universidad
			\makebox[\textwidth][c]{\includegraphics[width=16cm]{Portada_Logo.png}}
		\end{figure}
		\vspace{2.5cm}
		\begin{Large}
			Grado en Ingeniería Informática\\			
			2020-2021\\
			\vspace{2cm}		
			\textsl{Apuntes}\\
			\bigskip
		\end{Large}
		 	{\Huge Diseño de Sistemas Operativos}\\
		 	\vspace*{0.5cm}
	 		\rule{10.5cm}{0.1mm}\\
			\vspace*{0.9cm}
			{\LARGE Jorge Rodríguez Fraile\footnote{\href{mailto:100405951@alumnos.uc3m.es}{Universidad: 100405951@alumnos.uc3m.es}  |  \href{mailto:jrf1616@gmail.com}{Personal: jrf1616@gmail.com}}}\\ 
			\vspace*{1cm}
	\end{center}
	\vfill
	\color{black}
		\includegraphics[width=4.2cm]{img/creativecommons.png}\\
		Esta obra se encuentra sujeta a la licencia Creative Commons\\ \textbf{Reconocimiento - No Comercial - Sin Obra Derivada}
	\end{sffamily}
\end{titlepage}

%----------
%	ÍNDICES
%----------	

%--
% Índice general
%-
\tableofcontents
\thispagestyle{fancy}

%--
% Índice de figuras. Si no se incluyen, comenta las líneas siguientes
%-
\listoffigures
\thispagestyle{fancy}

%--
% Índice de tablas. Si no se incluyen, comenta las líneas siguientes
%-
\listoftables
\thispagestyle{fancy}

%----------
%	TRABAJO
%----------	
\clearpage
\pagenumbering{arabic} % numeración con números arábigos para el resto de la publicación	


%----------
%	COMENZAR A ESCRIBIR AQUI
%----------	


\section{Información}

\href{https://www.notion.so/Notas-2607b92b49e64106afd141326b010798}{Notas}

\begin{quote}
Teorías: Javier García Guzmán

Prácticas: Mat Max Montalvo Martínez
\end{quote}

\chapter{Tema 1: Arquitectura de Sistemas
IoT}

\href{https://learning.oreilly.com/library/view/internet-of-things/9781788470599/a7f866bd-4ac8-47f3-a175-0f10d91a5ce2.xhtml}{Definición
de Internet de las Cosas}

\href{https://learning.oreilly.com/library/view/internet-of-things/9781119456742/part04.xhtml\#part}{Aplicaciones
Actuales de Internet de las Cosas}

\href{https://learning.oreilly.com/library/view/build-your-own/9781484244982/html/474034_1_En_2_Chapter.xhtml}{Arquitectura
de Sistemas de Internet de las Cosas}

\textbf{IoT - Internet de las Cosas}: Consiste en conectar a internet,
cualquier dispositivo, vehículo, edificios o en general objetos a los
que se les haya dotado de sensores, actuadores y conexión a la red. Lo
que les permite obtener e intercambiar información. Red de objetos
conectados a internet que aporta valor añadidos a los usuarios que
interactúan con ellos.

Consiste en añadir \textbf{inteligencia computacional} a dispositivos
para mejorar las funcionalidades.

Permite que los dispositivos puedan intercambiar información.

Se busca que sean pequeños y tengan un chip que les permita
\textbf{conectarse a la red} y operar en ella. Lo estandarizo IBM.

El \textbf{top 6 áreas} de aplicación de IoT:

\begin{enumerate}
\def\labelenumi{\arabic{enumi}.}

\item
  \textbf{Industrial/Fabricación}: Automatizar, controlar la
  distribución, gestión de instalaciones.
\item
  \textbf{Transporte/Movilidad:} Coches, tráfico en ciudad, vehículos de
  transporte masivo y transporte industrial.
\item
  \textbf{Energía/Gestión eléctrica}: Predecir consumo, personalizar,
  bien estar de los ocupantes, monitorizar el consumo detallado,
  instalaciones con sensores.
\item
  \textbf{Gestión de inventarios/Comercial}: Saber cuánto queda y pedir,
  facilitar compras.
\item
  \textbf{Ciudad}: Recoge muchas áreas; basura, vigilancia, trafico,
  etc.
\item
  \textbf{Salud:} Investigación, la forma de tratar, las emergencias,
  distribución de información médica y dispositivos.
\end{enumerate}

\textbf{Agricultura de precisión}: Según las previsiones ambientales y
meteorológica, plagas y demás se calcula el mejor momento para sembrar.
Además, cuando ya está plantado, las condiciones del suelo, cuando regar
y abonar las tierras, así como la recolecta.

\newpage

\section{En IoT de vehículos hay distintos
niveles}

\textbf{0}: Sin automatización.

\textbf{1}: Asistencia en la conducción.

\textbf{2}: Control de carril, lateral y longitudinal.

\textbf{3}: Conducción autónoma, pero el conductor en su puesto, para
riesgos.

\textbf{4}: Conducción autónoma, pero el conductor en su puesto,
supervisar.

\textbf{5}: Conducción totalmente autónoma, sin conductor.

\section{Elementos IoT}

\textbf{Colector}: Recogen información, sensores, o se activan,
actuadores, que se intercambian en internet.

\textbf{Transmisor}: Puertas de enlace, pasarelas, pasan los datos a la
red desde los dispositivos.

\textbf{Agregación + Distribución}: Calculo y procesamiento de la
información.

\textbf{Consumidor}: Los usuarios/clientes acceden a los datos.

\section{Evolución}

3 generaciones.

\begin{enumerate}
\def\labelenumi{\arabic{enumi}.}
\item
  \textbf{RFID y sensores}

  Tecnología de detección por radio frecuencia.

  Las cosas contengan información, las etiquetas (como NFC, se
  estandarizó para indicar que datos contiene) y tener un dispositivo
  que al acercarlo podamos leerla.
\item
  \textbf{Web services e inter-networking} (2004-2012): Interconexión
  completa de las cosas y la red de las cosas.

  IPv4, HTTP, Bluetooth, TCP, UDP, etc.

  Pasan a tener una manera fácil de conectarse los dispositivos entre sí
  o con internet.
\item
  \textbf{Social, Cloud \& ICN:} La era de la computación en la nube y
  la Internet del futuro.

  En esta generación la lógica pasa a estar en la nube, no en el
  dispositivo.

  Gestión de grandes cantidades de información.

  Seguridad, evitar accesos fraudulentos.
\end{enumerate}

\section{Arquitectura de in Sistema
IoT}
\begin{description}
	\item[Dispositivos (Devices)] Sensores y actuadores FÍSICOS, que
	normalmente tienen un microprocesador, que mide el medio físico y
	transforma las mediciones a señales digitales. Su función es tomar
	medidas y procesarlas, pero su función no puede ser solo transmitir la
	información.
	
	Actuador, Sensores, LED, LCD, Beacon (la parte dispositivo), Termostato,
	RFID, Trampa para ratones inteligente, Dispositivos embebidos, etc.
	\item[Pasarela (Gateway)] Dispositivo o protocolo con la capacidad de
	comunicar con internet los dispositivos, para transmitir los datos
	tomados. Su única función es transmitir.
	
	Router, Wifi, GSM, Bluetooth, Zigbee, Raspberry a veces, AMQP, CoAP,
	LoRaWAN (sistema de radio), Wimax.
	
	Un móvil está entre Device y Gateway.
	\item[Plataforma IoT (IoT Platform)] Conjunto de servicios
	orquestados para gestionar una gran red de dispositivos interconectados
	y que proporcionan información a aplicaciones u otros tipos de sistemas
	de información. Gestiona y almacena grandes cantidades de datos y las
	redirige. Funciona com middleware.
	
	Es una nube de servidores que dan servicios:
	
	\begin{enumerate}
		\item \textbf{Message broker y Message bus}: Se encarga de conectar los
		dispositivos físicos con los distintos procesos que forman parte de la
		red de IoT. Manda los datos a todos los que estén conectados a su bus,
		suscritos por API Rest.
		\item
		\textbf{Message router}: Está suscrito al message broker, los mensajes
		que recibe los enriquece; dando información semántica, de contexto, de
		estado y los reenvía a aquellos componentes que van a gestionar la
		lógica de las aplicaciones relacionadas con la nube IoT. Otra cosa que
		hace es transformar datos, descomprimir y decodificar datos para
		hacerlos más fáciles de procesar y tratar.
		\item
		\textbf{Rest API}: Interfaz que usan otros programas para obtener los
		servicios o las funcionalidades de un componente. Esta API se
		caracteriza por ser accesible por http e independiente del estado del
		sistema.
		\item
		\textbf{Data Management}: Para almacenar y gestionar los datos tomados
		en la red.
		\item
		\textbf{Rule engine}: Permite monitorear los mensajes recibidos desde
		el router y permite lanzar distintas acciones en distintos elementos.
		Decide que acción tomar.
		
		Ejem: Si se abre la puerta, entonces avisar de intruso.
		\item
		\textbf{Microservicios}: Proporciona funcionalidades muy específicas a
		través de una interfaz API Rest bien definidas mediante un contrato de
		datos. Muchos los coordina el rule engine. Se busca que este muy
		cohesionado y poco acoplado.
		
		Ejem: El que actualiza la estación meteorológica en el móvil, es un
		proceso muy concreto.
		\item
		\textbf{Device manager}: Permite monitorizar algunos elementos de los
		sensores físicos como si está activo, la batería o si está conectado a
		la red.
		\item
		\textbf{App y User management}: Sistema de permisos que identifica y
		gestiona el acceso de usuarios y aplicaciones.
	\end{enumerate}

	\item[Aplicación (Application)] La interfaz que el usuario utiliza
	para controlar el sistema.

\end{description}

\textbf{iBeacon} es un ejemplo de protocolo y \textbf{Beacon} es el
dispositivo. Ambos están relacionados con dispositivos.

\textbf{Small Data}: Que solo proporcione la información de valor
añadido. Es un conjunto de datos con un volumen y un formato que hacen
que los datos sean accesibles, informativos y procesables.

\textbf{Wimax}: Conjunto de tecnologías y protocolos para aumentar el
alcance de las redes inalámbricas (en vez de 30 metros, 40 kilómetros).

\chapter{Tema 2: Sensores y
Actuadores}

\href{https://learning.oreilly.com/library/view/internet-of-things/9781788470599/d39be056-b166-476e-868e-c415e4dfa886.xhtml}{Introducción
a Sensores y Actuadores} 

Hasta la sección `Up to Functional examples
(putting it all together)' incluida.

\section{Sensores}

Conjunto de componentes electrónicos capaces de detectar cambios físicos
en el entorno y enviar información a otros componentes electrónicos,
generalmente un procesador de computadora.

\textbf{Ejemplos}: Sensor de luz (LDR), sensor de ultrasonidos,
giroscopio, fototransistor, Reed switch, \ldots{}

Los sensores se pueden clasificar en tipos según lo que miden: Gases,
velocidad, flujo, fugas, movimiento, electricidad, \ldots{}

Según la señal que produce:

\begin{itemize}

\item
  \textbf{Analógico}: Produce voltaje analógico constante de los medido.
  El grafico de voltaje sobre el tiempo debe ser continuo y suave.

\begin{figure}[H]
	\ffigbox[\FBwidth]
	{\caption{Diagrama de voltaje Sensor Analogico}}
	{\includegraphics[scale=.5]{image-20210307210139988.png}}
\end{figure}

    Sensor de presión, sensor de luz, sensor de temperatura,
    acelerómetro, sensor de sonido.

\item
  \textbf{Digital}: Produce un voltaje discreto, por lo general tendrá
  uno u otro de dos valores, 0V (apagado) a 5V (encendido). Gracias a la
  miniaturización hay más dado que se puede introducir un conversor.

\begin{figure}[H]
	\ffigbox[\FBwidth]
	{\caption{Diagrama de voltaje Sensor Digital}}
	{\includegraphics[scale=.5]{image-20210307210421078.png}}
\end{figure}

    Sensor de ultrasonidos, sensor de infrarrojos, acelerómetro, sensor
    de sonido (suele ser analógico), sensor de temperatura.
\end{itemize}

Según si necesitan energía:

\begin{itemize}

\item
  \textbf{Sensor activo}: Siempre \textbf{necesitan} su propia fuente de
  energía.

  \begin{itemize}
  
  \item
    Sensor de ultrasonidos, radar, LiDAR, sensor de humedad, cámara
    infrarroja.
  \end{itemize}
\item
  \textbf{Sensor pasivo}: \textbf{No necesitan} una fuente de energía,
  usan factores externos para alimentarse.

  \begin{itemize}
  
  \item
    Sensor infrarrojo (fotodiodo infrarrojo), sensor PIR, sensor de luz
    (LDR)
  \end{itemize}
\end{itemize}

\textbf{Sensor piezoeléctrico:}

\begin{enumerate}
\def\labelenumi{\arabic{enumi}.}

\item
  Un cristal piezoeléctrico se coloca entre dos placas de metal que
  están en perfecto equilibrio y conduce ninguna corriente eléctrica.
\item
  Las placas de metal aplican tensión o fuerza mecánica sobre el
  material que hace que las cargas eléctricas del cristal se
  desequilibren.
\item
  Las placas de metal recogen esas cargas y produce un voltaje y envía
  una corriente eléctrica a través de un circuito.
\end{enumerate}

\section{Actuadores}

Cualquier dispositivo capaz de intervenir para cambiar las condiciones
físicas del entorno generando los datos.

\textbf{Ejemplos}: Display, LED, servomotor, motor de paso a paso,
Relay, solenoide, actuadores lineales, \ldots{}

\section{Factores de selección de Sensores y
Actuadores}

\textbf{Factores ambientales}: Temperatura, Humedad, Corrosión,
Interferencia electromagnética, Tamaño, Rudeza y Consumo de energía.

\textbf{Factores económicos}: Coste, Disponibilidad y Tiempo de vida.

\textbf{Factores característicos del sensor}: Sensibilidad, Rango,
Estabilidad, Repetibilidad, Rango de error, Tiempo de respuesta y
Linealidad.

\chapter{Tema 3: Sistemas operativos embebidos para Dispositivos
IoT}

\href{https://aulaglobal.uc3m.es/mod/url/view.php?id=3123882}{Introducción
a los Sistemas Embebidos} Solo capítulo 1

\href{https://aulaglobal.uc3m.es/mod/url/view.php?id=3123883}{Sistemas
Operativos Embebidos} Capítulo 9

\section{Que es un sistema embebido o
integrado?}

Son sistemas que integran uno o más sensores y que son capaces de
comunicarse con la red, con capacidades limitadas, por lo que están
entre la capa de Dispositivos y Pasarelas.

Se aplican sobre sobre cosas cotidianas para mejorarla, pero no
proporciona una mayor complejidad del sistema, permite realizar la
mismas funciones o alguna más pero mejor.

Todo dispositivo IoT es un sistema embebido, pero no todo sistema
embebido es IoT. Los sistemas IoT son accesibles a través de internet y
puede enviar la información que registra en tiempo real por internet.

\textbf{Los sistema embebidos o integrados} son aquellos capaces de
interactuar con el usuario o con otra herramienta invisible para el
usuario. Es decir, no tiene por qué haber una interacción directa con el
usuario (un pendrive se enchufa al ordenador, no al usuario)

\begin{itemize}

\item
  Ejem: Memoria flash, pendrive, sistema antibloqueo de ruedas.
\end{itemize}

\textbf{Un sistema IoT} es aquel con el que podemos interactuar
directamente, acceder a sus datos o que nos los muestre, y tiene
capacidad de internet. Hoy en día es muy barato transformar un sistema
embebido a IoT.

\textbf{Factor clave de los sistemas embebidos}:

\begin{itemize}

\item
  La \textbf{eficiencia}, velocidad a la que responde o realiza la tarea
  específica). Para alcanzar la eficiencia \textbf{se cambia el enfoque
  de la programación}, no hay recursos ilimitados y hay que adaptarlo
  para que consuma poca energía y memoria.
\item
  El \textbf{consumo de energía}, si se encuentra en algún lugar remoto
  y tiene una batería debe durar mucho.
\item
  El \textbf{uso de memoria}, ya que afecta al rendimiento y son caras.
\item
  \textbf{Precio}, ya que ante productos similares se elige el más
  barato.
\item
  \textbf{Sistema critico}, aquel del que el tiempo de respuesta es
  clave, que si falla puede correr riesgo alguna vida humana.
\end{itemize}

\textbf{No podemos aprovechar la Ley de Moore}, nos tenemos que ajustar
al sistema como está actualmente, no podemos esperar a que pase el
tiempo suficiente para que compremos otro que de mejor rendimiento. Hay
que diseñar sistemas que sean rápidos con la tecnología actual y pueda
durar en el un largo periodo de tiempo.

\textbf{Del cuestionario:}

\begin{itemize}

\item
  Se dice que un \textbf{sistema es en tiempo real si el tiempo de
  respuesta es crítico}. Como el sistema ABS o de detección de colisión.
\item
  Es cierto que la mayoría de los sistemas informáticos integrados están
  diseñados por equipos pequeños con plazo ajustados.
\item
  Un sistema en tiempo real se define como un sistema cuya corrección de
  la puntualidad de su respuesta.
\item
  Es cierto que un sistema integrado puede definirse como un sistema de
  control o un sistema informático diseñado para realizar una tarea
  específica.
\end{itemize}

\subsection{Ordenador personal vs. Sistema
embebido}

\textbf{Sistema embebido}: Son específicos de una aplicación, se
focalizan en una tarea o conjunto de tareas relacionadas en todo
momento.

\begin{itemize}

\item
  Todos los recursos están dirigidos a realizar esa tarea, por lo que la
  realiza muy bien, pero no van sobrados de recursos y una aplicación es
  muy difícil o imposible. El software y hardware lo diseñan juntos por
  lo que es más eficiente y fiable, se adaptan al hardware
  perfectamente.
\item
  Utilizan arquitecturas muy variadas, con diferentes CPU, periféricos,
  SO y prioridades de diseño.
\item
  El tiempo de arranque es casi instantáneo, medido en segundos.
\end{itemize}

\textbf{Computadora de escritorio}: Puede ejecutar cualquier clase se
aplicación según las necesidades del usuario.

\begin{itemize}

\item
  Está listo para cualquier tarea por lo que consume más energía y
  recursos. El diseño de hardware lo desarrollan empresas distintas, por
  lo que sobran recursos o se requiere más de los que hay, sobreestima.
  Además, se pueden ampliar fácil y económicamente si es necesario.
\item
  Usan una arquitectura muy similar todos y ejecutan software en
  sistemas idénticos.
\item
  El tiempo de inicio se puede medir en minutos cuando se carga desde
  disco.
\end{itemize}

\textbf{Del cuestionario:}

\begin{itemize}

\item
  Un sistema embebido no necesita interacción humana para realizar
  tareas.
\item
  Un sistema embebido necesita menos potencia operativa que una
  computadora.
\item
  Los ordenadores se pueden reprogramar par aun nuevo propósito.
\item
  Los ordenadores son difíciles cuando se usan, en comparación con un
  sistema embebido.
\item
  Los ordenadores pueden realizar muchas tareas.
\end{itemize}

\chapter{Recursos}
\href{https://learning.oreilly.com/playlists/5a6c045f-e39c-465e-9e7c-60dcbb12aebb}{Lista
de libros de referencia}
\includepdf[pages=-]{docs/Referencias.pdf}



\end{document}
