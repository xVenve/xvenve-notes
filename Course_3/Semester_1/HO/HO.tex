\documentclass[12pt, twoside, openright]{report} %fuente a 12pt, formato doble pagina y chapter a la derecha
\raggedbottom % No ajustar el contenido con un salto de pagina

% MÁRGENES: 2,5 cm sup. e inf.; 3 cm izdo. y dcho.
\usepackage[
a4paper,
vmargin=2.5cm,
hmargin=3cm
]{geometry}

% INTERLINEADO: Estrecho (6 ptos./interlineado 1,15) o Moderado (6 ptos./interlineado 1,5)
\renewcommand{\baselinestretch}{1.15}
\parskip=6pt

% DEFINICIÓN DE COLORES para portada y listados de código
\usepackage[table]{xcolor}
\definecolor{azulUC3M}{RGB}{0,0,102}
\definecolor{gray97}{gray}{.97}
\definecolor{gray75}{gray}{.75}
\definecolor{gray45}{gray}{.45}

% Soporte para GENERAR PDF/A
\usepackage[a-1b]{pdfx}

% ENLACES
\usepackage{hyperref}
\hypersetup{colorlinks=true,
	linkcolor=black, % enlaces a partes del documento (p.e. índice) en color negro
	urlcolor=blue} % enlaces a recursos fuera del documento en azul

% Añadir pdfs como partes del documento
\usepackage{pdfpages}
\usepackage{multicol}


% Quitar la indentación de principio de los parrafos
\setlength{\parindent}{0em}

% EXPRESIONES MATEMATICAS
\usepackage{amsmath,amssymb,amsfonts,amsthm}

\usepackage{txfonts} 
\usepackage[T1]{fontenc}
\usepackage[utf8]{inputenc}

% Insertar graficas y fotos
\usepackage{tikz}
\usepackage{pgfplots}

\usepackage[spanish, es-tabla]{babel} 
\usepackage[babel, spanish=spanish]{csquotes}
\AtBeginEnvironment{quote}{\small}

% diseño de PIE DE PÁGINA
\usepackage{fancyhdr}
\pagestyle{fancy}
\fancyhf{}
\renewcommand{\headrulewidth}{0pt}
\fancyfoot[LE,RO]{\thepage}
\fancypagestyle{plain}{\pagestyle{fancy}}

% DISEÑO DE LOS TÍTULOS de las partes del trabajo (capítulos y epígrafes o subcapítulos)
\usepackage{titlesec}
\usepackage{titletoc}
\titleformat{\chapter}[block]
{\large\bfseries\filcenter}
{\thechapter.}
{5pt}
{\MakeUppercase}
{}
\titlespacing{\chapter}{0pt}{0pt}{*3}
\titlecontents{chapter}
[0pt]                                               
{}
{\contentsmargin{0pt}\thecontentslabel.\enspace\uppercase}
{\contentsmargin{0pt}\uppercase}                        
{\titlerule*[.7pc]{.}\contentspage}                 

\titleformat{\section}
{\bfseries}
{\thesection.}
{5pt}
{}
\titlecontents{section}
[5pt]                                               
{}
{\contentsmargin{0pt}\thecontentslabel.\enspace}
{\contentsmargin{0pt}}
{\titlerule*[.7pc]{.}\contentspage}

\titleformat{\subsection}
{\normalsize\bfseries}
{\thesubsection.}
{5pt}
{}
\titlecontents{subsection}
[10pt]                                               
{}
{\contentsmargin{0pt}                          
	\thecontentslabel.\enspace}
{\contentsmargin{0pt}}                        
{\titlerule*[.7pc]{.}\contentspage}  


% DISEÑO DE TABLAS.
\usepackage{multirow} % permite combinar celdas 
\usepackage{caption} % para personalizar el título de tablas y figuras
\usepackage{floatrow} % utilizamos este paquete y sus macros \ttabbox y \ffigbox para alinear los nombres de tablas y figuras de acuerdo con el estilo definido. Para su uso ver archivo de ejemplo 
\usepackage{array} % con este paquete podemos definir en la siguiente línea un nuevo tipo de columna para tablas: ancho personalizado y contenido centrado
\newcolumntype{P}[1]{>{\centering\arraybackslash}p{#1}}
\DeclareCaptionFormat{upper}{#1#2\uppercase{#3}\par}

% Diseño de tabla para ingeniería
\captionsetup[table]{
	format=hang,
	name=Tabla,
	justification=centering,
	labelsep=colon,
	width=.75\linewidth,
	labelfont=small,
	font=small,
}

% DISEÑO DE FIGURAS.
\usepackage{graphicx}
\graphicspath{{img/}} %ruta a la carpeta de imágenes

% Diseño de figuras para ingeniería
\captionsetup[figure]{
	format=hang,
	name=Fig.,
	singlelinecheck=off,
	labelsep=colon,
	labelfont=small,
	font=small		
}

% NOTAS A PIE DE PÁGINA
\usepackage{chngcntr} %para numeración continua de las notas al pie
\counterwithout{footnote}{chapter}

% LISTADOS DE CÓDIGO
% soporte y estilo para listados de código. Más información en https://es.wikibooks.org/wiki/Manual_de_LaTeX/Listados_de_código/Listados_con_listings
\usepackage{listings}

% definimos un estilo de listings
\lstdefinestyle{estilo}{ frame=Ltb,
	framerule=0pt,
	aboveskip=0.5cm,
	framextopmargin=3pt,
	framexbottommargin=3pt,
	framexleftmargin=0.4cm,
	framesep=0pt,
	rulesep=.4pt,
	backgroundcolor=\color{gray97},
	rulesepcolor=\color{black},
	%
	basicstyle=\ttfamily\footnotesize,
	keywordstyle=\bfseries,
	stringstyle=\ttfamily,
	showstringspaces = false,
	commentstyle=\color{gray45},     
	%
	numbers=left,
	numbersep=15pt,
	numberstyle=\tiny,
	numberfirstline = false,
	breaklines=true,
	xleftmargin=\parindent
}

\captionsetup[lstlisting]{font=small, labelsep=period}
% fijamos el estilo a utilizar 
\lstset{style=estilo}
\renewcommand{\lstlistingname}{\uppercase{Código}}

\pgfplotsset{compat=1.17} 
%-------------
%	DOCUMENTO
%-------------

\begin{document}
\pagenumbering{roman} % Se utilizan cifras romanas en la numeración de las páginas previas al cuerpo del trabajo

%----------
%	PORTADA
%----------	
\begin{titlepage}
	\begin{sffamily}
		\color{azulUC3M}
		\begin{center}
			\begin{figure}[H] %incluimos el logotipo de la Universidad
				\makebox[\textwidth][c]{\includegraphics[width=16cm]{Portada_Logo.png}}
			\end{figure}
			\vspace{2.5cm}
			\begin{Large}
				Grado en Ingeniería Informática\\
				2020-2021\\
				\vspace{2cm}
				\textsl{Apuntes}\\
				\bigskip
			\end{Large}
			{\Huge Heurística y Optimización}\\
			\vspace*{0.5cm}
			\rule{10.5cm}{0.1mm}\\
			\vspace*{0.9cm}
			{\LARGE Jorge Rodríguez Fraile\footnote{\href{mailto:100405951@alumnos.uc3m.es}{Universidad: 100405951@alumnos.uc3m.es}  |  \href{mailto:jrf1616@gmail.com}{Personal: jrf1616@gmail.com}}}\\
			\vspace*{1cm}
		\end{center}
		\vfill
		\color{black}
		\includegraphics[width=4.2cm]{img/creativecommons.png}\\
		Esta obra se encuentra sujeta a la licencia Creative Commons\\ \textbf{Reconocimiento - No Comercial - Sin Obra Derivada}
	\end{sffamily}
\end{titlepage}

%----------
%	ÍNDICES
%----------	

%--
% Índice general
%-
\tableofcontents
\thispagestyle{fancy}

%--
% Índice de figuras. Si no se incluyen, comenta las líneas siguientes
%-
\listoffigures
\thispagestyle{fancy}

%----------
%	TRABAJO
%----------	

\pagenumbering{arabic} % numeración con múmeros arábigos para el resto de la publicación	


%----------
%	COMENZAR A ESCRIBIR AQUI
%----------	

\chapter{Información}
\section{Profesores}
\begin{quote}
	Magistral: Carlos Linares López

	Reducido: Francisco Javier García Polo
\end{quote}



\chapter{Tema 1 Programación Lineal}

\section{Programación Lineal}

\subsection{Representación grafica}


Un problema de Programación Lineal está en Forma Canónica si y solo
si:

\begin{enumerate}
	\def\labelenumi{\arabic{enumi}.}
	\item El objetivo es de la forma de maximización.
	\item Si todas las restricciones son desigualdades son del tipo $\leq$.
	\item Si todas las variables de decisión son no negativas.
	      \begin{eqnarray*}
		      \max z = c_1x_1+c_2x_2 + \ldots +c_nx_n \\
		      ba_{11}x_1 +a_{12}x_2 +&\ldots& +a_{1n}x_n \leq b_1 \\
		      a_{21}x_1 +a_{22}x_2 +&\ldots& +a_{2n}x_n \leq b_2 \\
		      &\ldots& \\
		      a_{m1}x_1 +a_{m2}x_2 +&\ldots& +a_{mn}x_n \leq b_m \\
		      \text { donde } x_{i} \geq 0, \forall i=1, \ldots, n
	      \end{eqnarray*}
\end{enumerate}

\begin{itemize}
	\item Nomenclatura:

	      \begin{itemize}
		      \item z representa la función a maximizar.
		      \item Las restricciones son para un sujeto a.
		      \item Cada fila representa una restricción.
		      \item Las b's son términos escalares, racionales que deben ser menores. -
		      \item Las c's son coeficientes, números racionales.
	      \end{itemize}
\end{itemize}

\begin{itemize}
	\item Algebraicamente:

	      \begin{itemize}
		      \item OJO: Todas deben ser del mismo tipo \textless{} o \textgreater{}
		            \begin{eqnarray*}
			            \max z &=& C^{T}x \\
			            Ax\leq b&,& ejem: \begin{pmatrix}  a_{11} & a_{12} \\
				            a_{21} & a_{22}\end{pmatrix} \cdot \begin{pmatrix}  x_1 \\
				            x_2\end{pmatrix} \leq \begin{pmatrix}  b_1 \\
				            b_2\end{pmatrix}\\
			            \text { donde } x_{i} &\geq& 0, \forall i=1, \ldots, n
		            \end{eqnarray*}
		      \item z: función objetivo.
		      \item C: coeficientes de la función objetivo. nx1
		      \item X: variables de decisión. nx1
		      \item A: matriz de coeficientes tecnológicos. mxn
		      \item b: recursos. mx1
	      \end{itemize}
\end{itemize}

\subsubsection{Desarrollo}


Representar todas las restricciones en un plano como rectas,
también x, y\textgreater0.

\begin{itemize}
	\item Después de trazar las rectas dar valor a las variables de
	      decisión y ver que hiperplano es el que cumple cada restricción
	      y el área que encierren todas es la Región Factible.
\end{itemize}

Por el teorema de Dantzig:

\begin{itemize}
	\item La Región Factible es siempre un poliedro convexo.~Por lo tanto,
	      uno de los vértices es la solución óptima.
\end{itemize}

Solo evaluamos los puntos extremos, por lo que hay que hallar las
intersecciones de las rectas si todavía no las conocemos.
\begin{itemize}
	\item Para hallar una intersección:
	      \begin{itemize}
		      \item Se hace un sistema con ambas
		            ecuaciones de recta. Los valores obtenidos son las intersecciones.
		            \begin{eqnarray*}
			            \textit{Este es el método}\left\{\begin{matrix}
				            \left\{\begin{matrix}
					            -2x+y=-8 \\
					            -x+6y=18
				            \end{matrix}\right.
				            ,
				            \begin{pmatrix}
					            -2 & 1 \\
					            -1 & 6
				            \end{pmatrix}
				            \begin{pmatrix}
					            x \\
					            y
				            \end{pmatrix}
				            =
				            \begin{pmatrix}
					            -8 \\
					            18
				            \end{pmatrix}               \\
				            \begin{pmatrix}
					            -2 & 1 \\
					            -1 & 6
				            \end{pmatrix}^{-1}
				            \begin{pmatrix}
					            -2 & 1 \\
					            -1 & 6
				            \end{pmatrix}
				            \begin{pmatrix}
					            x \\
					            y
				            \end{pmatrix}
				            =
				            \begin{pmatrix}
					            -2 & 1 \\
					            -1 & 6
				            \end{pmatrix}^{-1}
				            \begin{pmatrix}
					            -8 \\
					            18
				            \end{pmatrix}               \\
				            \begin{pmatrix}
					            x \\
					            y
				            \end{pmatrix}
				            =

				            \begin{pmatrix}
					            -8 \\
					            18
				            \end{pmatrix}\end{matrix}\right.
		            \end{eqnarray*}
		      \item \(x= A^{-1}b\); A es la matriz de coeficientes de las dos rectas
		            y b los recursos de cada una.
	      \end{itemize}

\end{itemize}


Sustituimos los distintos puntos extremos (x, y, \ldots) en la
función objetivo.


Observamos todos los resultados y el máximo, será aquel de mayor
valor.

\begin{itemize}
	\item La solución óptima es la última vez que la curva de
	      isobeneficio toca la región factible.
	      \begin{figure}[H]
		      \ffigbox[\FBwidth]
		      {\caption{Representación PL}}
		      {\begin{tikzpicture}[scale=.8]
				      \begin{axis}[
						      xlabel={$x$},
						      ylabel={$y$},
						      legend pos=north west,
						      ymajorgrids=true,
						      xmajorgrids=true,
						      grid style=dashed,
						      axis lines=middle,
						      xmin=-1, xmax=7, ymin=-1, ymax=5,
						      axis x line=center,
						      axis y line=center,
					      ]
					      \addplot[thick, domain=-1:10, smooth, color=purple]{-8+2*x};
					      \addplot[thick, domain=-1:10, smooth, color=blue]{(x+18)/6};
					      \addplot[thick, domain=-1:10, smooth, color=orange]{(x-2)/-2};
					      \addplot[thick, domain=-1:10, smooth, color=pink]{(9+2*x)/3};
					      \addplot[thick, domain=-1:10, smooth, color=pink]{(3+2*x)/3};
					      \addplot[thick, domain=-1:10, smooth, color=pink]{(2*x)/3};
					      \addplot[thick, domain=-1:10, smooth, color=pink]{x};
				      \end{axis}
			      \end{tikzpicture}}
	      \end{figure}
\end{itemize}

\subsubsection{Región factible}
Es la intersección de las restricciones en forma de
semiplanos. Son los infinitos puntos que cumplen las restricciones,
cada uno es Solución factible.



\subsubsection{Soluciones}

\paragraph{Compatible Determinado}
Solución única.
\begin{figure}[H]
	\ffigbox[\FBwidth]
	{\caption{Compatible Determinado}}
	{\begin{tikzpicture}[scale=.8]
			\begin{axis}[
					xlabel={$x$},
					ylabel={$y$},
					legend pos=north west,
					ymajorgrids=true,
					xmajorgrids=true,
					grid style=dashed,
					axis lines=middle,
					xmin=-2, xmax=5, ymin=-1, ymax=4,
					axis x line=center,
					axis y line=center,
				]
				\addplot[thick, smooth, color=blue]
				{x*.4+1};
				\addplot[thick, smooth, color=blue]
				{-x+3};
			\end{axis}
		\end{tikzpicture}}
\end{figure}
\paragraph{Compatible Indeterminado}
Soluciones infinitas, están superpuestas.
\begin{itemize}
	\item Cuando la solución óptima no es única.
	\item Se detecta
	      con el método de resolución gráfica si la curva de
	      isobeneficio/isocoste es paralela o idéntica a una de las
	      restricciones cuyos puntos extremos son soluciones óptimas.

	      \begin{figure}[H]
		      \ffigbox[\FBwidth]
		      {\caption{Compatible Determinado}}
		      {\begin{tikzpicture}[scale=.8]
				      \begin{axis}[
						      xlabel={$x$},
						      ylabel={$y$},
						      legend pos=north west,
						      ymajorgrids=true,
						      xmajorgrids=true,
						      grid style=dashed,
						      axis lines=middle,
						      xmin=-2, xmax=5, ymin=-1, ymax=4,
						      axis x line=center,
						      axis y line=center,
					      ]
					      \addplot[thick, smooth, color=blue]
					      {x*.4+1};
				      \end{axis}
			      \end{tikzpicture}}
	      \end{figure}
\end{itemize}
\pagebreak

No acotado, faltan restricciones y hay soluciones infinitas
\begin{figure}[H]
	\ffigbox[\FBwidth]
	{\caption{Compatible Determinado}}
	{\begin{tikzpicture}[scale=.8]
			\begin{axis}[
					xlabel={$x$},
					ylabel={$y$},
					legend pos=north west,
					ymajorgrids=true,
					xmajorgrids=true,
					grid style=dashed,
					axis lines=middle,
					xmin=-2, xmax=5, ymin=-1, ymax=4,
					axis x line=center,
					axis y line=center,
				]
				\addplot[thick, smooth, color=blue]
				{x*1.4+1};
				\addplot[thick, smooth, color=blue]
				{x*1.8-3};
				\addplot[thick, smooth, color=blue]
				{-x+1};
			\end{axis}
		\end{tikzpicture}}
\end{figure}


Incompatible:
\begin{itemize}
	\item Infactible
	      \begin{itemize}
		      \item Si y solo si la región de soluciones
		            factibles es vacía: \(F=\emptyset\), ya sea porque no cortan o
		            porque cortan en zonas negativas.
		            \begin{figure}[H]
			            \ffigbox[\FBwidth]
			            {\caption{Compatible Determinado}}
			            {\begin{tikzpicture}[scale=.8]
					            \begin{axis}[
							            xlabel={$x$},
							            ylabel={$y$},
							            legend pos=north west,
							            ymajorgrids=true,
							            xmajorgrids=true,
							            grid style=dashed,
							            axis lines=middle,
							            xmin=-2, xmax=5, ymin=-1, ymax=4,
							            axis x line=center,
							            axis y line=center,
						            ]
						            \addplot[thick, smooth, color=blue]
						            {x*1.4+1};
						            \addplot[thick, smooth, color=blue]
						            {x*1.4-3};

					            \end{axis}
				            \end{tikzpicture}}
		            \end{figure}
	      \end{itemize}

\end{itemize}



El método de resolución gráfica solo es posible para como mucho 3
variables de decisión.
\pagebreak
\subsection{Transformaciones}


Pasar inecuaciones de un tipo a otro (para Forma Canónica)

$$\begin{aligned}  &\sum\limits_{j=1}^{n} a_{i j} x_{j} \geqslant b_{i} \triangleq -\sum\limits_{j=1}^{n} a_{i j} x_{j}\leq -b_{i}  \end{aligned}$$

Transformar maximización en minimización y viceversa:


$$\min z = C^{T}x \triangleq \max z =- C^{T}x$$

Quitar inecuación (para Forma Estándar)

$$\begin{aligned}  &\sum\limits_{j=1}^{n} a_{i j} x_{j} \leqslant b_{i} \triangleq \sum\limits_{j=1}^{n} a_{i j} x_{j}+s_{i}=b_{i}\\  &\sum\limits_{j=1}^{n} a_{i j} x_{j} \geqslant b_{i} \triangleq \sum\limits_{j=1}^{n} a_{i j} x_{j}-s_{i}=b_{i}  \end{aligned}s_i: \textit{variables de holgura. Son variables de decisión cuando}$$

operamos.


Poner inecuación a partir de igualdad:

\(\sum\limits_{j=1}^{n} a_{i j} x_{j} = b_{i}\) es
\(\sum\limits_{j=1}^{n} a_{i j} x_{j} \leqslant b_{i}\) y
\(\sum\limits_{j=1}^{n} a_{i j} x_{j} \geq b_{i}= -\sum\limits_{j=1}^{n} a_{i j} x_{j} \leqslant b_{i}\)

Si una variable de decisión \(x_i\) no está restringida se pone
entonces como la diferencia de dos variables no negativas
restringidas: \(x_i=x_i'-x_i''; x_i',x_i'' \geq 0\)


\subsection{Método Simplex}

Simplex: Poliedro convexo de n dimensiones.

Una tarea de Programación Lineal está en Forma Estándar ( de
maximización / minimización) si y solo si:


\begin{enumerate}
	\def\labelenumi{\arabic{enumi}.}
	\item La función objetivo es de maximización (minimización, si se dice
	      forma estándar se supone siempre maximización a menos que lo digan
	      explícitamente).
	\item Todas las restricciones son =.
	\item Todas las variables de decisión son no negativas.
	\item Todos los recursos son no negativos.
\end{enumerate}
\pagebreak

Algebraicamente:

$$\begin{aligned}  &\max z = C^{T}x \\  &Ax = b \\ &x, y\geqslant 0 \end{aligned}$$
\begin{figure}[H]
	\ffigbox[\FBwidth]
	{\caption{Representación Simplex}}
	{

		\tikzset{every picture/.style={line width=0.75pt}} %set default line width to 0.75pt        

	\begin{tikzpicture}[x=0.75pt,y=0.75pt,yscale=-1,xscale=1]
		%uncomment if require: \path (0,310); %set diagram left start at 0, and has height of 310

		%Shape: Rectangle [id:dp8027028035618593] 
		\draw   (200,100) -- (350,100) -- (350,170) -- (200,170) -- cycle ;
		%Straight Lines [id:da6323757019931138] 
		\draw    (300,100) -- (300,170) ;
		%Shape: Rectangle [id:dp12270154819774604] 
		\draw   (380,100) -- (410,100) -- (410,170) -- (380,170) -- cycle ;
		%Shape: Brace [id:dp7864677175701058] 
		\draw   (201.2,181) .. controls (201.24,185.67) and (203.59,187.98) .. (208.26,187.94) -- (265.96,187.48) .. controls (272.63,187.43) and (275.98,189.73) .. (276.01,194.4) .. controls (275.98,189.73) and (279.29,187.37) .. (285.96,187.32)(282.96,187.34) -- (343.66,186.86) .. controls (348.33,186.82) and (350.64,184.47) .. (350.6,179.8) ;
		%Shape: Brace [id:dp2214119405780277] 
		\draw   (190.2,100.2) .. controls (185.53,100.2) and (183.2,102.53) .. (183.2,107.2) -- (183.2,125) .. controls (183.2,131.67) and (180.87,135) .. (176.2,135) .. controls (180.87,135) and (183.2,138.33) .. (183.2,145)(183.2,142) -- (183.2,162.8) .. controls (183.2,167.47) and (185.53,169.8) .. (190.2,169.8) ;
		%Shape: Brace [id:dp3914378412087798] 
		\draw   (351.2,91.4) .. controls (351.17,86.73) and (348.82,84.42) .. (344.15,84.46) -- (335.55,84.52) .. controls (328.88,84.57) and (325.53,82.27) .. (325.49,77.6) .. controls (325.53,82.27) and (322.22,84.63) .. (315.55,84.68)(318.55,84.66) -- (306.95,84.75) .. controls (302.28,84.78) and (299.97,87.13) .. (300,91.8) ;

		% Text Node
		\draw (358,126.5) node [anchor=north west][inner sep=0.75pt]   [align=left] {=};
		% Text Node
		\draw (389.5,126.5) node [anchor=north west][inner sep=0.75pt]   [align=left] {b};
		% Text Node
		\draw (201,72) node [anchor=north west][inner sep=0.75pt]   [align=left] {$\displaystyle A_{mxn}$};
		% Text Node
		\draw (221,142) node [anchor=north west][inner sep=0.75pt]   [align=left] {$\displaystyle B_{mxm}$};
		% Text Node
		\draw (238.4,201) node [anchor=north west][inner sep=0.75pt]   [align=left] {n variables};
		% Text Node
		\draw (93,116) node [anchor=north west][inner sep=0.75pt]   [align=left] {\begin{minipage}[lt]{59.98pt}\setlength\topsep{0pt}
				\begin{center}
					m\\restricciones
				\end{center}

			\end{minipage}};
		% Text Node
		\draw (311.6,52) node [anchor=north west][inner sep=0.75pt]   [align=left] {n-m};
		% Text Node
		\draw (201,102) node [anchor=north west][inner sep=0.75pt]   [align=left] {\begin{minipage}[lt]{32.2pt}\setlength\topsep{0pt}
				\begin{center}
					var.\\basica
				\end{center}

			\end{minipage}};
		% Text Node
		\draw (301,102) node [anchor=north west][inner sep=0.75pt]   [align=left] {\begin{minipage}[lt]{33.34pt}\setlength\topsep{0pt}
				\begin{center}
					var. no\\basica
				\end{center}

			\end{minipage}};


	\end{tikzpicture}
	}
\end{figure}

Teorema George Dantzig: Dado una tarea de Programación lineal en forma
estándar, el valor óptimo si lo hubiera, se alcanza en un punto
extremo de la región factible.



Términos:
\begin{itemize}
	\item Variable básica: Que no toma valor 0.
	\item Variable no básicas: Son las que toman valor 0.
	\item \(a_i\): Vector columna, está formado por los coeficientes de
	      \(x_i\) de las restricciones.
	\item \(c_i\): Coeficientes de la \(x_i\) de la función objetivo.
	\item \(i\): variables básicas.
	\item \(j\): variables no básicas.
\end{itemize}

\pagebreak
Tipos de soluciones:

\begin{itemize}
	\item Definición 4

	      Un vector $\mathbf{x}_{\mathbf{B}}=\left\{x_{B 1}, x_{B 2}, \ldots, x_{B m}\right\}$ se denomina solución básica si satisface $\mathbf{B x}_{\mathbf{B}}=\mathbf{b}$, donde $\mathbf{B}_{m \times m}$ es una submatriz de $\mathbf{A}_{m \times n}$ y todas las $(n-m)$ variables de decisión que no están en la base son nulas.

	      Si además es factible, se denomina solución básica factible. Se dice que la base $\mathbf{B}_{m \times m}$, formada por las columnas $a_{i}$ de $\mathbf{A}$ asociadas con las variables básicas $x_{B i}$ es una base factible sí $\mathbf{B}^{-1} \mathbf{b} \geq 0$.
	      \begin{itemize}
		      \item Por lo tanto, $\mathbf{B}_{m \times m}$ debe ser no singular
		      \item Podría haber hasta $\left(\begin{array}{l}n \\ m\end{array}\right)$ soluciones básicas factibles
		      \item Típicamente se denotará $B=\left\{x_{B 1}, x_{B 2}, \ldots, x_{B m}\right\}$
	      \end{itemize}
	\item Un vector \(x\) que satisface \(Ax=b\) se llama Solución.
	\item Un vector \(x_B\) que satisface \(Bx_B=b\) se llama Solución Básica.
	\item Un vector \(x_B \geq 0\) que satisface \(Bx_B=b\) se llama Solución Básica Factible.
\end{itemize}


Cálculo de la función objetivo: \(z=C_B^Tx_B\), solo se consideran
las variables básicas.

Una solución factible óptima, \(x^*\), si y solo si:
\(c^Tx^* \geq c^Tx; \forall \in xF\)

Modelo iterativo:

\begin{enumerate}
	\def\labelenumi{\arabic{enumi}.}
	\item Calcular una solución básica factible inicial.
	\item Si existe un punto extremo adyacente que mejore z, transitamos a
	      él.
	\item En otro caso, detenerse.
\end{enumerate}
\pagebreak
Proceso normal:

\begin{itemize}
	\item Cálculo de las variables básicas:

	      \begin{enumerate}
		      \def\labelenumi{\arabic{enumi}.}
		      \item Seleccionar una base mxm (m: número de restricciones), matriz
		            cuadrada B, que tenga inversa y no haga los recursos negativos.

		            \begin{itemize}
			            \item Tratamos de usar la matriz identidad.
			            \item Variables Artificiales: Para empezar con base matriz
			                  identidad.

			                  \(t_i\) se denomina Variable artificial que se añade a la
			                  función objetivo con coeficiente -M, para que no salga en la
			                  solución.

			                  Añadimos a una de las restricciones que tenga una variable
			                  de holgura que no nos interesa para hacer la matriz básica.
			            \item \(B = \{ x_i \}\), se recomienda escribir en orden de índice.
		            \end{itemize}
		      \item Calculamos el valor de las variables básicas de la solución
		            básica factible, \(x_B=B^{-1}b\).
		      \item Hallamos el valor de la función objetivo para esta solución
		            básica factible, \(z_B=C_B^Tx_B\)
	      \end{enumerate}
	\item Selección de la variable de entrada: Buscamos un punto extremo
	      adyacente que mejore la z, evaluamos las variables no básicas.

	      \begin{enumerate}
		      \def\labelenumi{\arabic{enumi}.}
		      \item Calculamos los costes reducidos para las variables no básicas,
		            \(z_j-c_j\)

		            $$\begin{matrix}
				            z_j=C_B^Ty_j  \\
				            y_j=B^{-1}a_j
			            \end{matrix}$$
		      \item La variable de entrada será:

		            \begin{itemize}
			            \item En max. se coge el más negativo y el proceso terminará cuando
			                  todos son positivos.
			            \item En min. se coge el más positivo y el proceso termina cuando
			                  todos los costes sean negativos.
		            \end{itemize}
	      \end{enumerate}
	\item Regla de salida: Para que tenga dimensión m tenemos que sacar una
	      variable básica de la base.

	      \begin{itemize}
		      \item La variable que salga de la base, será:
		            \(min\{ \frac {x_i} {y_{i'}} \}\) , con \(y_{i'} \geq 0\).
	      \end{itemize}
\end{itemize}
\pagebreak

Casos de las distintas soluciones:
\begin{enumerate}
	\item Dada la solución factible
	      \(x_B=B^{-1}b\) y \((z_j-c_j) > 0\).
	      \begin{itemize}
		      \item Solución Óptima Única.
	      \end{itemize}
	\item Dada la solución factible \(x_B=B^{-1}b\) y \((z_j-c_j) > 0\) para todas
	      las variables no básicas salvo una o más para las que
	      \((z_j-c_j) = 0\).
	      \begin{itemize}
		      \item Soluciones Óptimas Infinitas.
	      \end{itemize}
	\item Dada la solución factible \(x_B=B^{-1}b\) y \((z_j-c_j) < 0\) pero algún \(x_j\) no
	      básica con \(y_j \leq 0\).
	      \begin{itemize}
		      \item Básicamente que al intentar sacar una
		            variable todas las componentes no sean válidas, ya sea porque son
		            divisiones entre 0 o entre negativos y dan negativos, ambas no
		            válidas.
		      \item No Acotado.
	      \end{itemize}
	\item Dada la solución factible \(x_B=B^{-1}b\) y
	      \((z_j-c_j) \geq 0\), pero alguna variable artificial toma valor
	      positivo, que tenga valor en la solución. Variable artificial no es
	      lo mismo que variable de holgura, las artificiales son para comenzar
	      por la identidad y aportan negativamente a la función objetivo.
	      \begin{itemize}
		      \item Infactible, región factible vacía.
	      \end{itemize}
\end{enumerate}

\subsection{Dualidad}

Una tarea de Programación Lineal está en Forma Simétrica o Forma
Canónica de maximización (o minimización si se dice explícitamente) si
y solo si:

\begin{enumerate}
	\item El objetivo es de la forma de maximización.
	\item Si todas las restricciones son desigualdades son del tipo $\leq$.
	\item Si todas las variables de decisión son no negativas.
\end{enumerate}

El Problema Dual del Problema Primal (el canónico o simétrico):

\begin{minipage}{.5\linewidth}
	$$\max z = C^{T}x \\
		Ax\leq b \\
		x_{i} \geq 0$$
\end{minipage}
es
\begin{minipage}{.5\linewidth}
	$$\min w = b^{T}x' \\
		A^Tx' \geq c \\
		x' \geq 0$$
\end{minipage}

Si la tarea de Programación Lineal en Forma Simétrica tiene una
solución óptima correspondiente a una base B, entonces:
\(x'^*= c_B^TB^{-1}\)

$$x'^*=c^T_BB^{-1}=\left( \begin{matrix} -7 & 8 & 0 \end{matrix} \right) \left( \begin{matrix} 0 & \frac 1 4 & 0 \\ - \frac 1 2 & \frac 5 8 & 0 \\ - \frac 1 2 & \frac {23} 8 & 1 \end{matrix} \right) = \left( \begin{matrix} -4 & \frac {13} 4 & 0 \end{matrix} \right)$$

\begin{itemize}
	\item c y B, son las del problema resuelto, los que ya conocíamos. No las
	      del problema simétrico.
	\item Teorema: la variable dual \(x_i'^*\) indica la contribución al
	      crecimiento de la función objetivo por unidad del recurso i-ésimo
	      (de la primal, no la dual).
	      \begin{itemize}
		      \item Nos permite saber cuándo aumenta la función objetivo si le sumamos
		            1 al recurso.
	      \end{itemize}

\end{itemize}

\subsection{Interpretación de resultados}

Interpretación de la solución.

\begin{enumerate}
	\item Factible o Infactible. Se justifica con que no haya salido un valor positivo en las variables artificiales en la solución óptima.
	\item Solución única o infinitas. Se justifica con que nos han salido
	      costes reducidos estrictamente positivos en la última iteración.
	\item Función objetivo acotable o no acotable. Se justifica con que el
	      \(y_i\) de \(x_i\) no son negativos o nulos, por lo que se ha
	      podido elegir una variable de salida.
\end{enumerate}

Interpretación de los recursos.

\begin{enumerate}
	\item Interpretación de las variables de holgura.
	      \begin{itemize}
		      \item Si la variable de holgura suma, es que sobran recursos.
		      \item Si resta la variable de holgura es que falta recurso.
	      \end{itemize}
	\item Dualidad: Contribución unitaria de cada recurso al crecimiento de
	      la función objetivo.

	      \begin{itemize}
		      \item Indicar de forma individual cuáles de las variables contribuyen,
		            cuanto, y cuáles no.
	      \end{itemize}
\end{enumerate}
\pagebreak
\subsection{Modelización}

\subsubsection{Problema de Transporte}
Dados m orígenes y n destinos tal que:
\begin{itemize}
	\item[] $a_i$: capacidad del origen i, $1 \leq i \leq m$
	\item[] $b_i$: demanda del destino j, $1 \leq j \leq m$
	\item[] $c_{ij}$: coste unitario $i \rightarrow j$
\end{itemize}
Se modeliza con:
\begin{flalign*}
	&x_{ij} \lessgtr \textit{el número de ítems de i a j}& \\
	&\min z = \sum\limits^m_{i=1} \sum\limits^n_{j=1} c_{ij}x_{ij}& \\
	&\sum\limits^n_{j=1} x_{ij} \leq a_i \; \; \; \textit{No sobrepasar la capacidad}& \\
	&\sum\limits^m_{i=1} x_{ij} \geq b_j \; \; \; \textit{Al menos superar demanda}& \\
	&x_{ij} \geq 0 \; \; \; \textit{A veces hay que restringir a que sean enteros}&
\end{flalign*}

\subsubsection{Problema de Asignación}
Dado m individuos a los que hay que asignar m tareas con $c_{ij}$ es el coste unitario de coste asignación $i \rightarrow j$

Este ejemplo \#tarea = \#individuos

Se modeliza:
\begin{flalign*}
	&x_{ij}\begin{cases}
		0 & si \; i \nrightarrow j \\
		b & si \; i \rightarrow j
	\end{cases} & \\
	&\min z = \sum\limits^m_{i=1} \sum\limits^m_{j=1} c_{ij}x_{ij}& \\
	&\sum\limits^m_{j=1} x_{ij} = 1 \; \; \; \forall i = 1, ..., m& \\
	&\sum\limits^m_{i=1} x_{ij} = 1 \; \; \; \forall i = 1, ..., m& \\
	&x_{ij} \in \{0, 1\}&
\end{flalign*}

\subsubsection{Condicionales}
\begin{lstlisting}[language=Python]
if a=1
  then b=1
  else b=0
\end{lstlisting}

Tecnica de la M-Grande: Consiste en añadir una variable binaria, y además en usar una constante M arbitrariamente grande. Acotar la variable y otra restricción que ajuste todo.

\begin{flalign*}
	&a \leq b + My& \\
	&a \geq b + 2y& \\
	&b + y =1& \\
	&y \in \{ 0, 1 \}& \\
\end{flalign*}

Lo primero es acotar la variable, $\leq$ y $\geq$.

Después con una variable binaria y la arbitrariamente grande debemos
obligar a la variable a tomar el valor que nosotros queramos.

Partiendo de la condición, miramos en que parte de la misma podemos
meter la variable binaria y que obligue a que tome 1 o 0, idealmente
ambos. Después con el valor arbitrariamente grande en la misma
condición buscamos obligar a tomar el que nos falte por obligar.



\section{Programación Lineal Entera}

Una tarea de Programación Lineal es de Programación Lineal Entera si
una o más variables de decisión tienen restricciones de integridad
(\(x_j \in N^+_0)\), es NP-hard (Karp, 1972):

La colección de puntos, son enteros, los puntos rojos.

La Programación Lineal Entera es NP-hard (Karp, 1972)

3 tipos:

\begin{itemize}
	\item Programación Entera Pura: \(x_i \in N^+_0; \forall i\) Todas las
	      variables están afectadas por una restricción de integridad.
	      \begin{itemize}
		      \item Pueden no tener solución, si no hay puntos enteros en las
		            subregiones factibles.
	      \end{itemize}
	\item Programación 0-1: \(x_i \in \{ 0, 1\}\)
	\item Programación Entera Mixta: Solo algunas variables de decisión
	      tienen restricción de integridad.
\end{itemize}

\subsection{Ramificación y Acotación en profundidad}

Relajamos el problema, ignorando las restricciones de integridad, y
vamos añadiendo las restricciones.

Ramificamos una de las variables de decisión, creando dos
subregiones factibles.

La solución óptima de las subregiones será peor o igual que la óptima
relajada, \(z_S \leq z_F\).

Dado el problema de Programación Lineal Entera:
\vspace{-0.5cm}

$$\max \space z= z(x)$$
\vspace{-1cm}
$$s.a \space x \in F; x_i \in N^+_0 \space \forall i$$

Método:
\vspace{-0.5cm}

\begin{enumerate}
	\item $B ← -\infty$ o Valor negativo muy alto
	\item Resolver el Problema Relajado.

	      \begin{itemize}
		      \vspace{-0.5cm}
		      \item Si \(x* \in N^+_0\) entonces HALT
		      \item En otro caso, ir al 3.
	      \end{itemize}
	\item Aplicamos alguna regla de ramificación sobre una variable de
	      decisión no entera: \(F_1, F_2\) (las llamaremos S, de subset)
	\item Determinar el valor de \(z_s\)
	\item Son nodos terminales:
	      \vspace{-0.5cm}

	      \begin{itemize}
		      \item \(S= \emptyset\) Infactible esa subregión, hacemos backtracking.
		      \item \(z_s \leq B\) Peor que alguno de los terminales, hacemos
		            backtracking
		      \item \(x* \in N^+_0 , z_s > B\), entonces, \(B←z_s\)
	      \end{itemize}
	\item Si todos los nodos son terminales, HALT, en otro caso, ir a 2.
\end{enumerate}
\pagebreak

Ejemplo:

\begin{flalign*}
	\max z = &4x_1+3x_2+2x_3& \\
	s.a. &5x_1-2x_2+3x_3<=10& \\
	&3x_1+3x_2-2x_3<=7& \\
	&-x_1+2x_2-x_3<=9& \\
\end{flalign*}



\tikzset{every picture/.style={line width=0.75pt}} %set default line width to 0.75pt        

\begin{tikzpicture}[x=0.75pt,y=0.75pt,yscale=-1,xscale=1]
	%uncomment if require: \path (0,701); %set diagram left start at 0, and has height of 701

	%Shape: Rectangle [id:dp9442620448860839] 
	\draw   (400,50) -- (600,50) -- (600,110) -- (400,110) -- cycle ;
	%Straight Lines [id:da15747262423174502] 
	\draw    (500,110) -- (600,160) ;
	%Straight Lines [id:da007768120489592745] 
	\draw    (400,160) -- (500,110) ;
	%Shape: Rectangle [id:dp9647830398971795] 
	\draw   (320,160) -- (470,160) -- (470,210) -- (320,210) -- cycle ;
	%Straight Lines [id:da29173162034937805] 
	\draw    (400,210) -- (500,260) ;
	%Straight Lines [id:da7620557535793135] 
	\draw    (300,260) -- (400,210) ;
	%Shape: Rectangle [id:dp89689152507722] 
	\draw   (420,260) -- (570,260) -- (570,310) -- (420,310) -- cycle ;
	%Shape: Rectangle [id:dp12305441175128462] 
	\draw   (220,260) -- (370,260) -- (370,310) -- (220,310) -- cycle ;
	%Straight Lines [id:da8389776446418404] 
	\draw    (300,310) -- (400,360) ;
	%Straight Lines [id:da6618058365785755] 
	\draw    (200,360) -- (300,310) ;
	%Shape: Rectangle [id:dp47081678621295353] 
	\draw   (120,360) -- (270,360) -- (270,410) -- (120,410) -- cycle ;
	%Straight Lines [id:da7595646031071146] 
	\draw    (200,410) -- (300,460) ;
	%Straight Lines [id:da23356472952864804] 
	\draw    (100,460) -- (200,410) ;
	%Shape: Rectangle [id:dp0650592285503444] 
	\draw   (20,460) -- (170,460) -- (170,510) -- (20,510) -- cycle ;

	% Text Node
	\draw (449.5,50.25) node [anchor=north west][inner sep=0.75pt]   [align=left] {{\scriptsize Sol. relajado Simplex}};
	% Text Node
	\draw (410,62.98) node [anchor=north west][inner sep=0.75pt]    {$ \begin{array}{l}
				{\textstyle x\ \in F\ \ \ x^{*} =( 0,\ 8'2,\ 8'8)} \\
				{\textstyle \ \ \ \ \ \ \ \ \ \ \ \ \ z^{*} =42'2}
			\end{array}$};
	% Text Node
	\draw (323.5,165.9) node [anchor=north west][inner sep=0.75pt]    {$ \begin{array}{l}
				{\textstyle x^{*} =( 0'05,\ 8,\ 8'58)} \\
				{\textstyle z^{*} =41'36}
			\end{array}$};
	% Text Node
	\draw (423.5,265.9) node [anchor=north west][inner sep=0.75pt]    {$ \begin{array}{l}
				{\textstyle x^{*} =( 1,\ 4'4,\ 4'6)} \\
				{\textstyle \ \ z^{*} =26'4}
			\end{array}$};
	% Text Node
	\draw (223.5,265.9) node [anchor=north west][inner sep=0.75pt]    {$ \begin{array}{l}
				{\textstyle x^{*} =( 0'8,\ 8,\ 8'67)} \\
				{\textstyle \ z^{*} =41'33}
			\end{array}$};
	% Text Node
	\draw (123.5,365.9) node [anchor=north west][inner sep=0.75pt]    {$ \begin{array}{l}
				{\textstyle x^{*} =( 0,\ 7'68,\ 8)} \\
				{\textstyle z^{*} =39}
			\end{array}$};
	% Text Node
	\draw (567,161) node [anchor=north west][inner sep=0.75pt]   [align=left] {Infactible};
	% Text Node
	\draw (447.5,311) node [anchor=north west][inner sep=0.75pt]   [align=left] {Menor que 37};
	% Text Node
	\draw (23.5,465.9) node [anchor=north west][inner sep=0.75pt]    {$ \begin{array}{l}
				{\textstyle x^{*} =( 0,\ 7,\ 8)} \\
				{\textstyle \ z^{*} =37}
			\end{array}$};
	% Text Node
	\draw (367,362) node [anchor=north west][inner sep=0.75pt]   [align=left] {Infactible};
	% Text Node
	\draw (221,462) node [anchor=north west][inner sep=0.75pt]   [align=left] {\begin{minipage}[lt]{104.21pt}\setlength\topsep{0pt}
			\begin{center}
				Infactible,\\hacemos backtracking
			\end{center}

		\end{minipage}};
	% Text Node
	\draw (381,122.4) node [anchor=north west][inner sep=0.75pt]    {$x_{2} \leq 8$};
	% Text Node
	\draw (571,122.4) node [anchor=north west][inner sep=0.75pt]    {$x_{2} \geq 9\ $};
	% Text Node
	\draw (574.5,180.4) node [anchor=north west][inner sep=0.75pt]    {$x\in F_{2}$};
	% Text Node
	\draw (421,190.4) node [anchor=north west][inner sep=0.75pt]    {$x\in F_{1}$};
	% Text Node
	\draw (471,222.4) node [anchor=north west][inner sep=0.75pt]    {$x_{1} \geq 1\ $};
	% Text Node
	\draw (287,222.4) node [anchor=north west][inner sep=0.75pt]    {$x_{1} \leq 0\ $};
	% Text Node
	\draw (316,290.4) node [anchor=north west][inner sep=0.75pt]    {$x\in F_{11}$};
	% Text Node
	\draw (516,290.4) node [anchor=north west][inner sep=0.75pt]    {$x\in F_{12}$};
	% Text Node
	\draw (191,320.4) node [anchor=north west][inner sep=0.75pt]    {$x_{3} \leq 8\ $};
	% Text Node
	\draw (357,322.4) node [anchor=north west][inner sep=0.75pt]    {$x_{3} \geq 9\ $};
	% Text Node
	\draw (211,390.4) node [anchor=north west][inner sep=0.75pt]    {$x\in F_{111}$};
	% Text Node
	\draw (370,380.4) node [anchor=north west][inner sep=0.75pt]    {$x\in F_{112}$};
	% Text Node
	\draw (81,430.4) node [anchor=north west][inner sep=0.75pt]    {$x_{2} \leq 7\ $};
	% Text Node
	\draw (257,422.4) node [anchor=north west][inner sep=0.75pt]    {$x_{2} \geq 8\ $};


\end{tikzpicture}

\chapter{Tema 2 Programación Dinámica}


Principio de Optimalidad: Una política óptima debe verificar que
independientemente del estado inicial y decisiones iniciales, el
resto de decisiones deben ser óptimas con respecto al estado que
resulta de la primera decisión. (Bellman, 1957)

\begin{itemize}
	\item Ejemplo que lo cumple: La ecuación de Bellman.
	      \(V(x)= max_{a \in A} \{ f(x,a)+V(T(x,a))\}\)

	      \begin{itemize}
		      \item El coste de x, será el máximo evaluando todas las transiciones
		            de: el coste de la acción a sobre más el coste del estado al que
		            transiciona con la acción a sobre x.
	      \end{itemize}
	\item Ejemplo que no lo cumple: Longest Path Problem - LPP.
\end{itemize}

Los problemas que verifican el principio de optimalidad también
verifican la propiedad de Subestructura Óptima (Coman, 2009).

\begin{itemize}
	\item Si es el camino óptimo entre s y t, no habrá otro con menor coste.
	      Además, para los nodos que se recorren continuar ese camino
	      también será el camino óptimo hasta t.
\end{itemize}

La programación dinámica sugiere:

\begin{enumerate}
	\item Descomponer el problema en subproblemas y caracterizar su
	      estructura.
	\item Definir una expresión de recurrencia para calcular la solución
	      óptima de los problemas.
	\item Derivar la solución óptima de cada problema. Calcular los valores
	      de las soluciones óptimas de los subproblemas.
	\item Calcular la solución óptima del problema global.
\end{enumerate}
\pagebreak

\section{Single-Source Shortest-Path}


Dado un grafo \(G=(V,E)\) y una función de costes \(c: e → \mathbb{Z}\),
calcular el coste del camino óptimo desde \(s \in V\) hasta todos los
demás vértices.

\subsection{Bellman-Ford-Moore (1958, 56, 57)}

El camino óptimo entre dos puntos será: \(\min (d[e.v], d[e.u]+ e.c)\)

\begin{itemize}
	\item e.u = origen
	\item e.v = destino
	\item e.c = coste del arco entre u y v
	\item Lo que quiere decir, que escogemos el menor entre, el coste de ir al
	      destino ya calculado o el coste de ir a otro punto y coger desde
	      este un arco al destino.
\end{itemize}
\begin{lstlisting}[language=Python]
def bellmanFordMoore (V,E,s):
	for v in V:
		if V==s: d[v]=0
		else: d[v]= +infinito
	for i in range(len(V)-1):
		for e in E:
			d[e.v]= min(d[e.v], d[e.u]+e.c)
	for e in E:
		if d[e.u]+e.c<d[e.v]
			raise(...) #Ciclos negativos
\end{lstlisting}

Complejidad:

\begin{itemize}
	\item Spurse: \(O(|V|^2)\)
	\item Dense: \(O(|V|^3)\)
\end{itemize}


\section{All Pairs Shortest-Path}

Dado un grafo \(G=(V,E)\) y una función de costes \(c: e → \mathbb{Z}\),
el coste del camino más corto entre cada par de vértices.

\begin{itemize}
	\item Aplicando Bellman-Ford-Moore, la complejidad es \(O(|V|^3)-O(|V|^4)\)
\end{itemize}

\subsection{Floyd-Warshall}

En cada iteración vamos añadiendo un vértice más que podemos visitar.

\begin{itemize}
	\item $D_{ij}^{(k} = \min \{ D_{ij}^{(k-1}, D_{ik}^{(k-1}+D_{kj}^{(k-1} \}$ k son los vértices auxiliares que vamos añadiendo.
	\item $D_{ii}^{(0} =0$
	\item $D_{ij}^{(0}= c(e(i,j))$ si hay arco entre i y j, si no es $D_{ij}^{(0}= + \infty$
\end{itemize}


Algoritmo:
\begin{itemize}
	\item Partimos de una matriz con los costes a los vértices
	\item Ahora en cada
	      iteración vamos añadiendo un vértice, k, que podemos usar como
	      intermediario.
	      \begin{itemize}
		      \item En cada una de esas iteraciones evaluamos todos los
		            vértices con todos los vértices, ir de i a j.

		            Para cada par evaluado
		            nos quedamos con el menor entre, el coste de ir de uno al otro previo,
		            dij, o el coste de ir del origen al vértice que hemos añadido más el
		            cose de ir desde el vértice añadido al destino, dik+dkj.
	      \end{itemize}

\end{itemize}

\begin{lstlisting}[language=Python]
def floydwarshall:
	for v in V:
		d[v][v]=0
	for e in E:
		d[e.u][e.v]=e.c
	for k in V:
		for i in V: 
			for j in V:
				d[i][j]= min(d[i][j],d[i][k]+d[k][j])
\end{lstlisting}

\begin{itemize}
	\item Complejidad: \(O(|V|^3)\)
\end{itemize}


\chapter{Tema 3 Satisfabilidad}

\begin{enumerate}
	\item Una fórmula está en Forma Normal Conjuntiva si y solo si:
	      \(F \equiv \bigwedge_{i=1}^{n} \zeta_i= \bigwedge_{i=1}^{n} (\bigvee_{j=1}^{\left | \zeta_i \right |} \ell_i)\)
	      \vspace{-0.5cm}
	      \begin{itemize}
		      \item Ejemplo:
		            \((x_1 \vee \overline{x_2})\wedge (\overline{x_1} \vee \overline{x_2} \vee x_3)\)

		            \begin{itemize}
			            \item 3 variables: \(x_1, x_2, x_3\)
			            \item 4 literales: \(x_1,\overline{x_1}, \overline{x_2}, x_3\)
			            \item 2 cláusulas:
			                  \((x_1 \vee \overline{x_2}),(\overline{x_1} \vee \overline{x_2} \vee x_3)\)
		            \end{itemize}
	      \end{itemize}
	\item Las variables x pueden estar afirmadas \((x)\) o negadas
	      \((\overline{x})\) , y se denomina literal a la asociación de una
	      variable y su signo.
	\item Un literal \(\ell\) es puro si y solo si \(\overline{\ell}\) no
	      aparece en \(F\) (\(\overline{\ell} \notin F\))

	      \begin{itemize}
		      \item Ejemplo: \(x_1\) y \(\overline{x_1}\) no son literales puros
	      \end{itemize}
	\item Un modelo M consiste en una asignación de las constantes
	      proposicionales (\(\perp falso,\top verdadero\) ) a todos o algunas
	      de las variables de \(F\) que le satisfagan, \(M \models F\)
\end{enumerate}

\begin{itemize}
	\item El problema de SAT (Satisfacibilidad) consiste en encontrar al menos
	      un modelo \(M \models F\), o probar que no existe ninguno. (K-SAT es
	      NP-complete \(K \geq 3\))
\end{itemize}
\vspace{-0.5cm}
\section{Método de Resolución}
\vspace{-0.5cm}
$$Res(F,x)\left\{\begin{matrix}
		F              & si & x\notin F                          \\
		F-x            & si & x \text{ es un literal puro en } F \\
		(c_1 \vee c_2) & si & \begin{matrix}  (x \vee c_1) \in F            \\
			(\overline{x} \vee c_1) \in F\end{matrix}\end{matrix}\right.$$

Tautología: Toma el valor cierto siempre.
\vspace{-0.5cm}


$$F \equiv (x_1 \vee \overline{x_2})\wedge (\overline{x_1} \vee x_2)$$
$$Res(F,x_1)= \emptyset \space SAT$$

Contradicción: Nunca será cierta.
\vspace{-0.5cm}

$$F \equiv (x_1)\wedge (\overline{x_1})$$
$$Res(F,x_1)= \emptyset \vee \emptyset = \{ \emptyset \} \space UNSAT$$


Si resulta la cláusula vacía, \(\{ \emptyset \}\), UNSAT


Si resulta la conjunto vacía, \(\emptyset\), SAT

El modelo de resolución reduce el número de variables, pero no
necesariamente el número de cláusulas será \(\frac {n^2} {4}\)


El método de resolución no genera modelos, aunque preserva la
identidad lógica.



Ejemplo UNSAT
\begin{multicols}{2}

	$C_{1}:(p \vee \bar{q} \vee \bar{r})$

	$C_{2}:(\bar{p} \vee s)$

	$C_{3}:(\bar{s} \vee q)$

	$C_{4}: \bar{q}$ igual que $(\bar{q})$

	$C_{5}:(p \vee \bar{r})$

	$C_{6}:(p \vee q)$

	$C_{7}: (q \vee s)$

	$C_{8}:(q \vee q)=(q)$

	\columnbreak

	\begin{align*}
		\textit{Paso 0: } & G_0 = \{C_i\}_{i=1}^6                                                 \\
		                  & Res(G_0, \bar{r}) = \{ C_2, C_3, C_4, C_6 \}                          \\
		                  & G_0 \setminus Res(G_0, \bar{r}) = \{ C_1, C_5 \}                      \\
		\textit{Paso 1: } & G_1 = \{C_2, C_3, C_4, C_6\}                                          \\
		                  & Res(G_1, p) = \{ C_3, C_4, C_7 \}                                     \\
		                  & G_1 \setminus Res(G_1, p) = \{ C_2, C_6 \}                            \\
		\textit{Paso 2: } & G_2 = \{C_3, C_4, C_7\}                                               \\
		                  & Res(G_2, s) = \{ C_4, C_8 \}                                          \\
		                  & G_2 \setminus Res(G_2, s) = \{ C_3, C_7 \}                            \\
		\textit{Paso 3: } & G_3 = \{C_4, C_8\}                                                    \\
		                  & Res(G_3, q) = \{ C9: \emptyset \} = \{ \emptyset \} \; \textit{UNSAT} \\
		                  & \textit{Insatisfacible, terminado}
	\end{align*}

\end{multicols}

\pagebreak
\section{Algoritmos}
\subsection{Davis-Putnam (DP)}

\begin{enumerate}
	\item Selección de un literal \(\ell \in F\) (empezando por los puros)
	\item Aplicar \(Res(F,\ell)\) y anotando la variable usada y las cláusulas
	      involucradas.

	      \begin{itemize}
		      \item Si creamos cláusulas que son Tautologías no las ponemos, se
		            satisfacen directamente.
	      \end{itemize}
	\item Si resulta $\{ \emptyset \}$, F es UNSAT. Entonces HALT

	      \begin{itemize}
		      \item Cuando hay que juntar cláusulas y la misma variable, pero cada una
		            con un signo y se unen los vacíos.
	      \end{itemize}
	\item Si resulta \(\emptyset\), F es SAT. Ir a 6.

	      \begin{itemize}
		      \item Cuando queda una sola cláusula con un único literal, por lo que se
		            elimina en la siguiente.
		      \item Cuando al juntar literales no puros queda una tautología.
	      \end{itemize}
	\item En otro caso, ir a 1.
	\item Considerar las variables usables y las cláusulas involucradas en
	      orden inverso, y asignar los valores \(\perp\) y \(\top\) a las
	      variables usadas (y otras si hiciera falta) para satisfacer esas
	      cláusulas.
\end{enumerate}

Del 1-5 es la Fase I, es de progreso.

La 6 es la Fase II, de regresión.
\pagebreak

Ejemplo

\begin{multicols}{2}

	$C_{1}:(p \vee r \vee \bar{s})$

	$C_{2}:(\bar{p} \vee \bar{s})$

	$C_{3}:(\bar{q} \vee \bar{r})$

	$C_{4}:(q \vee s)$

	$C_{5}:(r \vee \bar{s})$

	$C_{6}: (q \vee r)$

	\columnbreak

	\begin{align*}
		\textit{Paso 0: } & G_0 = \{C_i\}_{i=1}^4                                                   \\
		                  & Res(G_0, p) = \{ C_3, C_4, C_5 \}                                       \\
		                  & G_0 \setminus Res(G_0, p) = \{ C_1, C_2 \}                              \\
		\textit{Paso 1: } & G_1 = \{ C_3, C_4, C_5 \}                                               \\
		                  & Res(G_1, s) = \{ C_3, C_6 \}                                            \\
		                  & G_1 \setminus Res(G_1, s) = \{ C_4, C_5 \}                              \\
		\textit{Paso 2: } & G_2 = \{C_3, C_6 \}                                                     \\
		                  & Res(G_2, q) = \{ C_7:(\bar{r} \vee r) \} = \emptyset \; \; \textit{SAT} \\
		                  & G_2 \setminus Res(G_2, q) = \{ C_3, C_6 \}                              \\
	\end{align*}

\end{multicols}

\begin{table}[H]
	\centering
	\begin{tabular}{c|c|c|c}
		\textbf{Paso}           & \textbf{Var} & \textbf{Clausulas} & \textbf{M}                              \\ \hline
		\multicolumn{1}{|c|}{2} & s            & $C_4, C_6$         & \multicolumn{1}{c|}{$q=\bot$, $r=\top$} \\ \hline
		\multicolumn{1}{|c|}{1} & s            & $C_4, C_5$         & \multicolumn{1}{c|}{$s=\top$}           \\ \hline
		\multicolumn{1}{|c|}{0} & p            & $C_1, C_2$         & \multicolumn{1}{c|}{$p=\bot$}           \\ \hline
	\end{tabular}
\end{table}

Buscamos en cada paso ir dando valor a las variables para que cumpla las cláusulas, primero a la var y si necesitamos otra a esa, pero no más

Los valores no son únicos

Si hay una variable que su valor no importa hay que indicar que puede ser $\top$ o $\bot$, \enquote{Se le da cualquier valor}


\subsection{Davis-Putnam-Logemann-Loveland (DPLL)}

\begin{enumerate}
	\def\labelenumi{\arabic{enumi}.}
	\item Resolución unitaria: Resolución en la que una, al menos de las
	      cláusulas padre es unitaria.
	\item Sea F una CNF y \(\ell \in F\). Si F es SAT entonces
	      \(F \cup \{ \ell \}\) es SAT o \(F \cup \{ \overline{\ell} \}\) es
	      SAT.
\end{enumerate}

\begin{itemize}
	\item Se denomina reducción de una fórmula F por un modelo parcial v a la
	      fórmula resultante \(F_v = Red(F, v\)) en la que se han propagado
	      las asignaciones de v

	      \begin{itemize}
		      \item Si el modelo es completo y resulta el conjunto vacío, entonces la
		            formula F es satisfacible y el modelo v lo valida
	      \end{itemize}
\end{itemize}

Ejemplo
\tikzset{every picture/.style={line width=0.75pt}} %set default line width to 0.75pt        

\begin{tikzpicture}[x=0.75pt,y=0.75pt,yscale=-1,xscale=1]
	%uncomment if require: \path (0,359); %set diagram left start at 0, and has height of 359

	%Straight Lines [id:da8536565270692567] 
	\draw    (300,35) -- (400,85) ;
	%Straight Lines [id:da8143384481700704] 
	\draw    (300,35) -- (200,85) ;
	%Straight Lines [id:da9113860189379104] 
	\draw    (200,119) -- (250,169) ;
	%Straight Lines [id:da15965108508694592] 
	\draw    (200,119) -- (150,169) ;
	%Straight Lines [id:da16767489107231293] 
	\draw    (400,119) -- (450,169) ;
	%Straight Lines [id:da23862558881233875] 
	\draw    (400,119) -- (350,169) ;
	%Straight Lines [id:da38139706249059024] 
	\draw    (150,199) -- (200,249) ;
	%Straight Lines [id:da4040442684585883] 
	\draw    (150,199) -- (100,249) ;
	%Straight Lines [id:da42010124304423035] 
	\draw    (450,199) -- (500,249) ;
	%Straight Lines [id:da6129321632178557] 
	\draw    (450,199) -- (400,249) ;

	% Text Node
	\draw (208,12.4) node [anchor=north west][inner sep=0.75pt]    {$( p\lor \overline{t}) \land (\overline{p} \lor q) \land ( t\lor \overline{q})$};
	% Text Node
	\draw (165,91.4) node [anchor=north west][inner sep=0.75pt]    {$q\land ( t\lor \overline{q})$};
	% Text Node
	\draw (361,91.4) node [anchor=north west][inner sep=0.75pt]    {$\overline{t} \land ( t\lor \overline{q})$};
	% Text Node
	\draw (147,171.4) node [anchor=north west][inner sep=0.75pt]    {$t$};
	% Text Node
	\draw (441,171.4) node [anchor=north west][inner sep=0.75pt]    {$\overline{q}$};
	% Text Node
	\draw (232,171.4) node [anchor=north west][inner sep=0.75pt]  [color={rgb, 255:red, 208; green, 2; blue, 27 }  ,opacity=1 ]  {$\{\emptyset \}$};
	% Text Node
	\draw (331,171.4) node [anchor=north west][inner sep=0.75pt]  [color={rgb, 255:red, 208; green, 2; blue, 27 }  ,opacity=1 ]  {$\{\emptyset \}$};
	% Text Node
	\draw (182,252.4) node [anchor=north west][inner sep=0.75pt]  [color={rgb, 255:red, 208; green, 2; blue, 27 }  ,opacity=1 ]  {$\{\emptyset \}$};
	% Text Node
	\draw (381,251.4) node [anchor=north west][inner sep=0.75pt]  [color={rgb, 255:red, 208; green, 2; blue, 27 }  ,opacity=1 ]  {$\{\emptyset \}$};
	% Text Node
	\draw (89,251.4) node [anchor=north west][inner sep=0.75pt]  [color={rgb, 255:red, 126; green, 211; blue, 33 }  ,opacity=1 ]  {$\emptyset $};
	% Text Node
	\draw (81,281) node [anchor=north west][inner sep=0.75pt]  [color={rgb, 255:red, 126; green, 211; blue, 33 }  ,opacity=1 ] [align=left] {SAT};
	% Text Node
	\draw (489,252.4) node [anchor=north west][inner sep=0.75pt]  [color={rgb, 255:red, 126; green, 211; blue, 33 }  ,opacity=1 ]  {$\emptyset $};
	% Text Node
	\draw (481,282) node [anchor=north west][inner sep=0.75pt]  [color={rgb, 255:red, 126; green, 211; blue, 33 }  ,opacity=1 ] [align=left] {SAT};
	% Text Node
	\draw (204,42.4) node [anchor=north west][inner sep=0.75pt]    {$p=\top $};
	% Text Node
	\draw (361,42.4) node [anchor=north west][inner sep=0.75pt]    {$p=\perp $};
	% Text Node
	\draw (125,132.4) node [anchor=north west][inner sep=0.75pt]    {$q=\top $};
	% Text Node
	\draw (241,132.4) node [anchor=north west][inner sep=0.75pt]    {$q=\perp $};
	% Text Node
	\draw (371,212.4) node [anchor=north west][inner sep=0.75pt]    {$q=\top $};
	% Text Node
	\draw (481,212.4) node [anchor=north west][inner sep=0.75pt]    {$q=\perp $};
	% Text Node
	\draw (71,212.4) node [anchor=north west][inner sep=0.75pt]    {$t=\top $};
	% Text Node
	\draw (331,132.4) node [anchor=north west][inner sep=0.75pt]    {$t=\top $};
	% Text Node
	\draw (181,212.4) node [anchor=north west][inner sep=0.75pt]    {$t=\perp $};
	% Text Node
	\draw (437,132.4) node [anchor=north west][inner sep=0.75pt]    {$t=\perp $};


\end{tikzpicture}

Hay 2 modelos:

$M_1= \{ p= \top, q=\top, t=\top \}$

$M_2= \{ p= \perp, q=\perp, t=\perp \}$

\subsection{CSP - Procedimiento de Satisfacción de Restricciones}

\begin{enumerate}
	\def\labelenumi{\arabic{enumi}.}
	\item Una Red De Restricciones R consiste en un conjunto de variables
	      \(X=\{ X_i \}^n_{i=1}\) definidas sobre un dominio
	      \(D=\{ D_i \}^h_{j=1}\) que contienen los posibles valores de cada
	      variable \(D_i=\{ V_1^{(i}, V_2^{(i}, ..., V_k^{(i} \}\) y un
	      conjunto de restricciones \(C=\{ C_i \}^m_{n=1}\).

	      \begin{itemize}
		      \item \(R = (X, D, C)\)
	      \end{itemize}
\end{enumerate}


\begin{enumerate}
	\def\labelenumi{\arabic{enumi}.}
	\setcounter{enumi}{1}
	\item Unas restricciones \(C_i\) consiste en una relación (bidireccional
	      típicamente) definida sobre un subconjunto de variables
	      \(S \subseteq X\), que denota todas las asignaciones simultáneamente
	      legales.
	\item Una instanciación de un subconjunto de variables \(S \subseteq X\)
	      consiste en una asignación de valores de los dominios de las
	      variables en S que sea consistente con las restricciones.

	      \begin{enumerate}
		      \item \(S \subset X\): instanciación parcial.
		      \item \(S = X\): instanciación total.
	      \end{enumerate}


	      \begin{itemize}
		      \item No siempre es posible extender una instanciación parcial a otra
		            total.
		      \item Objetivo: Dada una red de restricciones \(R(X, D, C)\) encontrar una
		            instanciación total que sea compatible con todas las restricciones en
		            \(C\), si existe alguna, en otro caso, salir con una instanciación
		            vacía.
	      \end{itemize}
	\item Una red de restricciones \(R(C, D, C)\) puede representarse como un
	      grafo donde los vértices representan X, y hay un arco entre los
	      vértices \(x_i\) y \(x_j\) si \(R_{ij} \neq \emptyset\)

\end{enumerate}


\subsection{Arco-Consistencia}

Una variable \(x_i\) es arco-consistente con otra variable \(x_j\) si
y solo si para cada \(a_i \in D_i\), existe otro valor
\(a_j \in D_j\),, \((a_i,a_j) \in R_{ij}\)

AC-REVISE($x_i$, $x_j$):

\hspace{0.6cm} For $\sigma_i \in D_i$:

\hspace{1.2cm} If $\nexists a_j \in D_j ,, (\sigma_i, \sigma_j) \in R_{ij}$:

\hspace{1.8cm} $D_i = D_i \setminus \{ \sigma_i \}$

Arco-consistencia es Direccional.

AC(R):

\hspace{0.6cm} REPEAT:

\hspace{1.2cm} For ($x_i$, $x_j$),, $R_{ij} \neq \emptyset$

\hspace{1.8cm} AC-REVISE($x_i$, $x_j$)

\hspace{1.8cm} AC-REVISE($x_j$, $x_i$)

\hspace{0.6cm} UNTIL no domain is changed

La arco-consistencia NO sirve para verificar la consistencia global.
Ya que los arcos lo verifican para ese par solamente.



\subsection{Camino-Consistencia}


La variable \(x_i\) es camino-consistente con \(x_j\) con respecto
de \(x_k\) sí y solo si para cada \((a_i,a_j) \in R_{ij}\) y
\(a_k \in D_k\),, \((a_i,a_k) \in R_{ik}\) y
\((a_j,a_k) \in R_{jk}\).

\begin{itemize}
	\item La camino-consistencia NO es direccional.
	\item La camino-consistencia tampoco sirve para verificar la
	      consistencia global. (por ejemplo, si aumenta a 4)

\end{itemize}
\pagebreak
\subsection{Como lo calculamos realmente.}


Se hace mediante un árbol de búsqueda, como todos los métodos que
hemos visto hasta ahora.


Se escoge una variable y se crean tantas ramas como valores tenga el
dominio de la misma, pero cuando vamos avanzamos por una rama
calculamos la arco-consistencia con todas las variables anteriores y
así reducimos el número de ramas que hay que considerar. Además, hay
que calcular la camino consistencia de todas las variables que
llevamos y la que estamos contemplando en conjunto.

Si llegamos a vacío en algún nodo, tenemos un fallo, y hacemos
backtracking, si aun así todos salen vacíos no tenemos solución o
por el contrario si encontramos una no vacía tenemos una
sustanciación global, que cumpla todas las restricciones.



\chapter{Tema 4 Búsqueda}

\section{Espacio De Búsqueda}

\begin{enumerate}
	\item Estados: (estructuras de datos) Contienen la información de tipo
	      estático.
	\item Operadores: (funciones) Dado un estado nos devuelve los que son
	      inmediatamente accesibles.

	      \begin{itemize}
		      \item if <precond> then <effects>
	      \end{itemize}
	\item Un estado inicial, s (start o source).
	\item Uno o más estados finales, t (target).

	      \begin{itemize}
		      \item En satisfacibilidad conocemos las propiedades del estado final, pero
		            no lo conocemos explícitamente. Lo que buscamos es saber cuál es.
		      \item En optimización se da explícitamente la meta. Lo que buscamos es
		            saber cómo llegar a él.
		      \item Los grafos de búsqueda se recorren con Árboles De Búsqueda.
		      \item Si sabemos mucho de algoritmos de búsqueda, podríamos resolver
		            cualquier tarea de optimización y decibilidad, incluso si es
		            exponencialmente difícil.
	      \end{itemize}
	\item Factor de ramificación (b): número medio de sucesores de cada nodo.
	\item Profundidad: número mínimo de niveles hasta alguna solución.
\end{enumerate}

\section{Algoritmos de Fuerza Bruta (Búsqueda no informada)}
\subsection{Costes unitarios}

\begin{enumerate}
	\item Algoritmo general: (Paso 1. y 3. son operaciones atómicas, pero el
	      2. no)

	      \begin{enumerate}
		      \item Generar el estado inicial (s).
		      \item Expandir el primer nodo de la lista abierta.
		      \item Si alguno de los sucesores es un nodo final, entonces halt

		            \begin{itemize}
			            \item En otro caso, ir a 2.
		            \end{itemize}
	      \end{enumerate}
	\item Generar: Es el proceso de creación de un estado en memoria.

	      Expandir: Es el proceso de generar todos los sucesores de un nodo.
	\item Completitud: Si un algoritmo garantiza que encontrará una solución.

	      Admisibilidad: Si un algoritmo garantiza que encontrará una solución
	      óptima. No hay una solución más óptima que otra, depende de lo que
	      se evalúe.
	\item En general, se asume que no hay preferencia en la aplicación de
	      operaciones (costes unitarios, si no son iguales son costes
	      arbitrarios).
\end{enumerate}
\subsection{Primero en Amplitud}

\enquote{Nunca expande un nodo a profundidad d si no ha expandido todos los
nodos a profundidad (d-1)}

\begin{figure}[H]
	\ffigbox[\FBwidth]
	{\caption{Primero en Amplitud}}
	{

		\tikzset{every picture/.style={line width=0.75pt}} %set default line width to 0.75pt        

	\begin{tikzpicture}[x=0.75pt,y=0.75pt,yscale=-1,xscale=1]
		%uncomment if require: \path (0,300); %set diagram left start at 0, and has height of 300

		%Shape: Ellipse [id:dp18669847718653876] 
		\draw   (285.66,41.38) .. controls (285.66,34.26) and (291.09,28.5) .. (297.78,28.5) .. controls (304.48,28.5) and (309.91,34.26) .. (309.91,41.38) .. controls (309.91,48.49) and (304.48,54.25) .. (297.78,54.25) .. controls (291.09,54.25) and (285.66,48.49) .. (285.66,41.38) -- cycle ;

		%Shape: Ellipse [id:dp5065383528404799] 
		\draw   (217.59,96.38) .. controls (217.59,89.26) and (223.02,83.5) .. (229.72,83.5) .. controls (236.41,83.5) and (241.84,89.26) .. (241.84,96.38) .. controls (241.84,103.49) and (236.41,109.25) .. (229.72,109.25) .. controls (223.02,109.25) and (217.59,103.49) .. (217.59,96.38) -- cycle ;

		%Shape: Ellipse [id:dp5426834345553206] 
		\draw   (353.73,97.13) .. controls (353.73,90.01) and (359.16,84.25) .. (365.85,84.25) .. controls (372.55,84.25) and (377.98,90.01) .. (377.98,97.13) .. controls (377.98,104.24) and (372.55,110) .. (365.85,110) .. controls (359.16,110) and (353.73,104.24) .. (353.73,97.13) -- cycle ;

		%Shape: Ellipse [id:dp01748689504157186] 
		\draw   (184.03,152.13) .. controls (184.03,145.01) and (189.46,139.25) .. (196.16,139.25) .. controls (202.85,139.25) and (208.28,145.01) .. (208.28,152.13) .. controls (208.28,159.24) and (202.85,165) .. (196.16,165) .. controls (189.46,165) and (184.03,159.24) .. (184.03,152.13) -- cycle ;

		%Shape: Ellipse [id:dp9428098642277205] 
		\draw   (251.15,152.13) .. controls (251.15,145.01) and (256.58,139.25) .. (263.28,139.25) .. controls (269.98,139.25) and (275.4,145.01) .. (275.4,152.13) .. controls (275.4,159.24) and (269.98,165) .. (263.28,165) .. controls (256.58,165) and (251.15,159.24) .. (251.15,152.13) -- cycle ;


		%Shape: Ellipse [id:dp8284720105120542] 
		\draw   (320.83,152.13) .. controls (320.83,145.01) and (326.26,139.25) .. (332.95,139.25) .. controls (339.65,139.25) and (345.08,145.01) .. (345.08,152.13) .. controls (345.08,159.24) and (339.65,165) .. (332.95,165) .. controls (326.26,165) and (320.83,159.24) .. (320.83,152.13) -- cycle ;

		%Shape: Ellipse [id:dp7978478412339847] 
		\draw   (386.63,152.13) .. controls (386.63,145.01) and (392.05,139.25) .. (398.75,139.25) .. controls (405.45,139.25) and (410.88,145.01) .. (410.88,152.13) .. controls (410.88,159.24) and (405.45,165) .. (398.75,165) .. controls (392.05,165) and (386.63,159.24) .. (386.63,152.13) -- cycle ;


		%Shape: Ellipse [id:dp12937770899009537] 
		\draw   (167.25,206.44) .. controls (167.25,199.33) and (172.68,193.56) .. (179.38,193.56) .. controls (186.07,193.56) and (191.5,199.33) .. (191.5,206.44) .. controls (191.5,213.55) and (186.07,219.31) .. (179.38,219.31) .. controls (172.68,219.31) and (167.25,213.55) .. (167.25,206.44) -- cycle ;

		%Shape: Ellipse [id:dp04757183323164882] 
		\draw   (200.81,206.44) .. controls (200.81,199.33) and (206.24,193.56) .. (212.94,193.56) .. controls (219.63,193.56) and (225.06,199.33) .. (225.06,206.44) .. controls (225.06,213.55) and (219.63,219.31) .. (212.94,219.31) .. controls (206.24,219.31) and (200.81,213.55) .. (200.81,206.44) -- cycle ;


		%Shape: Ellipse [id:dp22772640796294663] 
		\draw   (234.37,206.44) .. controls (234.37,199.33) and (239.8,193.56) .. (246.5,193.56) .. controls (253.19,193.56) and (258.62,199.33) .. (258.62,206.44) .. controls (258.62,213.55) and (253.19,219.31) .. (246.5,219.31) .. controls (239.8,219.31) and (234.37,213.55) .. (234.37,206.44) -- cycle ;

		%Shape: Ellipse [id:dp17174669939626908] 
		\draw   (267.94,206.44) .. controls (267.94,199.33) and (273.37,193.56) .. (280.06,193.56) .. controls (286.76,193.56) and (292.19,199.33) .. (292.19,206.44) .. controls (292.19,213.55) and (286.76,219.31) .. (280.06,219.31) .. controls (273.37,219.31) and (267.94,213.55) .. (267.94,206.44) -- cycle ;


		%Straight Lines [id:da4745317766148347] 
		\draw    (264,165) -- (280.06,193.56) ;
		%Straight Lines [id:da07339283488170056] 
		\draw    (196.87,165) -- (212.94,193.56) ;
		%Straight Lines [id:da48454934430057994] 
		\draw    (264,165) -- (246.5,193.56) ;
		%Straight Lines [id:da6739639319907231] 
		\draw    (196.16,165) -- (178.65,193.56) ;
		%Straight Lines [id:da025742732922336753] 
		\draw    (229.72,109.25) -- (196.16,139.25) ;
		%Straight Lines [id:da1324756178738058] 
		\draw    (229.72,109.25) -- (263.28,139.25) ;
		%Straight Lines [id:da5822899387100338] 
		\draw    (365.85,110) -- (332.29,140) ;
		%Straight Lines [id:da5970792883885521] 
		\draw    (365.85,110) -- (398.75,139.25) ;
		%Straight Lines [id:da982961690294212] 
		\draw    (296.9,55) -- (231,82.5) ;
		%Straight Lines [id:da48070786942664956] 
		\draw    (296.9,55) -- (365.85,84.25) ;

		% Text Node
		\draw (360.35,88.63) node [anchor=north west][inner sep=0.75pt]   [align=left] {3};
		% Text Node
		\draw (224.22,87.88) node [anchor=north west][inner sep=0.75pt]   [align=left] {2};
		% Text Node
		\draw (190.66,143.63) node [anchor=north west][inner sep=0.75pt]   [align=left] {4};
		% Text Node
		\draw (173.88,197.94) node [anchor=north west][inner sep=0.75pt]   [align=left] {8};
		% Text Node
		\draw (207.44,197.94) node [anchor=north west][inner sep=0.75pt]   [align=left] {9};
		% Text Node
		\draw (257.78,143.63) node [anchor=north west][inner sep=0.75pt]   [align=left] {5};
		% Text Node
		\draw (237,197.94) node [anchor=north west][inner sep=0.75pt]   [align=left] {10};
		% Text Node
		\draw (271.06,197.94) node [anchor=north west][inner sep=0.75pt]   [align=left] {11};
		% Text Node
		\draw (327.45,143.63) node [anchor=north west][inner sep=0.75pt]   [align=left] {6};
		% Text Node
		\draw (393.25,143.63) node [anchor=north west][inner sep=0.75pt]   [align=left] {7};
		% Text Node
		\draw (291.78,32.88) node [anchor=north west][inner sep=0.75pt]   [align=left] {S};


	\end{tikzpicture}}
\end{figure}

Al ser un problema de satisfacibilidad solo nos interesa saber cuál es
el nodo final, no como llegamos a el. En optimización devuelve el
camino.

La lista abierta se implementa con una COLA.

Es COMPLETO y ADMISIBLE.

Tiempo: Depende del número de expansiones \(O(b^d)\)

Memoria: \(O(b^d)\)

El hecho que de que ambos sean exponenciales es bastante
desafortunado.
\pagebreak

\subsection{Primero en Profundidad}

\enquote{Se expande el primero de los nodos recién generados hasta que se
encuentra una solución o se ha alcanzado un \(d_{max}\)}


Se usa una profundidad máxima para evitar caer en una rama infinita
que nos aleja de la solución.

\begin{figure}[H]
	\ffigbox[\FBwidth]
	{\caption{Primero en Profundidad}}
	{

		\tikzset{every picture/.style={line width=0.75pt}} %set default line width to 0.75pt        

	\begin{tikzpicture}[x=0.75pt,y=0.75pt,yscale=-1,xscale=1]
		%uncomment if require: \path (0,300); %set diagram left start at 0, and has height of 300

		%Shape: Ellipse [id:dp7147448304890514] 
		\draw   (299.78,42.06) .. controls (299.78,34.95) and (305.21,29.19) .. (311.91,29.19) .. controls (318.61,29.19) and (324.03,34.95) .. (324.03,42.06) .. controls (324.03,49.17) and (318.61,54.94) .. (311.91,54.94) .. controls (305.21,54.94) and (299.78,49.17) .. (299.78,42.06) -- cycle ;

		%Shape: Ellipse [id:dp20243255870818078] 
		\draw   (231.72,97.06) .. controls (231.72,89.95) and (237.15,84.19) .. (243.84,84.19) .. controls (250.54,84.19) and (255.97,89.95) .. (255.97,97.06) .. controls (255.97,104.17) and (250.54,109.94) .. (243.84,109.94) .. controls (237.15,109.94) and (231.72,104.17) .. (231.72,97.06) -- cycle ;

		%Shape: Ellipse [id:dp06075207361727131] 
		\draw   (367.85,97.81) .. controls (367.85,90.7) and (373.28,84.94) .. (379.98,84.94) .. controls (386.67,84.94) and (392.1,90.7) .. (392.1,97.81) .. controls (392.1,104.92) and (386.67,110.69) .. (379.98,110.69) .. controls (373.28,110.69) and (367.85,104.92) .. (367.85,97.81) -- cycle ;

		%Shape: Ellipse [id:dp7288702595865146] 
		\draw   (198.16,152.81) .. controls (198.16,145.7) and (203.58,139.94) .. (210.28,139.94) .. controls (216.98,139.94) and (222.41,145.7) .. (222.41,152.81) .. controls (222.41,159.92) and (216.98,165.69) .. (210.28,165.69) .. controls (203.58,165.69) and (198.16,159.92) .. (198.16,152.81) -- cycle ;

		%Shape: Ellipse [id:dp15893340072025164] 
		\draw   (265.28,152.81) .. controls (265.28,145.7) and (270.71,139.94) .. (277.4,139.94) .. controls (284.1,139.94) and (289.53,145.7) .. (289.53,152.81) .. controls (289.53,159.92) and (284.1,165.69) .. (277.4,165.69) .. controls (270.71,165.69) and (265.28,159.92) .. (265.28,152.81) -- cycle ;


		%Shape: Ellipse [id:dp9662165256424733] 
		\draw   (335,152.88) .. controls (335,145.76) and (340.43,140) .. (347.13,140) .. controls (353.82,140) and (359.25,145.76) .. (359.25,152.88) .. controls (359.25,159.99) and (353.82,165.75) .. (347.13,165.75) .. controls (340.43,165.75) and (335,159.99) .. (335,152.88) -- cycle ;
		%Shape: Ellipse [id:dp0998222305838985] 
		\draw   (400.75,152.88) .. controls (400.75,145.76) and (406.18,140) .. (412.88,140) .. controls (419.57,140) and (425,145.76) .. (425,152.88) .. controls (425,159.99) and (419.57,165.75) .. (412.88,165.75) .. controls (406.18,165.75) and (400.75,159.99) .. (400.75,152.88) -- cycle ;
		%Shape: Ellipse [id:dp994427506944682] 
		\draw   (181.38,207.13) .. controls (181.38,200.01) and (186.8,194.25) .. (193.5,194.25) .. controls (200.2,194.25) and (205.63,200.01) .. (205.63,207.13) .. controls (205.63,214.24) and (200.2,220) .. (193.5,220) .. controls (186.8,220) and (181.38,214.24) .. (181.38,207.13) -- cycle ;

		%Shape: Ellipse [id:dp4625155723271066] 
		\draw   (214.94,207.13) .. controls (214.94,200.01) and (220.36,194.25) .. (227.06,194.25) .. controls (233.76,194.25) and (239.19,200.01) .. (239.19,207.13) .. controls (239.19,214.24) and (233.76,220) .. (227.06,220) .. controls (220.36,220) and (214.94,214.24) .. (214.94,207.13) -- cycle ;


		%Shape: Ellipse [id:dp9934619630000707] 
		\draw   (248.5,207.13) .. controls (248.5,200.01) and (253.92,194.25) .. (260.62,194.25) .. controls (267.32,194.25) and (272.75,200.01) .. (272.75,207.13) .. controls (272.75,214.24) and (267.32,220) .. (260.62,220) .. controls (253.92,220) and (248.5,214.24) .. (248.5,207.13) -- cycle ;
		%Shape: Ellipse [id:dp24628406315543883] 
		\draw   (282.06,207.13) .. controls (282.06,200.01) and (287.49,194.25) .. (294.19,194.25) .. controls (300.88,194.25) and (306.31,200.01) .. (306.31,207.13) .. controls (306.31,214.24) and (300.88,220) .. (294.19,220) .. controls (287.49,220) and (282.06,214.24) .. (282.06,207.13) -- cycle ;
		%Straight Lines [id:da8102467632383925] 
		\draw    (278.13,165.69) -- (294.19,194.25) ;
		%Straight Lines [id:da8149785410743664] 
		\draw    (211,165.69) -- (227.06,194.25) ;
		%Straight Lines [id:da917783493754426] 
		\draw    (278.13,165.69) -- (260.62,194.25) ;
		%Straight Lines [id:da9063123251939682] 
		\draw    (210.28,165.69) -- (192.78,194.25) ;
		%Straight Lines [id:da945253246951004] 
		\draw    (243.84,109.94) -- (210.28,139.94) ;
		%Straight Lines [id:da5250821856676404] 
		\draw    (243.84,109.94) -- (277.4,139.94) ;
		%Straight Lines [id:da7344096654954528] 
		\draw    (379.98,110.69) -- (346.42,140.69) ;
		%Straight Lines [id:da3071529216977307] 
		\draw    (379.98,110.69) -- (412.88,139.94) ;
		%Straight Lines [id:da6476589848449845] 
		\draw    (311.02,55.69) -- (245.13,83.19) ;
		%Straight Lines [id:da37331982934300845] 
		\draw    (311.02,55.69) -- (379.98,84.94) ;

		% Text Node
		\draw (221.56,198.63) node [anchor=north west][inner sep=0.75pt]   [align=left] {7};
		% Text Node
		\draw (188,198.63) node [anchor=north west][inner sep=0.75pt]   [align=left] {6};
		% Text Node
		\draw (271.9,144.31) node [anchor=north west][inner sep=0.75pt]   [align=left] {5};
		% Text Node
		\draw (204.78,144.31) node [anchor=north west][inner sep=0.75pt]   [align=left] {4};
		% Text Node
		\draw (374.48,89.31) node [anchor=north west][inner sep=0.75pt]   [align=left] {3};
		% Text Node
		\draw (238.34,88.56) node [anchor=north west][inner sep=0.75pt]   [align=left] {2};
		% Text Node
		\draw (305.91,33.56) node [anchor=north west][inner sep=0.75pt]   [align=left] {S};
		% Text Node
		\draw (255.12,198.63) node [anchor=north west][inner sep=0.75pt]   [align=left] {8};
		% Text Node
		\draw (288.69,198.63) node [anchor=north west][inner sep=0.75pt]   [align=left] {9};
		% Text Node
		\draw (337.63,144.38) node [anchor=north west][inner sep=0.75pt]   [align=left] {10};
		% Text Node
		\draw (403.88,144.38) node [anchor=north west][inner sep=0.75pt]   [align=left] {11};


	\end{tikzpicture}}
\end{figure}




Implementa Backtracking con una PILA.

Completo: No, puede entrar en un bucle.

Admisible: No, si no es completo no cumplirá el encontrar la óptima.

Tiempo: \(O(b^d)\)

Memoria: \(O(d)\) Si solo se almacena 1 sucesor, que lo expandirás
cada vez y si se hace backtracking ya se generará. Si se almacenan
todos \(O(bd)\)

Se usa sobre el de amplitud, nos comprometemos hacia una dirección y
esperamos encontrar la solución.


\subsection{Profundidad Iterativa - Iterative Deepening}

\enquote{Consiste en una serie de exploraciones en profundidad donde
\(d_{max}\) se incrementa en k en cada iteración}

En cada iteración anterior a la solución ha expandido todos los
nodos.

Has visto todos los nodos previos a la solución, por lo que habrá
solución en la otra parte del árbol que todavía no se ha recorrido
en una iteración.

Completo: Sí. Es como amplitud ve todo.

Admisible: Si y solo si \(d_{max}=k=1\) Para comprobarlo profundidad
por profundidad, y no soltarse un posible nivel en el otro lado.

Memoria: \(O(d)\) Va en profundidad.

Tiempo: \(\frac {Tiempo (ID)} {Tiempo(BFS)} =\frac {b} {b-1}\)

Es como hacer primero en amplitud, pero con memoria lineal, que
compensa que tarde más (aunque no mucho más).

\begin{itemize}
	\item \(b^d >> \sum\limits_{i=0}^{d-1} b^i\) (\textgreater\textgreater{} Mucho
	      más grande)
\end{itemize}

Expandir TODOS los nodos de TODOS los niveles precedentes no es nada
comparado con lo que se tarda en expandir todos los nodos de una
nueva profundidad d.


\section{Costes arbitrarios}
\subsection{Ramificación y Acotación en Profundidad}


Definimos el coste de un camino \(\pi<s,n>\) como la suma de los
costes de los arcos en \(\pi\).

\begin{itemize}
	\item \(g(\pi)= \sum\limits_{i=0}^{k-1} c(n_i,n_{i+1})\) donde
	      \(c(n_i,n_{i+1})\) es el coste del arco \(<n_i,n_{i+1}>\) (modelo
	      aditivo nosotros lo consideramos siempre, pero en la realidad no
	      tiene por qué). Con frecuencia \(g(\pi)\) se representa como
	      \(g(n)\).
	      \begin{figure}[H]
		      \ffigbox[\FBwidth]
		      {\caption{Ramificación y Acotación}}
		      {

			      \tikzset{every picture/.style={line width=0.75pt}} %set default line width to 0.75pt        

		      \begin{tikzpicture}[x=0.75pt,y=0.75pt,yscale=-1,xscale=1]
			      %uncomment if require: \path (0,300); %set diagram left start at 0, and has height of 300

			      %Shape: Ellipse [id:dp4003665532106575] 
			      \draw   (305.66,61.38) .. controls (305.66,54.26) and (311.09,48.5) .. (317.78,48.5) .. controls (324.48,48.5) and (329.91,54.26) .. (329.91,61.38) .. controls (329.91,68.49) and (324.48,74.25) .. (317.78,74.25) .. controls (311.09,74.25) and (305.66,68.49) .. (305.66,61.38) -- cycle ;

			      %Shape: Ellipse [id:dp07934250802611231] 
			      \draw   (237.59,116.38) .. controls (237.59,109.26) and (243.02,103.5) .. (249.72,103.5) .. controls (256.41,103.5) and (261.84,109.26) .. (261.84,116.38) .. controls (261.84,123.49) and (256.41,129.25) .. (249.72,129.25) .. controls (243.02,129.25) and (237.59,123.49) .. (237.59,116.38) -- cycle ;
			      %Shape: Ellipse [id:dp9367384394905647] 
			      \draw   (373.73,117.13) .. controls (373.73,110.01) and (379.16,104.25) .. (385.85,104.25) .. controls (392.55,104.25) and (397.98,110.01) .. (397.98,117.13) .. controls (397.98,124.24) and (392.55,130) .. (385.85,130) .. controls (379.16,130) and (373.73,124.24) .. (373.73,117.13) -- cycle ;
			      %Shape: Ellipse [id:dp7560627925325043] 
			      \draw   (203.77,172.13) .. controls (203.77,165.01) and (209.19,159.25) .. (215.89,159.25) .. controls (222.59,159.25) and (228.02,165.01) .. (228.02,172.13) .. controls (228.02,179.24) and (222.59,185) .. (215.89,185) .. controls (209.19,185) and (203.77,179.24) .. (203.77,172.13) -- cycle ;

			      %Shape: Ellipse [id:dp9931576053307947] 
			      \draw   (271.15,172.13) .. controls (271.15,165.01) and (276.58,159.25) .. (283.28,159.25) .. controls (289.98,159.25) and (295.4,165.01) .. (295.4,172.13) .. controls (295.4,179.24) and (289.98,185) .. (283.28,185) .. controls (276.58,185) and (271.15,179.24) .. (271.15,172.13) -- cycle ;

			      %Shape: Ellipse [id:dp300916356649114] 
			      \draw   (341,172.38) .. controls (341,165.26) and (346.43,159.5) .. (353.13,159.5) .. controls (359.82,159.5) and (365.25,165.26) .. (365.25,172.38) .. controls (365.25,179.49) and (359.82,185.25) .. (353.13,185.25) .. controls (346.43,185.25) and (341,179.49) .. (341,172.38) -- cycle ;

			      %Shape: Ellipse [id:dp43938209107999815] 
			      \draw   (407,172.13) .. controls (407,165.01) and (412.43,159.25) .. (419.13,159.25) .. controls (425.82,159.25) and (431.25,165.01) .. (431.25,172.13) .. controls (431.25,179.24) and (425.82,185) .. (419.13,185) .. controls (412.43,185) and (407,179.24) .. (407,172.13) -- cycle ;

			      %Shape: Ellipse [id:dp4020358419650045] 
			      \draw   (254.19,226.44) .. controls (254.19,219.33) and (259.61,213.56) .. (266.31,213.56) .. controls (273.01,213.56) and (278.44,219.33) .. (278.44,226.44) .. controls (278.44,233.55) and (273.01,239.31) .. (266.31,239.31) .. controls (259.61,239.31) and (254.19,233.55) .. (254.19,226.44) -- cycle ;
			      %Shape: Ellipse [id:dp9981864633543025] 
			      \draw   (288.09,226.44) .. controls (288.09,219.33) and (293.52,213.56) .. (300.22,213.56) .. controls (306.92,213.56) and (312.34,219.33) .. (312.34,226.44) .. controls (312.34,233.55) and (306.92,239.31) .. (300.22,239.31) .. controls (293.52,239.31) and (288.09,233.55) .. (288.09,226.44) -- cycle ;
			      %Straight Lines [id:da3059365570001984] 
			      \draw    (284,185) -- (300.06,213.56) ;
			      %Straight Lines [id:da5952206635127173] 
			      \draw    (284,185) -- (266.5,213.56) ;
			      %Straight Lines [id:da11491176850041174] 
			      \draw    (249.72,129.25) -- (216.16,159.25) ;
			      %Straight Lines [id:da42874946559440397] 
			      \draw    (249.72,129.25) -- (283.28,159.25) ;
			      %Straight Lines [id:da9331241391070975] 
			      \draw    (385.85,130) -- (352.29,160) ;
			      %Straight Lines [id:da6443612761052202] 
			      \draw    (385.85,130) -- (418.75,159.25) ;
			      %Straight Lines [id:da6568518738253144] 
			      \draw    (316.9,75) -- (251,102.5) ;
			      %Straight Lines [id:da3865035093082203] 
			      \draw    (316.9,75) -- (385.85,104.25) ;

			      % Text Node
			      \draw (413.63,163.63) node [anchor=north west][inner sep=0.75pt]   [align=left] {8};
			      % Text Node
			      \draw (347.63,163.88) node [anchor=north west][inner sep=0.75pt]   [align=left] {7};
			      % Text Node
			      \draw (277.78,163.63) node [anchor=north west][inner sep=0.75pt]   [align=left] {4};
			      % Text Node
			      \draw (210.39,163.63) node [anchor=north west][inner sep=0.75pt]   [align=left] {3};
			      % Text Node
			      \draw (311.78,52.88) node [anchor=north west][inner sep=0.75pt]   [align=left] {S};
			      % Text Node
			      \draw (380.35,108.63) node [anchor=north west][inner sep=0.75pt]   [align=left] {2};
			      % Text Node
			      \draw (244.22,107.88) node [anchor=north west][inner sep=0.75pt]   [align=left] {1};
			      % Text Node
			      \draw (260.81,217.94) node [anchor=north west][inner sep=0.75pt]   [align=left] {5};
			      % Text Node
			      \draw (294.72,217.94) node [anchor=north west][inner sep=0.75pt]   [align=left] {6};
			      % Text Node
			      \draw (358.5,57.4) node [anchor=north west][inner sep=0.75pt]    {$B\leftarrow +\infty \ /19\ /\ 16$};
			      % Text Node
			      \draw (206.39,189.4) node [anchor=north west][inner sep=0.75pt]    {$19$};
			      % Text Node
			      \draw (256.81,244.4) node [anchor=north west][inner sep=0.75pt]    {$21$};
			      % Text Node
			      \draw (290.72,244.4) node [anchor=north west][inner sep=0.75pt]    {$19$};
			      % Text Node
			      \draw (343.63,195.15) node [anchor=north west][inner sep=0.75pt]    {$20$};
			      % Text Node
			      \draw (409.63,191.4) node [anchor=north west][inner sep=0.75pt]    {$16$};


		      \end{tikzpicture}}
	      \end{figure}
\end{itemize}

Pseudocódigo (B, valor arbitrariamente grande)

\begin{enumerate}
	\def\labelenumi{\arabic{enumi}.}
	\item Generar el estado inicial, s.
	\item Coger el primer nodo de Abierta (pila), n.
	\item Si n=t entonces B ←\(g(n)\). Salir (hacer backtracking)
	\item Expandir \(n: n_1, n_2, ..., n_k\)
	\item Si \(g(n_i)<B\), insertar \(n_i\) en abierta,
	      \(\forall _i=1,..., k\)
	\item Volver a 2.
\end{enumerate}

\begin{itemize}
	\item Costes de \(g(\pi)\): \(g(\pi)\) es monótono creciente. Los costes
	      de los arcos son positivos \(c(n_i,n_{i+1}) \geq 0\).
\end{itemize}


\section{Heurísticas}

\begin{enumerate}
	\def\labelenumi{\arabic{enumi}.}
	\item Si no tenemos conocimiento → Búsqueda no informada.

	      Si tenemos información perfecta → no hay búsqueda.
	\item \(h(n,t)\): Devolver una estimación del coste del mejor camino para
	      llegar desde n a t.
	\item Sí \(h(n,t) \leq h^*(n,t)\) entonces h es ADMISIBLE.

	      El nodo terminal debe tener heurística 0.
\end{enumerate}

\subsection{Relajación de restricciones (Judea Pearl, 1983)}



Las restricciones las encontramos en la definición de las
precondiciones de las operaciones.

Se observan las precondiciones y vemos cuál podemos relajar, una vez
relajada hallamos algunas soluciones con esta condición y nos damos
cuenta de cuál es la función heurística en este caso.

Si \(h_1(n)\geq h_2(n) \forall n\) y ambas son admisibles, entonces
\(h_1(n)\) está más informada que \(h_2(n)\).

Preferimos las más informadas.

Si relajamos todas las restricciones se llama Heurística no
informada.

Típicas heurísticas al relajar restricciones:

\begin{itemize}
	\item Casillas mal dispuestas: Número de casillas que no ocupan la
	      posición del estado final.
	\item Distancia de Manhattan: La suma del valor absoluto de las
	      diferencias de las coordenadas.
	\item Distancia Euclídea: Hallar la hipotenusa que crea el estado
	      actual, el final y un punto a la misma de ambos.
\end{itemize}
\pagebreak

\section{Algoritmos de Búsqueda Heurística}

Se aplica tanto a costes unitarios como a costes arbitrarios, pero
necesitamos emplear 1 heurística.

\subsection{Hill-climbing (De escalada)}

\enquote{Se escoge para su expansión el sucesor heurísticamente más
prometedor descartando el resto de sucesores}

\begin{itemize}
	\item Desestima nodos por tener menor coste heurístico.
	\item No hay backtracking, al no almacenarse.
	\item Aun así, se usa frecuentemente.
	\item Completo: no
	\item Admisible: no.
	\item Memoria: \(O(d)\)
\end{itemize}

\subsection{Algoritmo de búsqueda en Haz (Beam search)}

\enquote{Se expanden simultáneamente los k sucesores más prometedores
heurísticamente}

k: ventana o amplitud del haz

Desechamos el resto que no son k. Si coinciden se elige
arbitrariamente.

Completo: No, juzga por la heurística.

Admisibilidad: No, al no ser completo.

\(BS(k=1)\): Es hill-climbing.

\(BS(k= \infty)\): Primero en amplitud.

El k permite decidir la memoria que consume.

Falta de monotonía de la búsqueda en haz, se obtiene una mejor
solución con k+1 que k.

Tiempo: Exponencial.

Memoria: O(k) lineal, solo almacena los k más prometedores.
\pagebreak

\subsection{Algoritmo de \enquote{El mejor primero} (Raphael, Hart, Nilsson, 1968)}


\begin{enumerate}
	\item Considerar:

	      \begin{enumerate}
		      \item Lista abierta: Contiene todos los nodos generados pendientes de
		            ser expandidos, ordena=dos ascendentes por f(n)
		      \item Lista cerrada: Contiene todos los nodos que ya han sido expandidos
		            (duplicate detection, evita expandir un nodo expandido
		            previamente)
		      \item Terminación: Procedemos al expandir t, no al generarlo.
	      \end{enumerate}
	\item Miembros:
	      \begin{itemize}
		      \item \(f(n)=h(n)\) Algoritmo de búsqueda heurística pura.
		      \item \(f(n)=g(n)\) Dijkstra (Realmente es fuerza bruta)
		      \item \(f(n) = g(n) + h(n)\) A*
	      \end{itemize}
	\item Si \(h(n) \leq h^*(n)\) entonces A* es admisible.
	\item Tiempo(A*): Exponencial.

	      Memoria: Exponencial.

	      Especialmente útil por la detección de nodos duplicados.

\end{enumerate}

\subsection{Iterative-deepening A* - IDA* (Korf, 1985)}

\enquote{Consiste en una serie de recorridos del \textbf{primero en
	profundidad hasta que \(f(n) > \eta\) o hemos encontrado la solución},
incrementando f(n) en cada iteración al menor exceso cometido}


\end{document}